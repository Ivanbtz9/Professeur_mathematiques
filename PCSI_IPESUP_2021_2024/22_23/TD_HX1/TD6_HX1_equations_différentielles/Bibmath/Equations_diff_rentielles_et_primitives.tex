\documentclass[11pt]{article}

 %Configuration de la feuille 
 
\usepackage{amsmath,amssymb,enumerate,graphicx,pgf,tikz,fancyhdr}
\usepackage[utf8]{inputenc}
\usetikzlibrary{arrows}
\usepackage{geometry}
\usepackage{tabvar}
\geometry{hmargin=2.2cm,vmargin=1.5cm}\pagestyle{fancy}
\lfoot{\bfseries http://www.bibmath.net}
\rfoot{\bfseries\thepage}
\cfoot{}
\renewcommand{\footrulewidth}{0.5pt} %Filet en bas de page

 %Macros utilisées dans la base de données d'exercices 

\newcommand{\mtn}{\mathbb{N}}
\newcommand{\mtns}{\mathbb{N}^*}
\newcommand{\mtz}{\mathbb{Z}}
\newcommand{\mtr}{\mathbb{R}}
\newcommand{\mtk}{\mathbb{K}}
\newcommand{\mtq}{\mathbb{Q}}
\newcommand{\mtc}{\mathbb{C}}
\newcommand{\mch}{\mathcal{H}}
\newcommand{\mcp}{\mathcal{P}}
\newcommand{\mcb}{\mathcal{B}}
\newcommand{\mcl}{\mathcal{L}}
\newcommand{\mcm}{\mathcal{M}}
\newcommand{\mcc}{\mathcal{C}}
\newcommand{\mcmn}{\mathcal{M}}
\newcommand{\mcmnr}{\mathcal{M}_n(\mtr)}
\newcommand{\mcmnk}{\mathcal{M}_n(\mtk)}
\newcommand{\mcsn}{\mathcal{S}_n}
\newcommand{\mcs}{\mathcal{S}}
\newcommand{\mcd}{\mathcal{D}}
\newcommand{\mcsns}{\mathcal{S}_n^{++}}
\newcommand{\glnk}{GL_n(\mtk)}
\newcommand{\mnr}{\mathcal{M}_n(\mtr)}
\DeclareMathOperator{\ch}{ch}
\DeclareMathOperator{\sh}{sh}
\DeclareMathOperator{\vect}{vect}
\DeclareMathOperator{\card}{card}
\DeclareMathOperator{\comat}{comat}
\DeclareMathOperator{\imv}{Im}
\DeclareMathOperator{\rang}{rg}
\DeclareMathOperator{\Fr}{Fr}
\DeclareMathOperator{\diam}{diam}
\DeclareMathOperator{\supp}{supp}
\newcommand{\veps}{\varepsilon}
\newcommand{\mcu}{\mathcal{U}}
\newcommand{\mcun}{\mcu_n}
\newcommand{\dis}{\displaystyle}
\newcommand{\croouv}{[\![}
\newcommand{\crofer}{]\!]}
\newcommand{\rab}{\mathcal{R}(a,b)}
\newcommand{\pss}[2]{\langle #1,#2\rangle}
 %Document 

\begin{document} 

\begin{center}\textsc{{\huge Equations différentielles et primitives}}\end{center}

% Exercice 398


\vskip0.3cm\noindent\textsc{Exercice 1} - Intégration par parties - Niveau 2
\vskip0.2cm
Déterminer une primitive des fonctions suivantes :
$$\mathbf{1.}\quad x\mapsto\arctan(x)\quad\quad\mathbf{2.}\quad x\mapsto (\ln x)^2\quad\quad\mathbf{3.} x\mapsto \sin(\ln x).$$


% Exercice 399


\vskip0.3cm\noindent\textsc{Exercice 2} - Intégration par parties - Niveau 3
\vskip0.2cm
Calculer les intégrales suivantes :
$$\mathbf{1.}\quad I=\int_1^2\frac{\ln(1+t)}{t^2}dt\quad \mathbf{2.}\quad J=\int_0^1 x(\arctan x)^2dx\quad\quad\mathbf{3.}\quad K=\int_0^1 \frac{x\ln x}{(x^2+1)^2}dx$$


% Exercice 401


\vskip0.3cm\noindent\textsc{Exercice 3} - Une suite d'intégrales
\vskip0.2cm
Pour $(n,p)$ éléments de $\mathbb N^*\times\mathbb N$, on pose 
$$I_{n,p}=\int_0^1 x^n (\ln x)^p dx.$$
Calculer $I_{n,p}$.



% Exercice 400


\vskip0.3cm\noindent\textsc{Exercice 4} - Une suite d'intégrales
\vskip0.2cm
Soient $(\alpha,\beta,n)\in\mathbb R^2\times\mathbb N$. Calculer
$$\int_\alpha^\beta(t-\alpha)^n (t-\beta)^n dt.$$


% Exercice 394


\vskip0.3cm\noindent\textsc{Exercice 5} - Changements de variables - Niveau 2
\vskip0.2cm
En effectuant le changement de variables indiqué, calculer les intégrales suivantes :
\begin{enumerate}
\item  $\displaystyle \int_0^1\frac{dt}{1+e^t}$ en posant $x=e^t$;
\item $\displaystyle \int_1^3\frac{\sqrt t}{t+1}dt$ en posant $x=\sqrt t$;
\item $\displaystyle \int_{-1}^1 \sqrt{1-t^2}dt$ en posant $t=\sin\theta$.
\end{enumerate}


% Exercice 3156


\vskip0.3cm\noindent\textsc{Exercice 6} - Double changement de variables
\vskip0.2cm
\begin{enumerate}
\item Calculer $\displaystyle\int_0^2 \frac{2u}{\sqrt{1+u}}du$.
\item En déduire $\displaystyle \int_0^{3}\frac{dt}{\sqrt{1+\sqrt{1+t}}}$.
\end{enumerate}


% Exercice 440


\vskip0.3cm\noindent\textsc{Exercice 7} - Intégrale trigonométrique - 1
\vskip0.2cm
Calculer les intégrales suivantes :
$$\mathbf{1.}\ \int_0^{\pi/4}\frac{\sin^3(t)}{1+\cos^2 t}dt\quad\quad\mathbf{2.}\ \int_{\pi/3}^{\pi/2}\frac{dx}{\sin x}\quad\quad\mathbf{3.}\ \int_0^{\pi/3}\big(1+\cos(x)\big)\tan(x)dx.$$


% Exercice 441


\vskip0.3cm\noindent\textsc{Exercice 8} - Intégrale trigonométrique - 2
\vskip0.2cm
Calculer les intégrales suivantes :
$$\mathbf{1.}\ \int_0^{\pi/4}\frac{\tan x}{\sqrt 2\cos x+2\sin^2 x}dx\quad\quad\mathbf{2.}\ \int_0^{\pi/2}\frac{dx}{2+\sin x}.$$


% Exercice 3159


\vskip0.3cm\noindent\textsc{Exercice 9} - \'Equations du second ordre à coefficients constants - second membre exponentiel
\vskip0.2cm
Résoudre les équations différentielles suivantes :
\begin{enumerate}
\item $y''-y=e^{2x}-e^x$;
\item $y''+y'+y=\cos(x)$;
\item $y''-2y'+y=\sin^2 x$;
\item $y''+y'+y=e^x\cos(x)$.
\end{enumerate}


% Exercice 3158


\vskip0.3cm\noindent\textsc{Exercice 10} - Un système différentiel qui se ramène à une équation du second ordre
\vskip0.2cm
Déterminer les fonctions $y,z:\mathbb R\to\mathbb R$ dérivables et qui vérifient le système suivant :
$$
\left\{
\begin{array}{rcl}
y'-y&=&z\\
z'+z&=&3y
\end{array}
\right.$$


% Exercice 1458


\vskip0.3cm\noindent\textsc{Exercice 11} - Changement de variables
\vskip0.2cm
On cherche à résoudre sur $\mathbb R_+^*$ l’équation différentielle :
$$x^2y"−3xy'+4y = 0.\ (E)$$
\begin{enumerate}
\item Cette équation est-elle linéaire ? Qu’est-ce qui change par rapport au cours ?
\item Analyse. Soit $y$ une solution de $(E)$ sur $\mathbb R_+^*$. Pour $t\in\mathbb R$, on pose $z(t)=y(e^t)$.
\begin{enumerate}
\item Calculer pour $t\in\mathbb R$, $z'(t)$ et $z''(t)$.
\item En déduire que $z$ vérifie une équation différentielle linéaire d’ordre 2 à coefficients  constants que l’on précisera (on pourra poser $x = e^t$ dans $(E)$). 
\item Résoudre l’équation différentielle trouvée à la question précédente.
\item En déduire le ”portrait robot” de $y$.
\end{enumerate}
\item Synthèse. Vérifier que, réciproquement, les fonctions trouvées à la fin de l’analyse sont bien toutes les solutions de (E) et conclure.
\end{enumerate}




\vskip0.5cm
\noindent{\small Cette feuille d'exercices a été conçue à l'aide du site \textsf{http://www.bibmath.net}}

%Vous avez accès aux corrigés de cette feuille par l'url : https://www.bibmath.net/ressources/justeunefeuille.php?id=25223
\end{document}