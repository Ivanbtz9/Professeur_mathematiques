\documentclass[a4paper,10pt]{article}



\usepackage{fancyhdr} % pour personnaliser les en-têtes
\usepackage[utf8]{inputenc}
\usepackage[T1]{fontenc}
\usepackage{lastpage}
\usepackage[frenchb]{babel}
\usepackage{amsfonts,amssymb}
\usepackage{amsmath,amsthm,mathtools}
\usepackage{paralist}
\usepackage{xspace}
\usepackage{xcolor,multicol}
\usepackage{variations}
\usepackage{xypic}
\usepackage{eurosym}
\usepackage{graphicx}
\usepackage{mathdots}%faire des points suspendus en diagonale
\usepackage[np]{numprint}
\usepackage{hyperref} 
\usepackage{listings} % pour écrire des codes avec coloration syntaxique  

\usepackage{tikz}
\usetikzlibrary{calc, arrows, plotmarks,decorations.pathreplacing}
\usepackage{colortbl}
\usepackage{multirow}
\usepackage[top=2cm,bottom=1.5cm,right=2cm,left=1.5cm]{geometry}

\newtheorem{thm}{Théorème}
\newtheorem*{pro}{Propriété}
\newtheorem*{exemple}{Exemple}

\theoremstyle{definition}
\newtheorem*{remarque}{Remarque}
\theoremstyle{definition}
\newtheorem{exo}{Exercice}
\newtheorem{definition}{Définition}


\newcommand{\vtab}{\rule[-0.4em]{0pt}{1.2em}}
\newcommand{\V}{\overrightarrow}
\renewcommand{\thesection}{\Roman{section} }
\renewcommand{\thesubsection}{\arabic{subsection} }
\renewcommand{\thesubsubsection}{\alph{subsubsection} }
\newcommand*{\transp}[2][-3mu]{\ensuremath{\mskip1mu\prescript{\smash{\mathrm t\mkern#1}}{}{\mathstrut#2}}}%

\newcommand{\K}{\mathbb{K}}
\newcommand{\C}{\mathbb{C}}
\newcommand{\R}{\mathbb{R}}
\newcommand{\Q}{\mathbb{Q}}
\newcommand{\Z}{\mathbb{Z}}
\newcommand{\N}{\mathbb{N}}

\renewcommand{\Im}{\mathop{\mathrm{Im}}\nolimits}



\definecolor{vert}{RGB}{11,160,78}
\definecolor{rouge}{RGB}{255,120,120}
% Set the beginning of a LaTeX document
\pagestyle{fancy}
\lhead{Optimal Sup Spé, groupe IPESUP}\chead{Année~2021-2022}\rhead{Niveau: Première année de PCSI }\lfoot{M. Botcazou}\cfoot{\thepage}\rfoot{mail: ibotca52@gmail.com }\renewcommand{\headrulewidth}{0.4pt}\renewcommand{\footrulewidth}{0.4pt}

\begin{document}
	
	
	\begin{center}
		\Large \sc colle 15 = intégration et Déterminants
	\end{center}








\section*{Intégration :}\hfill\\
\begin{minipage}{1\linewidth}
	\begin{minipage}[t]{0.48\linewidth}
		\raggedright
		
		
		
		\begin{exo}\quad\\
			Soit $f:[a,b]\rightarrow \R$ une fonction continue sur $[a,b]$ ($a<b$).
			\begin{enumerate}
				\item On suppose que $f(x) \ge 0$ pour tout $x\in [a,b]$, et que $f(x_0)>0$ en un point $x_0\in [a,b]$. 
				Montrer que $\int_a^b f(x) d x>0$. En déduire que : <<si $f$ est une fonction continue
				positive sur $[a,b]$ telle que $\int_a^b f(x) d x=0$ alors $f$ est
				identiquement nulle>>.
				\item On suppose que $\int_a^b f(x) d x=0$. Montrer qu'il existe $c\in [a,b]$ tel que $f(c)=0$. 
				\item Application: on suppose
				que $f$ est une fonction continue sur $[0,1]$ telle que $\int_0^1 f(x) dx=\frac 12$. 
				Montrer qu'il existe $d\in [0,1]$ tel que $f(d)=d$.
			\end{enumerate}
			
			\centering
			\rule{1\linewidth}{0.6pt}
		\end{exo}
		
			\begin{exo}\quad\\
			Soit $f$ une fonction de classe $C^1$ sur $[0,1]$ telle que $f(0)=0$. Montrer que $2\int_{0}^{1}f^2(t)\;dt\leq\int_{0}^{1}{f'}^2(t)\;dt$.
			
			\centering
			\rule{1\linewidth}{0.6pt}
		\end{exo}
	
	\end{minipage}	
	\hfill\vrule\hfill
	\begin{minipage}[t]{0.48\linewidth}
		\raggedright
		
	\begin{exo}\quad\\
		Soient les fonctions définies sur $\R$,
		$$f(x)=x \text{ , } g(x)=x^2 \text{ et  } h(x)=e^x,$$
		Justifier qu'elles sont intégrables sur tout intervalle fermé borné de $\R$. En utilisant les
		sommes de Riemann, calculer les intégrales $\int_0^1f(x)d x$, $\int_1^2 g(x)
		d x$ et $\int_0^x h(t) d t$.
		
		\centering
		\rule{1\linewidth}{0.6pt}
	\end{exo}	
		
	
	\begin{exo}\quad\\
		Calculer la limite des suites suivantes :
		\begin{enumerate}
			\item $\displaystyle u_n=n\sum_{k=0}^{n-1}\frac 1{k^2+n^2}$
			\item $\displaystyle v_n=\prod\limits_{k=1}^n\left(1+\frac{k^2}{n^2}\right) ^{\frac 1n}$
		\end{enumerate}
		
		\centering
		\rule{1\linewidth}{0.6pt}
	\end{exo}



		
		
	\end{minipage}
\end{minipage}


\section*{Déterminant:}\hfill\\
\begin{minipage}{1\linewidth}
	\begin{minipage}[t]{0.48\linewidth}
		\raggedright
		
		
		
		\begin{exo}\quad\\
			\begin{enumerate}
				\item Calculer l'aire du parall\'elogramme construit sur les vecteurs 
				$\vec{u} = \left(\begin{array}{c}2\\3\end{array}\right)$ et 
				$\vec{v} = \left(\begin{array}{c}1\\4\end{array}\right)$.
				\item Calculer le volume du  parall\'el\'epip\`ede construit sur les vecteurs \\
				$\vec{u} = \left(\begin{array}{c}1\\2\\0\end{array}\right)$, 
				$\vec{v} = \left(\begin{array}{c}0\\1\\3\end{array}\right)$ et 
				$\vec{w} = \left(\begin{array}{c}1\\1\\1\end{array}\right)$.
				\item Montrer que le volume d'un parall\'el\'epip\`ede dont les sommets sont des points de 
				$\R^3$ \`a coefficients entiers est un nombre entier.
			\end{enumerate}
			
			\centering
			\rule{1\linewidth}{0.6pt}
		\end{exo}
		
		
		
		\begin{exo}\quad\\
			Calculer les déterminants des matrices suivantes :
			
			$$
			\begin{pmatrix}
			a&b&c\\c&a&b\\b&c&a
			\end{pmatrix}
			\quad
			\begin{pmatrix}
			-1 & 1 & 1 & 1\\ 1 & -1 & 1 & 1\\ 1 & 1 & -1& 1\\ 1 & 1& 1&-1
			\end{pmatrix}
			$$
			\centering
			\rule{1\linewidth}{0.6pt}
		\end{exo}
	
		\begin{exo}\quad\\[0.25cm]
		Soit $(a_{0},...,a_{n-1})\in\C^{n}$, $x\in\C$. Calculer
		$$
		\Delta_{n}=
		\left|
		\begin{matrix}
		x &  0    &        & a_{0}   \\
		-1 &\ddots &\ddots  &\vdots  \\
		&\ddots &x      & a_{n-2} \\
		0 &       & -1      & x+a_{n-1}
		\end{matrix}
		\right|
		$$
		\centering
		\rule{1\linewidth}{0.6pt}
	\end{exo}
		
		
		
	\end{minipage}	
	\hfill\vrule\hfill
	\begin{minipage}[t]{0.48\linewidth}
		\raggedright
		
		\begin{exo}\quad\\
			Soit $a$ un réel.
			On note $\Delta_n$ le déterminant suivant : 
			$$
			\Delta_n = 
			\left\vert
			\begin{matrix}
			a   &    0   & \cdots & 0      & n-1 \\
			0   &    a   & \ddots & \vdots & \vdots \\
			\vdots & \ddots & \ddots & 0      & 2 \\
			0   & \cdots &   0    & a      & 1 \\
			n-1  & \cdots &   2    & 1      & a
			\end{matrix}
			\right\vert
			$$
			\begin{enumerate}
				\item Calculer $\Delta_n$ en fonction de $\Delta_{n-1}$.
				\item Démontrer que : $\displaystyle \forall n\geq2\quad
				\Delta_n=a^n-a^{n-2}\sum_{i=1}^{n-1}{i^2}$.
			\end{enumerate}
			
			\centering
			\rule{1\linewidth}{0.6pt}
		\end{exo}
		
		\begin{exo}\quad\\
		Calculer les déterminants des matrices suivantes :
		
		$$
		\begin{pmatrix}
		a&a&b&0 \\  a&a&0&b \\  c&0&a&a \\ 0&c&a&a
		\end{pmatrix}
		\quad
		\begin{pmatrix}
		1&0&0&1&0 \\ 0&-4&3&0&0 \\ -3&0&0&-3&-2 \\ 0&1&7&0&0 \\ 4&0&0&7&1  
		\end{pmatrix}
		$$
			
			\centering
			\rule{1\linewidth}{0.6pt}
		\end{exo}
		
		\begin{exo} \textit{\textbf{Déterminant de Vandermonde}}\quad\\
		Montrer que
		$$\left|
		\begin{array}{ccccc}
		1 & t_1 & t_1^2 & \ldots & t_1^{n-1} \\
		1 & t_2 & t_2^2 & \ldots & t_2^{n-1} \\\
		\ldots&\ldots&\ldots& \ldots & \ldots \\
		1 & t_n & t_n^2 & \ldots & t_n^{n-1}
		\end{array}\right|
		= \prod_{1 \le i < j \le n} (t_j - t_i) $$
			
			\centering
			\rule{1\linewidth}{0.6pt}
		\end{exo}

		
		
		
		
		
	\end{minipage}
\end{minipage}

\end{document}