\documentclass[t,12pt]{beamer}
\usepackage[utf8]{inputenc}
\usepackage[T1]{fontenc}
\usepackage[frenchb]{babel}
\usepackage{amssymb,amsmath,graphicx,amsthm}


\newcommand{\vtab}{\rule[-0.4em]{0pt}{1.2em}}
\newcommand{\V}{\overrightarrow}
\renewcommand{\thesection}{\Roman{section} }
\renewcommand{\thesubsection}{\arabic{subsection} }
\renewcommand{\thesubsubsection}{\alph{subsubsection} }
\newcommand{\C}{\mathbb{C}}
\newcommand{\R}{\mathbb{R}}
\newcommand{\Q}{\mathbb{Q}}
\newcommand{\Z}{\mathbb{Z}}
\newcommand{\N}{\mathbb{N}}


\usetheme{Warsaw}


\begin{document}

\begin{frame}
	\frametitle{Question 1}
Soit un pull en laine chez "H\&M" qui coûte $33$ euros, donner 33\% de sa valeur: 
\end{frame}

\begin{frame}
\frametitle{Question 2}
Soit une paire de "Newbalance" qui coûte 119 euros, après la première démarque des soldes d'hiver elle passe à 83,3 euros. Donner le pourcentage de remise associé à cette baisse. 

\end{frame}


\begin{frame}
\frametitle{Question 3}
Soient des écouteurs sans fils qui coûtent 84 euros, donner leur prix après une remise de 15\% suivie d'une remise de 10\%.  
\end{frame}


\end{document}