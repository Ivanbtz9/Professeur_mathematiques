\documentclass[a4paper,11pt]{article}

\usepackage{inputenc}
\usepackage[T1]{fontenc}
\usepackage[frenchb]{babel}
\usepackage{fancyhdr,fancybox} % pour personnaliser les en-têtes
\usepackage{lastpage,setspace}
\usepackage{amsfonts,amssymb,amsmath,amsthm,mathrsfs}
\usepackage{mathdots}
\usepackage{relsize,exscale,bbold}
\usepackage{paralist}
\usepackage{xspace,multicol,diagbox,array}
\usepackage{xcolor}
\usepackage{variations}
\usepackage{xypic}
\usepackage{eurosym,stmaryrd}
\usepackage{graphicx}
\usepackage[np]{numprint}
\usepackage{hyperref} 
\usepackage{tikz}
\usepackage{colortbl}
\usepackage{multirow}
\usepackage{MnSymbol,wasysym}
\usepackage[top=1.5cm,bottom=1.5cm,right=1.2cm,left=1.5cm]{geometry}
\usetikzlibrary{calc, arrows, plotmarks, babel,decorations.pathreplacing}
\setstretch{1.25}
%\usepackage{lipsum} %\usepackage{enumitem} %\setlist[enumerate]{itemsep=1mm} bug avec enumerate



\newtheorem{thm}{Théorème}
\newtheorem{rmq}{Remarque}
\newtheorem{prop}{Propriété}
\newtheorem{cor}{Corollaire}
\newtheorem{lem}{Lemme}
\newtheorem{prop-def}{Propriété-définition}

\theoremstyle{definition}

\newtheorem{defi}{Définition}
\newtheorem{ex}{Exemple}
\newtheorem*{rap}{Rappel}
\newtheorem{cex}{Contre-exemple}
\newtheorem{exo}{Exercice} % \large {\fontfamily{ptm}\selectfont EXERCICE}
\newtheorem{nota}{Notation}
\newtheorem{ax}{Axiome}
\newtheorem{appl}{Application}
\newtheorem{csq}{Conséquence}
\def\di{\displaystyle}



\renewcommand{\thesection}{\Roman{section}}\renewcommand{\thesubsection}{\arabic{subsection} }\renewcommand{\thesubsubsection}{\alph{subsubsection} }


\newcommand{\bas}{~\backslash}\newcommand{\ba}{\backslash}
\newcommand{\C}{\mathbb{C}}\newcommand{\K}{\mathbb{K}}\newcommand{\R}{\mathbb{R}}\newcommand{\Q}{\mathbb{Q}}\newcommand{\Z}{\mathbb{Z}}\newcommand{\N}{\mathbb{N}}\newcommand{\V}{\overrightarrow}\newcommand{\Cs}{\mathscr{C}}\newcommand{\Ps}{\mathscr{P}}\newcommand{\Rs}{\mathscr{R}}\newcommand{\Gs}{\mathscr{G}}\newcommand{\Ds}{\mathscr{D}}\newcommand{\happy}{\huge\smiley}\newcommand{\sad}{\huge\frownie}\newcommand{\danger}{\begin{tikzpicture}[x=1.5pt,y=1.5pt,rotate=-14.2]
	\definecolor{myred}{rgb}{1,0.215686,0}
	\draw[line width=0.1pt,fill=myred] (13.074200,4.937500)--(5.085940,14.085900)..controls (5.085940,14.085900) and (4.070310,15.429700)..(3.636720,13.773400)
	..controls (3.203130,12.113300) and (0.917969,2.382810)..(0.917969,2.382810)
	..controls (0.917969,2.382810) and (0.621094,0.992188)..(2.097660,1.359380)
	..controls (3.574220,1.726560) and (12.468800,3.984380)..(12.468800,3.984380)
	..controls (12.468800,3.984380) and (13.437500,4.132810)..(13.074200,4.937500)
	--cycle;
	\draw[line width=0.1pt,fill=white] (11.078100,5.511720)--(5.406250,11.875000)..controls (5.406250,11.875000) and (4.683590,12.812500)..(4.367190,11.648400)
	..controls (4.050780,10.488300) and (2.375000,3.675780)..(2.375000,3.675780)
	..controls (2.375000,3.675780) and (2.156250,2.703130)..(3.214840,2.964840)
	..controls (4.273440,3.230470) and (10.640600,4.847660)..(10.640600,4.847660)
	..controls (10.640600,4.847660) and (11.332000,4.953130)..(11.078100,5.511720)
	--cycle;
	\fill (6.144520,8.839900)..controls (6.460940,7.558590) and (6.464840,6.457090)..(6.152340,6.378910)
	..controls (5.835930,6.300840) and (5.320300,7.277400)..(5.003900,8.554750)
	..controls (4.683590,9.835940) and (4.679690,10.941400)..(4.996090,11.019600)
	..controls (5.312490,11.097700) and (5.824210,10.121100)..(6.144520,8.839900)
	--cycle;
	\fill (7.292960,5.261780)..controls (7.382800,4.898500) and (7.128900,4.523500)..(6.730460,4.421880)
	..controls (6.328120,4.324220) and (5.929680,4.535220)..(5.835930,4.898500)
	..controls (5.746080,5.261780) and (5.999990,5.640630)..(6.402340,5.738340)
	..controls (6.804690,5.839840) and (7.203110,5.625060)..(7.292960,5.261780)
	--cycle;
	\end{tikzpicture}}\newcommand{\alors}{\Large\Rightarrow}\newcommand{\equi}{\Leftrightarrow}
\newcommand{\fonction}[5]{\begin{array}{l|rcl}
		#1: & #2 & \longrightarrow & #3 \\
		& #4 & \longmapsto & #5 \end{array}}
\newcommand{\Pb}{\mathbf{P}}

\definecolor{vert}{RGB}{11,160,78}
\definecolor{rouge}{RGB}{255,120,120}
\definecolor{bleu}{RGB}{15,5,107}



\pagestyle{fancy}
\lhead{Groupe IPESUP}\chead{}\rhead{Année~2022-2023}\lfoot{M. Botcazou \& M.Dupré}\cfoot{\thepage/3
}\rfoot{PCSI }\renewcommand{\headrulewidth}{0.4pt}\renewcommand{\footrulewidth}{0.4pt}


\begin{document}
 %%%%BIBMATH%%%%
 
 %(1) https://www.bibmath.net/ressources/index.php?action=affiche&quoi=mpsi/feuillesexo/espaceproba&type=fexo

\noindent\shadowbox{
	\begin{minipage}{1\linewidth}
		\centering
		\huge{\textbf{ TD 19 : Probabilités sur un univers fini  }}
	\end{minipage}}

\bigskip
\section*{Connaître son cours:}
\begin{itemize}[$\bullet$]
	\item Soit $(\Omega, \mathcal P(\Omega))$ un espace probabilisable fini, exprimer la probabilité uniforme. Donner un exemple de votre choix.  
	\item Soit $(\Omega, \mathcal P(\Omega),\Pb)$ un espace probabilisé fini et $(A_k )_{k\leq n}$ une famille d’événements vérifiant $\displaystyle\Pb\left(\underset{k\leq n}{\bigcap} A_k \right) >   0$. Donner et démontrer la formule des probabilités composées. 
	\item Soit $(\Omega, \mathcal P(\Omega),\Pb)$ un espace probabilisé fini et $(A_1,\dots,A_n)_{k\leq n}$ une partition de $\Omega$ d’événements de probabilités non nulles. Soit $B \subset \Omega$, donner et démontrer la formule des probabilités totales.
	\item Soit $(\Omega, \mathcal P(\Omega),\Pb)$ un espace probabilisé fini, $A$ et $B$ deux éléments de $\mathcal P(\Omega)$ de probabilité non nulle. Donner et démontrer la formule de Bayes.


\end{itemize}
\raggedright
\medskip

\section*{Probabilité, dénombrement et indépendance:}\hfill\\%[-0.25cm]
\medskip
   
\begin{minipage}{1\linewidth}\begin{minipage}[c]{0.48\linewidth}\raggedright
	
\begin{exo}\textbf{(*)}\quad\\[0.1cm]
Si $30$ personnes sont présentes à un réveillon et si, à minuit, chaque personne fait $2$ bises à toutes les autres, combien de bises se sont-elles échangées en tout ? (On appelle bise un contact entre deux joues...)

	
\centering\rule{1\linewidth}{0.6pt}\end{exo}



\begin{exo}\textbf{(*)}\quad\\[0.1cm]
	Amédée, Barnabé, Charles tirent sur un oiseau; si les
	probabilités indépendantes de succès sont pour Amédée : $70$\%, Barnabé : $50$\%, Charles : $90$\%, quelle est la probabilité que l'oiseau soit
	touché?

	\centering\rule{1\linewidth}{0.6pt}\end{exo}

\begin{exo}\textbf{(*)}\quad\\[0.1cm]
	Dans un jeu de $52$ cartes, on prend une carte au hasard : les événements <<tirer un roi>> et 
	<<tirer un pique>> sont-ils indépendants? quelle est la probabilité de <<tirer un roi ou
	un pique>> ? 
	
	\centering\rule{1\linewidth}{0.6pt}\end{exo}

		\begin{exo}\textbf{(**)}\quad\\[0.1cm]
	Soit $A$ et $B$ deux événements d'un espace probabilisé tels que $\mathbf{P}(A)=\mathbf{P}(B)=\frac{3}{4}$. 
	
	Déterminer le meilleur encadrement pour $\mathbf{P}(A \cap B)$.
	
	
	\centering\rule{1\linewidth}{0.6pt}\end{exo}




%%%%%%%%%%%%%%%%%%%%%%%%%%%%%%%%%%%%%%%%%%%%%%%%%%%%%%%%%%%%%%%%%%%%%%%%%%%%%%%%%%%%%%%%%%
\end{minipage}\hfill\vrule\hfill\begin{minipage}[c]{0.48\linewidth}\raggedright
%%%%%%%%%%%%%%%%%%%%%%%%%%%%%%%%%%%%%%%%%%%%%%%%%%%%%%%%%%%%%%%%%%%%%%%%%%%%%%%%%%%%%%%%%%

\begin{exo}\textbf{(**)}\quad\\[0.2cm]
Soit $n\geq 1$. On lance $n$ fois un dé parfaitement équilibré. Quelle est la probabilité d'obtenir
\begin{enumerate}
	\item au moins une fois le chiffre 6?
	\item au moins deux fois le chiffre 6?
	\item au moins $k$ fois le chiffre 6?
\end{enumerate}

\centering\rule{1\linewidth}{0.6pt}\end{exo}


\begin{exo}\textbf{(**)}\quad\\[0.2cm]
On dispose d'un dé pipé tel que la probabilité d'obtenir une face soit proportionnelle au chiffre porté par cette face. On lance le dé pipé. 
\begin{enumerate}
	\item Donner un espace probabilisé modélisant l'expérience aléatoire.
	\item Quelle est la probabilité d'obtenir un chiffre pair?
	\item Reprendre les questions si cette fois le dé est pipé de sorte que la probabilité d'une face paire soit le double de la probabilité d'une face impaire.
\end{enumerate}

\centering\rule{1\linewidth}{0.6pt}\end{exo}


\end{minipage}\end{minipage} \newpage


\begin{minipage}{1\linewidth}\begin{minipage}[c]{0.48\linewidth}\raggedright
		
		\begin{exo}\textbf{(*)}\quad\\[0.2cm]
			Lors d'une loterie de Noël, $300$ billets sont
			vendus aux enfants de l'école ; $4$ billets sont gagnants.
			J'achète $10$ billets, quelle est la probabilité pour que je gagne au moins un lot?
			
			\centering\rule{1\linewidth}{0.6pt}\end{exo}
		

		
		
		
	\begin{exo}\textbf{(***)}\quad\\[0.2cm]
		
		Soit $A_{1}, \ldots, A_{n}$ des événements d'un espace probabilisé fini $(\Omega, \mathscr{P}(\Omega), \mathbf{P})$.
		\begin{enumerate}
			\item Montrer que
			$\mathbf{P}\left(A_{1} \cap \ldots \cap A_{n}\right) \geq \sum_{i=1}^{n} \mathbf{P}\left(A_{i}\right)-(n-1)$.
			\item Montrer que $\mathbf{P}\left(A_{1} \cap \ldots \cap A_{n}\right) \leq \min _{1 \leq i \leq n} \mathbf{P}\left(A_{i}\right)$. 
			
			Étudier le cas d'égalité.
			
		\end{enumerate}
		\centering\rule{1\linewidth}{0.6pt}\end{exo}
		
		
		
		
		%%%%%%%%%%%%%%%%%%%%%%%%%%%%%%%%%%%%%%%%%%%%%%%%%%%%%%%%%%%%%%%%%%%%%%%%%%%%%%%%%%%%%%%%%%
	\end{minipage}\hfill\vrule\hfill\begin{minipage}[c]{0.48\linewidth}\raggedright
		%%%%%%%%%%%%%%%%%%%%%%%%%%%%%%%%%%%%%%%%%%%%%%%%%%%%%%%%%%%%%%%%%%%%%%%%%%%%%%%%%%%%%%%%%%
		
		\begin{exo}\textbf{(**)}\quad\\[0.2cm]
			Soit $A_{1}, \ldots, A_{n}$ des événements indépendants d'un espace probabilisé $(\Omega, \mathscr{A}, \mathbf{P})$. Montrer que
			
			$$
			\mathbf{P}\left(\bigcap_{k=1}^{n} \bar{A}_{k}\right) \leq \exp \left(-\sum_{k=1}^{n} \mathbf{P}\left(A_{k}\right)\right) .
			$$
			
			\centering\rule{1\linewidth}{0.6pt}\end{exo}
		
		
		\begin{exo}\textbf{(**)}\quad\\[0.2cm]
			Une personne négligente extrait $2 r$ chaussettes d'un tiroir où se trouvent $n$ paires de chaussettes non triées. Quelle est la probabilité de n'obtenir aucune paire de chaussettes assortie? d'obtenir $k \leq r$ paires assorties
			
			\centering\rule{1\linewidth}{0.6pt}\end{exo}
		
		
\end{minipage}\end{minipage} 
			


\section*{Probabilité conditionnelle :}\hfill\\%[-0.25cm]


\begin{minipage}{1\linewidth}\begin{minipage}[t]{0.48\linewidth}\raggedright
		

		
		\begin{exo}\textbf{(*)}\quad\\[0.2cm]
			Une fête réunit $35$ hommes, $40$ femmes, $25$ enfants. Il y a $3$ urnes $H$, $F$, $E$ contenant des boules de couleurs dont
			respectivement $10$\%, $40$\%, $80$\% de boules noires. Un présentateur
			aux yeux bandés désigne une personne au hasard et lui demande de
			tirer une boule dans l'urne $H$ si cette personne est un homme, dans l'urne $F$
			si cette personne est une femme, dans l'urne $E$ si cette personne est un
			enfant. La boule tirée est noire : quelle est la probabilité pour que
			la boule ait été tirée par un homme? une femme? un enfant? Le présentateur pronostique le genre de la personne au hasard , que doit-il dire pour avoir le moins de risque d'erreur?
			
			
			\centering\rule{1\linewidth}{0.6pt}\end{exo}
		
			\begin{exo}\textbf{(*)}\quad\\[0.2cm]
			Dans les barres de chocolat N., on trouve des images équitablement
			réparties des cinq personnages du dernier Walt Disney, une image par
			tablette. Ma fille veut avoir le héros Prince charmant : combien dois-je
			acheter de barres pour que la probabilité d'avoir la figurine attendue dépasse $80$\%? 
			Même question pour être sûr à $90$\%.
			
			\centering\rule{1\linewidth}{0.6pt}\end{exo}
		
		
		
		
		
		
		
		
		%%%%%%%%%%%%%%%%%%%%%%%%%%%%%%%%%%%%%%%%%%%%%%%%%%%%%%%%%%%%%%%%%%%%%%%%%%%%%%%%%%%%%%%%%%
	\end{minipage}\hfill\vrule\hfill\begin{minipage}[t]{0.48\linewidth}\raggedright
		%%%%%%%%%%%%%%%%%%%%%%%%%%%%%%%%%%%%%%%%%%%%%%%%%%%%%%%%%%%%%%%%%%%%%%%%%%%%%%%%%%%%%%%%%%
		
			\begin{exo}\textbf{(*)}\quad\\[0.2cm]
			Un fumeur, après avoir lu une série de statistiques
			effrayantes sur les risques de cancer, problèmes cardio-vasculaires 
			liés au tabac, décide d'arrêter de fumer; toujours d'après des
			statistiques, on estime les probabilités suivantes : si cette personne
			n'a pas fumé un jour $J_{n}$, alors la probabilité
			pour qu'elle ne fume pas le jour suivant $J_{n+1}$ est $0.3$; 
			mais si elle a fumé un jour $J_{n}$, alors la probabilité 
			pour qu'elle ne fume pas le jour suivant $J_{n+1}$ est $0.9$; 
			quelle est la probabilité $P_{n+1}$ pour qu'elle
			fume le jour $J_{n+1}$ en fonction de la probabilité 
			$P_{n}$ pour qu'elle fume le jour $J_{n}$ ? Quelle est la
			limite de $P_{n}$ ? Va-t-il finir par s'arrêter?
			
			\centering\rule{1\linewidth}{0.6pt}\end{exo}
		
	\begin{exo}\textbf{(**)}\quad\\[0.2cm]
		
		Soit $(\Omega, \mathscr{P}(\Omega), \mathbf{P})$ un espace probabilisé fini, $A \subset \Omega, A_{1}, \ldots, A_{n}$ une partition de $A$ d'événements de probabilités non nulles, et $B \subset \Omega$, telle que la probabilité $\mathbf{P}\left(B \mid A_{k}\right)$ ne dépende pas de $k$. Montrer que, pour tout $k \in \llbracket 1, n \rrbracket, \mathbf{P}\left(B \mid A_{k}\right)=\mathbf{P}(B \mid A)$.
		
		
		\centering\rule{1\linewidth}{0.6pt}\end{exo}
		
		
		
		
\end{minipage}\end{minipage} \newpage

\begin{minipage}{1\linewidth}\begin{minipage}[c]{0.48\linewidth}\raggedright

\begin{exo}\textbf{(**)}\quad\textit{(Loi de \sc Hardy-Weinberg)}\\[0.2cm]
	Considérons un gène qui se présente sous deux allèles (c'est-à-dire deux variantes) $A$ et $a$. Un individu dispose de deux allèles d'un même gène donc $A A, A a$ ou $a a$. Un individu reçoit un allèle de chacun de ses parents au hasard. Notons $p, q$ et $r$ les proportions des génotypes dans une génération et $P, Q$ et $R$ les proportions dans la génération suivante. Montrer que $Q^{2}=4 P R$.
	
	
	\centering\rule{1\linewidth}{0.6pt}\end{exo}



\begin{exo}\textbf{(***)}\quad\textit{(Succession de \sc Laplace)}\\[0.2cm]
	
	Considérons $n+1$ urnes numérotées de 0 à $n$ telle que l'urne numérotée $k$ contienne $n-k$ boules colorées et $k$ boules blanches. On choisit une urne au hasard et on effectue des tirages avec remises au hasard dans cette urne.
	
	\begin{enumerate}
		\item Déterminer la probabilité $p_{n}$ que les $N$ premiers tirages amènent des boules blanches. Déterminer la limite de $p_{n}$ lorsque $n$ tend vers $+\infty$.
		
		\item Déterminer la probabilité $q_{n}$ que la $(N+1)^{\text {ième }}$ boule tirée soit colorée sachant que les $N$ premières boules tirées étaient blanches. Déterminer la limite de $q_{n}$ lorsque $n$ tend vers $+\infty$.
		
	\end{enumerate}
	
	\centering\rule{1\linewidth}{0.6pt}\end{exo}




%%%%%%%%%%%%%%%%%%%%%%%%%%%%%%%%%%%%%%%%%%%%%%%%%%%%%%%%%%%%%%%%%%%%%%%%%%%%%%%%%%%%%%%%%%
\end{minipage}\hfill\vrule\hfill\begin{minipage}[c]{0.48\linewidth}\raggedright
%%%%%%%%%%%%%%%%%%%%%%%%%%%%%%%%%%%%%%%%%%%%%%%%%%%%%%%%%%%%%%%%%%%%%%%%%%%%%%%%%%%%%%%%%%

\begin{exo}\textbf{(***)}\quad\\[0.2cm]
	Soit $n \in \mathbb{N}^{*}$ un entier fixé. Une urne contient $b$ boules blanches et $c$ boules colorées; tirons les boules avec les règles suivantes
	
	- si la boule est blanche, on la retire définitivement;
	
	- si la boule est colorée, on la replace dans l'urne.
	
	Déterminer la probabilité d'obtenir exactement une boule blanche en $n$ tirages.
	
	\centering\rule{1\linewidth}{0.6pt}\end{exo}


\begin{exo}\textbf{(****)}\quad\textit{(Urne de \sc Polya)}\\[0.2cm]
	
	On considère une urne contenant $a$ boules colorées et $b$ boules blanches. Après chaque tirage, la boule extraite est remise dans l'urne avec $c$ boules de la même couleur.
	\begin{enumerate}
		\item Pour $a=3, b=2, c=4$, faire un schéma pour le premier tirage
		\item Déterminer la probabilité que la n-ième boule tirée soit blanche.
	\end{enumerate}
	
	
	
	\centering\rule{1\linewidth}{0.6pt}\end{exo}


\end{minipage}\end{minipage} \newpage

\end{document}