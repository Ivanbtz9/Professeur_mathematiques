
%%%%%%%%%%%%%%%%%% PREAMBULE %%%%%%%%%%%%%%%%%%

\documentclass[11pt,a4paper]{article}

\usepackage{amsfonts,amsmath,amssymb,amsthm}
\usepackage[utf8]{inputenc}
\usepackage[T1]{fontenc}
\usepackage[francais]{babel}
\usepackage{mathptmx}
\usepackage{fancybox}
\usepackage{graphicx}
\usepackage{ifthen}

\usepackage{tikz}   

\usepackage{hyperref}
\hypersetup{colorlinks=true, linkcolor=blue, urlcolor=blue,
pdftitle={Exo7 - Exercices de mathématiques}, pdfauthor={Exo7}}

\usepackage{geometry}
\geometry{top=2cm, bottom=2cm, left=2cm, right=2cm}

%----- Ensembles : entiers, reels, complexes -----
\newcommand{\Nn}{\mathbb{N}} \newcommand{\N}{\mathbb{N}}
\newcommand{\Zz}{\mathbb{Z}} \newcommand{\Z}{\mathbb{Z}}
\newcommand{\Qq}{\mathbb{Q}} \newcommand{\Q}{\mathbb{Q}}
\newcommand{\Rr}{\mathbb{R}} \newcommand{\R}{\mathbb{R}}
\newcommand{\Cc}{\mathbb{C}} \newcommand{\C}{\mathbb{C}}
\newcommand{\Kk}{\mathbb{K}} \newcommand{\K}{\mathbb{K}}

%----- Modifications de symboles -----
\renewcommand{\epsilon}{\varepsilon}
\renewcommand{\Re}{\mathop{\mathrm{Re}}\nolimits}
\renewcommand{\Im}{\mathop{\mathrm{Im}}\nolimits}
\newcommand{\llbracket}{\left[\kern-0.15em\left[}
\newcommand{\rrbracket}{\right]\kern-0.15em\right]}
\renewcommand{\ge}{\geqslant} \renewcommand{\geq}{\geqslant}
\renewcommand{\le}{\leqslant} \renewcommand{\leq}{\leqslant}

%----- Fonctions usuelles -----
\newcommand{\ch}{\mathop{\mathrm{ch}}\nolimits}
\newcommand{\sh}{\mathop{\mathrm{sh}}\nolimits}
\renewcommand{\tanh}{\mathop{\mathrm{th}}\nolimits}
\newcommand{\cotan}{\mathop{\mathrm{cotan}}\nolimits}
\newcommand{\Arcsin}{\mathop{\mathrm{arcsin}}\nolimits}
\newcommand{\Arccos}{\mathop{\mathrm{arccos}}\nolimits}
\newcommand{\Arctan}{\mathop{\mathrm{arctan}}\nolimits}
\newcommand{\Argsh}{\mathop{\mathrm{argsh}}\nolimits}
\newcommand{\Argch}{\mathop{\mathrm{argch}}\nolimits}
\newcommand{\Argth}{\mathop{\mathrm{argth}}\nolimits}
\newcommand{\pgcd}{\mathop{\mathrm{pgcd}}\nolimits} 

%----- Structure des exercices ------

\newcommand{\exercice}[1]{\video{0}}
\newcommand{\finexercice}{}
\newcommand{\noindication}{}
\newcommand{\nocorrection}{}

\newcounter{exo}
\newcommand{\enonce}[2]{\refstepcounter{exo}\hypertarget{exo7:#1}{}\label{exo7:#1}{\bf Exercice \arabic{exo}}\ \  #2\vspace{1mm}\hrule\vspace{1mm}}

\newcommand{\finenonce}[1]{
\ifthenelse{\equal{\ref{ind7:#1}}{\ref{bidon}}\and\equal{\ref{cor7:#1}}{\ref{bidon}}}{}{\par{\footnotesize
\ifthenelse{\equal{\ref{ind7:#1}}{\ref{bidon}}}{}{\hyperlink{ind7:#1}{\texttt{Indication} $\blacktriangledown$}\qquad}
\ifthenelse{\equal{\ref{cor7:#1}}{\ref{bidon}}}{}{\hyperlink{cor7:#1}{\texttt{Correction} $\blacktriangledown$}}}}
\ifthenelse{\equal{\myvideo}{0}}{}{{\footnotesize\qquad\texttt{\href{http://www.youtube.com/watch?v=\myvideo}{Vidéo $\blacksquare$}}}}
\hfill{\scriptsize\texttt{[#1]}}\vspace{1mm}\hrule\vspace*{7mm}}

\newcommand{\indication}[1]{\hypertarget{ind7:#1}{}\label{ind7:#1}{\bf Indication pour \hyperlink{exo7:#1}{l'exercice \ref{exo7:#1} $\blacktriangle$}}\vspace{1mm}\hrule\vspace{1mm}}
\newcommand{\finindication}{\vspace{1mm}\hrule\vspace*{7mm}}
\newcommand{\correction}[1]{\hypertarget{cor7:#1}{}\label{cor7:#1}{\bf Correction de \hyperlink{exo7:#1}{l'exercice \ref{exo7:#1} $\blacktriangle$}}\vspace{1mm}\hrule\vspace{1mm}}
\newcommand{\fincorrection}{\vspace{1mm}\hrule\vspace*{7mm}}

\newcommand{\finenonces}{\newpage}
\newcommand{\finindications}{\newpage}


\newcommand{\fiche}[1]{} \newcommand{\finfiche}{}
%\newcommand{\titre}[1]{\centerline{\large \bf #1}}
\newcommand{\addcommand}[1]{}

% variable myvideo : 0 no video, otherwise youtube reference
\newcommand{\video}[1]{\def\myvideo{#1}}

%----- Presentation ------

\setlength{\parindent}{0cm}

\definecolor{myred}{rgb}{0.93,0.26,0}
\definecolor{myorange}{rgb}{0.97,0.58,0}
\definecolor{myyellow}{rgb}{1,0.86,0}

\newcommand{\LogoExoSept}[1]{  % input : echelle       %% NEW
{\usefont{U}{cmss}{bx}{n}
\begin{tikzpicture}[scale=0.1*#1,transform shape]
  \fill[color=myorange] (0,0)--(4,0)--(4,-4)--(0,-4)--cycle;
  \fill[color=myred] (0,0)--(0,3)--(-3,3)--(-3,0)--cycle;
  \fill[color=myyellow] (4,0)--(7,4)--(3,7)--(0,3)--cycle;
  \node[scale=5] at (3.5,3.5) {Exo7};
\end{tikzpicture}}
}


% titre
\newcommand{\titre}[1]{%
\vspace*{-4ex} \hfill \hspace*{1.5cm} \hypersetup{linkcolor=black, urlcolor=black} 
\href{http://exo7.emath.fr}{\LogoExoSept{3}} 
 \vspace*{-5.7ex}\newline 
\hypersetup{linkcolor=blue, urlcolor=blue}  {\Large \bf #1} \newline 
 \rule{12cm}{1mm} \vspace*{3ex}}

%----- Commandes supplementaires ------



\begin{document}

%%%%%%%%%%%%%%%%%% EXERCICES %%%%%%%%%%%%%%%%%%
\fiche{f00018, bodin, 2007/09/01} 

\titre{Applications linéaires}

\renewcommand{\ker}{\mathop{\mathrm{Ker}}\nolimits}

\section{Définition}          
\exercice{929, legall, 1998/09/01}
\video{4CS7MiS5AQA}
\enonce{000929}{}
D\'eterminer si les applications $f_{i}$ suivantes sont lin\'eaires :
$$\begin{array}{rl}
f_1 : \Rr^2 \to \Rr^2 & f_1(x,y)=(2x+y,x-y)  \\
f_2 : \Rr^3 \to \Rr^3 & f_2(x,y,z)=(xy,x,y) \\
f_3 : \Rr^3 \to \Rr^3 & f_3(x,y,z)=(2x+y+z,y-z,x+y) \\
f_4 : \Rr^2 \to \Rr^4 & f_4(x,y)=(y,0,x-7y,x+y) \\
f_5 : \Rr_3[X] \to \Rr^3 & f_5(P) = \big( P(-1), P(0), P(1) \big) \\
\end{array}
$$
\finenonce{000929} 


\finexercice
\exercice{930, legall, 1998/09/01}
\video{EQJ4bUFDuiQ}
\enonce{000930}{}
Soit $E$ un espace vectoriel de dimension $n$ et
$\phi $  une application lin\'eaire de  $E$  dans lui-m\^eme telle que  $\phi ^n=0$  et
$\phi ^{n-1}\not = 0$.
Soit  $x\in E$  tel que  $\phi ^{n-1}(x )\not = 0$. Montrer que la
famille  $\{ x,\phi(x),\phi^2(x), \ldots ,\phi ^{n-1}(x)\} $  est une base de $E$.

\finenonce{000930} 


\finexercice

\section{Image et noyau}
\exercice{934, cousquer, 2003/10/01}
\video{pNH1v_tXHvg}
\enonce{000934}{}
Soit $E$ un espace vectoriel et soient $E_1$ et $E_2$ deux sous-espaces vectoriels de dimension finie
de $E$, on d\'efinit l'application $f\colon E_1\times E_2 \to E$ par $f(x_1,x_2)=x_1+x_2$.
\begin{enumerate}
\item Montrer que $f$ est lin\'eaire.
\item D\'eterminer le noyau et l'image de $f$.
\item Que donne le th\'eor\`eme du rang ?
\end{enumerate}

\finenonce{000934} 


\finexercice
\exercice{943, legall, 1998/09/01}
\video{yGZYazY1EhM}
\enonce{000943}{}
Soit $E$  un espace vectoriel de dimension  $n$
et $f$ une application lin\' eaire de  $E$  dans lui-m\^eme.
Montrer que les deux assertions qui suivent sont \' equivalentes :
\begin{enumerate}
\item[(i)] $\ker f = \Im f$
\item[(ii)] $f^2=0 \ \text{ et } \  n=2\cdot \text{rg}(f)$
\end{enumerate}
\finenonce{000943}
 

\finexercice
\exercice{947, ridde, 1999/11/01}
\video{qUT7yUQxk4w}
\enonce{000947}{}
Soient $f$ et $g$ deux endomorphismes de $E$ tels que $f \circ g = g \circ f$.
Montrer que $\ker f$ et $\Im f$ sont stables par $g$.
\finenonce{000947} 


\finexercice
\exercice{1027, ridde, 1999/11/01}
\video{GU9xDUgfz48}
\enonce{001027}{}
Soit $E$ et $F$ de dimensions finies et $u, v \in \mathcal{L} (E, F)$.
\begin{enumerate}
\item Montrer que $\text{rg} (u + v) \leq \text{rg} (u) + \text{rg} (v)$.
\item En d\'eduire que $\left|\text{rg} (u) - \text{rg} (v)\right| \leq
\text{rg} (u + v)$.
\end{enumerate}

\finenonce{001027} 


\finexercice

\section{Injectivité, surjectivité, isomorphie}
\exercice{956, legall, 1998/09/01}
\video{DY3GrL-j6C4}
\enonce{000956}{}
Pour les applications lin\'eaires suivantes,  d\'eterminer $\ker f_i$ et 
$\Im f_i$. En d\'eduire si $f_i$ est injective, surjective, bijective.
$$\begin{array}{rl}
f_1 : \Rr^2 \to \Rr^2 & f_1(x,y)=(2x+y,x-y)  \\
f_2 : \Rr^3 \to \Rr^3 & f_2(x,y,z)=(2x+y+z,y-z,x+y) \\
f_3 : \Rr^2 \to \Rr^4 & f_3(x,y)=(y,0,x-7y,x+y) \\
f_4 : \Rr_3[X] \to \Rr^3 & f_4(P) = \big( P(-1), P(0), P(1) \big) \\
\end{array}
$$
\finenonce{000956} 


\finexercice
\exercice{954, cousquer, 2003/10/01}
\video{nAFU6wBhyrE}
\enonce{000954}{}
 Soit $E$ un espace vectoriel de dimension $3$, $\{e_1,e_2,e_3\}$ une base
de $E$, et $t$ un param\`etre r\'eel. \\
D\'emontrer que la donn\'ee de
$\left\{
\begin{array}{rcl}
    \phi(e_1) & = & e_1+e_2  \\
    \phi(e_2) & = & e_1-e_2  \\
    \phi(e_3) & = & e_1+t e_3
\end{array}\right.$
d\'efinit une application lin\'eaire
$\phi$ de $E$ dans $E$. \'Ecrire le transform\'e du vecteur 
$x=\alpha_1e_1+\alpha_2e_2+\alpha_3e_3$. Comment choisir $t$ pour que 
$\phi$ soit injective ? surjective ?
\finenonce{000954} 


\finexercice
\exercice{963, legall, 1998/09/01}
\video{ZIq3tKZZgfg}
\enonce{000963}{}
Soit  $E$  et  $F$  deux espaces vectoriels de dimension finie
et  $\phi $  une application lin\' eaire  de  $E$  dans  $F$.
Montrer que  $\phi $  est un isomorphisme si et seulement si l'image par  $\phi $  de
toute base de  $E$  est une base de  $F$.
\finenonce{000963} 


\finexercice

\section{Morphismes particuliers}
\exercice{974, gourio, 2001/09/01}
\video{6N7D6lPLPHc}
\enonce{000974}{}
Soit $E$ l'espace vectoriel des fonctions de $\Rr$ dans $\Rr$. Soient $P$ le
sous-espace des fonctions paires et $I$ le sous-espace des fonctions
impaires. Montrer que $E=P\bigoplus I$. Donner l'expression du projecteur sur
$P$ de direction $I$.

\finenonce{000974} 


\finexercice
\exercice{959, legall, 1998/09/01}
\video{ce0mL82x6Y0}
\enonce{000959}{}
Soit $E = \Rr_n[X]$ et soient $A$ et $B$ deux polyn\^omes \`a coefficients réels de
degr\'e $n+1$. On consid\`ere l'application $f$ qui \`a tout polyn\^ome $P$ de $E$, associe
le reste de la division euclidienne de $AP$ par $B$.
\begin{enumerate}
    \item Montrer que $f$ est un endomorphisme de $E$.
    \item Montrer l'\'equivalence
$$
f \hbox{ est bijective} \Longleftrightarrow \hbox{$A$ et $B$ sont premiers entre eux}.
$$
\end{enumerate}
\finenonce{000959} 


\finexercice
\exercice{976, gourio, 2001/09/01}
\video{yivtpWQkPFg}
\enonce{000976}{}
Soit $E={\Rr}_{n}[X]$ l'espace vectoriel des polyn\^{o}mes de degr\'{e} $%
\leq n$, et $f:E\rightarrow E$ d\'{e}finie par:
$$f(P)=P+(1-X)P'. $$
Montrer que $f$ est une application linéaire et donner une base de $\Im f$ et de $\ker f.$
\finenonce{000976} 


\finexercice

\finfiche

 \finenonces 



 \finindications 

\indication{000929}
Une seule application n'est pas lin\'eaire.
\finindication
\indication{000930}
Prendre une combinaison lin\'eaire nulle et l'\'evaluer par $\phi^{n-1}$.
\finindication
\indication{000934}
Faire un dessin de l'image et du noyau pour $f: \Rr\times \Rr \longrightarrow \Rr$.
Montrer que le noyau est isomorphe à $E_1 \cap E_2$.
\finindication
\indication{000943}
Pour chacune des implications utiliser la formule du rang.
\finindication
\indication{000947}
Dire qu'un sous-espace $F$ est stable par $g$ signifie que $g(F) \subset F$.
\finindication
\noindication
\noindication
\indication{000954}
$t=0$ est un cas à part.
\finindication
\indication{000963}
Pour une base $\mathcal{B} =\{ e_1, \ldots , e_n \}$ de $E$ considérer la famille 
$\{ \phi (e_1), \ldots, \phi (e_n) \} $.
\finindication
\indication{000974}
Pour une fonction $f$ on peut \'ecrire 
$$f(x)= \frac{f(x)+f(-x)}{2}+\frac{f(x)-f(-x)}{2}.$$

Le projecteur sur $P$ de direction $I$ est l'application $\pi : E \longrightarrow E$
qui vérifie $\pi(f)\in P$, $\pi \circ \pi = \pi$ et $\ker \pi = I$.
\finindication
\indication{000959}
Résultats utiles d'arithmétique des polynômes : la division euclidienne, le théorème de Bézout,
le lemme de Gauss.
\finindication
\indication{000976}
$P'$ désigne la dérivée de $P$.
Pour trouver le noyau, résoudre une équation différentielle.
Pour l'image calculer les $f(X^k)$.
\finindication


\newpage

\correction{000929}
\begin{enumerate}
  \item $f_1$ est linéaire. Pour $(x,y) \in \Rr^2$ et $(x',y')\in \Rr^2$ :
\begin{align*}
f_1\big( (x,y)+(x',y') \big) 
  & = f_1\big( x+x',y+y' \big) \\
  & = \big(2(x+x')+ (y+y'), (x+x')-(y+y') \big) \\
  & = \big(2x+y +2x'+y', x-y+x'-y' \big) \\
  & = \big(2x+y,x-y \big) + \big(2x'+y',x'-y' \big) \\
  & = f_1(x,y)+f_1(x',y')
\end{align*}

Pour $(x,y)\in \Rr^2$ et $\lambda\in \Rr$ :
$$f_1\big( \lambda \cdot (x,y)\big) = f_1\big(\lambda x,\lambda y\big)
 = \big( 2\lambda x+\lambda y,\lambda x - \lambda y \big) = \lambda \cdot\big(2x+y,x-y \big) 
= \lambda \cdot f_1(x,y).$$

  \item $f_2$ n'est pas lin\'eaire, en effet par exemple $f_2(1,1,0)+f_2(1,1,0)$ n'est pas \'egal \`a $f_2(2,2,0)$.

  \item $f_3$ est linéaire : il faut vérifier d'abord que pour tout $(x,y,z)$ et $(x',y',z')$ alors
$f_3\big( (x,y,z) + (x',y',z') \big) = f_3(x,y,z)+f_3(x',y',z')$. Et ensuite que pour tout $(x,y,z)$ et $\lambda$
on a $f_3\big(\lambda\cdot(x,y,z) \big) = \lambda \cdot f_3(x,y,z)$.


  \item $f_4$ est linéaire : il faut vérifier d'abord que pour tout $(x,y)$ et $(x',y')$ alors
$f_4\big( (x,y) + (x',y') \big) = f_4(x,y)+f_4(x',y')$. Et ensuite que pour tout $(x,y)$ et $\lambda$
on a $f_4\big(\lambda\cdot(x,y) \big) = \lambda \cdot f_4(x,y)$.

  \item $f_5$ est lin\'eaire : soient $P,P' \in \Rr_3[X]$ alors 

\begin{align*}
f_5\big(P+P'\big) 
  & = \big( (P+P')(-1), (P+P')(0), (P+P')(1) \big) \\
  & = \big( P(-1)+P'(-1), P(0)+P'(0), P(1)+P'(1) \big) \\
  & = \big( P(-1), P(0), P(1) \big)  + \big( P'(-1), P'(0), P'(1) \big) \\
  & = f_5(P)+f_5(P')  \\
\end{align*}

Et si $P\in \Rr_3[X]$ et $\lambda \in \Rr$ :
\begin{align*}
f_5\big( \lambda \cdot P\big) 
  & =  \big( (\lambda P)(-1), (\lambda P)(0), (\lambda P)(1) \big) \\
  & = \big( \lambda \times P(-1), \lambda \times P(0), \lambda \times P(1) \big) \\
  & = \lambda \cdot \big( P(-1), P(0), P(1) \big) \\
  & = \lambda \cdot f_5(P) \\
\end{align*}
\end{enumerate}
\fincorrection
\correction{000930}
Montrons que la famille  $\{ x,\phi(x),\phi^2(x), \ldots , \phi ^{n-1}(x) \}$ est
libre. Soient  $\lambda _0, \ldots , \lambda _{n-1} \in {\R}$ tels
que  $\lambda _0 x + \lambda_1\phi(x)+\cdots + \lambda _{n-1} \phi ^{n-1}(x)=0$.
Alors : $\phi ^{n-1} \big(\lambda _0 x+ \lambda_1\phi(x) + \cdots + \lambda _{n-1}
\phi ^{n-1}(x)\big)=0$. Mais comme de plus $\phi ^n=0 $, on a
l'\'egalit\'e $\phi ^{n-1} \big(\lambda _0 x+ \lambda_1\phi(x) + \cdots + \lambda
_{n-1} \phi ^{n-1}(x)\big)=\phi ^{n-1} (\lambda _0 x ) + \phi
^n \big(\lambda _1 x+ \cdots + \lambda _{n-1} \phi
^{n-2}(x)\big)= \phi ^{n-1}(\lambda _0x)=\lambda _0 \phi ^{n-1}(x)$. Comme  $\phi
^{n-1}(x) \not= 0$  on obtient  $\lambda _0=0$.

\noindent En calculant ensuite  $\phi ^{n-2}\big(\lambda _1 \phi
(x )+ \cdots + \lambda _{n-1} \phi ^{n-1}(x)\big)$  on obtient
$\lambda _1=0$  puis, de proche en proche,  $\lambda_2=2$,\ldots, $\lambda _{n-1}=0$. 
La famille  $\{ x, \phi(x), \ldots , \phi
^{n-1}(x) \}$  est donc libre. En plus elle compte  $n$  vecteurs, comme
$\dim E=n$  elle est libre et maximale et forme donc une base
de $E$.
\fincorrection
\correction{000934}
\begin{enumerate}
  \item Aucun problème...
  \item Par d\'efinition de $f$ et de ce qu'est la somme de deux sous-espaces vectoriels, l'image est
$$\Im f =  \{ f(x_1,x_2) \mid x_1 \in E_1, x_2\in E_2 \} = \{ x_1+x_2 \mid x_1 \in E_1, x_2\in E_2 \} = E_1 + E_2.$$

Pour le noyau :
$$\ker f = \{ (x_1,x_2) \mid f(x_1,x_2)=0 \} = \{ (x_1,x_2) \mid x_1+x_2=0 \}$$

Mais on peut aller un peu plus loin. En effet un \'el\'ement $(x_1,x_2) \in \ker f$,
v\'erifie $x_1\in E_1$, $x_2\in E_2$ et $x_1=-x_2$. Donc $x_1 \in E_2$. Donc $x_1\in E_1 \cap E_2$.
R\'eciproquement si $x\in E_1 \cap E_2$, alors $(x,-x)\in \ker f$.
Donc 
$$\ker f = \{ (x,-x) \mid x \in  E_1 \cap E_2 \}. $$
De plus l'application $x \mapsto (x,-x)$ montre que $\ker f$ est isomorphe \`a $E_1 \cap E_2$.

\item Le th\'eor\`eme du rang s'\'ecrit :
$$\dim \ker f+ \dim \Im f = \dim (E_1\times E_2).$$
Compte tenu de  l'isomorphisme entre $\ker f$ et $E_1 \cap E_2$ on obtient :
$$\dim (E_1 \cap E_2) + \dim (E_1+E_2) = \dim (E_1\times E_2).$$
Mais $\dim (E_1\times E_2) = \dim E_1 + \dim E_2$, donc on retrouve ce que l'on appelle le th\'eor\`eme des quatre dimensions :
$$\dim (E_1+E_2) = \dim E_1+\dim E_2-\dim (E_1 \cap E_2).$$

\end{enumerate}
\fincorrection
\correction{000943}
\begin{itemize}
  \item[(i) $\Rightarrow$ (ii)] Supposons
$\ker f = \Im f$. Soit $x\in E$, alors $f(x) \in \Im f$ donc $f(x)
\in \ker f$, cela entraîne $f(f(x)) = 0$ ; donc $f^2=0$. De plus
d'apr\`es la formule du rang $\dim \ker f + \text{rg}  (f) = n$, mais $\dim
\ker f = \dim \Im f = \text{rg}  f$, ainsi $2\text{rg}  (f)=n$.
  \item[(ii) $\Rightarrow$ (i)] Si $f^2 = 0$ alors
$\Im f \subset \ker f$ car pour $y\in \Im f$ il existe $x$ tel que
$y=f(x)$ et $f(y)=f^2(x)=0$. De plus si $2 \text{rg}  (f) = n$ alors la
formule du rang donne $\dim \ker f = \text{rg}  (f)$ c'est-\`a-dire $\dim \ker f =
\dim \Im f$. Nous savons donc que $\Im f$ est inclus dans $\ker f$
mais ces espaces sont de m\^eme dimension donc sont \'egaux :
$\ker f = \Im f$.
 \end{itemize}
\fincorrection
\correction{000947}
On va montrer $g(\ker f) \subset \ker f$.
Soit $y \in g(\ker f)$. Il existe $x\in \ker f$ tel que $y=g(x)$.
Montrons $y\in \ker f$:
$$f(y)=f(g(x))=f\circ g (x) = g\circ f(x) = g(0)=0.$$

\bigskip

On fait un raisonnement similaire pour montrer $g(\Im f) \subset \Im f$.
Soit $z\in g(\Im f)$, il existe $y \in \Im f$ tel que $z=g(y)$. Il existe alors $x\in E$ tel que $y=f(x)$.
Donc 
$$z=g(y)=g(f(x))=g\circ f(x)= f\circ g (x) = f(g(x)) \in \Im f.$$

\fincorrection
\correction{001027}
\begin{enumerate}
  \item Par la formule $\dim(G+H) = \dim(G)+\dim(H)-
\dim(G\cap H)$, on sait que $\dim(G+H) \leqslant \dim(G)+\dim(H)$.
Pour $G=\Im u$ et $H=\Im v$ on obtient :
$\dim (\Im u+\Im v) \leqslant \dim \Im u +\dim \Im v$.
Or $\Im (u+v) \subset \Im u+\Im v$.
Donc $\text{rg} (u + v) \leq \text{rg} (u) + \text{rg} (v)$.
  \item On applique la formule pr\'ec\'edente \`a $u+v$ et $-v$ :
$\text{rg}  ((u+v)+(-v)) \leqslant \text{rg}  (u+v)+\text{rg} (-v)$, or $\text{rg} (-v)=\text{rg} (v)$
donc $\text{rg} (u) \leqslant \text{rg} (u+v)+\text{rg} (v)$.
Donc $\text{rg} (u)-\text{rg} (v)\leqslant \text{rg} (u+v)$.
On recommence en \'echangeant $u$ et $v$ pour obtenir :
$\left|\text{rg} (u) - \text{rg} (v)\right| \leq \text{rg} (u + v)$.
\end{enumerate}
\fincorrection
\correction{000956}
Calculer le noyau revient à résoudre un système linéaire,
et calculer l'image aussi. On peut donc tout faire ``à la main''.

Mais on peut aussi appliquer un peu de théorie ! Noyau et image sont liés par la formule du rang : 
$\dim \ker f + \dim \Im f = \dim E$
pour $f : E \to F$. Donc si on a trouvé le noyau alors on connaît la dimension de l'image.
Et il suffit alors de trouver autant de vecteur de l'image.


\begin{enumerate}
  \item $f_1$ est injective, surjective (et donc bijective).

  \begin{enumerate}
  \item Faisons tout à la main. Calculons le noyau :
\begin{align*}
(x,y) \in \ker f_1
  & \iff f_1(x,y) = (0,0) 
  \iff (2x+y,x-y) = (0,0) \\
  &\iff \begin{cases}
         2x+y=0 \\
        x-y=0 \\   
        \end{cases}
  \iff (x,y)=(0,0) \\
\end{align*}
Ainsi $\ker f_1 = \{ (0,0) \}$ et donc $f_1$ est injective.

   \item Calculons l'image. Quels éléments $(X,Y)$ peuvent s'écrire $f_1(x,y)$ ?
\begin{align*}
f_1(x,y) = (X,Y) 
  & \iff (2x+y,x-y) = (X,Y) \\
  &\iff \begin{cases}
         2x+y=X \\
         x-y=Y \\   
        \end{cases} 
  \iff \begin{cases}
         x=\frac{X+Y}{3} \\
         y=\frac{X-2Y}{3} \\   
        \end{cases} \\
  & \iff (x,y)=\left(\frac{X+Y}{3},\frac{X-2Y}{3}\right) \\
\end{align*}
Donc pour n'importe quel $(X,Y)\in\Rr^2$ on trouve un antécédent $(x,y)=(\frac{X+Y}{3},\frac{X-2Y}{3})$
qui vérifie donc $f_1(x,y)=(X,Y)$. Donc $\Im f_1 = \Rr^2$. Ainsi $f_1$ est surjective.

  \item Conclusion : $f_1$ est injective et surjective donc bijective.
  \end{enumerate}


  \item 
\begin{enumerate}
  \item Calculons d'abord le noyau :
\begin{align*}
(x,y,z) \in \ker f_2
  & \iff f_2(x,y,z) = (0,0,0) \\
  & \iff (2x+y+z,y-z,x+y) = (0,0,0) \\
  &\iff \begin{cases}
         2x+y+z = 0 \\
         y-z = 0 \\   
         x+y = 0 \\ 
        \end{cases} \\
  & \iff  \begin{cases}
         x = -z \\
         y = z \\   
        \end{cases} \\
  & \iff 
  \begin{pmatrix}x\\y\\z\end{pmatrix} = \begin{pmatrix}-z\\z\\z\end{pmatrix}\\ 
  &\iff 
  \begin{pmatrix}x\\y\\z\end{pmatrix} \in 
  \text{Vect} \begin{pmatrix}-1\\1\\1\end{pmatrix} 
  = \left\lbrace \lambda\begin{pmatrix}-1\\1\\1\end{pmatrix} \mid \lambda \in \Rr \right\rbrace\\ 
\end{align*}
Ainsi $\ker f_2 =  \text{Vect}(-1,1,1) $ et donc $f_2$ n'est pas injective.

  \item Maintenant nous allons utiliser que $\ker f_2 = \text{Vect}(-1,1,1)$ , autrement dit
$\dim  \ker f_2 = 1$.
La formule du rang, appliquée à $f_2 : \Rr^3 \to \Rr^3$ s'écrit 
$\dim \ker f_2 + \dim \Im f_2 = \dim \Rr^3$. Donc $\dim \Im f_2 = 2$.
Nous allons trouver une base de $\Im f_2$. Il suffit donc de trouver deux vecteurs linéairement indépendants.
Prenons par exemple
$v_1 = f_2(1,0,0) =  (2,0,1) \in\Im f_2$
et $v_2 = f_2(0,1,0) = (1,1,1) \in \Im f_2$. Par construction ces vecteurs sont dans l'image de $f_2$ et 
il est clair qu'ils sont linéairement indépendants. Donc $\{v_1,v_2\}$ est une base de $\Im f_2$.

  \item $f_2$ n'est ni injective, ni surjective (donc pas bijective).
  \end{enumerate}


  \item Sans aucun calcul on sait $f_3: \Rr^2 \to \Rr^4$ ne peut être surjective 
car l'espace d'arrivée est de dimension strictement
supérieur à l'espace de départ. 
  \begin{enumerate}
  \item Calculons le noyau :
\begin{align*}
(x,y) \in \ker f_3
  & \iff f_3(x,y) = (0,0,0,0) \\
  & \iff  (y,0,x-7y,x+y) = (0,0,0,0) \\
  & \iff \begin{cases}
         y = 0 \\
         0  = 0 \\   
         x-7y = 0 \\ 
         x+y = 0 \\ 
        \end{cases} \\
  & \iff \cdots \\ 
  & \iff (x,y)=(0,0) \\
\end{align*}
Ainsi $\ker f_3 = \{ (0,0) \}$ et donc $f_3$ est injective.

  \item La formule du rang, appliquée à $f_3 : \Rr^2 \to \Rr^4$ s'écrit 
$\dim \ker f_3 + \dim \Im f_3 = \dim \Rr^2$. Donc $\dim \Im f_3 = 2$.
Ainsi $\Im f_3$ est un espace vectoriel de dimension $2$ inclus dans $\Rr^3$,
 $f_3$ n'est pas surjective.

Par décrire $\Im f_3$ nous allons trouver deux vecteurs indépendants de $\Im f_3$.
Il y a un nombre infini de choix : prenons par exemple 
$v_1 = f(1,0) = (0,0,1,1)$. Pour $v_2$ on cherche (un peu à tâtons) un vecteur linéairement indépendant de $v_1$.
Essayons $v_2 = f(0,1)=(1,0,-7,1)$. Par construction $v_1,v_2 \in \Im f$ ; ils sont clairement linéairement indépendants
et comme $\dim \Im f_3=2$ alors $\{v_1,v_2\}$ est une base de $\Im f_3$.

Ainsi $\Im f_3= \text{Vect}\{v_1,v_2\} =\big\{ \lambda(0,0,1,1) + \mu (1,0,-7,1) \mid \lambda,\mu \in \Rr \big\}$.
  \end{enumerate}

  
  
  \item $f_4 : \Rr_3[X] \to \Rr^3$ va d'un espace de dimension $4$ vers un espace de dimension strictement plus petit
et donc $f_4$ ne peut être injective.

\begin{enumerate}
  \item Calculons le noyau. \'Ecrivons un polynôme $P$ de degré $\le 3$ sous la forme
$P(X)= aX^3+bX^2+cX+d$. Alors $P(0) = d$, $P(1)=a+b+c+d$, $P(-1)=-a+b-c+d$.
\begin{align*}
P(X) \in \ker f_4 
  & \iff \big( P(-1), P(0), P(1) \big)= (0,0,0)  \\
  & \iff (-a+b-c+d,d,a+b+c+d) = (0,0,0) \\
  &\iff \begin{cases}
          -a+b-c+d = 0 \\
          d = 0 \\   
          a+b+c+d = 0 \\ 
        \end{cases}\\
 & \iff \cdots \\
 & \iff \begin{cases}
          a = -c \\
          b = 0 \\  
          d = 0 \\ 
        \end{cases} \\
  & \iff (a,b,c,d)=(t,0,-t,0) \quad t\in\Rr \\
  \end{align*}

Ainsi le noyau $\ker f_4 = \big\{ tX^3 - tX \mid t\in \Rr\big\} = \text{Vect}\{ X^3-X \}$.
$f_4$ n'est pas injective son noyau étant de dimension $1$.

  \item La formule du rang pour $f_4 : \Rr_3[X] \to \Rr^3$ s'écrit 
$\dim \ker f_4 + \dim \Im f_4 = \dim \Rr_3[4]$. Autrement dit
$1+ \dim \Im f_4 = 4$. Donc $\dim \Im f_4 = 3$. 
Ainsi $\Im f_4$ est un espace de dimension $3$ dans $\Rr^3$ donc 
$\Im f_4=\Rr^3$. Conclusion $f_4$ est surjective.
  \end{enumerate}

\end{enumerate}
\fincorrection
\correction{000954}
\begin{enumerate}
  \item Comment est d\'efinie $\phi$ \`a partir de la d\'efinition sur les \'el\'ements de la base ?
Pour  $x\in E$ alors $x$ s'\'ecrit dans la base  $\{e_1,e_2,e_3\}$, $x=\alpha_1e_1+\alpha_2e_2+\alpha_3e_3$. 
Et $\phi$ est d\'efinie sur $E$ par la formule
$$\phi(x)=\alpha_1 \phi(e_1) + \alpha_2 \phi(e_2) + \alpha_3 \phi(e_3).$$
Soit ici :
$$\phi(x) = (\alpha_1+\alpha_2+\alpha_3) e_1 + (\alpha_1-\alpha_2)e_2 + t\alpha_3e_3.$$

Cette d\'efinition rend automatiquement $\phi$ lin\'eaire (v\'erifiez-le si vous n'\^etes pas convaincu !).
  \item On cherche \`a savoir si $\phi$ est injective.
Soit $x\in E$ tel que $\phi(x)=0$ donc
$(\alpha_1+\alpha_2+\alpha_3) e_1 + (\alpha_1-\alpha_2)e_2 + t\alpha_3e_3=0$. Comme $\{e_1,e_2,e_3\}$ est une base alors tous les coefficients sont nuls :
$$\alpha_1+\alpha_2+\alpha_3=0, \quad \alpha_1-\alpha_2=0, \quad  t\alpha_3 = 0.$$
Si $t \neq 0$ alors en résolvant le syst\`eme on obtient $\alpha_1=0$, $\alpha_2=0$,
$\alpha_3=0$. Donc $x=0$ et $\phi$ est injective.

Si $t =0$, alors $\phi$ n'est pas injective, en résolvant le m\^eme syst\`eme on obtient
des solutions non triviales, par exemple $\alpha_1=1$, $\alpha_2=1$, $\alpha_3=-2$.
Donc pour $x= e_1+e_2-2e_3$ on obtient $\phi(x)=0$.

  \item Pour la surjectivité on peut soit faire des calculs, soit appliquer la formule du rang. 
Examinons cette deuxi\`eme m\'ethode. $\phi$ est surjective si et seulement si 
la dimension de $\Im \phi$ est \'egale
\`a la dimension de l'espace d'arriv\'ee (ici $E$ de dimension $3$).
Or on a une formule pour $\dim \Im \phi$ :
$$\dim \ker \phi + \dim \Im \phi = \dim E.$$
Si $t \neq 0$, $\phi$ est injective donc $\ker \phi = \{0\}$ est de dimension $0$.
Donc $\dim \Im \phi =3$ et $\phi$ est surjective.

Si $t = 0$ alors $\phi$ n'est pas injective donc $\ker \phi$ est de dimension au moins $1$
(en fait $1$ exactement), donc  $\dim \Im \phi \leqslant 2$. Donc $\phi$ n'est pas surjective.


On remarque que $\phi$ est injective si et seulement si elle est surjective.
Ce qui est un r\'esultat du cours pour les applications ayant l'espace 
de d\'epart et d'arriv\'ee de m\^eme dimension (finie).
\end{enumerate}
\fincorrection
\correction{000963}
\begin{enumerate}
    \item Montrons que si  $\phi $  est un isomorphisme,
l'image de toute base de  $E$  est une base de  $F$ : soit
$\mathcal{B} =\{ e_1, \ldots , e_n \} $  une base de  $E$  et
nommons  $\mathcal{B} ' $  la famille  $\{ \phi (e_1), \ldots ,
\phi (e_n) \} $.
    \begin{enumerate}
        \item $\mathcal{B} '$  est libre. Soient en effet  $\lambda _1 , \ldots , \lambda _n\in {\R}$  tels
que  $\lambda _1\phi (e_1)+ \cdots + \lambda _n \phi (e_n)
=0 $. Alors  $ \phi (\lambda _1e_1+ \cdots + \lambda _ne_n) =0
$  donc, comme  $\phi$  est injective,  $\lambda _1e_1+ \cdots
+ \lambda _ne_n=0$  puis, comme  $\mathcal{B} $  est libre,
$\lambda _1=\cdots =\lambda _n=0$.
        \item $\mathcal{B} '$  est g\' en\' eratrice. Soit  $y\in F$. Comme  $\phi $  est surjective, il
existe  $x\in E$  tel que  $y=\phi (x)$. Comme  $\mathcal{B}$
est g\' en\' eratrice, on peut choisir   $\lambda _1 , \cdots ,
\lambda _n\in {\R}$  tels que  $x=\lambda _1 e_1 +\cdots + \lambda
_n e_n $. Alors  $y=\lambda _1\phi (e_1)+ \cdots + \lambda _n
\phi (e_n)  $.
    \end{enumerate}
    \item Supposons que l'image par  $\phi $  de toute base de  $E$  soit une base  $F$. Soient  $\mathcal{B}
=\{ e_1,\ldots , e_n\} $  une base de  $E$  et  $\mathcal{B} ' $
la base  $\{ \phi (e_1), \ldots , \phi (e_n) \} $.
    \begin{enumerate}
        \item $\Im \phi$  contient  $\mathcal{B} '$  qui est une partie g\' en\' eratrice de $F$. Donc  $\phi $  est surjective.
        \item Soit maintenant  $x\in E$  tel que  $\phi (x)=0$.
Comme  $\mathcal{B} $  est une base, il existe  $\lambda _1 ,
\ldots , \lambda _n\in {\R}$  tels que  $x= \lambda _1e_1+ \cdots
+ \lambda _ne_n$. Alors  $\phi (x)=0=\lambda _1\phi (e_1)+
\cdots + \lambda _n \phi (e_n) $ donc puisque  $\mathcal{B} '$
est libre :  $\lambda _1=\cdots =\lambda _n=0$. En cons\' equence
si  $\phi (x)=0$  alors  $x=0$ : $\phi$  est injective.
    \end{enumerate}
\end{enumerate}

En fait on montrerait de la même façon que ``$\phi $  est un isomorphisme si et seulement si l'image par 
$\phi $  \textbf{d'une} base de  $E$  est une base de  $F$''.
\fincorrection
\correction{000974}
\begin{enumerate}
  \item La seule fonction qui est \`a la fois paire et impaire est la fonction nulle : $P\cap I = \{0\}$. Montrons qu'une fonction $f:\Rr \longrightarrow \Rr$ se d\'ecompose en une fonction paire et une fonction impaire. 
En effet : 
 $$f(x)= \frac{f(x)+f(-x)}{2}+\frac{f(x)-f(-x)}{2}.$$
La fonction $x \mapsto \frac{f(x)+f(-x)}{2}$ est paire (le v\'erifier !),
la fonction $x \mapsto \frac{f(x)-f(-x)}{2}$ est impaire.
Donc $P+I=E$.
Bilan : $E=P\oplus I.$
  \item Le projecteur sur $P$ de direction $I$ est l'application $\pi : E \longrightarrow E$
    qui \`a $f$ associe la fonction $x \mapsto \frac{f(x)+f(-x)}{2}$, c'est-à-dire à $f$ on associe la partie paire
 de $f$.
Nous avons bien 
\begin{itemize}
  \item $\pi(f)\in P$. Par définition de $\pi$, $\pi(f)$ est bien une fonction paire.
  \item $\pi \circ \pi = \pi$. Si $g$ est une fonction paire alors $\pi(g)=g$. 
Appliquons ceci avec $g=\pi(f)$ (qui est bien est une fonction paire) donc $\pi(\pi(f))=\pi(f)$. 
  \item $\ker \pi = I$. Si $\pi(f)=0$ alors cela signifie exactement que la fonction $x \mapsto \frac{f(x)+f(-x)}{2}$
est la fonction nulle. Donc pour tout $x$ : $\frac{f(x)+f(-x)}{2}=0$ donc $f(x)=-f(-x)$ ; 
cela implique que $f$ est une fonction impaire. Réciproquement si $f\in I$ est une fonction impaire, 
sa partie paire est nulle donc $f\in \ker f$.
\end{itemize}

\end{enumerate}
\fincorrection
\correction{000959}
\begin{enumerate}
  \item Soit $P \in E$ et $\lambda \in \Rr$, alors la division euclidienne de $AP$ par $B$ s'\'ecrit $AP= Q\cdot B + R$, donc en multipliant par $\lambda$ on obtient :
$A\cdot (\lambda P)= (\lambda Q)B+\lambda R$.
ce qui est la division euclidienne de $A\cdot (\lambda P)$ par $B$, 
donc si $f(P)= R$ alors $f(\lambda P)=\lambda R$. Donc $f(\lambda P)=\lambda f(P)$.

Soient $P, P' \in E$. On \'ecrit les divisions euclidiennes :
$$AP= Q\cdot B + R,\quad  AP'=Q'\cdot B+R'.$$
En additionnant :
$$A(P+P')=(Q+Q')B+(R+R')$$
qui est la division euclidienne de $A(P+P')$ par $B$.
Donc si $f(P)=R$, $f(P')=R'$ alors $f(P+P')=R+R'=f(P)+f(P')$.

Donc $f$ est lin\'eaire.
  \item Sens $\Rightarrow$. Supposons $f$ est bijective, donc en particulier $f$ est surjective,
    en particulier il existe $P\in E$ tel que $f(P)=1$ ($1$ est le polyn\^ome constant \'egale \`a $1$). La division euclidienne est donc $AP=BQ+1$, autrement dit 
$AP-BQ=1$. Par le th\'eor\`eme de B\'ezout, $A$ et $B$ sont premiers entre eux.

  \item Sens $\Leftarrow$. Supposons $A, B$ premiers entre eux. Montrons que $f$ est injective.
Soit $P\in E$ tel que $f(P)=0$. Donc la division euclidienne s'\'ecrit : $AP=BQ+0$.
Donc $B$ divise $AP$. Comme $A$ et $B$ sont premiers entre eux, par le lemme de Gauss,
alors $B$ divise $P$. Or $B$ est de degr\'e $n+1$ et $P$ de degr\'e moins que $n$, 
donc la seule solution est $P=0$. Donc $f$ est injective. 
Comme $f : E\longrightarrow E$ est injective et $E$ est de dimension finie, alors $f$ est bijective.
\end{enumerate}
\fincorrection
\correction{000976}
\begin{enumerate}
  \item $f$ est bien lin\'eaire...
  \item Soit $P$ tel que $f(P)=0$. Alors $P$ v\'erifie l'\'equation diff\'erentielle
$$P+(1-X)P'=0.$$ Dont la solution est $P = \lambda(X-1)$, $\lambda \in \Rr$.
Donc $\ker f$ est de dimension $1$ et une base est donn\'ee par un seul vecteur : $X-1$.

\item Par le th\'eor\`eme du rang la dimension de l'image est :
$$\dim \Im f = \dim \Rr_n[X]-\dim \ker f = (n+1) - 1 = n.$$
Il faut donc trouver $n$ vecteurs lin\'eairement ind\'ependants dans $\Im f$.
\'Evaluons $f(X^k)$, alors 
$$f(X^k) = (1-k)X^k+kX^{k-1}.$$
Cela donne $f(1)=1, f(X)=1, f(X^2)=-X^2+2X,...$
on remarque que pour $k= 2,\ldots n$, $f(X^k)$ est de degr\'e $k$ sans terme constant.
Donc l'ensemble 
$$\big\{ f(X), f(X^2), \ldots, f(X^n)\big\}$$
est une famille de $n$ vecteurs, appartenant \`a $\Im f$, et libre (car les degr\'es sont distincts).
Donc ils forment une base de $\Im f$.
\end{enumerate}
\fincorrection


\end{document}

