\documentclass[11pt]{article}

 %Configuration de la feuille 
 
\usepackage{amsmath,amssymb,enumerate,graphicx,pgf,tikz,fancyhdr}
\usepackage[utf8]{inputenc}
\usetikzlibrary{arrows}
\usepackage{geometry}
\usepackage{tabvar}
\geometry{hmargin=2.2cm,vmargin=1.5cm}\pagestyle{fancy}
\lfoot{\bfseries http://www.bibmath.net}
\rfoot{\bfseries\thepage}
\cfoot{}
\renewcommand{\footrulewidth}{0.5pt} %Filet en bas de page

 %Macros utilisées dans la base de données d'exercices 

\newcommand{\mtn}{\mathbb{N}}
\newcommand{\mtns}{\mathbb{N}^*}
\newcommand{\mtz}{\mathbb{Z}}
\newcommand{\mtr}{\mathbb{R}}
\newcommand{\mtk}{\mathbb{K}}
\newcommand{\mtq}{\mathbb{Q}}
\newcommand{\mtc}{\mathbb{C}}
\newcommand{\mch}{\mathcal{H}}
\newcommand{\mcp}{\mathcal{P}}
\newcommand{\mcb}{\mathcal{B}}
\newcommand{\mcl}{\mathcal{L}}
\newcommand{\mcm}{\mathcal{M}}
\newcommand{\mcc}{\mathcal{C}}
\newcommand{\mcmn}{\mathcal{M}}
\newcommand{\mcmnr}{\mathcal{M}_n(\mtr)}
\newcommand{\mcmnk}{\mathcal{M}_n(\mtk)}
\newcommand{\mcsn}{\mathcal{S}_n}
\newcommand{\mcs}{\mathcal{S}}
\newcommand{\mcd}{\mathcal{D}}
\newcommand{\mcsns}{\mathcal{S}_n^{++}}
\newcommand{\glnk}{GL_n(\mtk)}
\newcommand{\mnr}{\mathcal{M}_n(\mtr)}
\DeclareMathOperator{\ch}{ch}
\DeclareMathOperator{\sh}{sh}
\DeclareMathOperator{\vect}{vect}
\DeclareMathOperator{\card}{card}
\DeclareMathOperator{\comat}{comat}
\DeclareMathOperator{\imv}{Im}
\DeclareMathOperator{\rang}{rg}
\DeclareMathOperator{\Fr}{Fr}
\DeclareMathOperator{\diam}{diam}
\DeclareMathOperator{\supp}{supp}
\newcommand{\veps}{\varepsilon}
\newcommand{\mcu}{\mathcal{U}}
\newcommand{\mcun}{\mcu_n}
\newcommand{\dis}{\displaystyle}
\newcommand{\croouv}{[\![}
\newcommand{\crofer}{]\!]}
\newcommand{\rab}{\mathcal{R}(a,b)}
\newcommand{\pss}[2]{\langle #1,#2\rangle}
 %Document 

\begin{document} 

\begin{center}\textsc{{\huge }}\end{center}

% Exercice 935


\vskip0.3cm\noindent\textsc{Exercice 1} - D'un produit à l'autre
\vskip0.2cm
Soit $A\in\mathcal M_{3,2}(\mathbb R)$, $B\in\mathcal M_{2,3}(\mathbb R)$ tels que 
$$AB=\left(
\begin{array}{ccc}
0&0&0\\
0&1&0\\
0&0&1
\end{array}
\right).$$
Démontrer que $BA=I_2$.


% Exercice 3122


\vskip0.3cm\noindent\textsc{Exercice 2} - Base adaptée à un endomorphisme dont le carré est nul
\vskip0.2cm
Soit $f\in\mathcal L(\mathbb R^3)$ tel que $f\neq 0$ et $f^2=0$.
\begin{enumerate}
\item Démontrer que $\dim(\ker(f))=2$.
\item En déduire qu'il existe une base $\mathcal B$ de $\mathbb R^3$ dans laquelle la matrice de $f$ est $\begin{pmatrix}0&0&1\\0&0&0\\0&0&0\end{pmatrix}$. 
\end{enumerate}


% Exercice 403


\vskip0.3cm\noindent\textsc{Exercice 3} - Intégration par parties itérée
\vskip0.2cm
\begin{enumerate}
\item Soient $f,g:[a,b]\to\mathbb R$ deux fonctions de classe $C^n$. Montrer que
$$\int_{a}^b f^{(n)}g=\sum_{k=0}^{n-1}(-1)^k \big(f^{(n-k-1)}(b)g^{(k)}(b)-f^{(n-k-1)}(a)g^{(k)}(a)\big)+(-1)^n \int_a^b fg^{(n)}.$$
\item Application : On pose $Q_n(x)=(1-x^2)^n$ et $P_n(x)=Q_n^{(n)}(x)$. Justifier que $P_n$ est un polynôme de degré $n$, puis prouver
que $\int_{-1}^1 QP_n=0$ pour tout polynôme $Q$ de degré inférieur ou égal à $n-1$.
\end{enumerate}


% Exercice 386


\vskip0.3cm\noindent\textsc{Exercice 4} - Retrouver la fonction
\vskip0.2cm
Soit $f:[a,b]\to\mathbb R$ continue telle que $|f(x)|\leq 1$ pour tout $x\in[a,b]$
et $\int_a^b f(x)dx=b-a$. Que dire de $f$?


% Exercice 1104


\vskip0.3cm\noindent\textsc{Exercice 5} - Valeur moyenne
\vskip0.2cm
Soit $f:[a,b]\to\mathbb R$ continue. Démontrer que sa valeur moyenne est atteinte : il existe $c\in [a,b]$ tel que 
$$f(c)=\frac{1}{b-a}\int_a^b f(t)dt.$$




\vskip0.5cm
\noindent{\small Cette feuille d'exercices a été conçue à l'aide du site \textsf{https://www.bibmath.net}}

%Vous avez accès aux corrigés de cette feuille par l'url : https://www.bibmath.net/ressources/justeunefeuille.php?id=27976
\end{document}