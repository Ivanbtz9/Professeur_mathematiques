\documentclass[a4paper,11pt]{article}

\usepackage{inputenc}
\usepackage[T1]{fontenc}
\usepackage[frenchb]{babel}
\usepackage{fancyhdr,fancybox} % pour personnaliser les en-têtes
\usepackage{lastpage,setspace}
\usepackage{amsfonts,amssymb,amsmath,amsthm,mathrsfs}
\usepackage{relsize,exscale,bbold}
\usepackage{paralist}
\usepackage{xspace,multicol,diagbox,array}
\usepackage{xcolor}
\usepackage{variations}
\usepackage{xy,xypic}
\usepackage{eurosym,stmaryrd}
\usepackage{graphicx}
\usepackage[np]{numprint}
\usepackage{hyperref} 
\usepackage{tikz}
\usepackage{colortbl}
\usepackage{multirow}
\usepackage{MnSymbol,wasysym}
\usepackage[top=1.5cm,bottom=1.5cm,right=1.2cm,left=1.5cm]{geometry}
\usetikzlibrary{calc, arrows, plotmarks, babel,decorations.pathreplacing}
\setstretch{1.25}
%\usepackage{lipsum} %\usepackage{enumitem} %\setlist[enumerate]{itemsep=1mm} bug avec enumerate



\newtheorem{thm}{Théorème}
\newtheorem{rmq}{Remarque}
\newtheorem{prop}{Propriété}
\newtheorem{cor}{Corollaire}
\newtheorem{lem}{Lemme}
\newtheorem{prop-def}{Propriété-définition}

\theoremstyle{definition}

\newtheorem{defi}{Définition}
\newtheorem{ex}{Exemple}
\newtheorem*{rap}{Rappel}
\newtheorem{cex}{Contre-exemple}
\newtheorem{exo}{Exercice} % \large {\fontfamily{ptm}\selectfont EXERCICE}
\newtheorem{nota}{Notation}
\newtheorem{ax}{Axiome}
\newtheorem{appl}{Application}
\newtheorem{csq}{Conséquence}
\def\di{\displaystyle}



\renewcommand{\thesection}{\Roman{section}}\renewcommand{\thesubsection}{\arabic{subsection} }\renewcommand{\thesubsubsection}{\alph{subsubsection} }


\newcommand{\bas}{~\backslash}\newcommand{\ba}{\backslash}
\newcommand{\C}{\mathbb{C}}\newcommand{\R}{\mathbb{R}}\newcommand{\Q}{\mathbb{Q}}\newcommand{\Z}{\mathbb{Z}}\newcommand{\N}{\mathbb{N}}\newcommand{\V}{\overrightarrow}\newcommand{\Cs}{\mathscr{C}}\newcommand{\Ps}{\mathscr{P}}\newcommand{\Rs}{\mathscr{R}}\newcommand{\Gs}{\mathscr{G}}\newcommand{\Ds}{\mathscr{D}}\newcommand{\happy}{\huge\smiley}\newcommand{\sad}{\huge\frownie}\newcommand{\danger}{\begin{tikzpicture}[x=1.5pt,y=1.5pt,rotate=-14.2]
	\definecolor{myred}{rgb}{1,0.215686,0}
	\draw[line width=0.1pt,fill=myred] (13.074200,4.937500)--(5.085940,14.085900)..controls (5.085940,14.085900) and (4.070310,15.429700)..(3.636720,13.773400)
	..controls (3.203130,12.113300) and (0.917969,2.382810)..(0.917969,2.382810)
	..controls (0.917969,2.382810) and (0.621094,0.992188)..(2.097660,1.359380)
	..controls (3.574220,1.726560) and (12.468800,3.984380)..(12.468800,3.984380)
	..controls (12.468800,3.984380) and (13.437500,4.132810)..(13.074200,4.937500)
	--cycle;
	\draw[line width=0.1pt,fill=white] (11.078100,5.511720)--(5.406250,11.875000)..controls (5.406250,11.875000) and (4.683590,12.812500)..(4.367190,11.648400)
	..controls (4.050780,10.488300) and (2.375000,3.675780)..(2.375000,3.675780)
	..controls (2.375000,3.675780) and (2.156250,2.703130)..(3.214840,2.964840)
	..controls (4.273440,3.230470) and (10.640600,4.847660)..(10.640600,4.847660)
	..controls (10.640600,4.847660) and (11.332000,4.953130)..(11.078100,5.511720)
	--cycle;
	\fill (6.144520,8.839900)..controls (6.460940,7.558590) and (6.464840,6.457090)..(6.152340,6.378910)
	..controls (5.835930,6.300840) and (5.320300,7.277400)..(5.003900,8.554750)
	..controls (4.683590,9.835940) and (4.679690,10.941400)..(4.996090,11.019600)
	..controls (5.312490,11.097700) and (5.824210,10.121100)..(6.144520,8.839900)
	--cycle;
	\fill (7.292960,5.261780)..controls (7.382800,4.898500) and (7.128900,4.523500)..(6.730460,4.421880)
	..controls (6.328120,4.324220) and (5.929680,4.535220)..(5.835930,4.898500)
	..controls (5.746080,5.261780) and (5.999990,5.640630)..(6.402340,5.738340)
	..controls (6.804690,5.839840) and (7.203110,5.625060)..(7.292960,5.261780)
	--cycle;
	\end{tikzpicture}}\newcommand{\alors}{\Large\Rightarrow}\newcommand{\equi}{\Leftrightarrow}
\newcommand{\fonction}[5]{\begin{array}{l|rcl}
		#1: & #2 & \longrightarrow & #3 \\
		& #4 & \longmapsto & #5 \end{array}} \newcommand{\sh}{\text{sh}}\newcommand{\ch}{\text{ch}}%\newcommand{\arccos}{\text{arccos}}\newcommand{\arcsin}{\text{arcsin}} \newcommand{\arctan}{\text{arctan}}


\definecolor{vert}{RGB}{11,160,78}
\definecolor{rouge}{RGB}{255,120,120}
\definecolor{bleu}{RGB}{15,5,107}



\pagestyle{fancy}
\lhead{Groupe IPESUP}\chead{}\rhead{Année~2022-2023}\lfoot{M. Botcazou \& M.Dupré}\cfoot{\thepage/3}\rfoot{PCSI }\renewcommand{\headrulewidth}{0.4pt}\renewcommand{\footrulewidth}{0.4pt}


\begin{document}
 	
	

\noindent\shadowbox{
	\begin{minipage}{1\linewidth}
		\centering
		\huge{\textbf{ TD 5 : Fonctions usuelles }}
	\end{minipage}
}
\bigskip


\raggedright

\section*{Généralités sur les fonctions:}\hfill\\%[-0.25cm]

\begin{minipage}{1\linewidth}
	\begin{minipage}[t]{0.48\linewidth}
		\raggedright
	
\begin{exo}\textbf{(*)}\quad\\[0.2cm]
	\begin{enumerate}
		\item  Soit $f$ une fonction dérivable sur $\R$ à valeurs dans $\R$. Montrer que si $f$ est paire, $f'$ est
		impaire et si $f$ est impaire, $f'$ est paire.
		\item  Soient $n\in\N^*$ et $f$ une fonction $n$ fois dérivable sur $\R$ à valeurs dans $\R$. $f^{(n)}$
		désignant la dérivée $n$-ième de $f$, montrer que si $f$ est paire, $f^{(n)}$ est paire si $n$ est pair et impaire si
		$n$ est impair.
		\item  Soit $f$ une fonction continue sur $\R$ à valeurs dans $\R$. A-t-on des résultats analogues concernant les
		primitives de $f$~?
		\item  Reprendre les questions précédentes en remplaçant la condition \og~$f$ est paire (ou impaire)~\fg~par la
		condition \og~$f$ est $T$-périodique~\fg.
	\end{enumerate}
	
	\centering
	\rule{1\linewidth}{0.6pt}
\end{exo}


		\begin{exo}\textbf{(*)}\quad\\[0.2cm]
	\begin{enumerate}
		\item Montrer que la fonction $x \longmapsto x-\dfrac{1}{x}$ est injective
		sur $\R^{+*}$. Et sur $\R^*$?
		\item\begin{enumerate}
			\item La fonction $x \longmapsto xe^x$ est-elle injective sur $\R$ ?
			\item Déterminer son image.
		\end{enumerate}
		\item \begin{enumerate}
			\item La fonction $f : x \longmapsto \sqrt{x^2+x+1}$ est-elle injective
			sur $\R$ ?
			\item Déterminer $f(] - 2, 4])$ .
		\end{enumerate}
	\end{enumerate}
	
	
	
	\centering
	\rule{1\linewidth}{0.6pt}
\end{exo}



\end{minipage}	
\hfill\vrule\hfill
\begin{minipage}[t]{0.48\linewidth}
\raggedright




\begin{exo}\textbf{(*)}\quad\\[0.2cm]
	
Soit $a \in \R$. Définissons $f_a : \R \longrightarrow \R$ par
$$f_a(x):\left\{\begin{array}{l}
	x + a  \ \text{ si } x\geq 0\\
	
	x - a \ \text{ sinon. }
\end{array}\right.$$
	\begin{enumerate}
		\item Déterminer les réels $a$  pour lesquels  $f_a$ est surjective.
		\item Déterminer les réels $a$  pour lesquels  $f_a$ est injective.
	\end{enumerate}
	\centering
	\rule{1\linewidth}{0.6pt}
\end{exo}


\begin{exo}\textbf{(**)}\quad\\[0.2cm]
	$$f: x \mapsto \left\{\begin{array}{ll}
	x + 1  \ & \text{ si } x\in \Q\\
	
	x \ &\text{ sinon. }
\end{array}\right.$$

	La fonction est-elle surjective de $\R$ dans $\R$ ? injective ? bijective ?
	
	\centering
	\rule{1\linewidth}{0.6pt}
\end{exo}

\begin{exo}\textbf{(***)}\quad\\[0.2cm]
Pour une fonction $f : \R \rightarrow \R$, on définit les fonctions partie positive et partie négative de $f$ par $f_+ : x \mapsto \max(f(x),0)$ et $f_- : x \mapsto \max(-f(x),0)$
\begin{enumerate}
	\item Montrer que si $f : \R \rightarrow \R$ est $k$-lipschitzienne, alors les fonctions $f_+$ et
	$f_- $ sont aussi $k$-lipschitziennes.
	\item Soit $f : \R \rightarrow \R$ telle que $f_+$ et $f_-$ sont $k$-lipschitziennes. Montrer que $f$	est $k$-lipschitzienne.
\end{enumerate} 

	\centering
\rule{1\linewidth}{0.6pt}
\end{exo}

\end{minipage}
\end{minipage}
\newpage
%\hfill\\[0.5cm]
\section*{Exponentielle, logarithmes, puissance et fonction valeur absolue:}\hfill\\[1cm]
\begin{minipage}{1\linewidth}
	\begin{minipage}[t]{0.48\linewidth}
		\raggedright
		
			\begin{exo}\textbf{(*)}\quad\\[0.2cm]
			Résoudre dans $\R$ les équations ou inéquations suivantes~:
			\begin{enumerate}
				\item $\; \exp(x) \geq 1 +x$.
				\item $\;\ln|x+1|-\ln|2x+1|\leq\ln2$.
				\item $\;x^{\sqrt{x}}=\sqrt{x}^x$.
				
			\end{enumerate}
			
			\centering
			\rule{1\linewidth}{0.6pt}
		\end{exo}
	

		
		\begin{exo}\textbf{(*)}\quad\\[0.2cm]
			\begin{enumerate}
				\item  Etudier brièvement la fontion $x\mapsto\frac{\ln x}{x}$ et tracer son graphe.
				\item  Trouver tous les couples $(a,b)$ d'entiers naturels non nuls et distincts vérifiant $a^b=b^a$.
			\end{enumerate}
			
			\centering
			\rule{1\linewidth}{0.6pt}
		\end{exo}

		
	\begin{exo}\textbf{(*)}\quad\\[0.2cm]
		Tracer le graphe de $x \longmapsto 2|x - 1| - |x + 1|$.
		
		\centering
		\rule{1\linewidth}{0.6pt}
	\end{exo}

	\begin{exo}\textbf{(*)}\quad\\[0.2cm]
	Résoudre dans $\R$ :
	\begin{enumerate}
		\item $2|2x-1| = |x+2|+3x$
		\item $3-|2-3x| \geq |3x+4|x$
	\end{enumerate}
	\centering
	\rule{1\linewidth}{0.6pt}
\end{exo}
	
		
	\end{minipage}	
	\hfill\vrule\hfill
	\begin{minipage}[t]{0.48\linewidth}
		\raggedright
		
				\begin{exo}\textbf{(*)}\quad\\[0.2cm]
		Trouver la plus grande valeur de $\sqrt[n]{n}$, $n\in\N^*$.
		
		\centering
		\rule{1\linewidth}{0.6pt}
	\end{exo}
		
		\begin{exo}\textbf{(**)}\quad\\[0.2cm]
			\begin{enumerate}
				\item Montrer que, pour tout entier $n$, $4$ divise $n^2$ ou $n^2 - 1$.
				\item En déduire que, pour tout $n$, $\lfloor\sqrt{4n + 1}\rfloor=  \lfloor\sqrt{4n + 3}\rfloor$.
			\end{enumerate}
	
			\centering
			\rule{1\linewidth}{0.6pt}
		\end{exo}
		
		
		
		\begin{exo}\textbf{(**)}\quad\\[0.2cm]
			Montrer que pour tout $n\geq 2$ :
			$$\left(1 + \dfrac{1}{n}\right)^n\leq e \leq \left(1 - \dfrac{1}{n}\right)^{-n}$$
			\centering
			\rule{1\linewidth}{0.6pt}
		\end{exo}
		


\begin{exo}\textbf{(**)}\quad\\[0.2cm]
	Montrer pour tout $n \in \N$:
	$$\forall x \in \R^+, \ 	\exp(x) \ \geq  \ \sum\limits^n_{k=0}\dfrac{x^k}{k!}.$$
	
	Cette inégalité est-elle vraie sur $\R$?
	
	\centering
	\rule{1\linewidth}{0.6pt}
\end{exo}
		
		
		
	\end{minipage}
\end{minipage}

\newpage

\section*{Fonctions hyperboliques:}\hfill\\%[-0.25cm]
\begin{minipage}{1\linewidth}
	\begin{minipage}[t]{0.48\linewidth}
		\raggedright
		
		
		\begin{exo}\textbf{(*)}\quad\\[0.2cm]
			Simplifier l'expression $\displaystyle\frac{2\ch^2(x)-\sh(2x)}{x-\ln(\ch x)-\ln 2}$ 
			et donner ses limites en $-\infty$ et $+\infty$.
			
			\centering
			\rule{1\linewidth}{0.6pt}
		\end{exo}
		
		
		\begin{exo}\textbf{(**)}\quad\\[0.2cm]
			Soit $x$ un réel fixé. Pour $n\in\N^*$, on pose
			$$C_n=\sum_{k=1}^n\ch(kx)\qquad\text{ et }\qquad S_n=\sum_{k=1}^n\sh(kx).$$
			Calculer $C_n$ et $S_n$.
			
			\centering
			\rule{1\linewidth}{0.6pt}
		\end{exo}
		
			\begin{exo}\textbf{(*)}\quad\\[0.2cm]%Bertelaut
			Pour tous $n \in \N$ et $x \in \R$
			
			factoriser la somme $ \ \sum\limits_{k=0}^{n} \ch(2kx)$
			
			
			
			\centering
			\rule{1\linewidth}{0.6pt}
		\end{exo}
		
		
	\end{minipage}	
	\hfill\vrule\hfill
	\begin{minipage}[t]{0.48\linewidth}
		\raggedright
		
	
		
		
		
		\begin{exo}\textbf{(**)}\quad\\[0.2cm]
			Soit $a$ et $b$ deux réels positifs tels que $a^2-b^2=1$. Résoudre le système
			$$\left\{\begin{array}{l}
			\ch(x)+\ch(y)=2a\\
			\sh(x)+\sh(y)=2b
			\end{array}\right.$$
			
			\centering
			\rule{1\linewidth}{0.6pt}
		\end{exo}
		
		\begin{exo}\textbf{(**)}\quad\\[0.2cm]
			Montrer que, pour tout $x \neq 0$,
			
			$$\dfrac{1}{\sh (x)} = \dfrac{1}{ \text{th}( x)} - \dfrac{1}{ \text{th} \left(\frac{x}{2}\right)}$$
		
	 En déduire une expression simple de $\sum\limits_{k=0}^{n}\dfrac{1}{\sh (2^kx)}$
	 
	 puis sa limite quand $n$ tend vers $+\infty$.
			
			
			\centering
			\rule{1\linewidth}{0.6pt}
		\end{exo}
		
		
		
	\end{minipage}
\end{minipage}

\section*{Fonctions circulaires et réciproques:}\hfill\\%[-0.25cm]
\begin{minipage}{1\linewidth}
	\begin{minipage}[t]{0.48\linewidth}
		\raggedright
		
			\begin{exo}\textbf{(*)}\quad\\[0.2cm]
			Déterminer les réels $x$ tels que $(\cos x)^4 + (\sin x)^6 = 1$.
			\centering
			\rule{1\linewidth}{0.6pt}
		\end{exo}
	
		
		\begin{exo}\textbf{(*)}\quad\\[0.2cm]
			\'Ecrire sous forme d'expression algébrique
			\begin{enumerate}
				\item $ \sin(\arccos x),\ \cos(\arcsin x),\ \cos(2 \arcsin x)$.
				\item $ \sin(\arctan x),\ \cos(\arctan x),\ \sin(3 \arctan x)$.
			\end{enumerate}
			
			\centering
			\rule{1\linewidth}{0.6pt}
		\end{exo}
		
		\begin{exo}\textbf{(*)}\quad\\[0.2cm]
			Montrer que pour tout $x>0$, on a
			$$\arctan\left(\frac{1}{2x^2}\right)=\arctan\left(\frac{x}{x+1}\right)-\arctan\left(\frac{x-1}{x}\right).$$
			En déduire une expression de $\displaystyle S_n=\sum_{k=1}^n\arctan\left(\frac{1}{2k^2}\right)$ 
			et calculer $\displaystyle\lim_{n\to +\infty}S_n$.
			
			\centering
			\rule{1\linewidth}{0.6pt}
		\end{exo}
		
	\end{minipage}	
	\hfill\vrule\hfill
	\begin{minipage}[t]{0.48\linewidth}
		\raggedright
		
			\begin{exo}\textbf{(*)}\quad\\[0.2cm]
			Résoudre les équations suivantes:
			\begin{enumerate}
				\item $\arccos x = 2\arccos \frac{3}{4}$.
				\item $\arcsin x = \arcsin \frac{2}{5} + \arcsin \frac{3}{5}$.
				\item $\arctan {2x}+\arctan x=\frac{\pi}{4}$.
			\end{enumerate}
			
			
			\centering
			\rule{1\linewidth}{0.6pt}
		\end{exo}
		
		\begin{exo}\textbf{(**)}\quad\\[0.2cm]
			Vérifier lorsque l'on peut définir les expressions ci-dessous que:
			\begin{enumerate}
				\item $ \arcsin x + \arccos x = \frac{\pi}{2}$.
				\item $\arctan x + \arctan\frac{1}{x} = \text{sgn}(x)\frac{\pi}{2}.$.
			\end{enumerate}
			
			\centering
			\rule{1\linewidth}{0.6pt}
		\end{exo}
		
	
		
		\begin{exo}\textbf{(**)}\quad\\[0.2cm]
			
	Calculer $3 \arctan \dfrac{1}{3} + \arctan \dfrac{1}{7} $.
			
			\centering
			\rule{1\linewidth}{0.6pt}
		\end{exo}
		
		
		
	\end{minipage}
\end{minipage}


\end{document}