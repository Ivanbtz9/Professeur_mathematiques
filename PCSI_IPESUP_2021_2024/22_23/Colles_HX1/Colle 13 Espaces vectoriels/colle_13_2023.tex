\documentclass[a4paper,11pt]{article}

\usepackage{inputenc}
\usepackage[T1]{fontenc}
\usepackage[frenchb]{babel}
\usepackage{fancyhdr,fancybox} % pour personnaliser les en-têtes
\usepackage{lastpage,setspace}
\usepackage{amsfonts,amssymb,amsmath,amsthm,mathrsfs}
\usepackage{relsize,exscale,bbold}
\usepackage{paralist}
\usepackage{xspace,multicol,diagbox,array}
\usepackage{xcolor}
\usepackage{variations}
\usepackage{xypic}
\usepackage{eurosym,stmaryrd}
\usepackage{graphicx}
\usepackage[np]{numprint}
\usepackage{hyperref} 
\usepackage{tikz}
\usepackage{colortbl}
\usepackage{multirow}
\usepackage{MnSymbol,wasysym}
\usepackage[top=1.5cm,bottom=1.5cm,right=1.2cm,left=1.5cm]{geometry}
\usetikzlibrary{calc, arrows, plotmarks, babel,decorations.pathreplacing}
\setstretch{1.25}
%\usepackage{lipsum} %\usepackage{enumitem} %\setlist[enumerate]{itemsep=1mm} bug avec enumerate



\newtheorem{thm}{Théorème}
\newtheorem{rmq}{Remarque}
\newtheorem{prop}{Propriété}
\newtheorem{cor}{Corollaire}
\newtheorem{lem}{Lemme}
\newtheorem{prop-def}{Propriété-définition}

\theoremstyle{definition}

\newtheorem{defi}{Définition}
\newtheorem{ex}{Exemple}
\newtheorem*{rap}{Rappel}
\newtheorem{cex}{Contre-exemple}
\newtheorem{exo}{Exercice} % \large {\fontfamily{ptm}\selectfont EXERCICE}
\newtheorem{nota}{Notation}
\newtheorem{ax}{Axiome}
\newtheorem{appl}{Application}
\newtheorem{csq}{Conséquence}
\def\di{\displaystyle}



\renewcommand{\thesection}{\Roman{section}}\renewcommand{\thesubsection}{\arabic{subsection} }\renewcommand{\thesubsubsection}{\alph{subsubsection} }


\newcommand{\bas}{~\backslash}\newcommand{\ba}{\backslash}
\newcommand{\C}{\mathbb{C}}\newcommand{\R}{\mathbb{R}}\newcommand{\K}{\mathbb{K}}\newcommand{\Q}{\mathbb{Q}}\newcommand{\Z}{\mathbb{Z}}\newcommand{\N}{\mathbb{N}}\newcommand{\V}{\overrightarrow}\newcommand{\Cs}{\mathscr{C}}\newcommand{\Ps}{\mathscr{P}}\newcommand{\Rs}{\mathscr{R}}\newcommand{\Gs}{\mathscr{G}}\newcommand{\Ds}{\mathscr{D}}\newcommand{\happy}{\huge\smiley}\newcommand{\sad}{\huge\frownie}\newcommand{\danger}{\begin{tikzpicture}[x=1.5pt,y=1.5pt,rotate=-14.2]
	\definecolor{myred}{rgb}{1,0.215686,0}
	\draw[line width=0.1pt,fill=myred] (13.074200,4.937500)--(5.085940,14.085900)..controls (5.085940,14.085900) and (4.070310,15.429700)..(3.636720,13.773400)
	..controls (3.203130,12.113300) and (0.917969,2.382810)..(0.917969,2.382810)
	..controls (0.917969,2.382810) and (0.621094,0.992188)..(2.097660,1.359380)
	..controls (3.574220,1.726560) and (12.468800,3.984380)..(12.468800,3.984380)
	..controls (12.468800,3.984380) and (13.437500,4.132810)..(13.074200,4.937500)
	--cycle;
	\draw[line width=0.1pt,fill=white] (11.078100,5.511720)--(5.406250,11.875000)..controls (5.406250,11.875000) and (4.683590,12.812500)..(4.367190,11.648400)
	..controls (4.050780,10.488300) and (2.375000,3.675780)..(2.375000,3.675780)
	..controls (2.375000,3.675780) and (2.156250,2.703130)..(3.214840,2.964840)
	..controls (4.273440,3.230470) and (10.640600,4.847660)..(10.640600,4.847660)
	..controls (10.640600,4.847660) and (11.332000,4.953130)..(11.078100,5.511720)
	--cycle;
	\fill (6.144520,8.839900)..controls (6.460940,7.558590) and (6.464840,6.457090)..(6.152340,6.378910)
	..controls (5.835930,6.300840) and (5.320300,7.277400)..(5.003900,8.554750)
	..controls (4.683590,9.835940) and (4.679690,10.941400)..(4.996090,11.019600)
	..controls (5.312490,11.097700) and (5.824210,10.121100)..(6.144520,8.839900)
	--cycle;
	\fill (7.292960,5.261780)..controls (7.382800,4.898500) and (7.128900,4.523500)..(6.730460,4.421880)
	..controls (6.328120,4.324220) and (5.929680,4.535220)..(5.835930,4.898500)
	..controls (5.746080,5.261780) and (5.999990,5.640630)..(6.402340,5.738340)
	..controls (6.804690,5.839840) and (7.203110,5.625060)..(7.292960,5.261780)
	--cycle;
	\end{tikzpicture}}\newcommand{\alors}{\Large\Rightarrow}\newcommand{\equi}{\Leftrightarrow}
\newcommand{\fonction}[5]{\begin{array}{l|rcl}
		#1: & #2 & \longrightarrow & #3 \\
		& #4 & \longmapsto & #5 \end{array}}


\definecolor{vert}{RGB}{11,160,78}
\definecolor{rouge}{RGB}{255,120,120}
\definecolor{bleu}{RGB}{15,5,107}


\pagestyle{fancy}
\lhead{Optimal Sup Spé, groupe IPESUP}\chead{Année~2022-2023}\rhead{Niveau: Première année de PCSI }\lfoot{M. Botcazou}\cfoot{\thepage}\rfoot{mail: i.botcazou@gmx.fr }\renewcommand{\headrulewidth}{0.4pt}\renewcommand{\footrulewidth}{0.4pt}

\begin{document}
	
	
	\begin{center}
		\Large \sc colle 13 = Espaces vectoriels
	\end{center}
\raggedright

\section*{Connaître son cours:}
\begin{enumerate}
	\item Soit $u$ une application linéaire entre deux $\K$-espaces vectoriels $E$ et $F$.Montrer que l'image directe par $u$ d’un sous-espace vectoriel de $E$ est un sous-espace vectoriel de $F$.
	\item Montrer que la somme de deux sous-espaces vectoriels est directe si, et seulement si, leur intersection est égale à $\{0_E\} $.
	\item Soit $e_1 , \dots , e_p$ des vecteurs d’un $\K$-espace vectoriel $E $. 
	
	Montrer que pour tous $\lambda \in \K$ et $i\neq j \ \in \llbracket1, p\rrbracket$ , $\text{Vect}(e_1 ,\dots , e_p) = \text{Vect}(e_1 ,\dots , e_i + \lambda e_j , \dots, e_p) $.
	
\end{enumerate}

\section*{Exercices:} 	

\begin{exo}\textbf{(*)}\quad\\[0.25cm]
	Dans $\mathbb R[X]$, $P(X)=16X^3-7X^2+21X-4$ est-il combinaison linéaire de $P_1(X)=8X^3-5X^2+1$ et de $P_2(X)=X^2+7X-2$?
	
	\centering
	\rule{1\linewidth}{0.6pt}
\end{exo}
	
\begin{exo}\textbf{(**)}\quad\\[0.25cm]
	Soit $E = \Cs^0 (\R, \R)$ l’espace vectoriel des fonctions continues de $\R$ dans $\R$. Pour tout $a \in \R$, posons $E_a = \{ f \in E , f (a) = 0\}$.
	\begin{enumerate}
		\item Montrer, que pour tout $a \in \R$, $E_a$ est un sous-espace vectoriel de $E $.
		\item Soit $a \neq b $. Montrer que $E = E_a + E_b $.
		\item La somme de $E_a$ et de $E_b$ peut-elle être directe ?
	\end{enumerate}
	
	\centering
	\rule{1\linewidth}{0.6pt}
\end{exo}

	


\begin{exo}\textbf{(**)}\quad\\[0.25cm]%An114
1. \ Montrer par des opérations sur les Vect les égalités :\\[-0.5cm]

$$\R_2 [X ] \ = \  \text{Vect}\left((X - 1)^2 , (X - 1)(X + 1), (X + 1)^2\right)  .$$

2. \ Montrer que pour tout $n \in \N$ :
$$\underset{0\leq k\leq n}{\text{Vect}}\Big(\big(x\mapsto \cos(kx)\big)\Big)= \underset{0\leq k\leq n}{\text{Vect}}\Big(\big(x\mapsto \cos^k(x)\big)\Big).$$

	\centering
\rule{1\linewidth}{0.6pt}
\end{exo}

\begin{exo}\textbf{(**)}\quad\\[0.25cm]%An115
	Soit $E=\mathcal F(\R,\R)$ l'espace vectoriel des fonctions de $\R$ dans $\R$. On note $F$ le sous-espace vectoriel des fonctions paires et $G$ le sous-espace vectoriel des fonctions impaires. Montrer que $F$ et $G$ sont supplémentaires après avoir expliqué pour $F$ et $G$ étaient des sous-espaces vectoriels de $E$.
	
	\centering
	\rule{1\linewidth}{0.6pt}
\end{exo}


\newpage

\begin{exo}\textbf{(***)}\quad\\[0.25cm]
\noindent\textbf{ Partie A - Exemple d'un projecteur}
 
 Notons $E=\mathbb{R}[X]$ l'ensemble des polynômes réels, $\mathscr{P}$ et $\mathscr{I}$ les sous-espaces vectoriels des polynômes pairs et impairs respectivement.
 \begin{enumerate}
 	\item Montrer que $\mathscr{I}$ est un supplémentaire de $\mathscr{P}$ dans $E$.
 	\item   Soit l'application linéaire
 	$$\varphi : \left\{\begin{array}{ccl}
 	E &\longrightarrow & E\\
 	P &\longmapsto & \dfrac{P(X)+P(-X)}{2} + X\dfrac{P(X)-P(-X)}{2}
 	\end{array}\right.$$
 	\begin{enumerate}
 		\item Déterminer $\operatorname{Im} \varphi$ puis établir que
 		
 		$$
 		\operatorname{Ker} \varphi=\{(1-X) P(X), P \in \mathscr{I}\} .
 		$$
 		\item  Montrer que $\varphi$ est un projecteur de $E$.
 		\item En déduire que $\operatorname{Ker} \varphi$ est un supplémentaire de $\mathscr{P}$.
 	\end{enumerate}
 
 \end{enumerate}
 
\noindent\textbf{Partie B - sous-espaces qui admettent un supplémentaire commun}
 
 Soit $E$ un espace vectoriel, $F_{1}$ et $F_{2}$ deux sous-espaces vectoriels de $E$
 
 \begin{enumerate}
 	\item Supposons, dans cette question, que $F_{1}$ et $F_{2}$ sont supplémentaires dans $E$ et qu'il existe un isomorphisme $u: F_{1} \rightarrow F_{2}$.
 	
 	Montrer que $G=\left\{x-u(x), x \in F_{1}\right\}$ est un espace vectoriel puis qu'il est un supplémentaire commun à $F_{1}$ et $F_{2}$.
 	\item Réciproquement supposons dans cette question que $F_{1}$ et $F_{2}$ admettent un supplémentaire commun $G$. Montrer que $F_{1}$ et $F_{2}$ sont isomorphes.
 	
 	
 \end{enumerate}
 	\centering
	\rule{1\linewidth}{0.6pt}
\end{exo}






\end{document}
