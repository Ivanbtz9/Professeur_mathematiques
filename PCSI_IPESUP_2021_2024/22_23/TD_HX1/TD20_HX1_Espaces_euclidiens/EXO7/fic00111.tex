
%%%%%%%%%%%%%%%%%% PREAMBULE %%%%%%%%%%%%%%%%%%

\documentclass[11pt,a4paper]{article}

\usepackage{amsfonts,amsmath,amssymb,amsthm}
\usepackage[utf8]{inputenc}
\usepackage[T1]{fontenc}
\usepackage[francais]{babel}
\usepackage{mathptmx}
\usepackage{fancybox}
\usepackage{graphicx}
\usepackage{ifthen}

\usepackage{tikz}   

\usepackage{hyperref}
\hypersetup{colorlinks=true, linkcolor=blue, urlcolor=blue,
pdftitle={Exo7 - Exercices de mathématiques}, pdfauthor={Exo7}}

\usepackage{geometry}
\geometry{top=2cm, bottom=2cm, left=2cm, right=2cm}

%----- Ensembles : entiers, reels, complexes -----
\newcommand{\Nn}{\mathbb{N}} \newcommand{\N}{\mathbb{N}}
\newcommand{\Zz}{\mathbb{Z}} \newcommand{\Z}{\mathbb{Z}}
\newcommand{\Qq}{\mathbb{Q}} \newcommand{\Q}{\mathbb{Q}}
\newcommand{\Rr}{\mathbb{R}} \newcommand{\R}{\mathbb{R}}
\newcommand{\Cc}{\mathbb{C}} \newcommand{\C}{\mathbb{C}}
\newcommand{\Kk}{\mathbb{K}} \newcommand{\K}{\mathbb{K}}

%----- Modifications de symboles -----
\renewcommand{\epsilon}{\varepsilon}
\renewcommand{\Re}{\mathop{\mathrm{Re}}\nolimits}
\renewcommand{\Im}{\mathop{\mathrm{Im}}\nolimits}
\newcommand{\llbracket}{\left[\kern-0.15em\left[}
\newcommand{\rrbracket}{\right]\kern-0.15em\right]}
\renewcommand{\ge}{\geqslant} \renewcommand{\geq}{\geqslant}
\renewcommand{\le}{\leqslant} \renewcommand{\leq}{\leqslant}

%----- Fonctions usuelles -----
\newcommand{\ch}{\mathop{\mathrm{ch}}\nolimits}
\newcommand{\sh}{\mathop{\mathrm{sh}}\nolimits}
\renewcommand{\tanh}{\mathop{\mathrm{th}}\nolimits}
\newcommand{\cotan}{\mathop{\mathrm{cotan}}\nolimits}
\newcommand{\Arcsin}{\mathop{\mathrm{arcsin}}\nolimits}
\newcommand{\Arccos}{\mathop{\mathrm{arccos}}\nolimits}
\newcommand{\Arctan}{\mathop{\mathrm{arctan}}\nolimits}
\newcommand{\Argsh}{\mathop{\mathrm{argsh}}\nolimits}
\newcommand{\Argch}{\mathop{\mathrm{argch}}\nolimits}
\newcommand{\Argth}{\mathop{\mathrm{argth}}\nolimits}
\newcommand{\pgcd}{\mathop{\mathrm{pgcd}}\nolimits} 

%----- Structure des exercices ------

\newcommand{\exercice}[1]{\video{0}}
\newcommand{\finexercice}{}
\newcommand{\noindication}{}
\newcommand{\nocorrection}{}

\newcounter{exo}
\newcommand{\enonce}[2]{\refstepcounter{exo}\hypertarget{exo7:#1}{}\label{exo7:#1}{\bf Exercice \arabic{exo}}\ \  #2\vspace{1mm}\hrule\vspace{1mm}}

\newcommand{\finenonce}[1]{
\ifthenelse{\equal{\ref{ind7:#1}}{\ref{bidon}}\and\equal{\ref{cor7:#1}}{\ref{bidon}}}{}{\par{\footnotesize
\ifthenelse{\equal{\ref{ind7:#1}}{\ref{bidon}}}{}{\hyperlink{ind7:#1}{\texttt{Indication} $\blacktriangledown$}\qquad}
\ifthenelse{\equal{\ref{cor7:#1}}{\ref{bidon}}}{}{\hyperlink{cor7:#1}{\texttt{Correction} $\blacktriangledown$}}}}
\ifthenelse{\equal{\myvideo}{0}}{}{{\footnotesize\qquad\texttt{\href{http://www.youtube.com/watch?v=\myvideo}{Vidéo $\blacksquare$}}}}
\hfill{\scriptsize\texttt{[#1]}}\vspace{1mm}\hrule\vspace*{7mm}}

\newcommand{\indication}[1]{\hypertarget{ind7:#1}{}\label{ind7:#1}{\bf Indication pour \hyperlink{exo7:#1}{l'exercice \ref{exo7:#1} $\blacktriangle$}}\vspace{1mm}\hrule\vspace{1mm}}
\newcommand{\finindication}{\vspace{1mm}\hrule\vspace*{7mm}}
\newcommand{\correction}[1]{\hypertarget{cor7:#1}{}\label{cor7:#1}{\bf Correction de \hyperlink{exo7:#1}{l'exercice \ref{exo7:#1} $\blacktriangle$}}\vspace{1mm}\hrule\vspace{1mm}}
\newcommand{\fincorrection}{\vspace{1mm}\hrule\vspace*{7mm}}

\newcommand{\finenonces}{\newpage}
\newcommand{\finindications}{\newpage}


\newcommand{\fiche}[1]{} \newcommand{\finfiche}{}
%\newcommand{\titre}[1]{\centerline{\large \bf #1}}
\newcommand{\addcommand}[1]{}

% variable myvideo : 0 no video, otherwise youtube reference
\newcommand{\video}[1]{\def\myvideo{#1}}

%----- Presentation ------

\setlength{\parindent}{0cm}

\definecolor{myred}{rgb}{0.93,0.26,0}
\definecolor{myorange}{rgb}{0.97,0.58,0}
\definecolor{myyellow}{rgb}{1,0.86,0}

\newcommand{\LogoExoSept}[1]{  % input : echelle       %% NEW
{\usefont{U}{cmss}{bx}{n}
\begin{tikzpicture}[scale=0.1*#1,transform shape]
  \fill[color=myorange] (0,0)--(4,0)--(4,-4)--(0,-4)--cycle;
  \fill[color=myred] (0,0)--(0,3)--(-3,3)--(-3,0)--cycle;
  \fill[color=myyellow] (4,0)--(7,4)--(3,7)--(0,3)--cycle;
  \node[scale=5] at (3.5,3.5) {Exo7};
\end{tikzpicture}}
}


% titre
\newcommand{\titre}[1]{%
\vspace*{-4ex} \hfill \hspace*{1.5cm} \hypersetup{linkcolor=black, urlcolor=black} 
\href{http://exo7.emath.fr}{\LogoExoSept{3}} 
 \vspace*{-5.7ex}\newline 
\hypersetup{linkcolor=blue, urlcolor=blue}  {\Large \bf #1} \newline 
 \rule{12cm}{1mm} \vspace*{3ex}}

%----- Commandes supplementaires ------



\begin{document}

%%%%%%%%%%%%%%%%%% EXERCICES %%%%%%%%%%%%%%%%%%
\fiche{f00111, rouget, 2010/07/11}

\titre{Géométrie analytique (affine ou euclidienne)} 

Exercices de Jean-Louis Rouget.
Retrouver aussi cette fiche sur \texttt{\href{http://www.maths-france.fr}{www.maths-france.fr}}

\begin{center}
* très facile\quad** facile\quad*** difficulté moyenne\quad**** difficile\quad***** très difficile\\
I~:~Incontournable\quad T~:~pour travailler et mémoriser le cours
\end{center}


\exercice{5501, rouget, 2010/07/10}
\enonce{005501}{**T}
Dans $E_3$ rapporté à un repère $(O,i,j,k)$, on donne les points $A(1,2,-1)$, $B(3,2,0)$, $C(2,1,-1)$ et $D(1,0,4)$. Déterminer l'intersection des plans $(OAB)$ et $(OCD)$.
\finenonce{005501}


\finexercice
\exercice{5502, rouget, 2010/07/10}
\enonce{005502}{**T}
Dans $E_3$ rapporté à un repère $(O,i,j,k)$, on donne~:
la droite $(D)$ dont un système d'équations paramétriques est $\left\{
\begin{array}{l}
x=2+3t\\
y=-t\\
z=1+t
\end{array}
\right.
$,
le plan $P$ dont un système d'équations paramétriques est $\left\{
\begin{array}{l}
x=1+2\lambda+\mu\\
y=-1-3\lambda+2\mu\\
z=1+\lambda
\end{array}
\right.
$,
le plan $P'$ dont un système d'équations paramétriques est $\left\{
\begin{array}{l}
x=-5-\nu\\
y=3+\nu+3\eta\\
z=\nu+\eta
\end{array}
\right.
$,
Etudier $D\cap P$ et $P\cap P'$
\finenonce{005502}


\finexercice
\exercice{5503, rouget, 2010/07/10}
\enonce{005503}{**T}
Matrice dans la base canonique orthonormée directe de la rotation vectorielle de $\Rr^3$ autour de $(1,2,2)$ qui transforme $j$ en $k$.
\finenonce{005503}


\finexercice
\exercice{5504, rouget, 2010/07/15}
\enonce{005504}{**T}
Dans $\Rr^3$, espace vectoriel euclidien orienté rapporté à la base orthonormée directe $(i,j,k)$, déterminer l'image du plan d'équation $x+y=0$ par \begin{enumerate}
 \item  la symétrie orthogonale par rapport au plan d'équation $x-y+z=0$,  
 \item  la symétrie orthogonale par rapport au vecteur $(1,1,1)$,  
 \item  par la rotation d'angle $\frac{\pi}{4}$ autour du vecteur $(1,1,1)$.
\end{enumerate}
\finenonce{005504}


\finexercice
\exercice{5505, rouget, 2010/07/15}
\enonce{005505}{*T}
Dans $\Rr^3$ affine, déterminer un repère de la droite $(D)$ $\left\{
\begin{array}{l}
x-y+2z+7=0\\
2x+2y+3z-5=0
\end{array}
\right.$.
\finenonce{005505}


\finexercice
\exercice{5506, rouget, 2010/07/15}
\enonce{005506}{*T}
Dans $\Rr^3$, déterminer l'intersection de $(D)$ $\left\{
\begin{array}{l}
x=2+\lambda\\
y=3-\lambda\\
z=7
\end{array}
\right.$ et $(P)~:~x+3y-5z+2=0$.
\finenonce{005506}


\finexercice
\exercice{5507, rouget, 2010/07/15}
\enonce{005507}{**}
Dans $\Rr^3$ affine, déterminer le réel $a$ pour que les droites $\left\{
\begin{array}{l}
x+2=-2z\\
y=3x+z
\end{array}
\right.$ et $\left\{
\begin{array}{l}
x+y+z=1\\
2x+y-z=a
\end{array}
\right.$ soient coplanaires, puis déterminer une équation du plan les contenant.
\finenonce{005507}


\finexercice
\exercice{5508, rouget, 2010/07/15}
\enonce{005508}{**T}
Dans $\Rr^3$, équation du plan $P$ parallèle à la droite $(Oy)$ et passant par $A(0,-1,2)$ et $B(-1,2,3)$.
\finenonce{005508}


\finexercice
\exercice{5509, rouget, 2010/07/15}
\enonce{005509}{**T}
Dans $\Rr^3$, soient $(D)$ $\left\{
\begin{array}{l}
x+y+z=1\\
x-2y-z=0
\end{array}
\right.$ et $(\Delta)~:~6x=2y=3z$ puis $(P)~:~x+3y+2z=6$. Déterminer la projection de $(D)$ sur $(P)$ parallèlement à $(\Delta)$.

\finenonce{005509}


\finexercice
\exercice{5510, rouget, 2010/07/15}
\enonce{005510}{**}
Dans $\Rr^3$, soient $(D)$ $\left\{
\begin{array}{l}
x-z-a=0\\
y+3z+1=0
\end{array}
\right.$ et $(D')$ $\left\{
\begin{array}{l}
x+2y+z-2b=0\\
3x+3y+2z-7=0
\end{array}
\right.$.
Vérifier que $(D)$ et $(D')$ ne sont pas parallèles puis trouver $a$ et $b$ pour que $(D)$ et $(D')$ soient sécantes. Former alors une équation cartésienne de leur plan.
\finenonce{005510}


\finexercice
\exercice{5511, rouget, 2010/07/15}
\enonce{005511}{**}
Système d'équations cartésiennes de la droite $(\Delta)$ parallèle à la droite $(D)$~:~$2x=3y=6z$ et sécante aux droites $(D_1)$~:~$x=z-4=0$ et $(D_2)$~:~$y=z+4=0$.
\finenonce{005511}


\finexercice
\exercice{5512, rouget, 2010/07/15}
\enonce{005512}{***}
Trouver toutes les droites sécantes aux quatre droites $(D_1)$~$x-1=y=0$, $(D_2)$~:~$y-1=z=0$, $(D_3)$~:~$z-1=x=0$ et $(D_4)$~:~$x=y=-6z$.

\finenonce{005512}


\finexercice
\exercice{5513, rouget, 2010/07/15}
\enonce{005513}{**T}
Dans $\Rr^3$ euclidien rapporté à un repère orthonormé, on donne $A(2,-2,0)$, $B(4,2,6)$ et 
$C(-1,-3,0)$. Déterminer l'orthocentre, le centre de gravité, les centres des cercles circonscrits et inscrits au triangle $(A,B,C)$.
\finenonce{005513}


\finexercice\exercice{5514, rouget, 2010/07/15}
\enonce{005514}{**T}
Soit $M(x,y,z)$ un point de $\Rr^3$ rapporté à un repère orthonormé. Déterminer la distance de $M$ à la droite 
$(D)$ $\left\{
\begin{array}{l}
x+y+z+1=0\\
2x+y+5z=2
\end{array}
\right.$. En déduire une équation du cylindre de révolution d'axe $(D)$ et de rayon $2$. 
\finenonce{005514}


\finexercice
\exercice{5515, rouget, 2010/07/15}
\enonce{005515}{**T}
Dans $\Rr^3$ rapporté à un repère orthonormé, soient $(D)$ $\left\{
\begin{array}{l}
x+y+z+1=0\\
2x+y+5z=2
\end{array}
\right.$ et $(D')$ $\left\{
\begin{array}{l}
x+y+z=2\\
2x+y-5z=3
\end{array}
\right.$. Déterminer la distance de $(D)$ à $(D')$ puis la perpendiculaire commune à ces deux droites.
\finenonce{005515}


\finexercice
\exercice{5516, rouget, 2010/07/15}
\enonce{005516}{**}
Montrer que les plans $(P_1)$~:~$z-2y=5$, $(P_2)$~:~$2x-3z=0$ et $(P_3)~:~3y-x=0$ admettent une parallèle commune. Ils définissent ainsi un prisme. Déterminer l'aire d'une section perpendiculaire.
\finenonce{005516}


\finexercice
\exercice{5517, rouget, 2010/07/15}
\enonce{005517}{*T}
Angle des plans $x+2y+2z=3$ et $x+y=0$.
\finenonce{005517}


\finexercice
\exercice{5518, rouget, 2010/07/15}
\enonce{005518}{**T}
Soient $(P_1)~:~4x+4y-7z-1=0$ et $(P_2)~:~8x-4y+z+7=0$. Trouver une équation cartésienne des plans bissecteurs de $(P_1)$ et $(P_2)$.
\finenonce{005518}


\finexercice
\exercice{5519, rouget, 2010/07/15}
\enonce{005519}{**T}
Déterminer la perpendiculaire commune aux droites $(D)$ et $(D')$~:~$(D)$ $\left\{
\begin{array}{l}
x+y-3z+4=0\\
2x-z+1=0
\end{array}
\right.$ et $(D')$ $\left\{
\begin{array}{l}
x=z-1\\
y=z-1
\end{array}
\right.$. 
\finenonce{005519}


\finexercice
\exercice{5520, rouget, 2010/07/15}
\enonce{005520}{**}
Soient $(D)$ la droite dont un système d'équations cartésiennes est 
$\left\{
\begin{array}{l}
x+y+z=1\\
x-2y-z=0
\end{array}
\right.$ et $(P)$ le plan d'équation cartésienne $x+3y+2z=6$. Déterminer la projetée (orthogonale) de $(D)$ sur $(P)$.
\finenonce{005520}


\finexercice
\exercice{5521, rouget, 2010/07/15}
\enonce{005521}{**I}
Déterminer les différents angles d'un tétraèdre régulier (entre deux faces, entre deux arêtes et entre une arête et une face).
\finenonce{005521}


\finexercice
\exercice{5522, rouget, 2010/07/15}
\enonce{005522}{**T}
Déterminer la distance de l'origine $O$ à la droite $(D)$ dont un système d'équations cartésiennes est $\left\{
\begin{array}{l}
x-y-z=0\\
x+2y-z=10
\end{array}
\right.$.
\finenonce{005522}


\finexercice

\finfiche

 \finenonces 



 \finindications 

\noindication
\noindication
\noindication
\noindication
\noindication
\noindication
\noindication
\noindication
\noindication
\noindication
\noindication
\noindication
\noindication
\noindication
\noindication
\noindication
\noindication
\noindication
\noindication
\noindication
\noindication
\noindication


\newpage

\correction{005501}
\textbullet~$\overrightarrow{n}=\overrightarrow{OA}\wedge\overrightarrow{OB}$ a pour coordonnées $(2,-3,-4)$. Ce vecteur n'est pas nul. Par suite, les points $O$, $A$ et $B$ ne sont pas alignés et le plan $(OAB)$ est bien défini. C'est le plan passant par $O$ et de vecteur normal $\overrightarrow{n}(2,-3,-4)$. Une équation cartésienne du plan $(OAB)$ est donc $2x-3y-4z=0$.
\textbullet~$\overrightarrow{n}'=\overrightarrow{OC}\wedge\overrightarrow{OD}$ a pour coordonnées $(4,-9,-1)$. Ce vecteur n'est pas nul. Par suite, les points $O$, $C$ et $D$ ne sont pas alignés et le plan $(OCD)$ est bien défini. C'est le plan passant par $O$ et de vecteur normal $\overrightarrow{n}'(4,-9,-1)$. Une équation cartésienne du plan $(OAB)$ est donc $4x-9y-z=0$.
\textbullet$-\overrightarrow{n}\wedge\overrightarrow{n}'$ a pour coordonnées $(33,14,6)$. Ce vecteur n'est pas nul et on sait que les plans $(OAB)$ et $(OCD)$ sont sécants en une droite, à savoir la droite passant par $O(0,0,0)$ et de vecteur directeur $(33,14,6)$. Un système d'équations cartésiennes de cette droite est 
$\left\{
\begin{array}{l}
2x-3y-4z=0\\
4x-9y-z=0
\end{array}
\right.$.
\fincorrection
\correction{005502}
Les vecteurs $(2,-3,1)$ et $(1,2,0)$ ne sont pas colinéaires, de sorte que $(P)$ est bien un plan. Trouvons alors une équation cartésienne de $(P)$

\begin{align*}\ensuremath
M(x,y,z)\in(P)&\Leftrightarrow\exists(\lambda,\mu)\in\Rr^2/\;
\left\{
\begin{array}{l}
x=1+2\lambda+\mu\\
y=-1-3\lambda+2\mu\\
z=1+\lambda
\end{array}
\right.
\Leftrightarrow\exists(\lambda,\mu)\in\Rr^2/\;
\left\{
\begin{array}{l}
\lambda=z-1\\
x=1+2(z-1)+\mu\\
y=-1-3(z-1)+2\mu
\end{array}
\right.
\\
 &\Leftrightarrow\exists(\lambda,\mu)\in\Rr^2/\;
\left\{
\begin{array}{l}
\lambda=z-1\\
\mu=x-2z+1\\
y=-1-3(z-1)+2(x-2z+1)
\end{array}
\right.
\\
 &\Leftrightarrow
-2x+y+7z-4=0
\end{align*}
Soit alors $M(2+3t,-t,1+t)$, $t\in\Rr$, un point de $(D)$

\begin{align*}\ensuremath
M\in(P)&\Leftrightarrow-2(2+3t)+(-t)+7(1+t)-4=0
\Leftrightarrow
0\times t-1=0.
\end{align*}
Ce dernier système n'a pas de solution et donc $(D)\cap(P)=\varnothing$. La droite $(D)$ est strictement parallèle au plan $(P)$.

\begin{align*}\ensuremath
M(x,y,z)\in(P)\cap(P')&\Leftrightarrow\exists(\nu,\eta)\in\Rr^2/\;\left\{
\begin{array}{l}
x=-5-\nu\\
y=3+\nu+3\eta\\
z=\nu+\eta\\
-2x+y+7z-4=0
\end{array}
\right.\\
 &\Leftrightarrow\exists(\nu,\eta)\in\Rr^2/\;\left\{
\begin{array}{l}
x=-5-\nu\\
y=3+\nu+3\eta\\
z=\nu+\eta\\
-2(-5-\nu)+(3+\nu+3\eta)+7(\nu+\eta)-4=0
\end{array}
\right.\\
 &\Leftrightarrow\exists(\nu,\eta)\in\Rr^2/\;\left\{
\begin{array}{l}
\eta=-\nu-\frac{9}{10}\\
x=-5-\nu\\
y=3+\nu+3\left(-\nu-\frac{9}{10}\right)\\
z=\nu+\left(-\nu-\frac{9}{10}\right)
\end{array}
\right.
\Leftrightarrow
\exists\nu\in\Rr/\;
\left\{
\begin{array}{l}
x=-\nu-5\\
y=-2\nu+\frac{3}{10}\\
z=-\frac{9}{10}
\end{array}
\right.
\end{align*}
$(P)$ et $(P')$ sont donc sécants en la droite passant par le point $\left(-5,\frac{3}{10},-\frac{9}{10}\right)$ et de vecteur directeur $(1,2,0)$.
\fincorrection
\correction{005503}
Soit $r$ la rotation cherchée. Notons $u$ le vecteur $\frac{1}{3}(1,2,2)$ ($u$ est unitaire) et $\theta$ l'angle de $r$. $r$ est la rotation d'angle $\theta$ autour du vecteur unitaire $u$.  On sait que pour tout vecteur $v$ de $\Rr^3$

\begin{center}
$r(v)=(\cos\theta)v+(1-\cos\theta)(v.u)u+(\sin\theta) u\wedge v\quad(*)$
\end{center}
et en particulier que $[v,r(v),u]=\sin\theta\|v\wedge u\|^2$. L'égalité $r(j)=k$ fournit

\begin{center}
$\sin\theta\|j\wedge u\|^2=\left[j,r(j),u\right]=\left[u,j,k\right]=\frac{1}{3}\left|\begin{array}{ccc}
1&0&0\\
2&1&0\\
3&0&1
\end{array}
\right|=\frac{1}{3}$.
\end{center}

Comme $u\wedge j=\frac{1}{3}(i+2j+2k)\wedge j=-\frac{2}{3}j+\frac{1}{3}k$, on a $\|j\wedge u\|^2=\frac{5}{9}$ et donc $\sin\theta=\frac{3}{5}$. L'égalité $r(j)=k$ fournit ensuite

\begin{center}
$k=(\cos\theta)j+(1-\cos\theta)\times\frac{2}{3}\times\frac{1}{3}(i+2j+2k)+\frac{3}{5}\times\frac{1}{3}(i+2j+2k)\wedge j$
\end{center}
En analysant la composante en $i$, on en déduit que $\frac{2}{9}(1-\cos\theta)-\frac{2}{5}=0$ et donc $\cos\theta=-\frac{4}{5}$. Ainsi, pour tout vecteur $v=(x,y,z)$ de $\Rr^3$, l'égalité $(*)$ s'écrit

\begin{align*}\ensuremath
r(v)&=-\frac{4}{5}(x,y,z)+\frac{9}{5}\times\frac{1}{3}\times\frac{1}{3}(x+2y+2z)(1,2,2)+\frac{3}{5}\times\frac{1}{3}
(2z-2y,2x-z,-2x+y)\\
 &=\frac{1}{5}(-4x+(x+2y+2z)+(2z-2y),-4y+2(x+2y+2z)+(2x-z),-4z+2(x+2y+2z)+(-2x+y))\\
 &=\frac{1}{5}(-3x+4z,4x+3z,5y)=\frac{1}{5}\left(
 \begin{array}{ccc}
 -3&0&4\\
 4&0&3\\
 0&5&0
 \end{array}
 \right)\left(
 \begin{array}{c}
 x\\
 y\\
 z
 \end{array}
 \right)
\end{align*}
La matrice cherchée est

\begin{center}
\shadowbox{
$\left(
 \begin{array}{ccc}
 -\frac{3}{5}&0&\frac{4}{5}\\
\rule[-4mm]{0mm}{10mm} \frac{4}{5}&0&\frac{3}{5}\\
 0&1&0
 \end{array}
 \right)$.
 }
 \end{center}
\fincorrection
\correction{005504}
Notons $P$ le plan d'équation $x+y=0$ dans la base $\mathcal{B}=(i,j,k)$. $P$ est le plan de vecteur normal $n=i+j$.

\begin{enumerate}
 \item  Soit $s$ la symétrie orthogonale par rapport au plan $P'$ d'équation $x-y+z=0$. $s(P)$ est le plan de vecteur normal $s(n)$. 
Or, le vecteur $n$ est dans $P'$  et donc $s(n)=n$ puis $s(P)=P$.

\begin{center}
\shadowbox{
$s(P)$ est le plan $P$.
}
\end{center}
 \item  Notons $\sigma$ la symétrie orthogonale par rapport au vecteur $u=(1,1,1)$. $\sigma(P)$ est le plan de vecteur normal 

\begin{center}
$\sigma(n)=2\frac{n.u}{\|u\|^2}u-n=2\frac{2}{3}(1,1,1)-(1,1,0)=\frac{1}{3}(1,1,4)$.
\end{center}

\begin{center}
\shadowbox{
$\sigma(P)$ est le plan d'équation $x+y+4z=0$.
}
\end{center}
 \item  Notons $r$ la rotation d'angle $\frac{\pi}{4}$ autour du vecteur unitaire $u=\frac{1}{\sqrt{3}}(1,1,1)$. $r(P)$ est le plan de vecteur normal 
\begin{align*}\ensuremath
r(n)&=\left(\cos\frac{\pi}{4}\right)n+\left(1-\cos\frac{\pi}{4}\right)(n.u)u+\left(\sin\frac{\pi}{4}\right)u\wedge n\\
 &=\frac{1}{\sqrt{2}}(1,1,0)+\left(1-\frac{1}{\sqrt{2}}\right)\times\frac{2}{3}(1,1,1)+\frac{1}{\sqrt{2}}\times\frac{1}{\sqrt{3}}(-1,1,0)\\
 &=\frac{1}{3\sqrt{2}}(3+2(\sqrt{2}-1)-\sqrt{3},3+2(\sqrt{2}-1)+\sqrt{3},2(\sqrt{2}-1))=\frac{1}{3\sqrt{2}}(1+2\sqrt{2}-\sqrt{3},1+2\sqrt{2}+\sqrt{3},2(\sqrt{2}-1)).
\end{align*}

\begin{center}
\shadowbox{
$r(P)$ est le plan d'équation $(1+2\sqrt{2}-\sqrt{3})x+(1+2\sqrt{2}+\sqrt{3})y+2(\sqrt{2}-1)z=0$.
}
\end{center}
\end{enumerate}
\fincorrection
\correction{005505}
Puisque $\left|
\begin{array}{cc}
1&2\\
2&3
\end{array}\right|=-1$, on choisit d'exprimer $x$ et $z$ en fonction de $y$.
Soit $M(x,y,z)\in\Rr^3$. D'après les formules de \textsc{Cramer}, on a
\begin{align*}\ensuremath
M\in(D)&\Leftrightarrow\left\{
\begin{array}{l}
x-y+2z+7=0\\
2x+2y+3z-5=0
\end{array}
\right.\Leftrightarrow\left\{
\begin{array}{l}
x+2z=y-7\\
-2x-3z=2y-5
\end{array}
\right.\\
 &\Leftrightarrow x=\frac{1}{1}\left|
\begin{array}{cc}
y-7&2\\
2y-5&-3
\end{array}
\right|\;\text{et}\;z=\frac{1}{1}\left|
\begin{array}{cc}
1&y-7\\
-2&2y-5
\end{array}
\right|\\
 &\Leftrightarrow\left\{\begin{array}{l}
 x=31-7y\\
 z=-19+4y
 \end{array}\right..
\end{align*}

\begin{center}
\shadowbox{
$(D)$ est la droite passant par $A(31,0,-19)$ dirigée par le vecteur $u(-7,1,4)$.
}
\end{center}
\fincorrection
\correction{005506}
Soit $M(2+\lambda,3-\lambda,7)$, $\lambda\in\Rr$, un point quelconque de $(D)$.

\begin{center}
$M\in(P)\Leftrightarrow(2+\lambda)+3(3-\lambda)-5\times7+2=0\Leftrightarrow \lambda=12$.
\end{center}
$(P)\cap(D)$ est donc un singleton. Pour $\lambda=12$, on obtient les coordonnées du point d'intersection

\begin{center}
\shadowbox{
$(P)\cap(D)=\{(14,-9,7)\}$.
}
\end{center}
\fincorrection
\correction{005507}
\textbullet~\textbf{Repère de $(D)$.}

\begin{center}
$\left\{
\begin{array}{l}
x+2=-2z\\
y=3x+z
\end{array}
\right.\Leftrightarrow\left\{
\begin{array}{l}
x=-2-2z\\
y=3(-2-2z)+z
\end{array}
\right.\Leftrightarrow\left\{
\begin{array}{l}
x=-2-2z\\
y=-6-5z
\end{array}
\right.$
\end{center}

$(D)$ est la droite passant par $A(0,-1,-1)$ et dirigée par $u(2,5,-1)$.
 \textbullet~\textbf{Repère de $(D')$.}

\begin{center}$\left\{
\begin{array}{l}
x+y+z=1\\
2x+y-z=a
\end{array}
\right.\Leftrightarrow\left\{
\begin{array}{l}
-x-y=z-1\\
2x+y=z+a
\end{array}
\right.\Leftrightarrow\left\{
\begin{array}{l}
x=2z+a-1\\
y=2-a-3z
\end{array}
\right.$
\end{center}
$(D')$ est la droite passant par $A'(a-1,2-a,0)$ et dirigée par $u'(2,-3,1)$.
\textbullet~Déjà $u$ et $u'$ ne sont pas colinéaires et donc $(D)$ et $(D')$ sont ou bien sécantes en un point et dans ce cas coplanaires ou bien non coplanaires.
\textbullet~Le plan $(P)$ contenant $(D)$ et parallèle à $(D')$ est le plan de repère $(A,u,u')$. Déterminons une équation de ce plan.

\begin{center}
$M(x,y,z)\in(P)\Leftrightarrow
\left|
\begin{array}{ccc}
x&2&2\\
y+1&5&-3\\
z+1&-1&1
\end{array}\right|=0\Leftrightarrow 2x-4(y+1)-16(z+1)=0\Leftrightarrow-x+2y+8z=-10$.
\end{center}
\textbullet~Enfin, $(D)$ et $(D')$ sont coplanaires si et seulement si $(D')$ est contenue dans $(P)$. Comme $(D')$ est déjà parallèle à $(P)$, on a

\begin{center}
$(D)$ et $(D')$ coplanaires $\Leftrightarrow A'\in(P)\Leftrightarrow-(a-1)+2(2-a)=-10\Leftrightarrow a=\frac{5}{3}$.
\end{center}

\begin{center}
\shadowbox{
\begin{tabular}{c}
$(D)$ et $(D')$ sont coplanaires si et seulement si $a=\frac{5}{3}$ et dans ce cas, une équation du plan contenant $(D)$\\
 et $(D')$ est $-x+2y+8z=-10$.
 \end{tabular}
}
\end{center}
\fincorrection
\correction{005508}
Puisque $P$ parallèle à la droite $(Oy)$, le vecteur $\overrightarrow{j}=(0,1,0)$ est dans $\overrightarrow{P}$. De même, le vecteur $\overrightarrow{AB}=(-1,3,1)$ est dans $\overrightarrow{P}$.
$P$ est donc nécessairement le plan passant par $A(0,-1,2)$ et de vecteur normal $\overrightarrow{j}\wedge\overrightarrow{AB}=(1,0,1)$. Réciproquement, ce plan convient.
Une équation de $P$ est donc $(x-0)+(z-2)=0$ ou encore $x+z=2$.

\begin{center}
\shadowbox{
Une équation du plan parallèle à la droite $(Oy)$ et passant par $A(0,-1,2)$ et $B(-1,2,3)$ est $x+z=2$.
}
\end{center}
\fincorrection
\correction{005509}
Notons $p$ la projection sur $(P)$ parallèlement à $(\Delta)$.
\textbullet~Déterminons un repère de $(D)$. 
\begin{center}$\left\{
\begin{array}{l}
x+y+z=1\\
x-2y-z=0
\end{array}
\right.\Leftrightarrow
\left\{
\begin{array}{l}
y+z=-x+1\\
2y+z=x
\end{array}
\right.
\Leftrightarrow
\left\{
\begin{array}{l}
y=2x-1\\
z=-3x+2
\end{array}
\right.
$
\end{center}
$(D)$ est la droite de repère $(A,\overrightarrow{u})$ où $A(0,-1,2)$ et $\overrightarrow{u}(1,2,-3)$.
\textbullet~$(\Delta)$ est dirigée par le vecteur $\overrightarrow{u'}(1,3,2)$. $\overrightarrow{u}$ n'est pas colinéaire à $\overrightarrow{u'}$ et donc $(D)$ n'est pas parallèle à $(\Delta)$. On en déduit que $p(D)$ est une droite. 
Plus précisément, $p(D)$ est la droite intersection du plan $(P)$ et du plan $(P')$ contenant $(D)$ et parallèle à $(\Delta)$. Déterminons une équation de $(P')$. Un repère de $(P')$ est $(A,\overrightarrow{u},\overrightarrow{u'})$. Donc

\begin{center}
$M(x,y,z)\in(P')\Leftrightarrow\left|
\begin{array}{ccc}
x&1&1\\
y+1&2&3\\
z-2&-3&2
\end{array}
\right|=0\Leftrightarrow 13x-5(y+1)+(z-2)=0\Leftrightarrow13x-5y+z=7$.
\end{center}

Finalement

\begin{center}
\shadowbox{
$p(D)$ est la droite dont un système d'équations cartésiennes est $\left\{
\begin{array}{l}
13x-5y+z=7\\
x+3y+2z=6
\end{array}
\right.$
}
\end{center}
\fincorrection
\correction{005510}
\textbullet~\textbf{Repère de $(D)$.}

\begin{center}
$\left\{
\begin{array}{l}
x-z-a=0\\
y+3z+1=0
\end{array}
\right.\Leftrightarrow\left\{
\begin{array}{l}
x=a+z\\
y=-1-3z
\end{array}
\right.$.
\end{center}

$(D)$ est la droite passant par $A(a,-1,0)$ et dirigée par $u(1,-3,1)$.
 \textbullet~\textbf{Repère de $(D')$.}

\begin{center}$\left\{
\begin{array}{l}
x+2y+z-2b=0\\
3x+3y+2z-7=0
\end{array}
\right.\Leftrightarrow\left\{
\begin{array}{l}
2y+z=2b-x\\
3y+2z=7-3x
\end{array}
\right.\Leftrightarrow\left\{
\begin{array}{l}
y=4b-7+x\\
z=14-6b-3x
\end{array}
\right.$
\end{center}
$(D')$ est la droite passant par $A'(0,4b-7,-6b+14)$ et dirigée par $u'(1,1,-3)$.
\textbullet~Les vecteurs $u$ et $u'$ ne sont pas colinéaires et donc $(D)$ et $(D')$ ne sont pas parallèles.
\textbullet~Le plan $(P)$ contenant $(D)$ et parallèle à $(D')$ est le plan de repère $(A,u,u')$. Déterminons une équation de ce plan.

\begin{center}
$M(x,y,z)\in(P)\Leftrightarrow
\left|
\begin{array}{ccc}
x-a&1&1\\
y+1&-3&1\\
z&1&-3
\end{array}\right|=0\Leftrightarrow 8(x-a)+4(y+1)+4z=0\Leftrightarrow2x+y+z=2a-1$.
\end{center}
\textbullet~Enfin, $(D)$ et $(D')$ sont sécantes si et seulement si $(D')$ est contenue dans $(P)$. Comme $(D')$ est déjà parallèle à $(P)$, on a

\begin{center}
$(D)$ et $(D')$ sécantes $\Leftrightarrow A'\in(P)\Leftrightarrow(4b-7)+(-6b+14)=2a-1\Leftrightarrow b=-a+4$.
\end{center}

\begin{center}
\shadowbox{
\begin{tabular}{c}
$(D)$ et $(D')$ sont sécantes si et seulement si $b=-a+4$ et dans ce cas, une équation du plan contenant $(D)$\\
 et $(D')$ est $2x+y+z=2a-1$.
 \end{tabular}
}
\end{center}
\fincorrection
\correction{005511}
\textbullet~$(\Delta)$ est parallèle à $(D)$ si et seulement si $(\Delta)$ est dirigée par le vecteur $u(3,2,1)$ ou encore $(\Delta)$ admet un système d'équations paramétriques de la forme $\left\{
\begin{array}{l}
x=a+3\lambda\\
y=b+2\lambda\\
z=c+\lambda
\end{array}
\right.$. Ensuite, 
$(\Delta)$ est sécante à $(D_1)$ si et seulement si on peut choisir le point $(a,b,c)$ sur $(D_1)$ ou encore si et seulement si $(\Delta)$ admet un système d'équations paramétriques de la forme $\left\{
\begin{array}{l}
x=3\lambda\\
y=b+2\lambda\\
z=4+\lambda
\end{array}
\right.$.
Enfin,

\begin{center}
$(\Delta)$ et $(D_2)$ sécantes $\Leftrightarrow\exists\lambda\in\Rr/\;b+2\lambda=4+\lambda+4=0\Leftrightarrow b+2\times(-8)=0\Leftrightarrow b=16$.
\end{center}
Ceci démontre l'existence et l'unicité de $(\Delta)$ : un système d'équations paramétriques de $(\delta)$ est $\left\{
\begin{array}{l}
x=3\lambda\\
y=16+2\lambda\\
z=4+\lambda
\end{array}
\right.$. Un système d'équations cartésiennes de $(\Delta)$ est $\left\{
\begin{array}{l}
x=3(z-4)\\
y=16+2(z-4)
\end{array}
\right.$ ou encore

\begin{center}
\shadowbox{
$(\Delta)$ : $\left\{
\begin{array}{l}
x-3z+12=0\\
y-2z-8=0
\end{array}
\right.$.
}
\end{center}
\fincorrection
\correction{005512}
Notons $(\Delta)$ une éventuelle droite solution.
\textbullet~$(\Delta)$ est sécante à $(D_1)$ et $(D_2)$ si et seulement si $(\Delta)$ passe par un point de la forme $(1,0,a)$ et par un point de la forme $(b,1,0)$ ou encore si et seulement si $(\Delta)$ passe par un point de la forme $(1,0,a)$ et est dirigée par un vecteur de la forme $(b-1,1,-a)$.
Ainsi, $(\Delta)$ est sécante à $(D_1)$ et $(D_2)$ si et seulement si $(\Delta)$ admet un système d'équations paramétriques de la forme $\left\{
\begin{array}{l}
x=1+\lambda(b-1)\\
y=\lambda\\
z=a-\lambda a
\end{array}
\right.$ ou encore un système d'équations cartésiennes de la forme $\left\{
\begin{array}{l}
x-(b-1)y=1\\
ay+z=a
\end{array}
\right.$.

\textbullet~Ensuite, $(\Delta)$ et $(D_3)$ sécantes $\Leftrightarrow\exists y\in\Rr/\;\left\{
\begin{array}{l}
-(b-1)y=1\\
ay+1=a
\end{array}
\right.\Leftrightarrow b\neq 1\;\text{et}\;-\frac{a}{b-1}+1=a\Leftrightarrow b\neq0\;\text{et}\;b\neq1\;\text{et}\;a=1-\frac{1}{b}$.
En résumé, les droites sécantes à $(D_1)$, $(D_2)$ et $(D_3)$ sont les droites dont un système d'équations cartésiennes est 

\begin{center}
$\left\{
\begin{array}{l}
x-(b-1)y=1\\
\left(1-\frac{1}{b}\right)y+z=1-\frac{1}{b}
\end{array}
\right.$, $b\notin\{0,1\}$.
\end{center}
Enfin,

\begin{align*}\ensuremath
(\Delta)\;\text{et}\;(D)\;\text{sécantes}&\Leftrightarrow\exists(x,y,z)\in\Rr^3/\;\left\{
\begin{array}{l}
x-(b-1)y=1\\
\left(1-\frac{1}{b}\right)y+z=1-\frac{1}{b}\\
x=y=-6z
\end{array}
\right.\\
&\Leftrightarrow\exists(x,y,z)\in\Rr^3/\;\left\{
\begin{array}{l}
-6z+6(b-1)z=1\\
-6\left(1-\frac{1}{b}\right)z+z=1-\frac{1}{b}\\
x=y=-6z
\end{array}
\right.\\
 &\Leftrightarrow b\notin\{0,1,2\}\;\text{et}\;-6\left(1-\frac{1}{b}\right)\frac{1}{6(b-2)}+\frac{1}{6(b-2)}=1-\frac{1}{b}\\
 &\Leftrightarrow b\notin\{0,1,2\}\;\text{et}\;-6(b-1)+b=6(b-1)(b-2)\Leftrightarrow b\notin\{0,1,2\}\;\text{et}\;6b^2-13b+6=0\\
 &\Leftrightarrow b\in\left\{\frac{2}{3},\frac{3}{2}\right\}.
\end{align*}

\begin{center}
\shadowbox{
Les droites solutions sont $(\Delta_1)$ : $\left\{
\begin{array}{l}
3x+y=3\\
y-2z=1
\end{array}
\right.$ et $(\Delta_2)$ : $\left\{
\begin{array}{l}
2x-y=2\\
y+3z=1
\end{array}
\right.$.
}
\end{center}
\fincorrection
\correction{005513}
\textbullet~Déterminons le centre de gravité $G$.

\begin{center}
$G=\frac{1}{3}A+\frac{1}{3}B+\frac{1}{3}C=\frac{1}{3}(2,-2,0)+\frac{1}{3}(4,2,6)+\frac{1}{3}(-1,-3,0)=\left(\frac{5}{3},-1,2\right)$.
\end{center}
\textbullet~Déterminons le centre du cercle circonscrit $O$. Une équation du plan $(ABC)$ est $\left|
\begin{array}{ccc}
x-2&2&-3\\
y+2&4&-1\\
z&6&0
\end{array}
\right|=0$ ou encore $6(x-2)-18(y+2)+10z=0$ ou enfin $3x-9y+5z=24$. Posons alors $O(a,b,c)$.
Ensuite, $OA=OB\Leftrightarrow(a-2)^2+(b+2)^2+c^2=(a-4)^2+(b-2)^2+(c-6)^2\Leftrightarrow4a+8b+12c=48\Leftrightarrow a+2b+3c=16$ et 
$OA=OC\Leftrightarrow(a-2)^2+(b+2)^2+c^2=(a+1)^2+(b+3)^2+c^2\Leftrightarrow-6a-2b=2\Leftrightarrow3a+b=-1$.
D'où le système

\begin{align*}\ensuremath
\left\{
\begin{array}{l}
3a-9b+5c=24\\
a+2b+3c=16\\
3a+b=-1
\end{array}
\right.&\Leftrightarrow\left\{
\begin{array}{l}
b=-3a-1\\
3a-9(-3a-1)+5c=24\\
a+2(-3a-1)+3c=16
\end{array}
\right.\Leftrightarrow\left\{
\begin{array}{l}
b=-3a-1\\
6a+c=3\\
-5a+3c=18
\end{array}
\right.\\
 &\Leftrightarrow\left\{
\begin{array}{l}
b=-3a-1\\
c=3-6a\\
-5a+3(3-6a)=18
\end{array}
\right.
\Leftrightarrow\left\{
\begin{array}{l}
a=-\frac{9}{23}\\
b=\frac{4}{23}\rule{0mm}{7mm}\\
c=\frac{123}{23}\rule{0mm}{7mm}
\end{array}
\right.
\end{align*}
Donc $O\left(-\frac{9}{23},\frac{4}{23},\frac{123}{23}\right)$.
\textbullet~Déterminons l'orthocentre $H$. D'après la relation d'\textsc{Euler},

\begin{center}
$H=O+3\overrightarrow{OG}=\left(-\frac{9}{23},\frac{4}{23},\frac{123}{23}\right)+3\left(-\frac{9}{23}-\frac{5}{3},\frac{4}{23}+1,\frac{123}{23}-2\right)=\left(\frac{-151}{23},\frac{85}{23},\frac{354}{23}\right)$.
\end{center}
\textbullet~Déterminons le centre du cercle inscrit $I$. On sait que $I=\text{bar}\left\{A(a),B(b),C(c)\right\}$ où $a=BC=\sqrt{5^2+5^2+6^2}=\sqrt{86}$, $b=AC=\sqrt{3^2+1^2+0^2}=\sqrt{10}$ et $c=AB=\sqrt{2^2+4^2+6^2}=\sqrt{54}$. Donc

\begin{align*}\ensuremath
I&=\frac{\sqrt{86}}{\sqrt{86}+\sqrt{10}+\sqrt{54}}A
+\frac{\sqrt{10}}{\sqrt{86}+\sqrt{10}+\sqrt{54}}B+\frac{\sqrt{54}}{\sqrt{86}+\sqrt{10}+\sqrt{54}}C\\
 &=\left(
 \frac{2\sqrt{86}+4\sqrt{10}-\sqrt{54}}{\sqrt{86}+\sqrt{10}+\sqrt{54}},
 \frac{-2\sqrt{86}+2\sqrt{10}-3\sqrt{54}}{\sqrt{86}+\sqrt{10}+\sqrt{54}},
 \frac{6\sqrt{10}}{\sqrt{86}+\sqrt{10}+\sqrt{54}}
 \right).
\end{align*}

Dans $\Rr^3$ euclidien rapporté à un repère orthonormé, on donne $A(2,-2,0)$, $B(4,2,6)$ et 
$C(-1,-3,0)$. Déterminer l'orthocentre, le centre de gravité, les centres des cercles circonscrits et inscrits au triangle $(A,B,C)$.
\begin{center}
\shadowbox{
\begin{tabular}{c}
$G\left(\frac{5}{3},-1,2\right)$, $O\left(-\frac{9}{23},\frac{4}{23},\frac{123}{23}\right)$ et $H\left(\frac{-151}{23},\frac{85}{23},\frac{354}{23}\right)$ puis\\
$I\left(\frac{2\sqrt{86}+4\sqrt{10}-\sqrt{54}}{\sqrt{86}+\sqrt{10}+\sqrt{54}},\frac{-2\sqrt{86}+2\sqrt{10}-3\sqrt{54}}{\sqrt{86}+\sqrt{10}+\sqrt{54}},\frac{6\sqrt{10}}{\sqrt{86}+\sqrt{10}+\sqrt{54}}\right)$.\rule{0mm}{10mm}
\end{tabular}
}
\end{center}
\fincorrection
\correction{005514}
\textbullet~Déterminons un repère de $(D)$.
\begin{center}
$\left\{
\begin{array}{l}
x+y+z+1=0\\
2x+y+5z=2
\end{array}
\right.\Leftrightarrow\left\{
\begin{array}{l}
x+y=-1-z\\
2x+y=2-5z
\end{array}
\right.\Leftrightarrow\Leftrightarrow\left\{
\begin{array}{l}
x=3-4z\\
y=-4+3z
\end{array}
\right.$. 
\end{center}
Un repère de $(D)$ est $\left(A,\overrightarrow{u}\right)$ où $A(3,-4,0)$ et $\overrightarrow{u}(-4,3,1)$.
\textbullet~Soit $M(x,y,z)$ un point du plan. On sait que

\begin{center}
$d(A,(D))=\frac{\|\overrightarrow{AM}\wedge\overrightarrow{u}\|}{\|\overrightarrow{u}\|}=\frac{\sqrt{(y-3z+4)^2+(x+4z-3)^2+(3x+4y+7)^2}}{\sqrt{26}}$
\end{center}
\textbullet~Notons $\mathcal{C}$ le cylindre de révolution d'axe $(D)$ et de rayon $2$.
\begin{center}
$M(x,y,z)\in\mathcal{C}\Leftrightarrow d(A,(D))=2\Leftrightarrow(y-3z+4)^2+(x+4z-3)^2+(3x+4y+7)^2=104$
\end{center}

\begin{center}
\shadowbox{
\begin{tabular}{c}
Une équation cartésienne  du cylindre de révolution d'axe $(D)$ et de rayon $2$ est\\
$(y-3z+4)^2+(x+4z-3)^2+(3x+4y+7)^2=104$.
\end{tabular}
}
\end{center}
\fincorrection
\correction{005515}
\textbullet~Déterminons un repère de $(D)$.

\begin{center}
$\left\{
\begin{array}{l}
x+y+z+1=0\\
2x+y+5z=2
\end{array}
\right.\Leftrightarrow\left\{
\begin{array}{l}
x+y=-z-1\\
2x+y=-5z+2
\end{array}
\right.\Leftrightarrow\left\{
\begin{array}{l}
x=-4z+3\\
y=3z-4
\end{array}
\right.$
\end{center}
Un repère de $(D)$ est $\left(A,\overrightarrow{u}\right)$ où $A(3,-4,0)$ et $\overrightarrow{u}(-4,3,1)$.
\textbullet~Déterminons un repère de $(D')$.

\begin{center}
$\left\{
\begin{array}{l}
x+y+z=2\\
2x+y-5z=3
\end{array}
\right.\Leftrightarrow\left\{
\begin{array}{l}
x+y=-z+2\\
2x+y=5z+3
\end{array}
\right.\Leftrightarrow\left\{
\begin{array}{l}
x=6z+1\\
y=-7z+1
\end{array}
\right.$
\end{center}
Un repère de $(D')$ est $\left(A',\overrightarrow{u'}\right)$ où $A'(1,1,0)$ et $\overrightarrow{u'}(6,-7,1)$.
\textbullet~$\overrightarrow{u}\wedge\overrightarrow{u'}=\left(
\begin{array}{c}
-4\\
3\\
1
\end{array}
\right)\wedge\left(
\begin{array}{c}
6\\
-7\\
1
\end{array}
\right)=\left(
\begin{array}{c}
10\\
10\\
10
\end{array}
\right)\neq\overrightarrow{0}$.
\rule{0mm}{5mm}Puisque $\overrightarrow{u}$ et $\overrightarrow{u'}$ ne sont pas colinéaires, les droites $(D)$ et $(D')$ ne sont parallèles. Ceci assure l'unicité de la perpendiculaire commune  à $(D)$ et $(D')$.
\textbullet~On sait que la distance $d$ de $(D)$ à $(D')$ est donnée par

\begin{center}
$d=\frac{\text{abs}\left(\left[\overrightarrow{AA'},\overrightarrow{u},\overrightarrow{u'}\right]\right)}{\|\overrightarrow{u}\wedge\overrightarrow{u'}\|}$,
\end{center}
avec $[\overrightarrow{AA'},\overrightarrow{u},\overrightarrow{u'}]=\left|
\begin{array}{ccc}
-2&-4&6\\
5&3&-7\\
0&1&1
\end{array}
\right|=10\times(-2)+10\times5=30$ et donc $d=\frac{30}{10\sqrt{3}}=\sqrt{3}$.

\begin{center}
\shadowbox{
$d((D),(D'))=\sqrt{3}$.
}
\end{center}

\textbullet~Un système d'équations de la perpendiculaire commune est 
$\left\{
\begin{array}{l}
\left[\overrightarrow{AM},\overrightarrow{u},\overrightarrow{u}\wedge\overrightarrow{u'}\right]=0\\
\rule{0mm}{6mm}\left[\overrightarrow{A'M},\overrightarrow{u'},\overrightarrow{u}\wedge\overrightarrow{u'}\right]=0
\end{array}
\right.$. Or,

\begin{center}
$\frac{1}{10}\left[\overrightarrow{AM},\overrightarrow{u},\overrightarrow{u}\wedge\overrightarrow{u'}\right]=\left|
\begin{array}{ccc}
x-3&-4&1\\
y+4&3&1\\
z&1&1
\end{array}
\right|=2(x-3)+5(y+4)-7z=2x+5y-7z+14$,
\end{center}
et

\begin{center}
$\frac{1}{10}\left[\overrightarrow{A'M},\overrightarrow{u'},\overrightarrow{u}\wedge\overrightarrow{u'}\right]=\left|
\begin{array}{ccc}
x-1&6&1\\
y-1&-7&1\\
z&1&1
\end{array}
\right|=-8(x-1)-5(y-1)+13z=-8x-5y+13z+13$.
\end{center}
Donc

\begin{center}
\shadowbox{
\begin{tabular}{c}
un système d'équations cartésienne de la perpendiculaire commune à $(D)$ et $(D')$ est\\
$\left\{
\begin{array}{l}
2x+5y-7z=-14\\
8x+5y-13z=13
\end{array}
\right.$.
\end{tabular}
}
\end{center}
\fincorrection
\correction{005516}
$\overrightarrow{u}\in\overrightarrow{P_1}\cap\overrightarrow{P_2}\cap\overrightarrow{P_3}\Leftrightarrow\left\{
\begin{array}{l}
z-2y=0\\
2x-3z=0\\
3y-x=0
\end{array}
\right.\Leftrightarrow\left\{
\begin{array}{l}
x=3y\\
z=2y
\end{array}
\right.$.
Ainsi, les plans $(P_1)$, $(P_2)$ et $(P_3)$ sont tous trois parallèles à la droite affine $(D)$ d'équations $\left\{
\begin{array}{l}
x=3y\\
z=2y
\end{array}
\right.$. Ces plans définissent donc un prisme.
Déterminons alors l'aire d'une section droite. Le plan $(P)$ d'équation $3x+y+2z=0$ est perpendiculaire à la droite $(D)$. Son intersection avec les plans $(P_1)$, $(P_2)$ et $(P_3)$ définit donc une section droite du prisme.
\textbullet~Soit $M(x,y,z)$ un point de l'espace.

\begin{center}
$M\in(P_1)\cap(P_2)\cap(P)\Leftrightarrow
\left\{
\begin{array}{l}
z-2y=5\\
2x-3z=0\\
3x+y+2z=0
\end{array}
\right.\Leftrightarrow\left\{
\begin{array}{l}
y=\frac{z-5}{2}\\
x=\frac{3}{2}z\\
\frac{9}{2}z+\frac{z-5}{2}+2z=0
\end{array}
\right.\Leftrightarrow\left\{
\begin{array}{l}
z=\frac{5}{14}\\
y=-\frac{65}{28}\rule[-4mm]{0mm}{11mm}\\
x=\frac{15}{28}
\end{array}
\right.$
\end{center}
Notons $A\left(\frac{15}{28},-\frac{65}{28},\frac{5}{14}\right)$.
\textbullet~Soit $M(x,y,z)$ un point de l'espace.

\begin{center}
$M\in(P_1)\cap(P_3)\cap(P)\Leftrightarrow
\left\{
\begin{array}{l}
z-2y=5\\
3y-x=0\\
3x+y+2z=0
\end{array}
\right.\Leftrightarrow\left\{
\begin{array}{l}
z=2y+5\\
x=3y\\
9y+y+2(2y+5)=0
\end{array}
\right.\Leftrightarrow\left\{
\begin{array}{l}
y=-\frac{5}{7}\\
x=-\frac{15}{7}\rule[-4mm]{0mm}{11mm}\\
z=\frac{25}{7}
\end{array}
\right.$
\end{center}
Notons $B\left(-\frac{15}{7},-\frac{5}{7},\frac{25}{7}\right)$.

\textbullet~Soit $M(x,y,z)$ un point de l'espace.

\begin{center}
$M\in(P_2)\cap(P_3)\cap(P)\Leftrightarrow
\left\{
\begin{array}{l}
2x-3z=0\\
3y-x=0\\
3x+y+2z=0
\end{array}
\right.\Leftrightarrow x=y=z=0$
\end{center}
Une section droite est $OAB$ où $A\left(\frac{15}{28},-\frac{65}{28},\frac{5}{14}\right)$ et $B\left(-\frac{15}{7},-\frac{5}{7},\frac{25}{7}\right)$. De plus

\begin{align*}\ensuremath
\text{aire de}(OAB)&=\frac{1}{2}\left\|\overrightarrow{OA}\wedge\overrightarrow{OB}\right\|=\frac{1}{2}\times\frac{5}{28}\times\frac{5}{7}\left\|\left(
\begin{array}{c}
3\\
-13\\
2
\end{array}\right)
\wedge
\left(
\begin{array}{c}
-3\\
-1\\
5
\end{array}\right)\right\|=\frac{1}{2}\times\frac{5}{28}\times\frac{5}{7}\sqrt{63^2+21^2+42^2}\\
 &=\frac{1}{2}\times\frac{5}{28}\times\frac{5}{7}\times21\sqrt{3^2+1^2+2^2}=\frac{75}{4\sqrt{14}}
\end{align*}
\begin{center}
\shadowbox{
L'aire d'une section droite est $\frac{75}{4\sqrt{14}}$.
}
\end{center}
\fincorrection
\correction{005517}
Soient $(P)$ le plan d'équation $x+2y+2z=3$ et $(P')$ le plan d'équation $x+y=0$. L'angle entre $(P)$ et $(P')$ est l'angle entre les vecteurs normaux $\overrightarrow{n}(1,2,2)$ et $\overrightarrow{n'}(1,1,0)$ :

\begin{center}
$\left(\widehat{\overrightarrow{n},\overrightarrow{n'}}\right)=\Arccos\left(\frac{\overrightarrow{n}.\overrightarrow{n'}}{\|\overrightarrow{n}\|\|\overrightarrow{n'}\|}\right)=\Arccos\left(\frac{3}{3\sqrt{2}}\right)=\Arccos\left(\frac{1}{\sqrt{2}}\right)=\frac{\pi}{4}$.
\end{center}
\fincorrection
\correction{005518}
Soit $M(x,y,z)$ un point de l'espace. On a 
\begin{center}
$d(M,(P_1))=\frac{\left|4x+4y-7z-1\right|}{\sqrt{4^2+4^2+7^2}}=\frac{\left|4x+4y-7z-1\right|}{9}$ et $d(M,(P_2))=\frac{\left|8x-4y+z+7\right|}{\sqrt{8^2+4^2+1^2}}=\frac{\left|8x-4y+z+7\right|}{9}$.
\end{center}
Par suite,

\begin{align*}\ensuremath
d(M,(P_1))=d(M,(P_2)&\Leftrightarrow|4x+4y-7z-1|=|8x-4y+z+7|\Leftrightarrow(4x+4y-7z-1)^2=(8x-4y+z+7)^2\\
 &\Leftrightarrow\left((4x+4y-7z-1)-(8x-4y+z+7)\right)\left((4x+4y-7z-1)+(8x-4y+z+7)\right)=0\\
 &\Leftrightarrow(-4x+8y-8z-8)(12x-6z+6)=0\Leftrightarrow x-2y+2z+2=0\;\text{ou}\;2x-z+1=0.
\end{align*}
\begin{center}
\shadowbox{
Les plans bissecteurs de $(P_1)$ et $(P_2)$ admettent pour équation cartésienne $x-2y+2z+2=0$ et $2x-z+1=0$.
}
\end{center}
\fincorrection
\correction{005519}
\textbullet~Déterminons un repère de $(D)$.

\begin{center}
$\left\{
\begin{array}{l}
x+y-3z+4=0\\
2x-z+1=0
\end{array}
\right.\Leftrightarrow\left\{
\begin{array}{l}
y-3z=-x-4\\
z=2x+1
\end{array}
\right.\Leftrightarrow\left\{
\begin{array}{l}
y=5x-1\\
z=2x+1
\end{array}
\right.$
\end{center}
Un repère de $(D)$ est $\left(A,\overrightarrow{u}\right)$ où $A(0,-1,1)$ et $\overrightarrow{u}(1,5,2)$.
\textbullet~Puisque un système d'équations de $(D')$ est $\left\{
\begin{array}{l}
x=z-1\\
y=z-1
\end{array}
\right.$, un repère de $(D')$ est $\left(A',\overrightarrow{u'}\right)$ où $A'(-1,-1,0)$ et $\overrightarrow{u'}(1,1,1)$.
\textbullet~$\overrightarrow{u}\wedge\overrightarrow{u'}=\left(
\begin{array}{c}
1\\
5\\
2
\end{array}
\right)\wedge\left(
\begin{array}{c}
1\\
1\\
1
\end{array}
\right)=\left(
\begin{array}{c}
3\\
1\\
-4
\end{array}
\right)\neq\overrightarrow{0}$.
\rule{0mm}{5mm}Puisque $\overrightarrow{u}$ et $\overrightarrow{u'}$ ne sont pas colinéaires, les droites $(D)$ et $(D')$ ne sont parallèles. Ceci assure l'unicité de la perpendiculaire commune  à $(D)$ et $(D')$.

\textbullet~Un système d'équations de la perpendiculaire commune est 
$\left\{
\begin{array}{l}
\left[\overrightarrow{AM},\overrightarrow{u},\overrightarrow{u}\wedge\overrightarrow{u'}\right]=0\\
\rule{0mm}{6mm}\left[\overrightarrow{A'M},\overrightarrow{u'},\overrightarrow{u}\wedge\overrightarrow{u'}\right]=0
\end{array}
\right.$. Or,

\begin{center}
$\left[\overrightarrow{AM},\overrightarrow{u},\overrightarrow{u}\wedge\overrightarrow{u'}\right]=\left|
\begin{array}{ccc}
x&1&3\\
y+1&5&1\\
z-1&2&-4
\end{array}
\right|=-22x+10(y+1)-14(z-1)=-22x+10y-14z+24$,
\end{center}
et

\begin{center}
$\left[\overrightarrow{A'M},\overrightarrow{u'},\overrightarrow{u}\wedge\overrightarrow{u'}\right]=\left|
\begin{array}{ccc}
x+1&1&3\\
y+1&1&1\\
z&1&-4
\end{array}
\right|=-5(x+1)+7(y+1)-2z=-5x+7y-2z+2$.
\end{center}
Donc

\begin{center}
\shadowbox{
\begin{tabular}{c}
un système d'équations cartésienne de la perpendiculaire commune à $(D)$ et $(D')$ est\\
$\left\{
\begin{array}{l}
11x-5y+7z=12\\
5x-7y+2z=2
\end{array}
\right.$.
\end{tabular}
}
\end{center}

\fincorrection
\correction{005520}
Notons $p$ la projection orthogonale sur $(P)$.
Un repère de $(D)$ est $\left(A,\overrightarrow{u}\right)$ où $A(0,-1,2)$ et $\overrightarrow{u}(1,2,-3)$. Un vecteur normal à $(P)$ est $\overrightarrow{n}(1,3,2)$. $\overrightarrow{u}$ et $\overrightarrow{n}$ ne sont pas colinéaires et donc $p(D)$ est une droite du plan $(P)$.
Plus précisément, $p(D)$ est l'intersection du plan $(P)$ et du plan $(P')$ contenant $(D)$ et perpendiculaire à $(P)$. Un repère de $(P')$ est $\left(A,\overrightarrow{u},\overrightarrow{n}\right)$. Donc

\begin{align*}\ensuremath
M(x,y,z)\in(P')&\Leftrightarrow\left|
\begin{array}{ccc}
x&1&1\\
y+1&2&3\\
z-2&-3&2
\end{array}
\right|=0\Leftrightarrow13x-5(y+1)+(z-2)=0\Leftrightarrow13x-5y+z=7.
\end{align*}

\begin{center}
\shadowbox{
La projetée orthogonale de $(D)$ sur $(P)$ est la droite d'équations $\left\{
\begin{array}{l}
13x-5y+z=7\\
x+3y+2z=6
\end{array}
\right.$.
}
\end{center}
\fincorrection
\correction{005521}



%$$\includegraphics{../images/img005521-1}$$


\textbf{Angle entre deux arêtes.} Les faces du tétraèdre $ABCD$ sont des triangles équilatéraux et donc l'angle entre deux arêtes est $60^\circ$.

%$$\includegraphics{../images/img005521-2}$$

\textbf{Angle entre une  arête et une face.} C'est l'angle $\widehat{CDI}$ de la figure ci-dessus.

\begin{center}
$\widehat{CDI}=\Arccos\left(\frac{HD}{DI}\right)=\Arccos\left(\frac{a/2}{a\sqrt{3}/2}\right)=\Arccos\left(\frac{1}{\sqrt{3}}\right)=54,7\ldots^\circ$.
\end{center}
\textbf{Angle entre deux faces.} C'est l'angle $\widehat{CID}$ de la figure ci-dessus.

\begin{center}
$\widehat{CID}=\pi-2\widehat{CDI}=2\left(\frac{\pi}{2}-\Arccos\left(\frac{1}{\sqrt{3}}\right)\right)=2\Arcsin\left(\frac{1}{\sqrt{3}}\right)=70,5\ldots^\circ$.
\end{center}
\fincorrection
\correction{005522}
Déterminons un repère de $(D)$.

\begin{center}
$\left\{
\begin{array}{l}
x-y-z=0\\
x+2y-z=10
\end{array}
\right.\Leftrightarrow\left\{
\begin{array}{l}
x-z=y\\
y+2y=10
\end{array}
\right.\Leftrightarrow\left\{
\begin{array}{l}
y=\frac{10}{3}\\
z=x-\frac{10}{3}
\end{array}
\right.$.
\end{center}
Un repère de $(D)$ est $\left(A,\overrightarrow{u}\right)$ où $A\left(\frac{10}{3},\frac{10}{3},0\right)$ et $\overrightarrow{u}(1,0,1)$.
On sait alors que

\begin{center}
$d(O,(D))=\frac{\|\overrightarrow{AO}\wedge\overrightarrow{u}\|}{\|\overrightarrow{u}\|}=\frac{1}{\sqrt{2}}\times\frac{10}{3}\left\|
\left(
\begin{array}{c}
1\\
1\\
0
\end{array}
\right)\wedge\left(
\begin{array}{c}
1\\
0\\
1
\end{array}
\right)\right\|=\frac{10}{\sqrt{6}}
$.
\end{center}
\begin{center}
\shadowbox{
$d(O,(D))=\frac{10}{\sqrt{6}}$.
}
\end{center}
\fincorrection


\end{document}

