\documentclass[a4paper,11pt]{article}

\usepackage{inputenc}
\usepackage[T1]{fontenc}
\usepackage[frenchb]{babel}
\usepackage{fancyhdr,fancybox} % pour personnaliser les en-têtes
\usepackage{lastpage,setspace}
\usepackage{amsfonts,amssymb,amsmath,amsthm,mathrsfs}
\usepackage{mathdots}
\usepackage{relsize,exscale,bbold}
\usepackage{paralist}
\usepackage{xspace,multicol,diagbox,array}
\usepackage{xcolor}
\usepackage{variations}
\usepackage{xypic}
\usepackage{eurosym,stmaryrd}
\usepackage{graphicx}
\usepackage[np]{numprint}
\usepackage{hyperref} 
\usepackage{tikz}
\usepackage{colortbl}
\usepackage{multirow}
\usepackage{MnSymbol,wasysym}
\usepackage[top=1.5cm,bottom=1.5cm,right=1.2cm,left=1.5cm]{geometry}
\usetikzlibrary{calc, arrows, plotmarks, babel,decorations.pathreplacing}
\setstretch{1.25}
%\usepackage{lipsum} %\usepackage{enumitem} %\setlist[enumerate]{itemsep=1mm} bug avec enumerate



\newtheorem{thm}{Théorème}
\newtheorem{rmq}{Remarque}
\newtheorem{prop}{Propriété}
\newtheorem{cor}{Corollaire}
\newtheorem{lem}{Lemme}
\newtheorem{prop-def}{Propriété-définition}

\theoremstyle{definition}

\newtheorem{defi}{Définition}
\newtheorem{ex}{Exemple}
\newtheorem*{rap}{Rappel}
\newtheorem{cex}{Contre-exemple}
\newtheorem{exo}{Exercice} % \large {\fontfamily{ptm}\selectfont EXERCICE}
\newtheorem{nota}{Notation}
\newtheorem{ax}{Axiome}
\newtheorem{appl}{Application}
\newtheorem{csq}{Conséquence}
\def\di{\displaystyle}



\renewcommand{\thesection}{\Roman{section}}\renewcommand{\thesubsection}{\arabic{subsection} }\renewcommand{\thesubsubsection}{\alph{subsubsection} }


\newcommand{\bas}{~\backslash}\newcommand{\ba}{\backslash}\newcommand{\disp}{\displaystyle}
\newcommand{\C}{\mathbb{C}}\newcommand{\K}{\mathbb{K}}\newcommand{\R}{\mathbb{R}}\newcommand{\Q}{\mathbb{Q}}\newcommand{\Z}{\mathbb{Z}}\newcommand{\N}{\mathbb{N}}\newcommand{\V}{\overrightarrow}\newcommand{\Cs}{\mathscr{C}}\newcommand{\Ps}{\mathscr{P}}\newcommand{\Rs}{\mathscr{R}}\newcommand{\Gs}{\mathscr{G}}\newcommand{\Ds}{\mathscr{D}}\newcommand{\happy}{\huge\smiley}\newcommand{\sad}{\huge\frownie}\newcommand{\danger}{\begin{tikzpicture}[x=1.5pt,y=1.5pt,rotate=-14.2]
	\definecolor{myred}{rgb}{1,0.215686,0}
	\draw[line width=0.1pt,fill=myred] (13.074200,4.937500)--(5.085940,14.085900)..controls (5.085940,14.085900) and (4.070310,15.429700)..(3.636720,13.773400)
	..controls (3.203130,12.113300) and (0.917969,2.382810)..(0.917969,2.382810)
	..controls (0.917969,2.382810) and (0.621094,0.992188)..(2.097660,1.359380)
	..controls (3.574220,1.726560) and (12.468800,3.984380)..(12.468800,3.984380)
	..controls (12.468800,3.984380) and (13.437500,4.132810)..(13.074200,4.937500)
	--cycle;
	\draw[line width=0.1pt,fill=white] (11.078100,5.511720)--(5.406250,11.875000)..controls (5.406250,11.875000) and (4.683590,12.812500)..(4.367190,11.648400)
	..controls (4.050780,10.488300) and (2.375000,3.675780)..(2.375000,3.675780)
	..controls (2.375000,3.675780) and (2.156250,2.703130)..(3.214840,2.964840)
	..controls (4.273440,3.230470) and (10.640600,4.847660)..(10.640600,4.847660)
	..controls (10.640600,4.847660) and (11.332000,4.953130)..(11.078100,5.511720)
	--cycle;
	\fill (6.144520,8.839900)..controls (6.460940,7.558590) and (6.464840,6.457090)..(6.152340,6.378910)
	..controls (5.835930,6.300840) and (5.320300,7.277400)..(5.003900,8.554750)
	..controls (4.683590,9.835940) and (4.679690,10.941400)..(4.996090,11.019600)
	..controls (5.312490,11.097700) and (5.824210,10.121100)..(6.144520,8.839900)
	--cycle;
	\fill (7.292960,5.261780)..controls (7.382800,4.898500) and (7.128900,4.523500)..(6.730460,4.421880)
	..controls (6.328120,4.324220) and (5.929680,4.535220)..(5.835930,4.898500)
	..controls (5.746080,5.261780) and (5.999990,5.640630)..(6.402340,5.738340)
	..controls (6.804690,5.839840) and (7.203110,5.625060)..(7.292960,5.261780)
	--cycle;
	\end{tikzpicture}}\newcommand{\alors}{\Large\Rightarrow}\newcommand{\equi}{\Leftrightarrow}
\newcommand{\fonction}[5]{\begin{array}{l|rcl}
		#1: & #2 & \longrightarrow & #3 \\
		& #4 & \longmapsto & #5 \end{array}}


\definecolor{vert}{RGB}{11,160,78}
\definecolor{rouge}{RGB}{255,120,120}
\definecolor{bleu}{RGB}{15,5,107}



\pagestyle{fancy}
\lhead{Groupe IPESUP}\chead{}\rhead{Année~2022-2023}\lfoot{M. Botcazou \& M.Dupré}\cfoot{\thepage/3}\rfoot{PCSI }\renewcommand{\headrulewidth}{0.4pt}\renewcommand{\footrulewidth}{0.4pt}


\begin{document}
 %%%%BIBMATH%%%%
 
%(1) https://www.bibmath.net/ressources/index.php?action=affiche&quoi=bde/proba/denombrement&type=fexo


%(2) https://www.bibmath.net/ressources/index.php?action=affiche&quoi=bde/proba/denombrement-binomiaux&type=fexo
 
 
%(3) https://www.bibmath.net/ressources/index.php?action=affiche&quoi=bde/proba/denombrement-theo&type=fexo


\noindent\shadowbox{
	\begin{minipage}{1\linewidth}
		\centering
		\huge{\textbf{ TD 17 : Dénombrement }}
	\end{minipage}}

\smallskip
\section*{Connaître son cours:}
\begin{itemize}[$\bullet$]
	\item Que signifie qu'un ensemble E est fini. Montrer que toute partie de l'ensemble $\llbracket1,n\rrbracket$ pour $n\in\N^*$ est finie. En déduire que toute partie d’un ensemble fini est finie.
	\item Soit $p , q \in \N^*,$ montrer qu'il existe une injection de $\llbracket1,p\rrbracket$ dans $\llbracket1,q\rrbracket$ si, et seulement si, $p \leq q $.
	\item Soit $E$ et $F$ deux ensembles finis. Alors, $E \cup F$ est fini et $\displaystyle |E \cup F | = |E | + |F | - |E \cap F |$. En déduire le nombre d'entiers à, au plus, quatre chiffres qui ne sont divisibles ni par $3$ ni par $5$.
	\item Soit $E$ et $F$ deux ensembles finis de cardinal $n$ et $p$ respectivement, et $f : E \to F $. S'il existe un entier $k \in \N^*$ tel que $n > k p $, montrer qu'il existe un élément $y \in F$ tel que $| f^{-1}(\{y \})| > k $.
	\item Donner la formule de Vandermonde pour tous $a , b , n \in \N$.

\end{itemize}
\raggedright

\section*{Dénombrements pratiques:}\hfill\\%[-0.25cm]

   
\begin{minipage}{1\linewidth}\begin{minipage}[t]{0.48\linewidth}\raggedright
	
\begin{exo}\textbf{(*)}\quad\\[0.2cm]
Combien y a-t-il de nombres de $5$ chiffres où $0$ figure une fois et une seule~?
	
\centering\rule{1\linewidth}{0.6pt}\end{exo}


\begin{exo}\textbf{(*)}\quad\\[0.2cm]
On part du point de coordonnées $(0,0)$ pour rejoindre le point de coordonnées $(p,q)$

($p$ et $q$ entiers naturels donnés) en se déplaçant à chaque étape d'une unité vers la droite ou vers le haut. Combien y a-t-il de chemins possibles~?

\centering\rule{1\linewidth}{0.6pt}\end{exo}


\begin{exo}\textbf{(*)}\quad\\[0.2cm]
Dénombrer les anagrammes des mots suivants : IPESUP , SUCCES , ANANAS.

\centering\rule{1\linewidth}{0.6pt}\end{exo}

\begin{exo}\textbf{(**)}\quad\\[0.2cm]
Quelle est la probabilité $p_n$ pour que dans un groupe de $n$ personnes choisies au hasard, deux personnes au moins aient le même anniversaire 

(\textit{on supposera que l'année a toujours $365$ jours, tous équiprobables}). Montrer que pour $n\geq23$, on a $p_n\geq\frac{1}{2}$.

\centering\rule{1\linewidth}{0.6pt}\end{exo}


%%%%%%%%%%%%%%%%%%%%%%%%%%%%%%%%%%%%%%%%%%%%%%%%%%%%%%%%%%%%%%%%%%%%%%%%%%%%%%%%%%%%%%%%%%
\end{minipage}\hfill\vrule\hfill\begin{minipage}[t]{0.48\linewidth}\raggedright
%%%%%%%%%%%%%%%%%%%%%%%%%%%%%%%%%%%%%%%%%%%%%%%%%%%%%%%%%%%%%%%%%%%%%%%%%%%%%%%%%%%%%%%%%%

\begin{exo}\textbf{(*)}\quad\\[0.2cm]
Dans une ville, il y a quatre boulangeries qui ferment un jour par semaine.
\begin{enumerate}
	\item Déterminer le nombre de façons d'attribuer un jour de fermeture hebdomadaire?
	\item Reprendre la même question si plusieurs boulangeries ne peuvent fermer le même jour.
	\item Reprendre la même question si chaque jour, il doit y avoir au moins une boulangerie ouverte.
\end{enumerate}

\centering\rule{1\linewidth}{0.6pt}\end{exo}

\begin{exo}\textbf{(*)}\quad\\[0.2cm]
On considère un ensemble $X$ de $n+1$ entiers (distincts) choisis dans $\{1,\dots,2n\}$. Démontrer que parmi les éléments de $X$, on peut toujours trouver $2$ entiers dont la somme fait $2n+1$.

\centering\rule{1\linewidth}{0.6pt}\end{exo}


\begin{exo}\textbf{(**)}\quad\\[0.2cm]
Dans une entreprise, il y a 800 employés. 300 sont des hommes, 352 sont membres d'un syndicat, 424 sont mariés, 188 sont des hommes syndiqués, 166 sont des hommes mariés, 208 sont syndiqués et mariés, 144 sont des hommes mariés syndiqués. Combien y-a-t-il de femmes célibataires non syndiquées?

\centering\rule{1\linewidth}{0.6pt}\end{exo}



\end{minipage}\end{minipage} \newpage


\begin{minipage}{1\linewidth}\begin{minipage}[t]{0.48\linewidth}\raggedright

\begin{exo}\textbf{(**)}\quad\\[0.2cm]
Une main au poker est formée de 5 cartes extraites d'un jeu de 52 cartes. Traditionnellement,trèfle, carreau, coeur, pique sont appelées couleurs et les valeurs des cartes sont rangées dans l'ordre : as, roi, dame, valet, 10, 9, 8, 7, 6, 5, 4, 3, 2, de la plus forte à la plus faible. Dénombrer les mains suivantes :
\begin{enumerate}
	\item quinte flush : main formée de 5 cartes consécutives de la même couleur (la suite as, 2, 3, 4 et 5 est une quinte flush).
	\item carré : main contenant 4 cartes de la même valeur (4 as par exemple).
	\item full : main formée de 3 cartes de la même valeur et de deux autres cartes de même valeur (par exemple, 3 as et 2 rois).
	\item quinte : main formée de 5 cartes consécutives et qui ne sont pas toutes de la même couleur.
	\item brelan : main comprenant 3 cartes de même valeur et qui n'est ni un carré, ni un full (par exemple, 3 as, 1 valet, 1 dix).
\end{enumerate}

\centering\rule{1\linewidth}{0.6pt}\end{exo}



\begin{exo}\textbf{(***)}\quad\\[0.2cm]
De combien de façons peut-on payer $100$ euros avec des pièces de $10$, $20$ et $50$ centimes~?

\centering\rule{1\linewidth}{0.6pt}\end{exo}




%%%%%%%%%%%%%%%%%%%%%%%%%%%%%%%%%%%%%%%%%%%%%%%%%%%%%%%%%%%%%%%%%%%%%%%%%%%%%%%%%%%%%%%%%%
\end{minipage}\hfill\vrule\hfill\begin{minipage}[t]{0.48\linewidth}\raggedright
%%%%%%%%%%%%%%%%%%%%%%%%%%%%%%%%%%%%%%%%%%%%%%%%%%%%%%%%%%%%%%%%%%%%%%%%%%%%%%%%%%%%%%%%%%

\begin{exo}\textbf{(**)}\quad\\[0.2cm]
Soit $A$ l'ensemble des nombres à 7 chiffres ne comportant aucun "1". Déterminer le nombre d'éléments des ensembles suivants :
\begin{enumerate}
	\item $A$.
	\item $A_1$, ensemble des nombres de $A$ ayant 7 chiffres différents.
	\item $A_2$, ensemble des nombres pairs de $A$.
	\item $A_3$, ensemble des nombres de $A$ dont les chiffres forment une suite strictement croissante (dans l'ordre où ils sont écrits).
\end{enumerate}

\centering\rule{1\linewidth}{0.6pt}\end{exo}


\begin{exo}\textbf{(****)}\quad\textit{(Le problème des chapeaux)}\\[0.2cm]
Soit $n\in\N$, considérons $n$ personnes qui laissent leur chapeau à un vestiaire. En repartant, chaque personne reprend un chapeau au hasard. Montrer que la probabilité qu'aucune de ces personnes n'ait repris son propre chapeau est environ $\frac{1}{e}$ quand $n$ est grand.

\textit{(Indications:)}
\begin{enumerate}
	\item Rappeler la « formule du crible ». 
	\item Faire le lien avec l'ensemble des permutation à $n$ éléments noté $\mathcal S_n$. 
	
\end{enumerate}

\centering\rule{1\linewidth}{0.6pt}\end{exo}


\end{minipage}\end{minipage} 

\section*{Dénombrements théoriques:}\hfill\\%[-0.25cm]


\begin{minipage}{1\linewidth}\begin{minipage}[t]{0.48\linewidth}\raggedright

\begin{exo}\textbf{(**)}\quad\\[0.2cm]
Soit $n \in \N^*$ et $E$ un ensemble de cardinal $n $. Dénombrer les couples $(X , Y ) \in \mathcal P (E )^2$ tels que $X \subset Y $.	
	
	
	\centering\rule{1\linewidth}{0.6pt}\end{exo}



\begin{exo}\textbf{(**)}\quad\\[0.2cm]
Dénombrer les permutations de $\mathcal S_{20}$ dont la décomposition en cycle de supports disjoints contient trois 4-cycles, deux 3-cycles et deux points fixes.
	
\centering\rule{1\linewidth}{0.6pt}\end{exo}




%%%%%%%%%%%%%%%%%%%%%%%%%%%%%%%%%%%%%%%%%%%%%%%%%%%%%%%%%%%%%%%%%%%%%%%%%%%%%%%%%%%%%%%%%%
\end{minipage}\hfill\vrule\hfill\begin{minipage}[t]{0.48\linewidth}\raggedright
%%%%%%%%%%%%%%%%%%%%%%%%%%%%%%%%%%%%%%%%%%%%%%%%%%%%%%%%%%%%%%%%%%%%%%%%%%%%%%%%%%%%%%%%%%

\begin{exo}\textbf{(**)}\quad\\[0.2cm]
Montrer que, pour tout $n \geq 2$,
$$\disp \sum_{X \in \mathcal{P} (\llbracket 1,n \rrbracket) }\sum_{k \in X}k = n (n + 1)2^{n-2}$$	
	
	\centering\rule{1\linewidth}{0.6pt}\end{exo}


\begin{exo}\textbf{(**)}\quad\\[0.2cm]
Démontrer par un dénombrement que, pour $n\geq 1$, on a :
$$\binom{2n}{n}=\sum_{k=0}^n \binom{n}{k}^2.$$	
	
	\centering\rule{1\linewidth}{0.6pt}\end{exo}


\end{minipage}\end{minipage}\newpage

\hfill\\[2cm]

\begin{minipage}{1\linewidth}\begin{minipage}[c]{0.48\linewidth}\raggedright
		
		\begin{exo}\textbf{(**)}\quad\\[0.2cm]
			Soit $n,p\geq 1$ deux entiers.
			\begin{enumerate}
				\item Combien y-a-t-il de fonctions strictement croissantes de $\{1,\dots,p\}$ dans $\{1,\dots,n\}$?
				\item \begin{enumerate}
					\item Soit $f:\{1,\dots,p\}\to\{1,\dots,n\}$ une fonction croissante. On pose $\phi(f)$ la fonction définie sur $\{1,\dots,p\}$, à valeurs dans $\{1,\dots,n+p-1\}$, par $\phi(f)(k)=f(k)+k-1$. Démontrer que $\phi(f)$ est strictement croissante.
					\item Soit $g:\{1,\dots,p\}\to\{1,\dots,n+p-1\}$ une fonction strictement croissante. On pose $\psi(g)$ la fonction définie sur $\{1,\dots,p\}$, à valeurs dans $\{1,\dots,n\}$, par $\psi(g)(k)=g(k)-k+1$. Démontrer que $\psi(g)$ est croissante.
					\item Combien y-a-t-il de fonctions croissantes de $\{1,\dots,p\}$ dans $\{1,\dots,n\}$?
				\end{enumerate}
			\end{enumerate}\centering\rule{1\linewidth}{0.6pt}\end{exo}
		
				\begin{exo}\textbf{(**)}\quad\\[0.2cm]
			Pour $n\in\mathbb N$, on note $B_n$ le nombre de partitions d'un ensemble $E$ de cardinal $n$. On pose $B_0=1$.
			\begin{enumerate}
				\item Calculer $B_1$, $B_2$ et $B_3$.
				\item \'Etablir la formule de récurrence 
				$$B_{n+1}=\sum_{k=0}^n \binom nk B_k.$$
			\end{enumerate}
			
			\centering\rule{1\linewidth}{0.6pt}\end{exo}
		
		
		%%%%%%%%%%%%%%%%%%%%%%%%%%%%%%%%%%%%%%%%%%%%%%%%%%%%%%%%%%%%%%%%%%%%%%%%%%%%%%%%%%%%%%%%%%
	\end{minipage}\hfill\vrule\hfill\begin{minipage}[c ]{0.48\linewidth}\raggedright
		%%%%%%%%%%%%%%%%%%%%%%%%%%%%%%%%%%%%%%%%%%%%%%%%%%%%%%%%%%%%%%%%%%%%%%%%%%%%%%%%%%%%%%%%%%
		
			

		
		\begin{exo}\textbf{(**)}\quad\\[0.2cm]
			Soit $n,p$ des entiers naturels avec $n\geq p$. Démontrer par dénombrement que
			$$\sum_{k=p}^n \dbinom{k}{p}=\dbinom{n+1}{p+1}.$$	
			
			
			\centering\rule{1\linewidth}{0.6pt}\end{exo}
		
		
		\begin{exo}\textbf{(**)}\quad\\[0.2cm]
			Soit $E$ un ensemble fini de cardinal $n\geq 1$. Démontrer que
			le nombre de parties de $E$ de cardinal pair vaut $2^{n-1}$.
			
			\centering\rule{1\linewidth}{0.6pt}\end{exo}
		
		
		
		\begin{exo}\textbf{(***)}\quad\\[0.2cm]
			Pour $n\in\mathbb N^*$ et $p\in\mathbb N$, on note $\Gamma_n^p$ le nombre de $n$-uplets $(x_1,\dots,x_n)\in\mathbb N^n$ tels que $x_1+\dots+x_n=p$. 
			\begin{enumerate}
				\item Déterminer $\Gamma_n^0$, $\Gamma_n^1$, $\Gamma_n^2$, $\Gamma_1^p$ et $\Gamma_2^p$.
				\item Démontrer que, pour tout $n\in\mathbb N^*$, pour tout $p\in\mathbb N$, 
				$$\Gamma_{n+1}^p=\Gamma_n^0+\Gamma_n^1+\dots+\Gamma_n^p.$$
				\item En déduire que, pour tout $n\in\mathbb N^*$ et tout $p\in\mathbb N$, 
				$$\Gamma_n^p=\binom{n+p-1}p.$$
			\end{enumerate}
			
			\centering\rule{1\linewidth}{0.6pt}\end{exo}

\end{minipage}\end{minipage} 

\end{document}