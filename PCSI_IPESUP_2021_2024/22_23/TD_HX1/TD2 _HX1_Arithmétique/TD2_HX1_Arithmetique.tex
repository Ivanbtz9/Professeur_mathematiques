\documentclass[a4paper,11pt]{article}

\usepackage{inputenc}
\usepackage[T1]{fontenc}
\usepackage[frenchb]{babel}
\usepackage{fancyhdr,fancybox} % pour personnaliser les en-têtes
\usepackage{lastpage,setspace}
\usepackage{amsfonts,amssymb,amsmath,amsthm,mathrsfs}
\usepackage{relsize,exscale,bbold}
\usepackage{paralist}
\usepackage{xspace,multicol,diagbox,array}
\usepackage{xcolor}
\usepackage{variations}
\usepackage{xypic}
\usepackage{eurosym,stmaryrd}
\usepackage{graphicx}
\usepackage[np]{numprint}
\usepackage{hyperref} 
\usepackage{tikz}
\usepackage{colortbl}
\usepackage{multirow}
\usepackage{MnSymbol,wasysym}
\usepackage[top=1.5cm,bottom=1.5cm,right=1.2cm,left=1.5cm]{geometry}
\usetikzlibrary{calc, arrows, plotmarks, babel,decorations.pathreplacing}
\setstretch{1.25}
%\usepackage{lipsum} %\usepackage{enumitem} %\setlist[enumerate]{itemsep=1mm} bug avec enumerate



\newtheorem{thm}{Théorème}
\newtheorem{rmq}{Remarque}
\newtheorem{prop}{Propriété}
\newtheorem{cor}{Corollaire}
\newtheorem{lem}{Lemme}
\newtheorem{prop-def}{Propriété-définition}

\theoremstyle{definition}

\newtheorem{defi}{Définition}
\newtheorem{ex}{Exemple}
\newtheorem*{rap}{Rappel}
\newtheorem{cex}{Contre-exemple}
\newtheorem{exo}{Exercice} % \large {\fontfamily{ptm}\selectfont EXERCICE}
\newtheorem{nota}{Notation}
\newtheorem{ax}{Axiome}
\newtheorem{appl}{Application}
\newtheorem{csq}{Conséquence}
\def\di{\displaystyle}



\renewcommand{\thesection}{\Roman{section}}\renewcommand{\thesubsection}{\arabic{subsection} }\renewcommand{\thesubsubsection}{\alph{subsubsection} }


\newcommand{\bas}{~\backslash}\newcommand{\ba}{\backslash}
\newcommand{\C}{\mathbb{C}}\newcommand{\R}{\mathbb{R}}\newcommand{\Q}{\mathbb{Q}}\newcommand{\Z}{\mathbb{Z}}\newcommand{\N}{\mathbb{N}}\newcommand{\V}{\overrightarrow}\newcommand{\Cs}{\mathscr{C}}\newcommand{\Ps}{\mathscr{P}}\newcommand{\Rs}{\mathscr{R}}\newcommand{\Gs}{\mathscr{G}}\newcommand{\Ds}{\mathscr{D}}\newcommand{\happy}{\huge\smiley}\newcommand{\sad}{\huge\frownie}\newcommand{\danger}{\begin{tikzpicture}[x=1.5pt,y=1.5pt,rotate=-14.2]
	\definecolor{myred}{rgb}{1,0.215686,0}
	\draw[line width=0.1pt,fill=myred] (13.074200,4.937500)--(5.085940,14.085900)..controls (5.085940,14.085900) and (4.070310,15.429700)..(3.636720,13.773400)
	..controls (3.203130,12.113300) and (0.917969,2.382810)..(0.917969,2.382810)
	..controls (0.917969,2.382810) and (0.621094,0.992188)..(2.097660,1.359380)
	..controls (3.574220,1.726560) and (12.468800,3.984380)..(12.468800,3.984380)
	..controls (12.468800,3.984380) and (13.437500,4.132810)..(13.074200,4.937500)
	--cycle;
	\draw[line width=0.1pt,fill=white] (11.078100,5.511720)--(5.406250,11.875000)..controls (5.406250,11.875000) and (4.683590,12.812500)..(4.367190,11.648400)
	..controls (4.050780,10.488300) and (2.375000,3.675780)..(2.375000,3.675780)
	..controls (2.375000,3.675780) and (2.156250,2.703130)..(3.214840,2.964840)
	..controls (4.273440,3.230470) and (10.640600,4.847660)..(10.640600,4.847660)
	..controls (10.640600,4.847660) and (11.332000,4.953130)..(11.078100,5.511720)
	--cycle;
	\fill (6.144520,8.839900)..controls (6.460940,7.558590) and (6.464840,6.457090)..(6.152340,6.378910)
	..controls (5.835930,6.300840) and (5.320300,7.277400)..(5.003900,8.554750)
	..controls (4.683590,9.835940) and (4.679690,10.941400)..(4.996090,11.019600)
	..controls (5.312490,11.097700) and (5.824210,10.121100)..(6.144520,8.839900)
	--cycle;
	\fill (7.292960,5.261780)..controls (7.382800,4.898500) and (7.128900,4.523500)..(6.730460,4.421880)
	..controls (6.328120,4.324220) and (5.929680,4.535220)..(5.835930,4.898500)
	..controls (5.746080,5.261780) and (5.999990,5.640630)..(6.402340,5.738340)
	..controls (6.804690,5.839840) and (7.203110,5.625060)..(7.292960,5.261780)
	--cycle;
	\end{tikzpicture}}\newcommand{\alors}{\Large\Rightarrow}\newcommand{\equi}{\Leftrightarrow}
\newcommand{\fonction}[5]{\begin{array}{l|rcl}
		#1: & #2 & \longrightarrow & #3 \\
		& #4 & \longmapsto & #5 \end{array}}


\definecolor{vert}{RGB}{11,160,78}
\definecolor{rouge}{RGB}{255,120,120}
\definecolor{bleu}{RGB}{15,5,107}



\pagestyle{fancy}
\lhead{Groupe IPESUP}\chead{}\rhead{Année~2022-2023}\lfoot{M. Botcazou \& M.Dupré}\cfoot{\thepage/2}\rfoot{PCSI }\renewcommand{\headrulewidth}{0.4pt}\renewcommand{\footrulewidth}{0.4pt}


\begin{document}
 	
	

\noindent\shadowbox{
	\begin{minipage}{1\linewidth}
		\centering
		\huge{\textbf{ TD 2 : Arithmétique  }}
	\end{minipage}
}
\medskip




\section*{Divisibilité, division euclidienne:}\hfill\\%[-0.25cm]
\begin{minipage}{1\linewidth}
	\begin{minipage}[t]{0.48\linewidth}
		\raggedright
	
\begin{exo}\textbf{(*)}\quad\\[0.15cm]
	 Montrer que $\forall n\in \N$ :
	 
	\noindent$n(n+1)(n+2)(n+3)  \text {  est divisible par }24,$
	
		\noindent$n(n+1)(n+2)(n+3)(n+4) \text{  est divisible par }120.$
	\centering
	\rule{1\linewidth}{0.6pt}
\end{exo}



\begin{exo}\textbf{(*)}\quad\\[0.15cm]
 	Montrer que si $n$ est un entier naturel somme de deux carr\'es d'entiers 
 	alors le reste de la division euclidienne de $n$ par $4$ n'est jamais \'egal \`a $3$.
 	
	\centering
	\rule{1\linewidth}{0.6pt}
\end{exo}

\begin{exo}\textbf{(*)}\quad\\[0.15cm]
	Montrer que le reste de la division euclidienne de $2^{65362}$ par $7$ est $2$.
	
	\centering
	\rule{1\linewidth}{0.6pt}
\end{exo}

\begin{exo}\textbf{(*)}\quad\\[0.15cm]
	
	Donner le reste de la division de  $100^{1000}$ par $13$.
	
	
	\centering
	\rule{1\linewidth}{0.6pt}
\end{exo}


\end{minipage}	
\hfill\vrule\hfill
\begin{minipage}[t]{0.48\linewidth}
\raggedright

\begin{exo}\textbf{(**)}\quad\\[0.15cm]
	\begin{enumerate}
		\item Montrer que le reste de la division euclidienne par $8$ du carr\'{e} de tout
		nombre impair est $1$.
		\item Montrer de m\^{e}me que tout nombre pair v\'{e}rifie $x^{2}=0 \pmod{8} $ ou
		$x^{2}=4 \pmod{8}.$
		\item Soient $a,b,c$ trois entiers impairs. D\'{e}terminer le reste modulo $8$ de
		$a^{2}+b^{2}+c^{2}$ et celui de $2(ab+bc+ca).$
		\item En d\'{e}duire que les deux nombres précédents ne sont pas des carr\'{e}s. Montrer ensuite que
		$ab+bc+ca$ n'en est pas un non plus.
	\end{enumerate}
	\centering
	\rule{1\linewidth}{0.6pt}
\end{exo}


\begin{exo}\textbf{(***)}\quad\\[0.15cm]
	Soient $x, y, z \in \Z$ trois entiers solutions de l'équation de Fermat $x^3 + y^3 = z^3 $. Montrer que l'un des entiers $x, y$ ou $z$ est
	multiple de $3$.

	
	
	\centering
	\rule{1\linewidth}{0.6pt}
\end{exo}



\end{minipage}
\end{minipage}


\section*{Algorithme d’Euclide, pgcd, ppcm:}\hfill\\%[-0.25cm]
\begin{minipage}{1\linewidth}
	\begin{minipage}[t]{0.48\linewidth}
		\raggedright
		
		

				
		\begin{exo}\textbf{(*)}\quad\\[0.15cm]
			\begin{enumerate}
				\item Trouver $u,v\in\Z$ tels que $126u+230v = 126 ~^\wedge~ 230$.
				\item Calculer le pgcd des nombres suivants : 
				
				 a)\quad$390, 720, 450.$ \quad\quad b) \quad$180, 606, 750.$
			\end{enumerate}
		
			\centering
			\rule{1\linewidth}{0.6pt}
		\end{exo}
		
						\begin{exo}\textbf{(**)}\quad\\[0.15cm]
			Soit $n\in\Z$.
			\begin{enumerate}
				\item Donner $(3n + 1) \wedge (2n + 5) $.
				\item Montrer que les entiers $n^3 + 3n^2 - 5$ et $n + 2$ sont premiers entre eux.
				
				
			\end{enumerate}  
			
			\centering
			\rule{1\linewidth}{0.6pt}
		\end{exo}
		
		
	
		

		
		
		
	\end{minipage}	
	\hfill\vrule\hfill
	\begin{minipage}[t]{0.48\linewidth}
		\raggedright
		
			\begin{exo}\textbf{(**)}\quad\\[0.2cm]
			D\'eterminer les couples d'entiers naturels de pgcd
			18 et de somme 360. De m\^eme avec pgcd 18 et produit 6480.
			
			\centering
			\rule{1\linewidth}{0.6pt}
		\end{exo}
		
		\begin{exo}\textbf{(**)}\quad\\[0.2cm]
			Soient $a=1\;111\;111\;111$ et $b=123\;456\;789$. Trouver $u,v\in\Z$ tels que $au+bv= \text{pgcd}(a,b)$.

		
			\centering
			\rule{1\linewidth}{0.6pt}
		\end{exo}
		


	\begin{exo}\textbf{(*)}\quad\\[0.2cm]
		
		D\'emontrer que le nombre $7^n+1$ est divisible par
		$8$ si $n$ est impair ; dans le cas $n$ pair, donner le reste de
		sa division par $8$.
		
		
		\centering
		\rule{1\linewidth}{0.6pt}
	\end{exo}	
		
		
		
	\end{minipage}
\end{minipage}

\section*{Nombres premiers, nombres premiers entre eux:}\hfill\\%[-0.25cm]
\begin{minipage}{1\linewidth}
	\begin{minipage}[t]{0.48\linewidth}
		\raggedright
		
		
		
		\begin{exo}\textbf{(*)}\quad\\[0.2cm]
	Montrer que tout entier composé $n \in \N^*$ possède un diviseur premier inférieur ou égal à $\sqrt{n}$.
	
	\centering
	\rule{1\linewidth}{0.6pt}
\end{exo}
		
		
		\begin{exo}\textbf{(*)}\quad\\[0.2cm]
			Combien $15!$ admet-il de diviseurs ?
			
			\centering
			\rule{1\linewidth}{0.6pt}
		\end{exo}
	
		\begin{exo}\textbf{(**)}\quad\\[0.2cm]
			Démontrer que $\sqrt[5]{\dfrac{4}{3}}$ est un irrationnel.
			
			\centering
			\rule{1\linewidth}{0.6pt}
		\end{exo}
		
			\begin{exo}\textbf{(***)}\quad\\[0.2cm]
			Soit $(x, y) \in \N^2$, justifier que l'équation $$x^2 = y^2 + (x~ ^\wedge~ y) + 2$$ admet $(2, 1)$ et $(2, 0)$ pour seules solutions.
			%Cours arithmétique Bertault
			
			\centering
			\rule{1\linewidth}{0.6pt}
		\end{exo}
		
	\end{minipage}	
	\hfill\vrule\hfill
	\begin{minipage}[t]{0.48\linewidth}
		\raggedright
		
				\begin{exo}\textbf{(**)}\quad\\[0.2cm]
			
			Soit $X$ l'ensemble des nombres premiers de la forme $4k + 3$ avec $k \in \N$.
			\begin{enumerate}
				\item Montrer que $X$ est non vide.
				\item Montrer que le produit de nombres de la forme $4k + 1$ est encore de cette forme.
				\item On suppose que $X$ est fini et on l'\'ecrit alors $X = \left\{
				p_1, \ldots, p_n\right\}$.\\  Soit $a = 4p_1 p_2 \ldots p_n  - 1$. Montrer par l'absurde
				que $a$ admet un diviseur premier de la forme $4k + 3$.
				\item Montrer que ceci est impossible et donc que $X$ est infini.
			\end{enumerate}
			
			\centering
			\rule{1\linewidth}{0.6pt}
		\end{exo}
	
		
		
		\begin{exo}\textbf{(***)}\quad\\[0.2cm]
			Montrer qu'il existe une infinité de nombres premiers de la forme $6k+5$.
			
			\centering
			\rule{1\linewidth}{0.6pt}
		\end{exo}
		

		
	\end{minipage}
\end{minipage}
	


\end{document}


%Ceci n'est pas dans le code exécuté 

\section*{Exercice supplémentaire de recherche}
\begin{exo}\textbf{(****)}\quad\\[0.2cm]
	Dans cet exercice nous souhaitons montrer que l'équation $$(\star)\quad \quad y^2 = x^3 - 3$$ d'inconnue $(x, y) \in \Z^2$ n'a pas de solution.
	
	\begin{enumerate}
		\item On suppose que $(x, y) \in \Z^2$ est solution de $(\star)$.
		\begin{enumerate}
			\item Montrer que $x$ est impair et $y$ est pair.
			\item Montrer qu'il existe $\alpha\in \Z$ tel que: \  $x^3+1 = 4(\alpha^2+1)$. 
		\end{enumerate}
		
	\end{enumerate}
	
	\centering
	\rule{1\linewidth}{0.6pt}
\end{exo}