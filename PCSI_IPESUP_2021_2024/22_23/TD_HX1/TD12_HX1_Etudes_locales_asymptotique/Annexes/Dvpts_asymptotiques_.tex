\documentclass[11pt]{article}

 %Configuration de la feuille 
 
\usepackage{amsmath,amssymb,enumerate,graphicx,pgf,tikz,fancyhdr}
\usepackage[utf8]{inputenc}
\usetikzlibrary{arrows}
\usepackage{geometry}
\usepackage{tabvar}
\geometry{hmargin=2.2cm,vmargin=1.5cm}\pagestyle{fancy}
\lfoot{\bfseries http://www.bibmath.net}
\rfoot{\bfseries\thepage}
\cfoot{}
\renewcommand{\footrulewidth}{0.5pt} %Filet en bas de page

 %Macros utilisées dans la base de données d'exercices 

\newcommand{\mtn}{\mathbb{N}}
\newcommand{\mtns}{\mathbb{N}^*}
\newcommand{\mtz}{\mathbb{Z}}
\newcommand{\mtr}{\mathbb{R}}
\newcommand{\mtk}{\mathbb{K}}
\newcommand{\mtq}{\mathbb{Q}}
\newcommand{\mtc}{\mathbb{C}}
\newcommand{\mch}{\mathcal{H}}
\newcommand{\mcp}{\mathcal{P}}
\newcommand{\mcb}{\mathcal{B}}
\newcommand{\mcl}{\mathcal{L}}
\newcommand{\mcm}{\mathcal{M}}
\newcommand{\mcc}{\mathcal{C}}
\newcommand{\mcmn}{\mathcal{M}}
\newcommand{\mcmnr}{\mathcal{M}_n(\mtr)}
\newcommand{\mcmnk}{\mathcal{M}_n(\mtk)}
\newcommand{\mcsn}{\mathcal{S}_n}
\newcommand{\mcs}{\mathcal{S}}
\newcommand{\mcd}{\mathcal{D}}
\newcommand{\mcsns}{\mathcal{S}_n^{++}}
\newcommand{\glnk}{GL_n(\mtk)}
\newcommand{\mnr}{\mathcal{M}_n(\mtr)}
\DeclareMathOperator{\ch}{ch}
\DeclareMathOperator{\sh}{sh}
\DeclareMathOperator{\vect}{vect}
\DeclareMathOperator{\card}{card}
\DeclareMathOperator{\comat}{comat}
\DeclareMathOperator{\imv}{Im}
\DeclareMathOperator{\rang}{rg}
\DeclareMathOperator{\Fr}{Fr}
\DeclareMathOperator{\diam}{diam}
\DeclareMathOperator{\supp}{supp}
\newcommand{\veps}{\varepsilon}
\newcommand{\mcu}{\mathcal{U}}
\newcommand{\mcun}{\mcu_n}
\newcommand{\dis}{\displaystyle}
\newcommand{\croouv}{[\![}
\newcommand{\crofer}{]\!]}
\newcommand{\rab}{\mathcal{R}(a,b)}
\newcommand{\pss}[2]{\langle #1,#2\rangle}
 %Document 

\begin{document} 

\begin{center}\textsc{{\huge Dvpts asymptotiques }}\end{center}

% Exercice 579


\vskip0.3cm\noindent\textsc{Exercice 1} - Tangente
\vskip0.2cm
Soit $n\geq 1$.
\begin{enumerate}
\item Montrer que l'équation $\tan x=x$ possède une solution unique $x_n$ dans $\left]n\pi-\frac\pi2,n\pi+\frac\pi2\right[$.
\item Quelle relation lie $x_n$ et $\arctan(x_n)$?
\item Montrer que $x_n=n\pi+\frac{\pi}{2}+o(1)$.
\item En écrivant $x_n=n\pi+\frac{\pi}{2}+\veps_n$ et en utilisant le résultat de la question 2., en déduire que
$$x_n=n\pi+\frac{\pi}{2}-\frac{1}{n\pi}+\frac{1}{2n^2\pi}+o\left(\frac1{n^2}\right).$$
\end{enumerate}


% Exercice 1153


\vskip0.3cm\noindent\textsc{Exercice 2} - Développement asymptotique de la série harmonique
\vskip0.2cm
On pose $H_n=1+\frac12+\dots+\frac1n$. 
\begin{enumerate}
\item Prouver que $H_n\sim_{+\infty}\ln n$.
\item On pose $u_n=H_n-\ln n$, et $v_n=u_{n+1}-u_n$.
\'Etudier la nature de la série $\sum_n v_n$. En déduire que la suite $(u_n)$ est convergente. On notera $\gamma$ sa limite.
\item Soit $R_n=\sum_{k=n}^{+\infty} \frac{1}{k^2}$. Donner un équivalent de $R_n$.
\item Soit $w_n$ tel que $H_n=\ln n+\gamma+w_n$, et soit 
$t_n=w_{n+1}-w_n$. Donner un équivalent du reste $\sum_{k\geq n}t_k$.
En déduire que $H_n=\ln n+\gamma+\frac{1}{2n}+o\left(\frac1n\right)$.
\end{enumerate}


% Exercice 592


\vskip0.3cm\noindent\textsc{Exercice 3} - Equivalence de sommes partielles
\vskip0.2cm
Soient $(u_n)$ et $(v_n)$ deux suites réelles positives telles que $u_n\sim_{+\infty}v_n$. On pose
$$U_n=\sum_{k=1}^n u_k\textrm{ et }V_n=\sum_{k=1}^n v_k,$$
et on suppose de plus que $V_n\to+\infty$. Démontrer que $U_n\sim_{+\infty} V_n.$




\vskip0.5cm
\noindent{\small Cette feuille d'exercices a été conçue à l'aide du site \textsf{https://www.bibmath.net}}

%Vous avez accès aux corrigés de cette feuille par l'url : https://www.bibmath.net/ressources/justeunefeuille.php?id=26713
\end{document}