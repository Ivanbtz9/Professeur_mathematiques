
 %Macros utilisées dans la base de données d'exercices 

\newcommand{\mtn}{\mathbb{N}}
\newcommand{\mtns}{\mathbb{N}^*}
\newcommand{\mtz}{\mathbb{Z}}
\newcommand{\mtr}{\mathbb{R}}
\newcommand{\mtk}{\mathbb{K}}
\newcommand{\mtq}{\mathbb{Q}}
\newcommand{\mtc}{\mathbb{C}}
\newcommand{\mch}{\mathcal{H}}
\newcommand{\mcp}{\mathcal{P}}
\newcommand{\mcb}{\mathcal{B}}
\newcommand{\mcl}{\mathcal{L}}
\newcommand{\mcm}{\mathcal{M}}
\newcommand{\mcc}{\mathcal{C}}
\newcommand{\mcmn}{\mathcal{M}}
\newcommand{\mcmnr}{\mathcal{M}_n(\mtr)}
\newcommand{\mcmnk}{\mathcal{M}_n(\mtk)}
\newcommand{\mcsn}{\mathcal{S}_n}
\newcommand{\mcs}{\mathcal{S}}
\newcommand{\mcd}{\mathcal{D}}
\newcommand{\mcsns}{\mathcal{S}_n^{++}}
\newcommand{\glnk}{GL_n(\mtk)}
\newcommand{\mnr}{\mathcal{M}_n(\mtr)}
\DeclareMathOperator{\ch}{ch}
\DeclareMathOperator{\sh}{sh}
\DeclareMathOperator{\vect}{vect}
\DeclareMathOperator{\card}{card}
\DeclareMathOperator{\comat}{comat}
\DeclareMathOperator{\imv}{Im}
\DeclareMathOperator{\rang}{rg}
\DeclareMathOperator{\Fr}{Fr}
\DeclareMathOperator{\diam}{diam}
\DeclareMathOperator{\supp}{supp}
\newcommand{\veps}{\varepsilon}
\newcommand{\mcu}{\mathcal{U}}
\newcommand{\mcun}{\mcu_n}
\newcommand{\dis}{\displaystyle}
\newcommand{\croouv}{[\![}
\newcommand{\crofer}{]\!]}
\newcommand{\rab}{\mathcal{R}(a,b)}
\newcommand{\pss}[2]{\langle #1,#2\rangle}
 %Document 

\begin{document} 

\begin{center}\textsc{{\huge }}\end{center}

% Exercice 1198


\vskip0.3cm\noindent\textsc{Exercice 1} - Dans une entreprise...
\vskip0.2cm
Dans une entreprise, il y a 800 employés. 300 sont des hommes, 352 sont membres d'un syndicat, 424 sont mariés, 188 sont des hommes syndiqués, 166 sont des hommes mariés, 208 sont syndiqués et mariés, 144 sont des hommes mariés syndiqués. Combien y-a-t-il de femmes célibataires non syndiquées?


% Exercice 2362


\vskip0.3cm\noindent\textsc{Exercice 2} - Les boulangeries
\vskip0.2cm
Dans une ville, il y a quatre boulangeries qui ferment un jour par semaine.
\begin{enumerate}
\item Déterminer le nombre de façons d'attribuer un jour de fermeture hebdomadaire?
\item Reprendre la même question si plusieurs boulangeries ne peuvent fermer le même jour.
\item Reprendre la même question si chaque jour, il doit y avoir au moins une boulangerie ouverte.
\end{enumerate}


% Exercice 1200


\vskip0.3cm\noindent\textsc{Exercice 3} - Nombres et chiffres
\vskip0.2cm
Soit $A$ l'ensemble des nombres à 7 chiffres ne comportant aucun "1". Déterminer le nombre d'éléments des ensembles suivants :
\begin{enumerate}
\item $A$.
\item $A_1$, ensemble des nombres de $A$ ayant 7 chiffres différents.
\item $A_2$, ensemble des nombres pairs de $A$.
\item $A_3$, ensemble des nombres de $A$ dont les chiffres forment une suite strictement croissante (dans l'ordre où ils sont écrits).
\end{enumerate}


% Exercice 1204


\vskip0.3cm\noindent\textsc{Exercice 4} - Anagrammes
\vskip0.2cm
Dénombrer les anagrammes des mots suivants : MATHS, RIRE, ANANAS.


% Exercice 2361


\vskip0.3cm\noindent\textsc{Exercice 5} - Le poker
\vskip0.2cm
Une main au poker est formée de 5 cartes extraites d'un jeu de 52 cartes. Traditionnellement,trèfle, carreau, coeur, pique sont appelées couleurs et les valeurs des cartes sont rangées dans l'ordre : as, roi, dame, valet, 10, 9, 8, 7, 6, 5, 4, 3, 2, de la plus forte à la plus faible. Dénombrer les mains suivantes :
\begin{enumerate}
\item quinte flush : main formée de 5 cartes consécutives de la même couleur (la suite as, 2, 3, 4 et 5 est une quinte flush).
\item carré : main contenant 4 cartes de la même valeur (4 as par exemple).
\item full : main formée de 3 cartes de la même valeur et de deux autres cartes de même valeur (par exemple, 3 as et 2 rois).
\item quinte : main formée de 5 cartes consécutives et qui ne sont pas toutes de la même couleur.
\item brelan : main comprenant 3 cartes de même valeur et qui n'est ni un carré, ni un full (par exemple, 3 as, 1 valet, 1 dix).
\end{enumerate}


% Exercice 3129


\vskip0.3cm\noindent\textsc{Exercice 6} - $n+1$ entiers parmi $2n$
\vskip0.2cm
On considère un ensemble $X$ de $n+1$ entiers (distincts) choisis dans $\{1,\dots,2n\}$. Démontrer que parmi les éléments de $X$, on peut toujours trouver $2$ entiers dont la somme fait $2n+1$.


% Exercice 1212


\vskip0.3cm\noindent\textsc{Exercice 7} - Bizarre, bizarre,...
\vskip0.2cm
Démontrer par un dénombrement que, pour $n\geq 1$, on a :
$$\binom{2n}{n}=\sum_{k=0}^n \binom{n}{k}^2.$$


% Exercice 1213


\vskip0.3cm\noindent\textsc{Exercice 8} - Une somme
\vskip0.2cm
Soit $n,p$ des entiers naturels avec $n\geq p$. Démontrer par dénombrement que
$$\sum_{k=p}^n \dbinom{k}{p}=\dbinom{n+1}{p+1}.$$


% Exercice 1217


\vskip0.3cm\noindent\textsc{Exercice 9} - Combinaisons avec répétitions
\vskip0.2cm
Pour $n\in\mathbb N^*$ et $p\in\mathbb N$, on note $\Gamma_n^p$ le nombre de $n$-uplets $(x_1,\dots,x_n)\in\mathbb N^n$ tels que $x_1+\dots+x_n=p$. 
\begin{enumerate}
\item Déterminer $\Gamma_n^0$, $\Gamma_n^1$, $\Gamma_n^2$, $\Gamma_1^p$ et $\Gamma_2^p$.
\item Démontrer que, pour tout $n\in\mathbb N^*$, pour tout $p\in\mathbb N$, 
$$\Gamma_{n+1}^p=\Gamma_n^0+\Gamma_n^1+\dots+\Gamma_n^p.$$
\item En déduire que, pour tout $n\in\mathbb N^*$ et tout $p\in\mathbb N$, 
$$\Gamma_n^p=\binom{n+p-1}p.$$
\end{enumerate}


% Exercice 1209


\vskip0.3cm\noindent\textsc{Exercice 10} - Parties de cardinal pair
\vskip0.2cm
Soit $E$ un ensemble fini de cardinal $n\geq 1$. Démontrer que
le nombre de parties de $E$ de cardinal pair vaut $2^{n-1}$.


% Exercice 2743


\vskip0.3cm\noindent\textsc{Exercice 11} - Nombres de Bell
\vskip0.2cm
Pour $n\in\mathbb N$, on note $B_n$ le nombre de partitions d'un ensemble $E$ de cardinal $n$. On pose $B_0=1$.
\begin{enumerate}
 \item Calculer $B_1$, $B_2$ et $B_3$.
 \item \'Etablir la formule de récurrence 
 $$B_{n+1}=\sum_{k=0}^n \binom nk B_k.$$
\end{enumerate}


% Exercice 3131


\vskip0.3cm\noindent\textsc{Exercice 12} - Nombre de fonctions (strictement) croissantes
\vskip0.2cm
Soit $n,p\geq 1$ deux entiers.
\begin{enumerate}
\item Combien y-a-t-il de fonctions strictement croissantes de $\{1,\dots,p\}$ dans $\{1,\dots,n\}$?
\item \begin{enumerate}
\item Soit $f:\{1,\dots,p\}\to\{1,\dots,n\}$ une fonction croissante. On pose $\phi(f)$ la fonction définie sur $\{1,\dots,p\}$, à valeurs dans $\{1,\dots,n+p-1\}$, par $\phi(f)(k)=f(k)+k-1$. Démontrer que $\phi(f)$ est strictement croissante.
\item Soit $g:\{1,\dots,p\}\to\{1,\dots,n+p-1\}$ une fonction strictement croissante. On pose $\psi(g)$ la fonction définie sur $\{1,\dots,p\}$, à valeurs dans $\{1,\dots,n\}$, par $\psi(g)(k)=g(k)-k+1$. Démontrer que $\psi(g)$ est croissante.
\item Combien y-a-t-il de fonctions croissantes de $\{1,\dots,p\}$ dans $\{1,\dots,n\}$?
\end{enumerate}
\end{enumerate}




\vskip0.5cm
\noindent{\small Cette feuille d'exercices a été conçue à l'aide du site \textsf{https://www.bibmath.net}}

%Vous avez accès aux corrigés de cette feuille par l'url : https://www.bibmath.net/ressources/justeunefeuille.php?id=27881
\end{document}