\documentclass[t,12pt]{beamer}
\usepackage[utf8]{inputenc}
\usepackage[T1]{fontenc}
\usepackage{fancyhdr} % pour personnaliser les en-têtes
\usepackage{lastpage}
\usepackage[frenchb]{babel}
\usepackage{amsfonts,amssymb}
\usepackage{amsmath,amsthm}
\usepackage{paralist}
\usepackage{enumerate}
\usepackage{xspace}
\usepackage{xcolor}
\usepackage{variations}
\usepackage{xypic}
\usepackage{eurosym,multicol}
\usepackage{graphicx}
\usepackage[np]{numprint}
\usepackage{hyperref} 
\usepackage{setspace}
\usepackage{listings} % pour écrire des codes avec coloration syntaxique  

\usepackage{tikz}
\usetikzlibrary{calc, arrows, plotmarks,decorations.pathreplacing}
\usepackage{colortbl}
\usepackage{multirow}


\newtheorem{defi}{Définition}
\newtheorem{thm}{Théorème}
\newtheorem{thm-def}{Théorème/Définition}
\newtheorem{rmq}{Remarque}
\newtheorem{prop}{Propriété}
\newtheorem{cor}{Corollaire}
\newtheorem{lem}{Lemme}
\newtheorem{ex}{Exemple}
\newtheorem{cex}{Contre-exemple}
\newtheorem{prop-def}{Propriété-définition}
\newtheorem{exer}{Exercice}
\newtheorem{nota}{Notation}
\newtheorem{ax}{Axiome}
\newtheorem{appl}{Application}
\newtheorem{csq}{Conséquence}
%\def\di{\displaystyle}


\newcommand{\vtab}{\rule[-0.4em]{0pt}{1.2em}}
\newcommand{\V}{\overrightarrow}
\renewcommand{\thesection}{\Roman{section} }
\renewcommand{\thesubsection}{\arabic{subsection} }
\renewcommand{\thesubsubsection}{\alph{subsubsection} }
\newcommand{\C}{\mathbb{C}}
\newcommand{\R}{\mathbb{R}}
\newcommand{\Q}{\mathbb{Q}}
\newcommand{\Z}{\mathbb{Z}}
\newcommand{\N}{\mathbb{N}}



\usetheme{Warsaw}

\title{ Quatre Questions sur les fonctions affines}
\author{Évaluation de 1 heure \\mercredi 25 mai 2022}
\date{Programme de révisions: Fonctions affines + Probabilités}
\begin{document}
	\maketitle	
\begin{frame}
	
	\frametitle{Question 1: }
Soit $F$ une fonction affine définie sur $\R$.\hfill\\[0.2cm]

On sait que la courbe de la fonction $F$ passe par les points de coordonnées $X(3;4)$ et $Y(-2;0)$. 
\begin{enumerate}
	\item Donner l'expression de la fonction $F$.
	\item Donner le tableau de signe de la fonction $F$. 
\end{enumerate}
		

	

\end{frame}

\begin{frame}
	\frametitle{Question 2: }
	
Soit $G$ une fonction affine définie sur $\R$.\hfill\\[0.2cm]

On sait que la courbe de la fonction $G$ passe par les points de coordonnées $U(-10;-4)$ et $V(-2;10)$. 
\begin{enumerate}
	\item Donner l'expression de la fonction $G$.
	\item Calculer le nombre suivant:
	$$\dfrac{y_U-y_V}{x_U-x_V}$$ 
	Que remarquez-vous?
	\item Donner une règle qui vous permet de trouver le coefficient directeur d'une droite lorsque l'on connaît les coordonnées de deux points sur cette droite. Donner la preuve de cette propriété. 
	
\end{enumerate}


\end{frame}

\begin{frame}
	\frametitle{Question 3: }
	
	Soit $a\in\R$. On considère les points suivants:
	$$K\left(-1,3\right) ; L\left(2,a\right)$$  \hfill\\[0.2cm]
	 
	\begin{enumerate}  
		\item Donner la valeur de l'inconnue $a$ pour que le coefficient de la droite $\left(KL\right)$ soit égal à 2. 
		\item Donner la valeur de l'inconnue $a$ pour que le coefficient de la droite $\left(KL\right)$ soit égal à -3.
		
	\end{enumerate}
	
	
\end{frame}

\begin{frame}
	\frametitle{Question 4: }
	
	Soit $H$ une fonction affine définie sur $\R$.\hfill\\[0.2cm]
	
	On sait que la courbe de la fonction $H$ passe par les points de coordonnées $A(4;5)$ et $B(-2;5)$. 
	\begin{enumerate}
		\item Donner le coefficient directeur de la droite associée à la courbe de la fonction H et en déduire l'expression de la fonction $H$. Quelle est la nature de la fonction $H$.  
		\item Donner les coordonnées d'un point $C$ du plan telles que le coefficient de la droite $\left(AC\right)$ soit égal à 5. 
		\item Donner les coordonnées d'un point $D$ du plan telles que le coefficient de la droite $\left(BD\right)$ soit égal à -2. 
		
	\end{enumerate}
	
	
\end{frame}




\end{document}