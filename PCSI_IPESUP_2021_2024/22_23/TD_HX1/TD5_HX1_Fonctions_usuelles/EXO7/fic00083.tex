
%%%%%%%%%%%%%%%%%% PREAMBULE %%%%%%%%%%%%%%%%%%

\documentclass[11pt,a4paper]{article}

\usepackage{amsfonts,amsmath,amssymb,amsthm}
\usepackage[utf8]{inputenc}
\usepackage[T1]{fontenc}
\usepackage[francais]{babel}
\usepackage{mathptmx}
\usepackage{fancybox}
\usepackage{graphicx}
\usepackage{ifthen}

\usepackage{tikz}   

\usepackage{hyperref}
\hypersetup{colorlinks=true, linkcolor=blue, urlcolor=blue,
pdftitle={Exo7 - Exercices de mathématiques}, pdfauthor={Exo7}}

\usepackage{geometry}
\geometry{top=2cm, bottom=2cm, left=2cm, right=2cm}

%----- Ensembles : entiers, reels, complexes -----
\newcommand{\Nn}{\mathbb{N}} \newcommand{\N}{\mathbb{N}}
\newcommand{\Zz}{\mathbb{Z}} \newcommand{\Z}{\mathbb{Z}}
\newcommand{\Qq}{\mathbb{Q}} \newcommand{\Q}{\mathbb{Q}}
\newcommand{\Rr}{\mathbb{R}} \newcommand{\R}{\mathbb{R}}
\newcommand{\Cc}{\mathbb{C}} \newcommand{\C}{\mathbb{C}}
\newcommand{\Kk}{\mathbb{K}} \newcommand{\K}{\mathbb{K}}

%----- Modifications de symboles -----
\renewcommand{\epsilon}{\varepsilon}
\renewcommand{\Re}{\mathop{\mathrm{Re}}\nolimits}
\renewcommand{\Im}{\mathop{\mathrm{Im}}\nolimits}
\newcommand{\llbracket}{\left[\kern-0.15em\left[}
\newcommand{\rrbracket}{\right]\kern-0.15em\right]}
\renewcommand{\ge}{\geqslant} \renewcommand{\geq}{\geqslant}
\renewcommand{\le}{\leqslant} \renewcommand{\leq}{\leqslant}

%----- Fonctions usuelles -----
\newcommand{\ch}{\mathop{\mathrm{ch}}\nolimits}
\newcommand{\sh}{\mathop{\mathrm{sh}}\nolimits}
\renewcommand{\tanh}{\mathop{\mathrm{th}}\nolimits}
\newcommand{\cotan}{\mathop{\mathrm{cotan}}\nolimits}
\newcommand{\Arcsin}{\mathop{\mathrm{arcsin}}\nolimits}
\newcommand{\Arccos}{\mathop{\mathrm{arccos}}\nolimits}
\newcommand{\Arctan}{\mathop{\mathrm{arctan}}\nolimits}
\newcommand{\Argsh}{\mathop{\mathrm{argsh}}\nolimits}
\newcommand{\Argch}{\mathop{\mathrm{argch}}\nolimits}
\newcommand{\Argth}{\mathop{\mathrm{argth}}\nolimits}
\newcommand{\pgcd}{\mathop{\mathrm{pgcd}}\nolimits} 

%----- Structure des exercices ------

\newcommand{\exercice}[1]{\video{0}}
\newcommand{\finexercice}{}
\newcommand{\noindication}{}
\newcommand{\nocorrection}{}

\newcounter{exo}
\newcommand{\enonce}[2]{\refstepcounter{exo}\hypertarget{exo7:#1}{}\label{exo7:#1}{\bf Exercice \arabic{exo}}\ \  #2\vspace{1mm}\hrule\vspace{1mm}}

\newcommand{\finenonce}[1]{
\ifthenelse{\equal{\ref{ind7:#1}}{\ref{bidon}}\and\equal{\ref{cor7:#1}}{\ref{bidon}}}{}{\par{\footnotesize
\ifthenelse{\equal{\ref{ind7:#1}}{\ref{bidon}}}{}{\hyperlink{ind7:#1}{\texttt{Indication} $\blacktriangledown$}\qquad}
\ifthenelse{\equal{\ref{cor7:#1}}{\ref{bidon}}}{}{\hyperlink{cor7:#1}{\texttt{Correction} $\blacktriangledown$}}}}
\ifthenelse{\equal{\myvideo}{0}}{}{{\footnotesize\qquad\texttt{\href{http://www.youtube.com/watch?v=\myvideo}{Vidéo $\blacksquare$}}}}
\hfill{\scriptsize\texttt{[#1]}}\vspace{1mm}\hrule\vspace*{7mm}}

\newcommand{\indication}[1]{\hypertarget{ind7:#1}{}\label{ind7:#1}{\bf Indication pour \hyperlink{exo7:#1}{l'exercice \ref{exo7:#1} $\blacktriangle$}}\vspace{1mm}\hrule\vspace{1mm}}
\newcommand{\finindication}{\vspace{1mm}\hrule\vspace*{7mm}}
\newcommand{\correction}[1]{\hypertarget{cor7:#1}{}\label{cor7:#1}{\bf Correction de \hyperlink{exo7:#1}{l'exercice \ref{exo7:#1} $\blacktriangle$}}\vspace{1mm}\hrule\vspace{1mm}}
\newcommand{\fincorrection}{\vspace{1mm}\hrule\vspace*{7mm}}

\newcommand{\finenonces}{\newpage}
\newcommand{\finindications}{\newpage}


\newcommand{\fiche}[1]{} \newcommand{\finfiche}{}
%\newcommand{\titre}[1]{\centerline{\large \bf #1}}
\newcommand{\addcommand}[1]{}

% variable myvideo : 0 no video, otherwise youtube reference
\newcommand{\video}[1]{\def\myvideo{#1}}

%----- Presentation ------

\setlength{\parindent}{0cm}

\definecolor{myred}{rgb}{0.93,0.26,0}
\definecolor{myorange}{rgb}{0.97,0.58,0}
\definecolor{myyellow}{rgb}{1,0.86,0}

\newcommand{\LogoExoSept}[1]{  % input : echelle       %% NEW
{\usefont{U}{cmss}{bx}{n}
\begin{tikzpicture}[scale=0.1*#1,transform shape]
  \fill[color=myorange] (0,0)--(4,0)--(4,-4)--(0,-4)--cycle;
  \fill[color=myred] (0,0)--(0,3)--(-3,3)--(-3,0)--cycle;
  \fill[color=myyellow] (4,0)--(7,4)--(3,7)--(0,3)--cycle;
  \node[scale=5] at (3.5,3.5) {Exo7};
\end{tikzpicture}}
}


% titre
\newcommand{\titre}[1]{%
\vspace*{-4ex} \hfill \hspace*{1.5cm} \hypersetup{linkcolor=black, urlcolor=black} 
\href{http://exo7.emath.fr}{\LogoExoSept{3}} 
 \vspace*{-5.7ex}\newline 
\hypersetup{linkcolor=blue, urlcolor=blue}  {\Large \bf #1} \newline 
 \rule{12cm}{1mm} \vspace*{3ex}}

%----- Commandes supplementaires ------



\begin{document}

%%%%%%%%%%%%%%%%%% EXERCICES %%%%%%%%%%%%%%%%%%
\fiche{f00083, rouget, 2010/07/11}

\titre{Fonctions usuelles} 

Exercices de Jean-Louis Rouget.
Retrouver aussi cette fiche sur \texttt{\href{http://www.maths-france.fr}{www.maths-france.fr}}

\begin{center}
* très facile\quad** facile\quad*** difficulté moyenne\quad**** difficile\quad***** très difficile\\
I~:~Incontournable\quad T~:~pour travailler et mémoriser le cours
\end{center}


\exercice{5097, rouget, 2010/06/30}
\enonce{005097}{**I}
\begin{enumerate}
\item  Soit $f$ une fonction dérivable sur $\Rr$ à valeurs dans $\Rr$. Montrer que si $f$ est paire, $f'$ est
impaire et si $f$ est impaire, $f'$ est paire.
\item  Soient $n\in\Nn^*$ et $f$ une fonction $n$ fois dérivable sur $\Rr$ à valeurs dans $\Rr$. $f^{(n)}$
désignant la dérivée $n$-ième de $f$, montrer que si $f$ est paire, $f^{(n)}$ est paire si $n$ est pair et impaire si
$n$ est impair.
\item  Soit $f$ une fonction continue sur $\Rr$ à valeurs dans $\Rr$. A-t-on des résultats analogues concernant les
primitives de $f$~?
\item  Reprendre les questions précédentes en remplaçant la condition \og~$f$ est paire (ou impaire)~\fg~par la
condition \og~$f$ est $T$-périodique~\fg.
\end{enumerate}
\finenonce{005097}


\finexercice
\exercice{5098, rouget, 2010/06/30}
\enonce{005098}{**}
Trouver la plus grande valeur de $\sqrt[n]{n}$, $n\in\Nn^*$.
\finenonce{005098}


\finexercice
\exercice{5099, rouget, 2010/06/30}
\enonce{005099}{**I}
\begin{enumerate}
\item  Etudier brièvement la fontion $x\mapsto\frac{\ln x}{x}$ et tracer son graphe.
\item  Trouver tous les couples $(a,b)$ d'entiers naturels non nuls et distincts vérifiant $a^b=b^a$.
\end{enumerate}
\finenonce{005099}


\finexercice
\exercice{5100, rouget, 2010/06/30}
\enonce{005100}{}
Résoudre dans $\Rr$ les équations ou inéquations suivantes~:
\begin{enumerate}
 \item $(**)\;\ln|x+1|-\ln|2x+1|\leq\ln2$,
 \item $(*)\;x^{\sqrt{x}}=\sqrt{x}^x$,
 \item $(**)\;2\Argsh x=\Argch3-\Argth\frac{7}{9}$,
 \item $(**)\;\mbox{ln}_x(10)+2\mbox{ln}_{10x}(10)+3\mbox{ln}_{100x}(10)=0$,
 \item $(**)\;2^{2x}-3^{x-\frac{1}{2}}=3^{x+\frac{1}{2}}-2^{2x-1}$.
\end{enumerate}
\finenonce{005100}


\finexercice\exercice{5101, rouget, 2010/06/30}
\enonce{005101}{**}
Trouver $\lim_{x\rightarrow +\infty}\frac{(x^x)^x}{x^{(x^x)}}$.
\finenonce{005101}


\finexercice
\exercice{5102, rouget, 2010/06/30}
\enonce{005102}{}
Construire le graphe des fonctions suivantes~:
\begin{enumerate}
\item (*) $f_1(x)=2|2x-1|-|x+2|+3x$.
\item (**) $f_{2}(x)=\ln(\ch x)$.
\item (***) $f_3(x)=x+\sqrt{|x^2-1|}$.
\item (**) $f_4(x)=|\tan x|+\cos x$.
\item (***) $f_5(x)=\left(1+\frac{1}{x}\right)^x$ (à étudier sur $]0,+\infty[$).
\item (**) $f_6(x)=\mbox{log}_2(1-\mbox{log}_{\frac{1}{2}}(x^2-5x+6))$.
\end{enumerate}
\finenonce{005102}


\finexercice

\finfiche


 \finenonces 



 \finindications 

\noindication
\noindication
\noindication
\noindication
\noindication
\noindication


\newpage

\correction{005097}

\begin{enumerate}
 \item  Soit $f$ une fonction dérivable sur $\Rr$ à valeurs dans $\Rr$. Si $f$ est paire, alors, pour tout réel $x$,
$f(-x)=f(x)$. En dérivant cette égalité, on obtient

$$\forall x\in\Rr,\;-f'(-x)=f'(x),$$
et donc $f'$ est impaire. De même, si $f$ est impaire, pour tout réel $x$, on a $f(-x)=-f(x)$, et par dérivation on
obtient pour tout réel $x$, $f'(-x)=f'(x)$. $f'$ est donc paire.

\begin{center}
\shadowbox{
$(f\;\mbox{paire}\Rightarrow f'\;\mbox{impaire) et}\;(f\;\mbox{impaire}\Rightarrow f'\;\mbox{paire.})$
}
\end{center}
 \item  Soient $n\in\Nn^*$ et $f$ une fonction $n$ fois dérivable sur $\Rr$ à valeurs dans $\Rr$. Supposons $f$
paire. Par suite, pour tout réel $x$, $f(-x)=f(x)$. Immédiatement par récurrence, on a

$$\forall x\in\Rr,\;f^{(n)}(-x)=(-1)^nf(x).$$
Ceci montre que $f^{(n)}$ a la parité de $n$, c'est-à-dire que $f^{(n)}$ est une fonction paire quand $n$ est un
entier pair et est une fonction impaire quand $n$ est un entier impair.
De même, si $f$ est impaire et $n$ fois dérivable sur $\Rr$, $f^{(n)}$ a la parité contraire de celle de $n$.
 \item  Soit $f$ une fonction continue sur $\Rr$ et impaire et $F$ une primitive de $f$. Montrons que $F$ est paire.
Pour $x$ réel, posons $g(x)=F(x)-F(-x)$. $g$ est dérivable sur $\Rr$ et pour tout réel $x$,

$$g'(x)=F'(x)+F'(-x)=f(x)+f(-x)=0.$$
$g$ est donc constante sur $\Rr$ et par suite, pour tout réel $x$, $g(x)=g(0)=F(0)-F(0)=0$. Ainsi, $g$ est la fonction nulle et donc, pour
tout réel $x$, $F(x)=F(-x)$. On a montré que $F$ est paire.
Par contre, si $f$ est paire, $F$ n'est pas nécessairement impaire. Par exemple, la fonction $f~:~x\mapsto1$ est paire,
mais $F~:~x\mapsto x+1$ est une primitive de $f$ qui n'est pas impaire.
 \item  On montre aisément en dérivant une ou plusieurs fois l'égalité~:~$\forall x\in\Rr,\;f(x+T)=f(x)$, que les
dérivées successives d'une fonction $T$-périodique sont $T$-périodiques. Par contre, il n'en est pas de même des
primitives. Par exemple, si pour tout réel $x$, $f(x)=\cos^2x=\frac{1}{2}(1+\cos(2x))$, $f$ est $\pi$-périodique, mais
la fonction $F~:~x\mapsto\frac{x}{2}+\frac{\sin(2x)}{4}$, qui est une primitive de $f$ sur $\Rr$, n'est pas
$\pi$-périodique ni même périodique tout court.
\end{enumerate}
\fincorrection
\correction{005098}
Pour $n\in\Nn^*$, posons $u_n=\sqrt[n]{n}$ puis, pour $x$ réel strictement positif, $f(x)=x^{1/x}$
de sorte que pour tout naturel non nul $n$, on a $u_n=f(n)$.
$f$ est définie sur $]0,+\infty[$ et pour $x>0$, $f(x)=e^{\ln x/x}$. $f$ est dérivable sur $]0,+\infty[$ et pour
$x>0$,

$$f'(x)=\frac{1-\ln x}{x^2}e^{\ln x/x}.$$
Pour $x>0$, $f'(x)$ est du signe de $1-\ln x$ et donc $f'$ est strictement positive sur $]0,e[$ et strictement négative
sur $]e,+\infty[$. $f$ est donc strictement croissante sur $]0,e]$ et strictement décroissante sur $[e,+\infty[$. En
particulier, pour $n\geq3$,

$$u_n=f(n)\leq f(3)=u_3=\sqrt[3]{3}.$$
Comme $u_2=\sqrt{2}>1=u_1$, on a donc $\mbox{Max}\{u_n,\;n\in\Nn^*\}=\mbox{Max}\{\sqrt{2},\sqrt[3]{3}\}$. Enfin,
$\sqrt{2}=1,41...<1,44..=\sqrt[3]{3}$ (on peut aussi constater que $(\sqrt{2})^6=8<9=(\sqrt[3]{3})^6$). Finalement,

\begin{center}
\shadowbox{
$\text{Max}\left\{\sqrt[n]{n},\;n\in\Nn^*\right\}=\sqrt[3]{3}=1,44...$
}
\end{center}
\fincorrection
\correction{005099}
\begin{enumerate}
 \item  Pour $x>0$, posons $f(x)=\frac{\ln x}{x}$. $f$ est définie et dérivable sur $]0,+\infty[$ et, pour $x>0$,
$f'(x)=\frac{1-\ln x}{x^2}$. $f$ est donc strictement croissante sur $]0,e]$ et strictement décroissante sur
$[e,+\infty[$. Le graphe de $f$ s'en déduit facilement~:

%$$\includegraphics{../images/img005099-1}$$


 \item  Soient $a$ et $b$ deux entiers naturels non nuls tels que $a<b$. On a alors

$$a^b=b^a\Leftrightarrow\ln(a^b)=\ln(b^a)\Leftrightarrow b\ln a=a\ln b\Leftrightarrow\frac{\ln a}{a}=\frac{\ln b}{b}\Leftrightarrow f(a)=f(b).$$
Si $a\geq3$, puisque $f$ est strictement décroissante sur $[e,+\infty[$, on a alors $f(a)>f(b)$ et en particulier,
$f(a)\neq f(b)$. $a$ n'est donc pas solution.
$a=1$ n'est évidemment pas solution. Par exemple, $a^b=b^a\Rightarrow1^b=b^1\Rightarrow b=1=a$ ce qui est exclu.
Donc, nécessairement $a=2$ et $b$ est un entier supérieur ou égal à $3$, et donc à $e$, vérifiant $f(b)=f(2)$. Comme $f$
est strictement décroissante sur $[e,+\infty[$, l'équation $f(b)=f(2)$ a au plus une solution dans $[e,+\infty[$. Enfin,
comme $2^4=16=4^2$, on a montré que~:~il existe un et un seul couple $(a,b)$ d'entiers naturels non nuls tel que $a<b$
et $a^b=b^a$, à savoir $(2,4)$.
\end{enumerate}
\fincorrection
\correction{005100}
\begin{enumerate}
 \item  Soit $x\in\Rr$,

\begin{align*}
\ln|x+1|-\ln|2x+1|\leq\ln2&\Leftrightarrow\ln\left|\frac{x+1}{2x+1}\right|\leq\ln2\Leftrightarrow\left|\frac{x+1}{2x+1}\right|\leq2\;
\mbox{et}\;x+1\neq0\\
 &\Leftrightarrow-2\leq\frac{x+1}{2x+1}\leq2\;\mbox{et}\;x\neq-1\Leftrightarrow\frac{x+1}{2x+1}+2\geq0\;\mbox{et}\;\frac{x+1}{2x+1}-2
 \leq0\;\mbox{et}\;x\neq-1\\
 &\Leftrightarrow\frac{5x+3}{2x+1}\geq0\;\mbox{et}\;\frac{-3x-1}{2x+1}\leq0\;\mbox{et}\;x\neq-1\\
 &\Leftrightarrow\left(x\in\left]-\infty,-\frac{3}{5}\right]\cup\left]-\frac{1}{2},+\infty\right[\right)\;\mbox{et}\;\left(\left]-\infty,-\frac{1}{2}\right[\cup\left[-
\frac{1}{3},+\infty\right[\right)\;\mbox{et}\;x\neq-1\\
 &\Leftrightarrow x\in]-\infty,-1[\cup\left]-1,-\frac{3}{5}\right]\cup\left[-\frac{1}{3},+\infty\right[
\end{align*}

 \item  Pour $x>0$

\begin{align*}
x^{\sqrt{x}}=\sqrt{x}^x&\Leftrightarrow\sqrt{x}\ln x=x\ln\sqrt{x}\Leftrightarrow\ln x(\sqrt{x}-\frac{x}{2})=0\\
 &\Leftrightarrow\ln x\times\sqrt{x}(2-\sqrt{x})=0\Leftrightarrow x=1\;\mbox{ou}\;x=4.
\end{align*}

 \item  $\Argch3=\ln(3+\sqrt{3^2-1})=\ln(3+\sqrt{8})$ et
$\Argth\frac{7}{9}=\frac{1}{2}\ln\left(\frac{1+\frac{7}{9}}{1-\frac{7}{9}}\right)=\ln\sqrt{8}$. Donc,
$\Argch3-\Argth\frac{7}{9}=\ln\left(1+\frac{3}{\sqrt{8}}\right)$.
Par suite,

\begin{align*}
2\Argsh x=\Argch3-\Argth\frac{7}{9}&\Leftrightarrow x=\sh\left(\frac{1}{2}\ln\left(1+\frac{3}{\sqrt{8}}\right)\right)\\
 &\Leftrightarrow
x=\frac{1}{2}\left(\sqrt{1+\frac{3}{\sqrt{8}}}-\frac{1}{\sqrt{1+\frac{3}{\sqrt{8}}}}\right)=\frac{3}{2\sqrt{8}}
\frac{1}{\sqrt{1+\frac{3}{\sqrt{8}}}}=\frac{3}{2\sqrt[4]{8}}
\frac{1}{\sqrt{3+2\sqrt{2}}}\\
 &\Leftrightarrow x=\frac{3\sqrt[4]{2}}{4}\frac{1}{\sqrt{(1+\sqrt{2})^2}}=\frac{3\sqrt[4]{2}(\sqrt{2}-1)}{4}.
\end{align*}

 \item  Pour $x\in]0,+\infty[\setminus\left\{\frac{1}{100},\frac{1}{10},1\right\}$,

\begin{align*}
\mbox{ln}_x(10)+2\mbox{ln}_{10x}(10)&+3\mbox{ln}_{100x}(10)=0\Leftrightarrow\frac{\ln(10)}{\ln
x}+2\frac{\ln(10)}{\ln(10x)}+3\frac{\ln(10)}{\ln(100x)}=0\\
 &\Leftrightarrow\frac{(\ln x+\ln(10))(\ln x+2\ln(10))+2\ln x(\ln x+2\ln(10))+3\ln x(\ln x+\ln(10))}{\ln x(\ln x+\ln(10))(\ln
x+2\ln(10))}=0\\
 &\Leftrightarrow6\ln^2x+10\ln(10)\times\ln x+2\ln^2(10)=0\\
 &\Leftrightarrow\ln x\in\left\{\frac{-5\ln(10)+\sqrt{13\ln^2(10)}}{6},\frac{-5\ln(10)-\sqrt{13\ln^2(10)}}{6}\right\}\\
 &\Leftrightarrow x\in\left\{10^{(-5-\sqrt{13})/6},10^{(-5+\sqrt{13})/6}\right\}.
\end{align*}
Comme aucun de ces deux nombres n'est dans $\left\{\frac{1}{100},\frac{1}{10},1\right\}$, $\mathcal{S}=\left\{10^{(-5-\sqrt{13})/6},10^{(-5+\sqrt{13})/6}\right\}$.

 \item  Soit $x\in\Rr$.

\begin{align*}
2^{2x}-3^{x-\frac{1}{2}}=3^{x+\frac{1}{2}}-2^{2x-1}&\Leftrightarrow2^{2x}+2^{2x-1}=3^{x+\frac{1}{2}}+3^{x-\frac{1}{2}}\\
 &\Leftrightarrow2^{2x-1}(2+1)=3^{x-\frac{1}{2}}(3+1)\Leftrightarrow3\times2^{2x-1}=4\times3^{x-\frac{1}{2}}\\
 &\Leftrightarrow2^{2x-3}=3^{x-\frac{3}{2}}\Leftrightarrow(2x-3)\ln2=\left(x-\frac{3}{2}\right)\ln3\\
 &\Leftrightarrow x=\frac{3\ln2-\frac{3}{2}\ln3}{2\ln2-\ln3}\Leftrightarrow x=\frac{3}{2}.
\end{align*}
\end{enumerate}
\fincorrection
\correction{005101}
Pour $x>0$, $(x^x)^x=e^{x\ln(x^x)}=e^{x^2\ln x}$ et $x^{(x^x)}=e^{x^x\ln x}$. Par suite,

$$\forall x>0,\;\frac{(x^x)^x}{x^{(x^x)}}=\text{exp}(\ln x(x^2-x^x)).$$
Or, $x^2-x^x=-x^x(1-x^{2-x})=-e^{x\ln x}(1-e^{(2-x)\ln x})$. Quand $x$ tend vers $+\infty$, $(2-x)\ln x$ tend vers
$-\infty$. Donc, $1-e^{(2-x)\ln x}$ tend vers $1$ puis $x^2-x^x$ tend vers $-\infty$. Mais alors, $\ln x(x^2-x^x)$ tend
vers $-\infty$, puis $\frac{(x^x)^x}{x^{(x^x)}}=\mbox{exp}(\ln x(x^2-x^x))$ tend vers $0$.

\begin{center}
\shadowbox{
$\lim_{x\rightarrow +\infty}\frac{(x^x)^x}{x^{(x^x)}}=0.$
}
\end{center}
\fincorrection
\correction{005102}
On notera $\mathcal{C}_i$ le graphe de $f_i$.
\begin{enumerate}
 \item  $f_1$ est définie et continue sur $\Rr$, dérivable sur $\Rr\setminus\left\{-2,\frac{1}{2}\right\}$. On précise dans
un tableau l'expression de $f_1(x)$ suivant les valeurs de $x$.

$$
\begin{array}{|c|lcc|ccc|ccr|}
\hline
x&-\infty& &\multicolumn{2}{c}{-2}& &\multicolumn{2}{c}{1/2}& &+\infty\\
\hline
|2x-1|& &-2x+1& & &-2x+1& & &2x-1& \\
\hline
|x+2|& &-x-2& & &x+2& & &x+2& \\
\hline
f_1(x)& &4& & &-2x& & &6x-4& \\
\hline
\end{array}
$$
On en déduit $\mathcal{C}_1$.

%$$\includegraphics{../images/img005102-1}$$

 \item  Soit $x\in\Rr$. $\ch x\geq1$ et donc $f_2(x)$ existe et $f_2(x)\geq0$. $f_2$ est donc définie sur $\Rr$. De
plus, $f_2$ est continue et dérivable sur $\Rr$, paire.
Puisque la fonction $x\mapsto\ch x$ est strictement croissante sur $\Rr^+$ à valeurs dans $]0,+\infty[$ et que la fonction $x\mapsto\ln x$ est
strictement croissante sur $]0,+\infty[$, $f_2$ est strictement croissante sur $\Rr^+$ et, par parité, strictement
décroissante sur $\Rr^-$.
$f_2$ est paire et donc $f_2'$ est impaire. Par suite, $f_2'(0)=0$ et $\mathcal{C}_2$ admet l'axe des abscisses pour
tangente en $(0,f_2(0))=(0,0)$.
\textbf{Etude en} $+{\infty}$. Pour $x\geq0$,

$$f_2(x)=\ln\left(\frac{1}{2}(e^x+e^{-x}))=\ln(e^x+e^{-x}\right)-\ln2=\ln(e^x(1+e^{-2x}))-\ln2=x-\ln2+\ln(1+e^{-2x}).$$
Quand $x$ tend vers $+\infty$, $e^{-2x}$ tend vers $0$ et donc, $\ln(1+e^{-2x})$ tend vers $0$. On en déduit que
$\lim_{x\rightarrow +\infty}f_2(x)=+\infty$. De plus, $\lim_{x\rightarrow +\infty}(f_2(x)-(x-\ln2))=0$ et la droite $(D)$ d'équation
$y=x-\ln2$ est asymptote à $\mathcal{C}_2$ en $+\infty$. Par symétrie par rapport à la droite $(Oy)$, la droite $(D')$ d'équation $y=-x-\ln2$ est asymptote à $\mathcal{C}_2$ en $-\infty$. Enfin, pour tout réel $x$,

$$f_2(x)-(x-\ln2)=\ln(1+e^{-2x})>\ln1=0,$$
et $\mathcal{C}_2$ est strictement au-dessus de $(D)$ sur $\Rr$. De même, $\mathcal{C}_2$ est strictement au-dessus de
$(D')$ sur $\Rr$.
On en déduit $\mathcal{C}_2$.


%$$\includegraphics{../images/img005102-2}$$

 \item  $f_3$ est définie et continue sur $\Rr$, dérivable sur $\Rr\setminus\{-1,1\}$.
\textbf{Etude en} $-\infty$. Soit $x\leq-1$.

$$f_3(x)=x+\sqrt{x^2-1}=\frac{(x+\sqrt{x^2-1})(x-\sqrt{x^2-1})}{x-\sqrt{x^2-1}}=\frac{1}{x-\sqrt{x^2-1}}.$$
Or, quand $x$ tend vers $-\infty$, $x-\sqrt{x^2-1}$ tend vers $-\infty$ et donc $\lim_{x\rightarrow -\infty}f_3(x)=0$.
\textbf{Etude en} $+\infty$. Immédiatement, $\lim_{x\rightarrow +\infty}f_3(x)=+\infty$. Ensuite, pour $x\geq1$,

$$\frac{f_3(x)}{x}=\frac{x+\sqrt{x^2-1}}{x}=1+\sqrt{1-\frac{1}{x^2}},$$
qui tend vers $2$ quand $x$ tend vers $+\infty$. Mais alors,

$$f_3(x)-2x=-x+\sqrt{x^2-1}=\frac{(-x+\sqrt{x^2-1})(-x-\sqrt{x^2-1})}{-x-\sqrt{x^2-1}}=-\frac{1}{x+\sqrt{x^2-1}}.$$
On en déduit que $\lim_{x\rightarrow +\infty}(f_3(x)-2x)=0$ et donc que la droite $(D)$ d'équation $y=2x$ est asymptote à $\mathcal{C}_3$ en $+\infty$.
\textbf{Etude en} $1$. Pour $x>1$,

$$\frac{f_3(x)-f_3(1)}{x-1}=\frac{(x-1)+\sqrt{(x-1)(x+1)}}{x-1}=1+\sqrt{\frac{x+1}{x-1}},$$
et pour $x\in]-1,1[$,

$$\frac{f_3(x)-f_3(1)}{x-1}=\frac{(x-1)+\sqrt{(-x+1)(x+1)}}{-(-x+1)}=1-\sqrt{\frac{x+1}{-x+1}}.$$
Par suite, $\lim_{x\rightarrow 1,\;x>1}\frac{f_3(x)-f_3(1)}{x-1}=+\infty$ et
$\lim_{x\rightarrow 1,\;x<1}\frac{f_3(x)-f_3(1)}{x-1}=-\infty$. On en déduit que $f_3$ n'est pas dérivable en $1$, mais que
$\mathcal{C}_3$ admet deux demi-tangentes parallèles à $(Oy)$ au point de $\mathcal{C}_3$ d'abscisse $1$. Les résultats
sont analogues en $-1$.
\textbf{Etude des variations de} $\bf{f_3}$. Pour $x\in]-\infty,-1[\cup]1,+\infty[$, $f_3(x)=x+\sqrt{x^2-1}$ et donc

$$f_3'(x)=1+\frac{x}{\sqrt{x^2-1}}=\frac{x+\sqrt{x^2-1}}{\sqrt{x^2-1}}.$$

Si $x>1$, on a $x+\sqrt{x^2-1}>0$ et donc, $f_3'(x)>0$. Si $x<-1$, on a

$$\sqrt{x^2-1}<\sqrt{x^2}=|x|=-x,$$
et donc, $x+\sqrt{x^2-1}<0$ puis $f_3'(x)<0$. Ainsi, $f_3$ est strictement décroissante sur $]-\infty,-1[$ et
strictement croissante sur $]1,+\infty[$.
Pour $x\in]-1,1[$, $f_3(x)=x+\sqrt{-x^2+1}$ et donc

$$f_3'(x)=1-\frac{x}{\sqrt{-x^2+1}}=\frac{\sqrt{-x^2+1}-x}{\sqrt{-x^2+1}}.$$
Si $x\in]-1,0]$, on a clairement $f_3'(x)>0$. Si x$\in[0,1[$, par stricte croissance de la fonction $x\mapsto x^2$ sur $\Rr^+$, on a

$$\mbox{sgn}(f_3'(x))=\mbox{sgn}(\sqrt{-x^2+1}-x)=\mbox{sgn}((-x^2+1)-x^2)=\mbox{sgn}(1-2x^2)=\mbox{sgn}((1-x\sqrt{2})
(1+x\sqrt{2}))=\mbox{sgn}[\frac{1}{\sqrt{2}}-x).$$
Donc, $f_3'$ est strictement positive sur $\left[0,\frac{1}{\sqrt{2}}\right[$, strictement négative sur $\left]\frac{1}{\sqrt{2}},1\right[$
et s'annule en $\frac{1}{\sqrt{2}}$.
En résumé, $f_3'$ est strictement négative sur $]-\infty,-1[$ et sur $\left]\frac{1}{\sqrt{2}},1\right[$ et strictement positive
sur $\left]-1,\frac{1}{\sqrt{2}}\right[$ et sur $]1,+\infty[$. $f_3$ est donc strictement croissante sur $]-\infty,-1]$ et sur
$\left[\frac{1}{\sqrt{2}},1\right[$ et strictement décroissante sur $\left[-1,\frac{1}{\sqrt{2}}\right]$ et sur $[1,+\infty[$.
On en déduit $\mathcal{C}_3$.

%$$\includegraphics{../images/img005102-3}$$

 \item  $f_4$ est définie sur $\Rr\setminus\left(\frac{\pi}{2}+\pi\Zz\right)$, $2\pi$-périodique et paire. On étudie
donc $f_4$ sur $\left[0,\frac{\pi}{2}\right[\cup\left]\frac{\pi}{2},\pi\right]$.
\textbf{Etude des variations de} $\bf{f_4}$. Pour $x\in\left[0,\frac{\pi}{2}\right[$, $f_4(x)=\tan x+\cos x$ et donc,

$$f_4'(x)=\frac{1}{\cos^2x}-\sin x\geq 1-1=0,$$
avec égalité si et seulement si $\sin x=\cos^2x=1$ ce qui est impossible. Donc, $f_4'$ est strictement positive sur
$\left[0,\frac{\pi}{2}\right[$ et $f_4$ est strictement croissante sur $\left[0,\frac{\pi}{2}\right[$.
Pour $x\in\left]\frac{\pi}{2},\pi\right]$, $f_4(x)=-\tan x+\cos x$ et $f_4$ est strictement décroissante sur
$\left]\frac{\pi}{2},\pi\right]$ en tant que somme de deux fonctions strictement décroissantes sur $\left]\frac{\pi}{2},\pi\right]$.
On a immédiatement
$\displaystyle\lim_{\substack{x\rightarrow\frac{\pi}{2}\\
x<\frac{\pi}{2}}}f_4(x)=\displaystyle\lim_{\substack{x\rightarrow\frac{\pi}{2}\\
x>\frac{\pi}{2}}}f_4(x)=+\infty$.
On en déduit $\mathcal{C}_4$.

%$$\includegraphics{../images/img005102-4}$$

 \item  Soit $x>0$. $x$ n'est pas nul donc $\frac{1}{x}$ existe puis $1+\frac{1}{x}>0$ et $f_6(x)$ existe.
\textbf{Etude en} $\bf{0.}$ Pour $x>0$, $x\ln(1+\frac{1}{x})=-x\ln x+x\ln(1+x)$. Par suite, $x\ln(1+\frac{1}{x})$ tend
vers $0$ quand $x$ tend vers $0$ par valeurs supérieures et donc $f_5(x)=\mbox{exp}(x\ln(1+\frac{1}{x}))$ tend vers
$1$.
Posons encore $f_5(0)=1$ et étudions la dérivabilité de $f_5$ en $0$. Pour $x>0$,

$$\frac{f_5(x)-f_5(0)}{x-0}=\frac{1}{x}\left(\text{exp}(x\ln(1+\frac{1}{x}))-1\right)
=\frac{\text{exp}\left(x\ln(1+\frac{1}{x})\right)-1}{x\ln\left(1+\frac{1}{x}\right)}\ln\left(1+\frac{1}{x}\right).$$
Or, $x\ln\left(1+\frac{1}{x}\right)$ tend vers $0$ quand $x$ tend vers $0$, et donc

$$\lim_{\substack{x\rightarrow0\\
x>0}}\frac{\mbox{exp}(x\ln\left(1+\frac{1}{x})\right)-1}{x\ln\left(1+\frac{1}{x}\right)}=\lim_{y\rightarrow 0}\frac{e^y-1}{y}=1.$$
D'autre part, $\ln\left(1+\frac{1}{x}\right)$ tend vers $+\infty$ quand $x$ tend vers $0$ par valeurs supérieures. Finalement,

$$\lim_{\substack{x\rightarrow0\\
x>0}}\frac{f_5(x)-f_5(0)}{x-0}=+\infty.$$
Ainsi, $f_5$ n'est pas dérivable en $0$ mais $\mathcal{C}_5$ admet l'axe des ordonnées pour tangente en
$(0,f_5(0))=(0,1)$.
\textbf{Etude en} $+\infty.$ Pour $x>0$,
$x\ln\left(1+\frac{1}{x}\right)=\frac{\ln\left(1+\frac{1}{x}\right)}{\frac{1}{x}}$ et 
donc $\lim_{x\rightarrow +\infty}x\ln\left(1+\frac{1}{x}\right)=\lim_{y\rightarrow 0}\frac{\ln(1+y)}{y}=1$.
Par suite,

$$\lim_{x\rightarrow +\infty}f_5(x)=e.$$
\textbf{Etude des variations de} $\bf{f_5.}$ Pour $x>0$, $f_5(x)>0$ puis $\ln(f_5(x))=x\ln(1+\frac{1}{x})$. Par suite,
pour $x>0$,

$$f_5'(x)=f_5(x)\ln(f_5)'(x)=f_5(x)\left(\ln\left(1+\frac{1}{x}\right)+\frac{x(-\frac{1}{x^2})}{1+\frac{1}{x}}\right)=f_5(x)g(x),$$
où $g(x)=\ln\left(1+\frac{1}{x}\right)-\frac{1}{1+x}$. Sur $]0,+\infty[$, $f_5'$ est du signe de $g$.
Pour déterminer le signe de $g$, étudions d'abord les variations de $g$ sur $]0,+\infty[$. $g$ est dérivable sur
$]0,+\infty[$ et pour $x>0$,

$$g'(x)=\frac{-\frac{1}{x^2}}{1+\frac{1}{x}}+\frac{1}{(x+1)^2}=-\frac{1}{x(x+1)}+\frac{1}{(x+1)^2}
=\frac{-1}{x(x+1)^2}<0.$$
$g$ est donc strictement décroissante sur $]0,+\infty[$, et puisque $\lim_{x\rightarrow +\infty}g(x)=0$, $g$ est strictement
positive sur $]0,+\infty[$. Il en est de même de $f_5'$. $f_5$ est strictement croissante sur $]0,+\infty($.
On en déduit $\mathcal{C}_5$.

%$$\includegraphics{../images/img005102-5}$$


 \item  \textbf{Domaine de définition de} $\bf{f_6.}$ Soit $x\in\Rr$.

\begin{align*}
f_6(x)\;\mbox{existe}&\Leftrightarrow x^2-5x+6>0\;\mbox{et}\;1-\mbox{log}_{\frac{1}{2}}(x^2-5x+6)>0
\Leftrightarrow x^2-5x+6>0\;\mbox{et}\;\frac{\ln(x^2-5x+6)}{\ln\frac{1}{2}}<1\\
 &\Leftrightarrow x^2-5x+6>0\;\mbox{et}\;\ln(x^2-5x+6)>\ln\frac{1}{2}\Leftrightarrow x^2-5x+6>\frac{1}{2}\\
 &\Leftrightarrow x^2-5x+\frac{11}{2}>0\Leftrightarrow x\in]-\infty,\frac{5-\sqrt{3}}{2}[\cup]\frac{5+\sqrt{3}}{2},+\infty[=\mathcal{D}_f.
\end{align*}
\textbf{Variations de} $\bf{f_6.}$ La fonction $x\mapsto x^2-5x+6$ est strictement décroissante sur
$\left]-\infty,\frac{5}{2}\right]$ et strictement croissante sur $\left[\frac{5}{2},+\infty\right[$. Comme
$\frac{5+\sqrt{3}}{2}>\frac{5}{2}$ et que $\frac{5-\sqrt{3}}{2}<\frac{5}{2}$, la fonction $x\mapsto x^2-5x+6$ est
strictement décroissante sur$\left]-\infty,\frac{5-\sqrt{3}}{2}\right]$ et strictement croissante sur
$\left[\frac{5+\sqrt{3}}{2},+\infty\right[$, à valeurs dans $]0,+\infty[$, intervalle sur lequel la fonction logarithme néperien
est strictement croissante. La fonction $x\mapsto1+\frac{\ln(x^2-5x+6)}{\ln2}$ a le même sens de variations et
finalement $f_6$ est strictement décroissante sur $\left]-\infty,\frac{5-\sqrt{3}}{2}\right]$ et strictement croissante
sur $\left[\frac{5+\sqrt{3}}{2},+\infty\right[$.
\textbf{Axe de symétrie} Soit $x\in\Rr$. $x\in\mathcal{D}_f\Leftrightarrow\frac{5}{2}-x\in\mathcal{D}_f$ et de 
plus, $\left(\frac{5}{2}-x\right)^2-5\left(\frac{5}{2}-x\right)+6=x^2-5x+6$. Par suite,

$$\forall x\in D,\;f_6(\frac{5}{2}-x)=f_6(x).$$

$\mathcal{C}_6$ admet donc la droite d'équation $x=\frac{5}{2}$ pour axe de symétrie.

Le calcul des limites étant immédiat, on en déduit $\mathcal{C}_6$.

%$$\includegraphics{../images/img005102-6}$$

\end{enumerate}
\fincorrection


\end{document}

