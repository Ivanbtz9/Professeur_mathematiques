\documentclass[11pt]{article}

 %Configuration de la feuille 
 
\usepackage{amsmath,amssymb,enumerate,graphicx,pgf,tikz,fancyhdr}
\usepackage[utf8]{inputenc}
\usetikzlibrary{arrows}
\usepackage{geometry}
\usepackage{tabvar}
\geometry{hmargin=2.2cm,vmargin=1.5cm}\pagestyle{fancy}
\lfoot{\bfseries http://www.bibmath.net}
\rfoot{\bfseries\thepage}
\cfoot{}
\renewcommand{\footrulewidth}{0.5pt} %Filet en bas de page

 %Macros utilisées dans la base de données d'exercices 

\newcommand{\mtn}{\mathbb{N}}
\newcommand{\mtns}{\mathbb{N}^*}
\newcommand{\mtz}{\mathbb{Z}}
\newcommand{\mtr}{\mathbb{R}}
\newcommand{\mtk}{\mathbb{K}}
\newcommand{\mtq}{\mathbb{Q}}
\newcommand{\mtc}{\mathbb{C}}
\newcommand{\mch}{\mathcal{H}}
\newcommand{\mcp}{\mathcal{P}}
\newcommand{\mcb}{\mathcal{B}}
\newcommand{\mcl}{\mathcal{L}}
\newcommand{\mcm}{\mathcal{M}}
\newcommand{\mcc}{\mathcal{C}}
\newcommand{\mcmn}{\mathcal{M}}
\newcommand{\mcmnr}{\mathcal{M}_n(\mtr)}
\newcommand{\mcmnk}{\mathcal{M}_n(\mtk)}
\newcommand{\mcsn}{\mathcal{S}_n}
\newcommand{\mcs}{\mathcal{S}}
\newcommand{\mcd}{\mathcal{D}}
\newcommand{\mcsns}{\mathcal{S}_n^{++}}
\newcommand{\glnk}{GL_n(\mtk)}
\newcommand{\mnr}{\mathcal{M}_n(\mtr)}
\DeclareMathOperator{\ch}{ch}
\DeclareMathOperator{\sh}{sh}
\DeclareMathOperator{\vect}{vect}
\DeclareMathOperator{\card}{card}
\DeclareMathOperator{\comat}{comat}
\DeclareMathOperator{\imv}{Im}
\DeclareMathOperator{\rang}{rg}
\DeclareMathOperator{\Fr}{Fr}
\DeclareMathOperator{\diam}{diam}
\DeclareMathOperator{\supp}{supp}
\newcommand{\veps}{\varepsilon}
\newcommand{\mcu}{\mathcal{U}}
\newcommand{\mcun}{\mcu_n}
\newcommand{\dis}{\displaystyle}
\newcommand{\croouv}{[\![}
\newcommand{\crofer}{]\!]}
\newcommand{\rab}{\mathcal{R}(a,b)}
\newcommand{\pss}[2]{\langle #1,#2\rangle}
 %Document 

\begin{document} 

\begin{center}\textsc{{\huge }}\end{center}

% Exercice 2030


\vskip0.3cm\noindent\textsc{Exercice 1} - Somme de deux lois de Poisson
\vskip0.2cm
Soit $X$ et $Y$ deux variables aléatoires indépendantes suivant des lois de Poisson de paramètre respectif $\lambda$ et $\mu$. Démontrer, à l'aide des fonctions génératrices, que $Z=X+Y$, suit une loi de Poisson de paramètre $\lambda+\mu$.


% Exercice 2035


\vskip0.3cm\noindent\textsc{Exercice 2} - Sur la variance
\vskip0.2cm
Soit $X$ une variable aléatoire admettant un moment d'ordre 2. Démontrer que $E\big((X-a)^2\big)$ est minimal
pour $a=E(X)$.


% Exercice 2033


\vskip0.3cm\noindent\textsc{Exercice 3} - Variable aléatoire quasi-certaine
\vskip0.2cm
On dit qu'une variable aléatoire réelle $X$ est quasi-certaine lorsqu'il existe un réel $a$ tel que $P(X=a)=1$. Soit $X$ une variable aléatoire réelle telle que $X(\Omega)$ soit fini ou dénombrable. Démontrer que $X$ est quasi-certaine si et seulement si $V(X)=0$.


% Exercice 2026


\vskip0.3cm\noindent\textsc{Exercice 4} - Lancer de dé
\vskip0.2cm
On jette 3600 fois un dé équilibré. Minorer la probabilité que le nombre d'apparitions du numéro 1 soit compris entre 480 et 720.


% Exercice 2040


\vskip0.3cm\noindent\textsc{Exercice 5} - Pièces défectueuses
\vskip0.2cm
Une usine fabrique des pièces dont une proportion inconnue $p$ est défectueuse, et on souhaite trouver une valeur approchée de $p$. On effectue un prélèvement de $n$ pièces. On suppose que le prélèvement se fait sur une population très grande, et donc qu'il peut s'apparenter à une suite de $n$  tirages indépendants avec remise. On note $X_n$ la variable aléatoire égale au nombre de pièces défectueuses et on souhaite quantifier le fait que $X_n/n$ approche $p$.
\begin{enumerate}
\item Quelle est la loi de $X_n$? Sa moyenne? Sa variance?
\item Démontrer que, pour tout $\veps>0$, $P\left(\left|\frac{X_n}n-p\right|\geq\veps\right)\leq\frac 1{4n\veps^2}.$
\item En déduire une condition sur $n$ pour que $X_n/n$ soit une valeur approchée de $p$ à $10^{-2}$ près avec une probabilité supérieure ou égale à $95\%$.
\end{enumerate}


% Exercice 3145


\vskip0.3cm\noindent\textsc{Exercice 6} - Une variante de l'inégalité de Markov
\vskip0.2cm
Soit $X$ une variable aléatoire réelle finie à valeurs dans $\mathbb R_+$, $f:\mathbb R_+\to\mathbb R_+^*$ une fonction croissante. Démontrer que
$$P(X\geq a)\leq \frac{E(f(X))}{f(a)}.$$




\vskip0.5cm
\noindent{\small Cette feuille d'exercices a été conçue à l'aide du site \textsf{https://www.bibmath.net}}

%Vous avez accès aux corrigés de cette feuille par l'url : https://www.bibmath.net/ressources/justeunefeuille.php?id=29647
\end{document}