\documentclass[a4paper,10pt]{article}



\usepackage{fancyhdr} % pour personnaliser les en-têtes
\usepackage[utf8]{inputenc}
\usepackage[T1]{fontenc}
\usepackage{lastpage}
\usepackage[frenchb]{babel}
\usepackage{amsfonts,amssymb}
\usepackage{amsmath,amsthm,mathtools}
\usepackage{paralist}
\usepackage{xspace,xypic}
\usepackage{xcolor,multicol,tabularx}
\usepackage{variations}
\usepackage{xypic}
\usepackage{eurosym,multicol}
\usepackage{graphicx}
\usepackage{mathdots}%faire des points suspendus en diagonale
\usepackage[np]{numprint}
\usepackage{hyperref} 
\usepackage{listings} % pour écrire des codes avec coloration syntaxique  

\usepackage{tikz}
\usetikzlibrary{calc, arrows, plotmarks,decorations.pathreplacing}
\usepackage{colortbl}
\usepackage{multirow}
\usepackage[top=2cm,bottom=1.5cm,right=2cm,left=1.5cm]{geometry}

\newtheorem{thm}{Théorème}
\newtheorem*{pro}{Propriété}
\newtheorem*{exemple}{Exemple}

\theoremstyle{definition}
\newtheorem*{remarque}{Remarque}
\theoremstyle{definition}
\newtheorem{exo}{Exercice}
\newtheorem{definition}{Définition}


\newcommand{\vtab}{\rule[-0.4em]{0pt}{1.2em}}
\newcommand{\V}{\overrightarrow}
\renewcommand{\thesection}{\Roman{section} }
\renewcommand{\thesubsection}{\arabic{subsection} }
\renewcommand{\thesubsubsection}{\alph{subsubsection} }
\newcommand*{\transp}[2][-3mu]{\ensuremath{\mskip1mu\prescript{\smash{\mathrm t\mkern#1}}{}{\mathstrut#2}}}%

\newcommand{\K}{\mathbb{K}}
\newcommand{\C}{\mathbb{C}}
\newcommand{\R}{\mathbb{R}}
\newcommand{\Q}{\mathbb{Q}}
\newcommand{\Z}{\mathbb{Z}}
\newcommand{\N}{\mathbb{N}}
\newcommand{\p}{\mathbb{P}}

\renewcommand{\Im}{\mathop{\mathrm{Im}}\nolimits}



\definecolor{vert}{RGB}{11,160,78}
\definecolor{rouge}{RGB}{255,120,120}
% Set the beginning of a LaTeX document
\pagestyle{fancy}
\lhead{Optimal Sup Spé, groupe IPESUP}\chead{Année~2021-2022}\rhead{Niveau: Première année de PCSI }\lfoot{M. Botcazou}\cfoot{\thepage}\rfoot{mail: ibotca52@gmail.com }\renewcommand{\headrulewidth}{0.4pt}\renewcommand{\footrulewidth}{0.4pt}

\begin{document}
 	
	
	\begin{center}
		\Large \sc colle 22 = Variables aléatoires et calculs différentiels
	\end{center}




\section*{Variables aléatoires:}\hfill\\%[-0.25cm]
\begin{minipage}{1\linewidth}
	\begin{minipage}[t]{0.48\linewidth}
		\raggedright
		
			\begin{exo}\quad\\
			On dispose de $n$ urnes numérotées de $1$ à $n$, l'urne numérotée $k$ comprenant $k$ boules numérotées de $1$ à $k$. On choisit d'abord une urne, puis une boule dans cette urne, et on note $Y$ la variable aléatoire du numéro obtenu. Quelle est la loi de Y? Son espérance?
			
			\centering
			\rule{1\linewidth}{0.6pt}
		\end{exo}
	\begin{exo}\quad\\
		On jette $3600$ fois un dé équilibré. Minorer la probabilité que le nombre d'apparitions du numéro $1$ soit compris entre $480$ et $720$.
		
		\centering
		\rule{1\linewidth}{0.6pt}
	\end{exo}
	
	\begin{exo}\quad\\
		Soit $X$ une variable aléatoire réelle définie sur un espace probabilisé fini. Démontrer que $$E(X)^2\leq E(X^2)$$
		\centering
		\rule{1\linewidth}{0.6pt}
	\end{exo}
	
\begin{exo}\quad\\
	
	Soit $X$ une variable aléatoire prenant ses valeurs dans $\{0,1,...,N\}$. Démontrer que 
	$$E(X) = \sum_{n=0}^{N-1} P(X>n)$$
	
	\centering
	\rule{1\linewidth}{0.6pt}
\end{exo}
	



		
		
		
	\end{minipage}	
	\hfill\vrule\hfill
	\begin{minipage}[t]{0.48\linewidth}
		\raggedright
		
					
	
		\begin{exo}\quad\\[0.2cm]
		Soit $X,Y$ deux variables aléatoires indépendantes suivant la loi uniforme sur $\{1,...,n\}$.
		\begin{enumerate}
			\item Déterminer $P(X=Y)$.
			\item Déterminer $P(X\geq Y)$.
			\item Déterminer la loi de $X+Y$.
		\end{enumerate}
		
		\centering
		\rule{1\linewidth}{0.6pt}
	\end{exo}
	
	\begin{exo}\quad\\[0.2cm]
		Une entreprise souhaite recrute un cadre. $n$ personnes se présentent pour le poste. Chacun d'entre eux passe à tour de rôle un test, et le premier qui réussit le test est engagé. La probabilité de réussir le test est $p\in ]0,1[$. On pose également $q=1-p$. On définit la variable aléatoire $X$ par $X=k$ si le k-ième candidat qui réussit le test est engagé, et $X=n+1$ si personne n'est engagé.
		\begin{enumerate}
			\item Déterminer la loi de $X$.
			\item En dérivant la fonction  $x\mapsto \sum_{k=0}^{n}x^k$.\\[0.1cm]
			En déduire l'espérance de $X$. 
			\item Quelle est la valeur minimale de $p$ pour avoir plus d'une chance sur deux de recruter l'un des candidats? 
		\end{enumerate}
		\centering
		\rule{1\linewidth}{0.6pt}
	\end{exo}	

	


		
			
		
		
	\end{minipage}
\end{minipage}	


\section*{Fonctions à plusieurs variables:}\hfill\\%[-0.25cm]
\begin{minipage}{1\linewidth}
	\begin{minipage}[t]{0.48\linewidth}
		\raggedright
		
		\begin{exo}\quad\\[0.2cm]
			Etudier l'existence et la valeur éventuelle des limites suivantes :
			\begin{multicols}{2}
				\begin{enumerate}
					\item  $ \frac{xy}{x^2+y^2}$ en $(0,0)$
					
					\item  $ \frac{x^3+y^3}{x^2+y^4}$  en $(0,0)$
					
					
					\item[4.]  $ \frac{x^2y^2}{x^2+y^2}$ en $(0,0)$
					\item[5.]   $ \frac{1-\cos\sqrt{|xy|}}{|y|}$ en $(0,0)$	
				\end{enumerate}

			\end{multicols}
			\begin{enumerate}
				\item[3.]  $ \frac{\sqrt{x^2+y^2}}{|x|\sqrt{|y|}+|y|\sqrt{|x|}}$  en $(0,0)$
			\end{enumerate}
			\centering
			\rule{1\linewidth}{0.6pt}
		\end{exo}
	
	\begin{exo}\quad\\[0.2cm]
		Soit $\begin{array}[t]{cccc}f~:&\R^2&\longrightarrow&\R\\
		&(x,y)&\mapsto&\left\{
		\begin{array}{l}
		0\;\text{si}\;y=0\\
		y^2\sin\left(\frac{x}{y}\right)\;\text{si}\;y\neq0
		\end{array}
		\right.
		\end{array}
		$.
		\begin{enumerate}
			\item  Etudier la continuité de $f$.
			
			\item  Etudier l'existence et la valeur éventuelle de dérivées partielles d'ordre 1 sur $\R^2$.
			\item Étudier $\frac{\partial^2f}{\partial x\partial y}$ et $\frac{\partial^2f}{\partial y\partial x}$ en $(0,0)$.
		\end{enumerate}
		
		\centering
		\rule{1\linewidth}{0.6pt}
	\end{exo}

		
	\end{minipage}	
	\hfill\vrule\hfill
	\begin{minipage}[t]{0.48\linewidth}
		\raggedright
		
			
		
		\begin{exo}\quad\\[0.2cm]
			Pour  $(x,y)\in\R^2$, on pose \\[0.2cm]$f(x,y) =\left\{
			\begin{array}{l}
			\frac{xy(x^2-y^2)}{x^2+y^2}\;\text{si}\;(x,y)\neq(0,0)\\
			\rule{0mm}{5mm}0\;\text{si}\;(x,y)=(0,0)
			\end{array}
			\right.$.\hfil\\[0.2cm] Montrer que $f$ est de classe $C^1$ (au moins) sur $\R^2$.
			
			
			\centering
			\rule{1\linewidth}{0.6pt}
		\end{exo}
		
		\begin{exo}\quad\\[0.2cm]
			Soit $a$ un réel strictement positif donné. Trouver le minimum de $f(x,y)=\sqrt{x^2+(y-a)^2}+\sqrt{y^2+(x-a)^2}$.
			
			\centering
			\rule{1\linewidth}{0.6pt}
		\end{exo}
	
		\begin{exo}\quad\\[0.2cm]
			Trouver les extrema locaux de 
			
			\begin{enumerate}
				\item  $\begin{array}[t]{cccc}
				f~:&\R^2&\rightarrow&\R\\
				&(x,y)&\mapsto&x^2+xy+y^2+2x+3y
				\end{array}$ 
				\item  $\begin{array}[t]{cccc}
				f~:&\R^2&\rightarrow&\R\\
				&(x,y)&\mapsto&x^4+y^4-4xy
				\end{array}$ 
			\end{enumerate}
			
			\centering
			\rule{1\linewidth}{0.6pt}
		\end{exo}	
		
	

		
		
		
		
	\end{minipage}
\end{minipage}	



\end{document}