\documentclass[11pt,a4paper]{article}
\oddsidemargin 0.1 cm \evensidemargin 0.1cm \textwidth 16cm
\textheight24cm
\setlength{\topmargin}{0pt}\setlength{\headsep}{0pt}\pagestyle{empty}

\usepackage{graphicx}
\usepackage{variations}
\usepackage{enumerate}
\usepackage{amssymb}
\usepackage[latin1]{inputenc}  
\usepackage{fontenc}   
\usepackage{amsmath}
\usepackage{amsthm}
\usepackage{amsthm}
\usepackage{xypic}
\usepackage{variations}
%\usepackage{fancyhdr}
%\usepackage{xcolor}
%\usepackage{pstricks-add}
\usepackage[francais]{babel}
%\usepackage[french]{babel}
\newtheorem{defi}{D\'{e}finition}
\newtheorem{thm}{Th\'{e}or\`{e}me}
\newtheorem{rmq}{Remarque}
\newtheorem{prop}{Propri\'{e}t\'{e}}
\newtheorem{prop-def}{Propri\'{e}t\'{e}-D\'{e}finition}
\newtheorem{ex}{Exemple}
\newtheorem{exs}{Exemples}
\newtheorem{exer}{Exercice}
%\newtheorem{proof}{D\'{e}monstration}
\def\di{\displaystyle}
\newcommand{\vtab}{\rule[-0.4em]{0pt}{1.2em}}
\usepackage[top=0.5cm,bottom=0.5cm,right=1.5cm,left=1.5cm]{geometry}
\usepackage{pstricks,pst-plot} 
%\usepackage{framed}
\usepackage{amsmath}
%\usepackage{amssymb}
\usepackage{fancyhdr}
\usepackage{fancybox}
\usepackage{multicol}
%\usepackage{xcolor}
\usepackage{epsfig}
\usepackage{pifont}
%\usepackage[framed]{ntheorem}
%\usepackage[frenchb]{babel}
\usepackage{tabularx}
\def\R{{\mathbb R}}
\newtheorem{Rem}{Remarque}
\newcommand{\V}{\overrightarrow}
\newcommand{\Rep}{(O;\V{\imath};\V{\jmath})}
\newcommand{\Coor}[2]{\begin{pmatrix} #1\\#2 \end{pmatrix}}

\begin{document}
\title{Vecteurs - Activit� d'introduction}         % Enter your title between curly braces
\author{}        % Enter your name between curly braces
\date{}          % Enter your date or \today between curly braces
\maketitle



\noindent On consid�re la figure ci-dessous :

\psset{unit=0.8}
    $$\begin{pspicture}(6,5.3)
      \psgrid[gridlabels=0,subgriddiv=0,griddots=10](13,5)
      \rput(2.6,3.3){$A$}
      \rput(8.6,1.5){$B$}
			\rput(4.6,0.7){$O$}
      \rput(5.6,0.7){$I$}
      \rput(4.6,2.3){$J$}
      \psdots(3,3)
      \psdots(8,2)
			\psdots(5,1)
			\psdots(6,1)
			\psdots(5,2)
			\psline{->}(0,1)(13,1)
			\psline{->}(5,0)(5,5)
   \end{pspicture}$$
\hfill\\

\begin{enumerate}
\item Placer le point $A'$ obtenu � partir de $A$ en d�pla�ant $A$ de $4$ carreaux vers la droite, puis de $2$ carreaux vers le haut.\\
Faire subir � $B$ les m�mes d�placements. On notera $B'$ le point obtenu.\\[0.2cm]
\item Quelle est la nature exacte du quadrilat�re $AA'B'B$ ? Le d�montrer en utilisant le rep�re orthonorm� $(O;I,J)$.\\

\noindent\fcolorbox{green}{white}{
\begin{minipage}{1\linewidth}	
\begin{defi}\hfill\\
On dit que $A'$ et $B'$ sont associ�s respectivement � $A$ et � $B$ par une m�me translation.\\
On dit que $A'$ est l'image de $A$ par cette translation.
\end{defi}
\end{minipage}
}

\hfill\\

\item 
	\begin{enumerate}[$a.$]
	\item Placer le point C obtenu en d�pla�ant $B$ de $4$ carreaux vers la gauche et de $2$ carreaux vers le bas.
	\item Placer le point $D$ image de $B'$ par la translation qui transforme $A$ en $C$.
	\item Quelle est la nature du quadrilat�re $ACDB'$ ? Que repr�sente $B$ pour ce quadrilat�re ?
	\end{enumerate}
	\hfill\\[0.2cm]
\item 
	\begin{enumerate}[$a.$]
	\item Quelle est l'image de $B$ par la translation qui transforme $A$ en $B$?\\
	Que peut-on dire des points $A$, $B$ et $D$ ?
	\item Existe-t-il une translation qui transforme $B'$ en $B$ et $B$ en $C$? Si oui, la d�finir � l'aide des carreaux.
	\end{enumerate}
	\hfill\\[0.2cm]
\item La translation qui transforme $A'$ en $A$ est-elle la m�me que celle qui transforme $A$ en $A'$ ? Donner l'image de $B$ par chacune d'elles.\\[0.2cm]
\item On effectue successivement sur le point $C$ la translation qui transforme $A$ en $A'$ puis celle qui transforme $A'$ en $B'$. Quel point obtient-on? Par quelle translation unique pourrait-on remplacer ces deux translations successives ?
\end{enumerate}
\end{document}