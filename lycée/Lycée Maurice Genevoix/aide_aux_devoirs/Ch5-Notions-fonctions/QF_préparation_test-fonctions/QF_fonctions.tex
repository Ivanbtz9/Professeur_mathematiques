\documentclass[t,12pt]{beamer}
\usepackage[utf8]{inputenc}
\usepackage[T1]{fontenc}
\usepackage[frenchb]{babel}
\usepackage{amssymb,amsmath,graphicx,amsthm}


\newtheorem{defi}{Définition}
\newtheorem{thm}{Théorème}
\newtheorem{thm-def}{Théorème/Définition}
\newtheorem{rmq}{Remarque}
\newtheorem{prop}{Propriété}
\newtheorem{cor}{Corollaire}
\newtheorem{lem}{Lemme}
\newtheorem{ex}{Exemple}
\newtheorem{cex}{Contre-exemple}
\newtheorem{prop-def}{Propriété-définition}
\newtheorem{exer}{Exercice}
\newtheorem{nota}{Notation}
\newtheorem{ax}{Axiome}
\newtheorem{appl}{Application}
\newtheorem{csq}{Conséquence}
%\def\di{\displaystyle}


\newcommand{\vtab}{\rule[-0.4em]{0pt}{1.2em}}
\newcommand{\V}{\overrightarrow}
\renewcommand{\thesection}{\Roman{section} }
\renewcommand{\thesubsection}{\arabic{subsection} }
\renewcommand{\thesubsubsection}{\alph{subsubsection} }
\newcommand{\C}{\mathbb{C}}
\newcommand{\R}{\mathbb{R}}
\newcommand{\Q}{\mathbb{Q}}
\newcommand{\Z}{\mathbb{Z}}
\newcommand{\N}{\mathbb{N}}



\usetheme{Warsaw}

\title{Questions sur la notion de fonction}
\author{Évaluation de 30 minutes \\jeudi 17 mars 2022}
\begin{document}
\maketitle	

\begin{frame}
	\frametitle{Question 1: La forme factorisée d'une fonction polynomiale de degré 2}
	Pour tout  $x\in\R$ on définit: $$f(x)= 120x^2+42x-36 $$ 
	\begin{enumerate}
		\item Justifier que pour tout $x\in\R$ on a: $$f(x)= 6(4x+3)(5x-2)$$ 
		\item Donner les antécédents de zéro par la fonction $f$. 
	\end{enumerate}

\end{frame}

\begin{frame}
	\frametitle{Question 2: La forme canonique d'une fonction polynomiale de degré 2 }
	Pour tout  $x\in\R$ on définit: $$g(x)= 9x^2-30x+19 $$ 
	\begin{enumerate}
		\item Justifier que pour tout $x\in\R$ on a: $$g(x)= (3x-5)^2-6$$ 
		\item En utilisant la forme la plus adaptée, trouver les antécédents de $10$ par la fonction $g$. 
	\end{enumerate}

\end{frame}

\begin{frame}
	\frametitle{Question 3: Double tableau de signes}
	Pour tout  $x\in\R$ on définit: $$h(x) = (x+2)(3x-4)$$ \hfill\\[-0.2cm]
		\begin{enumerate}
		\item Donner le signe de $h(x)$ en fonction de $x$.\\ \small\textit{(Indication: tableau de signes sur deux niveaux) }
		\item En déduire pour quelles valeurs de $x\in\R$ on a:
		$$h(x) \geq 0$$
	\end{enumerate}
	

\end{frame}

\begin{frame}
	\frametitle{Question 4: inéquation et fonction affine}
		Pour tout  $x\in\R$ on définit: $$m(x) = -\dfrac{5}{3}x - 5 $$ \hfill\\[-0.2cm]
	\begin{enumerate}
		\item Quelle est la nature de la fonction $m$ ? 
		\item Donner pour quelles valeurs de $x$, l'image de $x$ par la fonction $m$ est supérieure strictement à $5$.
		
	\end{enumerate}
	
 

\end{frame}

\begin{frame}
	\frametitle{Question 5: Double tableau de signes}
	Pour tout  $x\in\R$ on définit: $$n(x) = (-3x+2)(-5x-2)$$ \hfill\\[-0.2cm]
	\begin{enumerate}
		\item Donner le signes de $n(x)$ en fonction de $x$.\\ 
		\item En déduire pour quelles valeurs de $x\in\R$ on a:
		$$n(x) \leq 0$$
	\end{enumerate}
	
	
\end{frame}


\begin{frame}
	\frametitle{Question 6}
	Pour tout  $x\in\R$ on définit:$$f(x) = (-x-2)(7x-4) \quad \text{et} \quad g(x) = (7x-4)^2$$
Résoudre dans $\R$ l'équation suivante: 
$$f(x) = g(x) $$
\end{frame}


\begin{frame}
	\frametitle{Question 7}
	Pour tout  $x\in\R$ on définit:$$m(x) = (3x-8) \quad \text{et} \quad n(x) = (x+5)(3x-8)$$
	Résoudre dans $\R$ l'équation suivante: 
	$$m(x) = n(x) $$
\end{frame}
\begin{frame}
	\frametitle{Question 8}
	Pour tout  $x\in\R$ on définit: $$k(x) = (-\dfrac{5}{3}x - 5)(-\dfrac{7}{5}x - 14)$$ \hfill\\[-0.2cm]
	\begin{enumerate}
		\item Donner le signe de $k(x)$ en fonction de $x$.\\ 
		\item En déduire pour quelles valeurs de $x\in\R$ on a:
		$$k(x) \leq 0$$
	\end{enumerate}
	
	
\end{frame}





\end{document}