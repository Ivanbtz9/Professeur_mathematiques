\documentclass[11pt]{article}

 %Configuration de la feuille 
 
\usepackage{amsmath,amssymb,enumerate,graphicx,pgf,tikz,fancyhdr}
\usepackage[utf8]{inputenc}
\usetikzlibrary{arrows}
\usepackage{geometry}
\usepackage{tabvar}
\geometry{hmargin=2.2cm,vmargin=1.5cm}\pagestyle{fancy}
\lfoot{\bfseries http://www.bibmath.net}
\rfoot{\bfseries\thepage}
\cfoot{}
\renewcommand{\footrulewidth}{0.5pt} %Filet en bas de page

 %Macros utilisées dans la base de données d'exercices 

\newcommand{\mtn}{\mathbb{N}}
\newcommand{\mtns}{\mathbb{N}^*}
\newcommand{\mtz}{\mathbb{Z}}
\newcommand{\mtr}{\mathbb{R}}
\newcommand{\mtk}{\mathbb{K}}
\newcommand{\mtq}{\mathbb{Q}}
\newcommand{\mtc}{\mathbb{C}}
\newcommand{\mch}{\mathcal{H}}
\newcommand{\mcp}{\mathcal{P}}
\newcommand{\mcb}{\mathcal{B}}
\newcommand{\mcl}{\mathcal{L}}
\newcommand{\mcm}{\mathcal{M}}
\newcommand{\mcc}{\mathcal{C}}
\newcommand{\mcmn}{\mathcal{M}}
\newcommand{\mcmnr}{\mathcal{M}_n(\mtr)}
\newcommand{\mcmnk}{\mathcal{M}_n(\mtk)}
\newcommand{\mcsn}{\mathcal{S}_n}
\newcommand{\mcs}{\mathcal{S}}
\newcommand{\mcd}{\mathcal{D}}
\newcommand{\mcsns}{\mathcal{S}_n^{++}}
\newcommand{\glnk}{GL_n(\mtk)}
\newcommand{\mnr}{\mathcal{M}_n(\mtr)}
\DeclareMathOperator{\ch}{ch}
\DeclareMathOperator{\sh}{sh}
\DeclareMathOperator{\vect}{vect}
\DeclareMathOperator{\card}{card}
\DeclareMathOperator{\comat}{comat}
\DeclareMathOperator{\imv}{Im}
\DeclareMathOperator{\rang}{rg}
\DeclareMathOperator{\Fr}{Fr}
\DeclareMathOperator{\diam}{diam}
\DeclareMathOperator{\supp}{supp}
\newcommand{\veps}{\varepsilon}
\newcommand{\mcu}{\mathcal{U}}
\newcommand{\mcun}{\mcu_n}
\newcommand{\dis}{\displaystyle}
\newcommand{\croouv}{[\![}
\newcommand{\crofer}{]\!]}
\newcommand{\rab}{\mathcal{R}(a,b)}
\newcommand{\pss}[2]{\langle #1,#2\rangle}
 %Document 

\begin{document} 

\begin{center}\textsc{{\huge }}\end{center}

% Exercice 3151


\vskip0.3cm\noindent\textsc{Exercice 1} - Une fonction lipschitzienne
\vskip0.2cm
\begin{enumerate}
\item Démontrer que la fonction $\sin$ est lipschitzienne sur $\mathbb R$.
\item Soit $f:[a,b]\to\mathbb R$ continue. Démontrer que la fonction $F:\mathbb R\to\mathbb R$ définie par 
$$F(x)=\int_a^b f(t)\sin(xt)dt$$
est lipschitzienne.
\end{enumerate}


% Exercice 3151


\vskip0.3cm\noindent\textsc{Exercice 2} - Une fonction lipschitzienne
\vskip0.2cm
\begin{enumerate}
\item Démontrer que la fonction $\sin$ est lipschitzienne sur $\mathbb R$.
\item Soit $f:[a,b]\to\mathbb R$ continue. Démontrer que la fonction $F:\mathbb R\to\mathbb R$ définie par 
$$F(x)=\int_a^b f(t)\sin(xt)dt$$
est lipschitzienne.
\end{enumerate}


% Exercice 408


\vskip0.3cm\noindent\textsc{Exercice 3} - Limites de suites
\vskip0.2cm
Calculer la limite des suites suivantes :
\begin{enumerate}
\item $\dis u_n=\frac 1n\left(\sin\left(\frac{\pi}{n}\right)+\sin\left(\frac{2\pi}{n}\right)+\dots+\sin\left(\frac{n\pi}{n}\right)\right).$
\item $\dis u_n=n\left(\frac{1}{(n+1)^2}+\dots+\frac{1}{(n+n)^2}\right).$
\item $\dis u_n=\frac{\sqrt{1}+\sqrt{2}+\dots+\sqrt{n-1}}{n\sqrt{n}}.$
\item $\dis u_n=\sqrt[n]{\left(1+\left(\frac{1}{n}\right)^2\right)\left(1+\left(\frac{2}{n}\right)^2\right)\dots\left(1+\left(\frac{n}{n}\right)^2\right)}$.
\end{enumerate}


% Exercice 409


\vskip0.3cm\noindent\textsc{Exercice 4} - Produit
\vskip0.2cm
Déterminer la limite de 
$$v_n=\frac1n\prod_{k=1}^n (k+n)^{1/n}.$$


% Exercice 410


\vskip0.3cm\noindent\textsc{Exercice 5} - Inégalité de Jensen
\vskip0.2cm
Soit $f:[a,b]\to\mathbb R$ continue et $g:\mathbb R\to\mathbb R$ continue et convexe. Démontrer que 
$$g\left(\frac{1}{b-a}\int_a^b f(t)dt \right)\leq \frac{1}{b-a}\int_a^b g(f(t))dt.$$


% Exercice 418


\vskip0.3cm\noindent\textsc{Exercice 6} - Cesaro pour les intégrales
\vskip0.2cm
Soit $f:[0,+\infty[\to\mathbb R$ une fonction continue admettant une limite finie $a$ en $+\infty$.
Montrer que
$$\frac 1x\int_0^x f(t)dt\to a\textrm{ quand }x\to+\infty.$$




\vskip0.5cm
\noindent{\small Cette feuille d'exercices a été conçue à l'aide du site \textsf{https://www.bibmath.net}}

%Vous avez accès aux corrigés de cette feuille par l'url : https://www.bibmath.net/ressources/justeunefeuille.php?id=27787
\end{document}