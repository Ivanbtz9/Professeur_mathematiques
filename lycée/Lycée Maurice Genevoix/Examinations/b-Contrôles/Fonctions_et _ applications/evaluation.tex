\documentclass[a4paper,10.9pt]{article}
\usepackage[utf8]{inputenc}
\usepackage[T1]{fontenc}
\usepackage{fancyhdr} % pour personnaliser les en-têtes
\usepackage{lastpage}
\usepackage[frenchb]{babel}
\usepackage{amsfonts,amssymb}
\usepackage{amsmath,amsthm}
\usepackage{paralist}
\usepackage{xspace}
\usepackage{xcolor}
\usepackage{variations}
\usepackage{xypic}
\usepackage{eurosym,multicol}
\usepackage{graphicx}
\usepackage[np]{numprint}
\usepackage{hyperref} 
\usepackage{listings} % pour écrire des codes avec coloration syntaxique  
\usepackage{diagbox,makecell,setspace}

\usepackage{tikz}
\usetikzlibrary{calc, arrows, plotmarks,decorations.pathreplacing}
\usepackage{colortbl}
\usepackage{multirow}
\usepackage[top=1.5cm,bottom=1.5cm,right=1.5cm,left=1.5cm]{geometry}


\setcellgapes{10pt}
\makegapedcells

\newtheorem{defi}{Définition}
\newtheorem{thm}{Théorème}
\newtheorem{thm-def}{Théorème/Définition}
\newtheorem{rmq}{Remarque}
\newtheorem{prop}{Propriété}
\newtheorem{cor}{Corollaire}
\newtheorem{lem}{Lemme}
\newtheorem{ex}{Exemple}
\newtheorem{cex}{Contre-exemple}
\newtheorem{prop-def}{Propriété-définition}
\newtheorem{exer}{Exercice}
\newtheorem{nota}{Notation}
\newtheorem{ax}{Axiome}
\newtheorem{appl}{Application}
\newtheorem{csq}{Conséquence}
\theoremstyle{definition}
\newtheorem{exo}{Exercice}


\newcommand{\vtab}{\rule[-0.4em]{0pt}{1.2em}}
\newcommand{\V}{\overrightarrow}
\renewcommand{\thesection}{\Roman{section} }
\renewcommand{\thesubsection}{\arabic{subsection} }
\renewcommand{\thesubsubsection}{\alph{subsubsection} }
\newcommand{\C}{\mathbb{C}}
\newcommand{\R}{\mathbb{R}}
\newcommand{\Q}{\mathbb{Q}}
\newcommand{\Z}{\mathbb{Z}}
\newcommand{\N}{\mathbb{N}}


\definecolor{vert}{RGB}{11,160,78}
\definecolor{rouge}{RGB}{255,120,120}
% Set the beginning of a LaTeX document
\pagestyle{fancy}


\begin{document}
%\lhead{}\chead{}\rhead{}\lfoot{}\cfoot{\thepage/2}\rfoot{\textbf{}}\renewcommand{\headrulewidth}{0.0pt}\renewcommand{\footrulewidth}{0.0pt}
%\quad
%\newpage

\lhead{Lycée Le Maurice Genevoix}\chead{}\rhead{Année~2021-2022}\lfoot{M. Botcazou}\cfoot{\thepage/2}\rfoot{\textbf{Tourner la page S.V.P.}}\renewcommand{\headrulewidth}{0.4pt}\renewcommand{\footrulewidth}{0.4pt}

\hfill\\[-0.7cm]
$$	\fbox{\text{\Large{ \sc Contrôle sur les fonctions }}}$$
\centering \Large{ (55 minutes) }\\[0.5cm]

\flushleft\normalsize


\textbf{\textit{Note aux lecteurs:}} \textit{ce contrôle devra être rédigé sur une copie avec un stylo de couleur foncée. La présentation et la qualité de rédaction seront des points importants d'appréciation des copies.  Les calculatrices sont autorisées mais un résultat sans l'expression des calculs qui lui est associé ne rapportera pas la totalité des points.}\\[0.7cm]	
\begin{exo}\textbf{"Lecture graphique"}\hfill\textbf{/7pts}\\\hfil\\
\begin{minipage}[t]{1.0\linewidth}
\begin{minipage}[c]{0.45\linewidth}

\begin{enumerate}
 
\item  Donner l'image de $-3$ par la fonction $V$ et l'image de $0$ par la fonction $H$. 
\item  Donner les antécédents de $0$ pour la fonction $V$ et tracer le tableau de signes de la fonction $H$ à l'aide du graphique donné. 
\item Résoudre l'inéquation : $H(x)\geq V(x)$. 
\item Donner le tableau de variations de la fonction $V$ et préciser sur quel intervalle la fonction $V$ est décroissante.
\item Donner pour quelles valeurs de $x\in[-3.5;3.5]$ le nombre $H(x)\times V(x)$ est positif à l'aide d'un tableau de signes.   
\end{enumerate}
\end{minipage}\hfill
\begin{minipage}[c]{0.5\linewidth}
	
$$\shorthandoff{:}\begin{tikzpicture}[xscale=0.9, yscale = 0.9]
\clip (-3.5,-3.5) rectangle (3.5,3.5);
\draw[step=0.5,dotted] (-3.5,-3.5) grid (3.5,3.5);
\draw [->, line width = 1pt, >=latex'](-3.5,0) -- (3.5,0);
\draw (3.4,0) node[below]{\footnotesize$x$};
\draw (0,3.2) node[right]{\footnotesize$y$};
\foreach \x in {-3,-2,-1,1,2,3}
\draw[shift={(\x,0)}] (0pt,2pt) -- (0pt,-2pt) node[below] {\tiny $\x$};
\draw [->, line width = 1pt, >=latex'](0,-3.5) -- (0,3.5);
\foreach \y in {-3,-2,-1,1,2,3}
\draw[shift={(0,\y)}] (2pt,0pt) -- (-2pt,0pt) node[left] {\tiny $\y$};
\draw (0,0) node[below right]{\tiny $0$};
\draw[domain=-3.5:3.5,samples=100,color=blue] plot ({\x},{0.178*(\x+2.5)*(\x-3.7)*(\x+0.5)});
\draw[color=blue] (3.4,-1.5) node[left]{\footnotesize$\mathcal{C}_V$};
\draw[domain=-3.5:3.5,samples=100,color=red] plot ({\x},{0.48*(\x+2.5)*(\x-2.5)});
\draw[color=red] (-2.4,1.5) node[left]{\footnotesize$\mathcal{C}_H$};
\end{tikzpicture}\shorthandon{:}$$

\end{minipage}
\end{minipage}
\end{exo}
\begin{exo} \textbf{"Étude d'une fonction polynomiale du second degré"}\hfill\textbf{/10pts}\\\hfil\\

\begin{minipage}[t]{1.0\linewidth}
 \noindent On considère la fonction $f$ définie par la relation:
 $$\forall x\in \R,\ f(x) = 2x^2 +\dfrac{3}{5}x- \dfrac{1}{5}$$
\begin{enumerate}
\item Grâce à un tableau de valeurs à remplir en \textbf{(\textit{Annexe 1})}, tracer dans un repère orthogonal en \textbf{(\textit{Annexe 1})} la courbe associée à la fonction $f$ sur l’intervalle $[-1;1]$.\\
\textit{(Indication: Faire des pas de 0.25 pour choisir vos nombres)}
\item Justifier que pour tout $x\in\R$ on a: $$f(x)= 2(x-\dfrac{1}{5})(x+\dfrac{1}{2})$$
\item Donner le tableau de signes de la fonction $f$ grâce à la question précédente.\\
\end{enumerate}
\noindent On considère la fonction $g$ définie par la relation:
$$\forall x\in \R,\ g(x) = 2(-3x-\dfrac{9}{5})(x+\dfrac{1}{2})$$
\begin{enumerate}
	\item[4.] Résoudre dans $\R$ l'équation suivante: 
	$$f(x) = g(x) $$
	\item[5.] Répondre aux affirmations suivantes par (Vrai/Faux) avec justifications:
	\begin{enumerate}
		\item La fonction $f$ est paire sur [-1;1].
		\item La fonction $f$ admet un maximum local pour $x= -\dfrac{3}{20}$.
	\end{enumerate}
\end{enumerate}
\end{minipage}

\end{exo}

\begin{exo} \textbf{"Une fonction affine avec ses inconnues"}\hfill\textbf{/3pts}\\\hfil\\
Soit $a,b \in \R$ \\
Pour tout  $x\in\R$ on définit: $$m(x) = ax + b $$ \hfill\\[-0.2cm]
\begin{enumerate}[$\square$]
	\item On sait que la courbe de la fonction $m$ passe par les points de coordonnées $A(0;-2)$ et $B(3;2)$. \\[0.2cm] En déduire les valeurs de $a$ et de $b$.\\ 
	%\item Sans résoudre d'inéquation, en déduire le tableau de signes et le tableau de variations de la fonction $m$. 
	
\end{enumerate}
\end{exo}

\newpage
\lhead{Lycée Le Maurice Genevoix\\[0.2cm]
\bfseries Nom:\\[0.2cm]\bfseries Prénom:}\chead{}\rhead[t]{Année~2021-2022\\[0.2cm]$Seconde \ .....$\\}\lfoot{M. Botcazou}\cfoot{\thepage/2}\rfoot{\textbf{Fin}}\renewcommand{\headrulewidth}{0.4pt}\renewcommand{\footrulewidth}{0.4pt}

\hfill\\[0.5cm]
	
\section*{Annexe 1}\hfill\\[0.2cm]

\begin{center}
	\begin{tabular*}{0.7\textwidth}{@{\extracolsep{\fill}}|c|c|c|c|c|c|c|c|c|c|}
		\hline
		$x$ &&&&&&&&& \\
		\hline
		
		$f(x)$ &&&&&&&&&\\
		\hline
	\end{tabular*} 
	\hfill\\[1cm]
\end{center}
$$\begin{tikzpicture}[scale=1]
\draw[step=0.5,dotted] (-6,-6) grid (6,6); 	
\end{tikzpicture}$$

\hfil\\[1cm]
\subsection*{Bonus: \textit{\small(À faire que si tout a déjà été traité)}}
Soit $a,b \in \R$, pour tout  $x\in\R$ on définit: $$n(x) = ax + b $$ \hfill\\[-0.2cm]
\begin{enumerate}[$\square$]
	\item On sait que la courbe de la fonction $n$ passe par les points de coordonnées $A(1;-2)$ et $B(3;2)$. \\[0.2cm] En déduire les valeurs de $a$ et de $b$ en résolvant des équations.\\ 
	 
	
\end{enumerate}
\end{document}