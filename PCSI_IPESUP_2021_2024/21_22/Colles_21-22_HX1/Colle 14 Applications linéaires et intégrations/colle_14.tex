\documentclass[a4paper,10pt]{article}



\usepackage{fancyhdr} % pour personnaliser les en-têtes
\usepackage[utf8]{inputenc}
\usepackage[T1]{fontenc}
\usepackage{lastpage}
\usepackage[frenchb]{babel}
\usepackage{amsfonts,amssymb}
\usepackage{amsmath,amsthm,mathtools}
\usepackage{paralist}
\usepackage{xspace}
\usepackage{xcolor,multicol}
\usepackage{variations}
\usepackage{xypic}
\usepackage{eurosym}
\usepackage{graphicx}
\usepackage{mathdots}%faire des points suspendus en diagonale
\usepackage[np]{numprint}
\usepackage{hyperref} 
\usepackage{listings} % pour écrire des codes avec coloration syntaxique  

\usepackage{tikz}
\usetikzlibrary{calc, arrows, plotmarks,decorations.pathreplacing}
\usepackage{colortbl}
\usepackage{multirow}
\usepackage[top=2cm,bottom=1.5cm,right=2cm,left=1.5cm]{geometry}

\newtheorem{thm}{Théorème}
\newtheorem*{pro}{Propriété}
\newtheorem*{exemple}{Exemple}

\theoremstyle{definition}
\newtheorem*{remarque}{Remarque}
\theoremstyle{definition}
\newtheorem{exo}{Exercice}
\newtheorem{definition}{Définition}


\newcommand{\vtab}{\rule[-0.4em]{0pt}{1.2em}}
\newcommand{\V}{\overrightarrow}
\renewcommand{\thesection}{\Roman{section} }
\renewcommand{\thesubsection}{\arabic{subsection} }
\renewcommand{\thesubsubsection}{\alph{subsubsection} }
\newcommand*{\transp}[2][-3mu]{\ensuremath{\mskip1mu\prescript{\smash{\mathrm t\mkern#1}}{}{\mathstrut#2}}}%

\newcommand{\K}{\mathbb{K}}
\newcommand{\C}{\mathbb{C}}
\newcommand{\R}{\mathbb{R}}
\newcommand{\Q}{\mathbb{Q}}
\newcommand{\Z}{\mathbb{Z}}
\newcommand{\N}{\mathbb{N}}

\renewcommand{\Im}{\mathop{\mathrm{Im}}\nolimits}



\definecolor{vert}{RGB}{11,160,78}
\definecolor{rouge}{RGB}{255,120,120}
% Set the beginning of a LaTeX document
\pagestyle{fancy}
\lhead{Optimal Sup Spé, groupe IPESUP}\chead{Année~2021-2022}\rhead{Niveau: Première année de PCSI }\lfoot{M. Botcazou}\cfoot{\thepage}\rfoot{mail: ibotca52@gmail.com }\renewcommand{\headrulewidth}{0.4pt}\renewcommand{\footrulewidth}{0.4pt}

\begin{document}
	
	
	\begin{center}
		\Large \sc colle 14 = Espaces vectoriels, applications linéaires et intégration
	\end{center}





\section*{Espaces vectoriels et applications linéaires:}\hfill\\\hfill\\
\begin{minipage}{1\linewidth}
	\begin{minipage}[t]{0.48\linewidth}
		\raggedright
		
		
		
	\begin{exo}\quad\\
		Soit $f$ l'endomorphisme de $\R^2$ de matrice\\[0.25cm] $A=\begin{pmatrix} 2&\frac 23\\[0.25cm]
		-\frac 52&-\frac 23 \end{pmatrix}$ dans la base canonique. Soient\\[0.25cm] 
		$e_1 = \begin{pmatrix} -2 \\ 3\end{pmatrix}$
		et $e_2 = \begin{pmatrix} -2 \\ 5 \end{pmatrix}$.
		\begin{enumerate}
			\item Montrer que $\mathcal{B}'= (e_1, e_2)$ est une base de $\R^2$ et déterminer 
			$\text{Mat}_{\mathcal{B}'}(f)$.
			\item Calculer $A^n$ pour $n \in \N$.
			\item Déterminer l'ensemble des suites réelles qui \\[0.25cm]vérifient $\forall n \in \N$
			$\begin{cases} x_{n + 1} = 2x_n + \dfrac 23 y_n \\ y_{n + 1} = -\dfrac 52 x_n -
			\dfrac 23 y_n \end{cases}$.\\[0.25cm]
		\end{enumerate}
		
		
		\centering
		\rule{1\linewidth}{0.6pt}
	\end{exo}
	
	\begin{exo}\quad\\
	
			
		Soient $A, B$ deux matrices semblables (i.e. il existe $P$
		inversible telle que $B = P^{-1} A P$). Montrer que si l'une est
		inversible, l'autre aussi\,; que si l'une est idempotente, l'autre
		aussi\,; que si l'une est nilpotente, l'autre aussi\,; que si $A =
		\lambda I$, alors $A = B$.
		
		\centering
		\rule{1\linewidth}{0.6pt}
	\end{exo}
	
	\begin{exo}\quad\\
		Soit $f$ l'endomorphisme de $\R^2$ dont la matrice par
		rapport \`a la base canonique $(e_1, e_2)$ est
		$$A= \left( 
		\begin{array}{cc}
		11 & -6  \\
		12 & -6  \\
		\end{array}
		\right).$$
		Montrer que les vecteurs
		$$ e'_1 = 2e_1+3e_2,\quad e'_2 = 3e_1+4e_2,$$
		forment une base de $\R^2$ et calculer la matrice de $f$ par
		rapport \`a cette base.	
		
		\centering
		\rule{1\linewidth}{0.6pt}
	\end{exo}

		
		
	\end{minipage}	
	\hfill\vrule\hfill
	\begin{minipage}[t]{0.48\linewidth}
		\raggedright
	
		\begin{exo}\quad\\[0.25cm]
		% anti diagonal dots
		\def\Ddots{\mathinner{\mkern2mu\raise1pt\hbox{.}\mkern2mu
				\newline \raise4pt\hbox{.}\mkern2mu\raise7pt\hbox{.}\mkern1mu}}
		Soit $A = \begin{pmatrix} 
		0&&\dots&0&1 \\ 
		\vdots&&&1&0 \\
		& &  \Ddots && \\
		0&1& & &\vdots \\ 
		1&0&&\dots&0
		\end{pmatrix}$. 
		\hfil\\[0.25cm] En utilisant l'application linéaire associée de 
		$\mathcal{L} (\R^n,\R^n)$, calculer $A^p$ pour $p \in \Z$.
		
		\centering
		\rule{1\linewidth}{0.6pt}
	\end{exo}
	
	
	
	\begin{exo}\quad\\
		Soit $\R^2$ muni de la base canonique $\mathcal{B}=(\vec{i}, \vec{j})$.\\
		Soit $f : \R^2 \to \R^2$ la projection sur l'axe des abscisses $\R \vec{i}$ 
		parall\`element à $\R (\vec{i} + \vec{j})$.
		Déterminer $\textrm{Mat}_{\mathcal{B},\mathcal{B}}(f)$, la matrice de $f$ dans la base $(\vec{i}, \vec{j})$.
		
		Même question avec $\textrm{Mat}_{\mathcal{B}',\mathcal{B}}(f)$ où $\mathcal{B'}$ est la base 
		$(\vec{i} - \vec{j}, -2\vec{i}+3\vec{j})$ de $\R^2$.
		Même question avec $\textrm{Mat}_{\mathcal{B}',\mathcal{B}'}(f)$.	
		
		\centering
		\rule{1\linewidth}{0.6pt}
	\end{exo}
	
\begin{exo}\quad\\
	Soit  $E$  un espace vectoriel et  $f$  une application linéaire de
	$E$  dans  lui-m\^eme telle que  $f^2=f$.
	\begin{enumerate}
		\item Montrer que  $E= \ker f \oplus \Im f$.
		
		\item Supposons que  $E$ soit de dimension finie  $n$. 
		Posons  $r= \dim \Im f$. 
		Montrer qu'il existe une base 
		$\mathcal{B}= ( e_1, \ldots ,e_n)$ de  $E$  telle que : 
		$f(e_i)=e_i$ si $i\le r$ et $f(e_i)=0$ si $i>r$. 
		Déterminer la matrice de  $f$ dans cette base $\mathcal{B}$.
	\end{enumerate}
	
	\centering
	\rule{1\linewidth}{0.6pt}
\end{exo}
		
		\begin{exo}\quad\\
			Trouver toutes les matrices de $\mathcal{M}_3(\R)$ qui vérifient
			\begin{enumerate}
				\item $M^2 = 0$ ;
				\item $M^2 = M$ ; 
				\item $M^2 = I$. 
			\end{enumerate}
			
			\centering
			\rule{1\linewidth}{0.6pt}
		\end{exo}

		
		
		
	\end{minipage}
\end{minipage}
\newpage


\section*{Intégration :}\hfill\\\hfill\\
\begin{minipage}{1\linewidth}
	\begin{minipage}[t]{0.48\linewidth}
		\raggedright
		
		
		
		\begin{exo}\quad\\
			Soit $f:[a,b]\rightarrow \R$ une fonction continue sur $[a,b]$ ($a<b$).
			\begin{enumerate}
				\item On suppose que $f(x) \ge 0$ pour tout $x\in [a,b]$, et que $f(x_0)>0$ en un point $x_0\in [a,b]$. 
				Montrer que $\int_a^b f(x) d x>0$. En déduire que : <<si $f$ est une fonction continue
				positive sur $[a,b]$ telle que $\int_a^b f(x) d x=0$ alors $f$ est
				identiquement nulle>>.
				\item On suppose que $\int_a^b f(x) d x=0$. Montrer qu'il existe $c\in [a,b]$ tel que $f(c)=0$. 
				\item Application: on suppose
				que $f$ est une fonction continue sur $[0,1]$ telle que $\int_0^1 f(x) dx=\frac 12$. 
				Montrer qu'il existe $d\in [0,1]$ tel que $f(d)=d$.
			\end{enumerate}
			
			\centering
			\rule{1\linewidth}{0.6pt}
		\end{exo}
		
		
		
		\begin{exo}\quad\\
			Soient les fonctions définies sur $\R$,
			$$f(x)=x \text{ , } g(x)=x^2 \text{ et  } h(x)=e^x,$$
			Justifier qu'elles sont intégrables sur tout intervalle fermé borné de $\R$. En utilisant les
			sommes de Riemann, calculer les intégrales $\int_0^1f(x)d x$, $\int_1^2 g(x)
			d x$ et $\int_0^x h(t) d t$.
			
			\centering
			\rule{1\linewidth}{0.6pt}
		\end{exo}
		
				\begin{exo}\quad\\
			Calculer les intégrales suivantes :
			$$\int_0^{\frac \pi 2}\frac 1{1+\sin x}d x \quad \mbox{ et } \quad \int_0^{\frac \pi 2}\frac{\sin
				x}{1+\sin x}d x.$$
			
			\centering
			\rule{1\linewidth}{0.6pt}
		\end{exo}
	
	\end{minipage}	
	\hfill\vrule\hfill
	\begin{minipage}[t]{0.48\linewidth}
		\raggedright
		
		\begin{exo}\quad\\[0.25cm]
		Soit $\displaystyle I_{n} = \int_0^1 \frac{x^n}{1 + x}d x$.
		\begin{enumerate}
			\item En majorant la fonction int\'egr\'ee, montrer que
			$\lim_{n\to +\infty} I_{n}=0$.
			\item  Calculer $I_n + I_{n + 1}$.
			\item D\'eterminer $\displaystyle \lim_{n \rightarrow  + \infty} \left(\sum_{k = 1}^n \frac{
				(-1)^{k + 1}}k\right)$.
		\end{enumerate}
			
			\centering
			\rule{1\linewidth}{0.6pt}
		\end{exo}
		
		
	
	\begin{exo}\quad\\
		Calculer la limite des suites suivantes :
		\begin{enumerate}
			\item $\displaystyle u_n=n\sum_{k=0}^{n-1}\frac 1{k^2+n^2}$
			\item $\displaystyle v_n=\prod\limits_{k=1}^n\left(1+\frac{k^2}{n^2}\right) ^{\frac 1n}$
		\end{enumerate}
		
		\centering
		\rule{1\linewidth}{0.6pt}
	\end{exo}

	\begin{exo}\quad\\
	Soit $f$ une fonction de classe $C^1$ sur $[0,1]$ telle que $f(0)=0$. Montrer que $2\int_{0}^{1}f^2(t)\;dt\leq\int_{0}^{1}{f'}^2(t)\;dt$.
	
	\centering
	\rule{1\linewidth}{0.6pt}
\end{exo}

	\begin{exo}\quad\\
		Soit $f$ une fonction de classe $C^2$ sur $[0,1]$. Déterminer le réel $a$ tel que :
		
		$$\int_{0}^{1}f(t)\;dt-\frac{1}{n}\sum_{k=1}^{n-1}f(\frac{k}{n})\underset{n\rightarrow+\infty}{=}\frac{a}{n}+o(\frac{1}{n}).$$
		
		\centering
		\rule{1\linewidth}{0.6pt}
	\end{exo}


		
	
		
		
	\end{minipage}
\end{minipage}

\end{document}