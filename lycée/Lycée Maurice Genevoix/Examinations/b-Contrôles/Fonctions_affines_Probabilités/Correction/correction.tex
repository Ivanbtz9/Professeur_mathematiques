\documentclass[a4paper,11pt]{article}
\usepackage[utf8]{inputenc}
\usepackage[T1]{fontenc}
\usepackage{fancyhdr} % pour personnaliser les en-têtes
\usepackage{lastpage}
\usepackage[frenchb]{babel}
\usepackage{amsfonts,amssymb}
\usepackage{amsmath,amsthm}
\usepackage{paralist}
\usepackage{xspace}
\usepackage{xcolor}
\usepackage{variations}
\usepackage{xypic}
\usepackage{eurosym,multicol}
\usepackage{graphicx}
\usepackage[np]{numprint}
\usepackage{hyperref} 
\usepackage{listings} % pour écrire des codes avec coloration syntaxique  
\usepackage{diagbox,makecell,setspace}

\usepackage{pgf,tikz,pgfplots,tkz-tab}
\usetikzlibrary{calc, arrows, plotmarks,decorations.pathreplacing}
\usepackage{colortbl}
\usepackage{multirow}
\usepackage[top=1.5cm,bottom=1.5cm,right=1.5cm,left=1.5cm]{geometry}
\usepackage{tabularx}
\parindent=0cm




\lstset{
	literate=
	{�}{{\'a}}1 {�}{{\'e}}1 {�}{{\'i}}1 {�}{{\'o}}1 {�}{{\'u}}1
	{�}{{\'A}}1 {�}{{\'E}}1{�}{{\'I}}1 {�}{{\'O}}1 {�}{{\'U}}1
	{�}{{\`a}}1 {�}{{\`e}}1 {�}{{\`i}}1 {�}{{\`o}}1{�}{{\`u}}1
	{�}{{\`A}}1 {�}{{\'E}}1 {�}{{\`I}}1 {�}{{\`O}}1 {�}{{\`U}}1
	{�}{{\"a}}1 {�}{{\"e}}1 {�}{{\"i}}1 {�}{{\"o}}1 {�}{{\"u}}1
	{�}{{\"A}}1 {�}{{\"E}}1 {�}{{\"I}}1 {�}{{\"O}}1 {�}{{\"U}}1
	{�}{{\^a}}1 {�}{{\^e}}1 {�}{{\^i}}1 {�}{{\^o}}1 {�}{{\^u}}1
	{�}{{\^A}}1 {�}{{\^E}}1 {�}{{\^I}}1 {�}{{\^O}}1 {�}{{\^U}}1
	{?}{{\oe}}1 {?}{{\OE}}1 {�}{{\ae}}1 {�}{{\AE}}1 {�}{{\ss}}1
	{?}{{\H{u}}}1 {?}{{\H{U}}}1 {?}{{\H{o}}}1 {?}{{\H{O}}}1
	{�}{{\c c}}1 {�}{{\c C}}1 {�}{{\o}}1 {�}{{\r a}}1 {�}{{\r A}}1
	{?}{{\EUR}}1 {�}{{\pounds}}1 {?}{{?}}1
}
\lstdefinestyle{stylepython}{
	language=Python, 
	basicstyle=\ttfamily,
	%       name=iciNOM,title={Un programme Python}, 
	%       caption={iciTitre},
	%       label={iciNom},
	commentstyle=\footnotesize\color{green!50!black}, 
	keywordstyle=\color{blue},   
	stringstyle=\color{olive},   
	numberstyle=\tiny,  
	%       mathescape,  
	%       showstringspaces=false,   
	tabsize=3,   
	%framexleftmargin=5mm,  
	framexrightmargin=5pt,
	framexbottommargin=5pt
	xleftmargin=0mm,  
	%keepspaces=false,   
	classoffset=1,     
	numbers=left,    
	%stepnumber=1,    
	numbersep=8pt,   
	%showstringspac0ptes=false,  
	%frame=single,
	framerule=1pt,
	%rulecolor=\color{yellow}, 
	%       breaklines=true,  
	%       rulesepcolor=\color{blue}, %avec frame=shadowsbox
	%backgroundcolor=\color{yellow}
}



\setcellgapes{10pt}
\makegapedcells

\newtheorem{defi}{Définition}
\newtheorem{thm}{Théorème}
\newtheorem{thm-def}{Théorème/Définition}
\newtheorem{rmq}{Remarque}
\newtheorem{prop}{Propriété}
\newtheorem{cor}{Corollaire}
\newtheorem{lem}{Lemme}
\newtheorem{ex}{Exemple}
\newtheorem{cex}{Contre-exemple}
\newtheorem{prop-def}{Propriété-définition}
\newtheorem{exer}{Exercice}
\newtheorem{nota}{Notation}
\newtheorem{ax}{Axiome}
\newtheorem{appl}{Application}
\newtheorem{csq}{Conséquence}
\theoremstyle{definition}
\newtheorem{exo}{Exercice}


\newcommand{\vtab}{\rule[-0.4em]{0pt}{1.2em}}
\newcommand{\V}{\overrightarrow}
\renewcommand{\thesection}{\Roman{section} }
\renewcommand{\thesubsection}{\arabic{subsection} }
\renewcommand{\thesubsubsection}{\alph{subsubsection} }
\newcommand{\C}{\mathbb{C}}
\newcommand{\R}{\mathbb{R}}
\newcommand{\Q}{\mathbb{Q}}
\newcommand{\Z}{\mathbb{Z}}
\newcommand{\N}{\mathbb{N}}
\newcommand{\eq}{\Longleftrightarrow}


\definecolor{vert}{RGB}{11,160,78}
\definecolor{rouge}{RGB}{255,120,120}
% Set the beginning of a LaTeX document
\pagestyle{fancy}


\begin{document}


\lhead{Lycée Le Maurice Genevoix}\chead{}\rhead{Année~2021-2022}\lfoot{M. Botcazou}\cfoot{\thepage/2}\rfoot{\textbf{Tourner la page S.V.P.}}\renewcommand{\headrulewidth}{0.4pt}\renewcommand{\footrulewidth}{0.4pt}

\hfill\\[-0.7cm]
$$	\fbox{\text{\Large{ \sc Correction du Contrôle sur les fonctions affines et les probabilités }}}$$


\flushleft\normalsize



\begin{exo} \textbf{"Pièces équilibrée VS pièces truquée"}\\\hfil\\

	\begin{enumerate}%[$\square$]
		\item Voir en \textbf{Annexe 1}  \hfill\textbf{/1pts}
		\item On modélise cette expérience à l'aide des valeurs $\{0,1\}$ telles que: $\left\{\begin{array}{c}
		"Pile" \rightarrow 1\\
		"Face" \rightarrow 0
		\end{array}\right.$ \\
		 Soit $X$ la valeur de la pièce truquée.\hfill\textbf{/3pts}\\
		La pièce truquée donne 11 fois plus de "Face" que de "Pile".\\
		Donc $$P(\{X=0\}) = 11 \times P(\{X=1\})$$
		Or la somme des probabilités de chaque issue est égale à 1.\\
		Donc $$P(\{X=0\}) + P(\{X=1\}) = 1$$
		Ainsi par ce qui précède on peut en déduire que:
		$$11 \times P(\{X=1\}) + P(\{X=1\}) = 1$$
		Donc: 	$$12 \times P(\{X=1\})= 1$$
		Donc: $$\times P(\{X=1\})= \dfrac{1}{12}$$
		Or: $$P(\{X=0\}) = 1 - P(\{X=1\})$$
		Donc: $$P(\{X=0\}) = \dfrac{11}{12}$$
		$\underline{Conclusion}:$ la loi de cette expérience aléatoire est donnée par le tableau ci-dessous:
\begin{center}
			\begin{tabular}{|c|c|c|}
			\hline
			\text{Issue} "a" & 0 & 1\\
			\hline
			$P(\{X=a\})$ & $\dfrac{11}{12}$ & $\dfrac{1}{12}$\\
			\hline
		\end{tabular}
\end{center}

		\item  Voir en \textbf{Annexe 2}  \hfill\textbf{/1.5pts}\\[0.3cm]
		\item Oui nous pouvons affirmer que la probabilité d'obtenir un "Pile" avec la pièce équilibrée est égale à 6 fois la probabilité d'obtenir un "Pile" avec la pièce truquée de l'expérience précédente car: \hfill\textbf{/1pts}\\
		$$ P(\{"\text{Pile avec la pièce équlibrée}"\}) = \dfrac{1}{2} = \dfrac{6}{12} = 6 \times \dfrac{1}{12} = 6\times P(\{"\text{Pile avec la pièce truquée}"\})$$
	\end{enumerate}
\end{exo}


\newpage	
\begin{exo}\textbf{"Fonctions affines et courbes représentatives"}\\\hfil\\
Soit $F$ une fonction affine définie sur $\R$, il existe $a,b \in\R$ tel que: $\forall x\in \R \ ;\  F(x)= ax+b$\hfill\\[0.2cm]

$U(-9;-3)$ et $V(-1;5)$.\hfill\\[0.2cm]
\begin{enumerate}
	\item Par la méthode du taux d'accroissement: 
	$$a = \dfrac{y_U - y_V}{x_U - x_V} = \dfrac{-3 - 5}{-9  -(-1)} = \dfrac{-8}{-8} = 1$$
	Donc $F(x) = 1\times x +b = x+b $. De plus grace au point $V$ on sait que:

		 $$F(-1) = 5 \quad 
		\Longleftrightarrow \quad -1+b = 5\quad 
		\Longleftrightarrow \quad b = 6$$
	$\underline{Conclusion}: \forall x\in \R \ ;\  F(x)= x+6$\hfill\textbf{/2.5pts}\\[0.3cm]
	\item  \hfill\textbf{/1pts}
	\begin{minipage}[t]{1\linewidth}
		\begin{minipage}[t]{0.2\linewidth}
			\scriptsize
			\begin{align*}
			&\quad\quad F(x) \leq 0 \\
			&\eq x+6 \leq 0 \\
			&\eq x \leq -6 
			\end{align*}
		\end{minipage}\hfil\vrule\hfil
		\begin{minipage}[t]{0.4\linewidth}
		$$\begin{tikzpicture}[scale=0.9]
		\tkzTabInit{$x$ / 1 , \text{signe de:} $F(x)$ / 1}{$-\infty$, $-6$, $+\infty$}
		\tkzTabLine{, -, z, + }
		\end{tikzpicture}$$
		\end{minipage}
	\end{minipage}
\hfill\\[0.5cm]
	
	\item Soit $\alpha\in\R$ et $W(6;\alpha)$, on veut que: \hfill\textbf{/1.5pts}
$$\dfrac{y_U - y_W}{x_U - x_W} = -4 \quad 
\Longleftrightarrow \quad  \dfrac{-3-\alpha }{-9-6} = -4 \quad 
\Longleftrightarrow \quad \dfrac{-3-\alpha }{-15} = -4$$ 
	$$\Longleftrightarrow \quad -3-\alpha  = -4 \times (-15) \quad 
	\Longleftrightarrow \quad -3-\alpha  = 60  \quad \Longleftrightarrow \quad \alpha = -63 $$ 
\end{enumerate}

\end{exo}
\begin{exo}\textbf{"Des activités à l'université"}\\\hfil\\
	On choisit au hasard un étudiant de cette université nous sommes donc dans une situation d'équiprobabilité.

		\begin{enumerate}
			\item Voir \textbf{Annexe 3} \hfill\textbf{/2pts}
			
			\item  
			\begin{enumerate}
				\item $P(A) = \dfrac{\#A}{\#\Omega} = \dfrac{5500}{12000} = \dfrac{11}{24}$ \hfill\textbf{/1pts}\\[0.2cm]
				\item $P(A\cap S) = \dfrac{\#A\cap S}{\#\Omega} = \dfrac{1500}{12000} = \dfrac{1}{8}$ \hfill\textbf{/1pts}\\[0.2cm]
				\item $P(A\cup S) = \dfrac{\#A\cup S}{\#\Omega} = \dfrac{10500}{12000} = \dfrac{7}{8}$ \hfill\textbf{/1pts}\\[0.2cm]
				\item $P(\overline{A\cup S}) = 1 -P(A\cup S) =1- \dfrac{7}{8} = \dfrac{1}{8}$\hfill\textbf{/1pts}\\[0.2cm]
				\item  $P(S\backslash A) = \dfrac{\#S\backslash A}{\#\Omega} = \dfrac{5000}{12000} = \dfrac{5}{12}$ \hfill\textbf{/1pts}
				
				
			\end{enumerate}
			\item \begin{enumerate}
				\item $P(S) + P(A) = \dfrac{\#S}{\#\Omega} + \dfrac{\#A}{\#\Omega} = \dfrac{13}{24} + \dfrac{11}{24} = 1 $.\hfill\textbf{/0.5pts}\\[0.25cm]
				\item $P(S\cup A) + P(S\cap A) = \dfrac{7}{8} + \dfrac{1}{8} = 1 $.\hfill\textbf{/0.5pts}\\[0.25cm]
				\item On remarque que:\hfill\textbf{/0.5pts}\\ 
				$$P(S) + P(A) = P(S\cup A) + P(S\cap A)$$\\
				 Donc : $$P(S\cup A) = P(S) + P(A) - P(S\cap A)$$
			\end{enumerate}\end{enumerate}

	
\end{exo}
\lhead{Lycée Le Maurice Genevoix}\chead{}\rhead[t]{Année~2021-2022}\lfoot{M. Botcazou}\cfoot{\thepage/2}\rfoot{\textbf{Fin}}\renewcommand{\headrulewidth}{0.4pt}\renewcommand{\footrulewidth}{0.4pt}	

\newpage

		\section*{Annexe 1: une pièce équilibrée}
\begin{lstlisting}[style=stylepython]
def piece_equilibree():
	k = rd.randint(0,1)	
	return k			
\end{lstlisting}

\section*{Annexe 2: une pièce truquée}
\begin{lstlisting}[style=stylepython]
def piece_truquee():
	k = rd.randint(1,12)
	if k <= 11:
		return 0
	else:
		return 1			
\end{lstlisting}


\section*{Annexe 3:}
\hfil\\[1cm]

$$\begin{tikzpicture}[scale=2]
\draw[red](-1,0) ellipse [x radius=2cm, y radius=1cm];
\draw(2,0.05) node{$\#S\backslash A = 5000 $};
\draw(-2,0.05) node{$\#A\backslash S = 4000$};
\draw(0,-1.5) node[]{$\#\overline{A\cup S} = 1500$};
\draw(0,0.05) node{$\#A\cap S = 1500$};
\draw(2.5,2) node{$\#\Omega = 12000$};
\draw(-1.5,1.15) node{$\#A = 5500$};
\draw(1.5,1.15) node{$\#S = 6500$};
\draw[blue](1,0) ellipse [x radius=2cm, y radius=1cm];
\draw[black](0,0) ellipse [x radius=4cm, y radius=2cm];
\end{tikzpicture}$$



\end{document}



