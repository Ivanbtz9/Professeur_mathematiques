\documentclass[a4paper,11pt]{article}

\usepackage{inputenc}
\usepackage[T1]{fontenc}
\usepackage[frenchb]{babel}
\usepackage{fancyhdr,fancybox} % pour personnaliser les en-têtes
\usepackage{lastpage,setspace}
\usepackage{amsfonts,amssymb,amsmath,amsthm,mathrsfs}
\usepackage{relsize,exscale,bbold}
\usepackage{paralist}
\usepackage{xspace,multicol,diagbox,array}
\usepackage{xcolor}
\usepackage{variations}
\usepackage{xypic}
\usepackage{eurosym,stmaryrd}
\usepackage{graphicx}
\usepackage[np]{numprint}
\usepackage{hyperref} 
\usepackage{tikz}
\usepackage{colortbl}
\usepackage{multirow}
\usepackage{MnSymbol,wasysym}
\usepackage[top=1.5cm,bottom=1.5cm,right=1.2cm,left=1.5cm]{geometry}
\usetikzlibrary{calc, arrows, plotmarks, babel,decorations.pathreplacing}
\setstretch{1.25}
%\usepackage{lipsum} %\usepackage{enumitem} %\setlist[enumerate]{itemsep=1mm} bug avec enumerate



\newtheorem{thm}{Théorème}
\newtheorem{rmq}{Remarque}
\newtheorem{prop}{Propriété}
\newtheorem{cor}{Corollaire}
\newtheorem{lem}{Lemme}
\newtheorem{prop-def}{Propriété-définition}

\theoremstyle{definition}

\newtheorem{defi}{Définition}
\newtheorem{ex}{Exemple}
\newtheorem*{rap}{Rappel}
\newtheorem{cex}{Contre-exemple}
\newtheorem{exo}{Exercice} % \large {\fontfamily{ptm}\selectfont EXERCICE}
\newtheorem{nota}{Notation}
\newtheorem{ax}{Axiome}
\newtheorem{appl}{Application}
\newtheorem{csq}{Conséquence}
\def\di{\displaystyle}



\renewcommand{\thesection}{\Roman{section}}\renewcommand{\thesubsection}{\arabic{subsection} }\renewcommand{\thesubsubsection}{\alph{subsubsection} }


\newcommand{\bas}{~\backslash}\newcommand{\ba}{\backslash}
\newcommand{\C}{\mathbb{C}}\newcommand{\R}{\mathbb{R}}\newcommand{\Q}{\mathbb{Q}}\newcommand{\Z}{\mathbb{Z}}\newcommand{\N}{\mathbb{N}}\newcommand{\V}{\overrightarrow}\newcommand{\Cs}{\mathscr{C}}\newcommand{\Ps}{\mathscr{P}}\newcommand{\Rs}{\mathscr{R}}\newcommand{\Gs}{\mathscr{G}}\newcommand{\Ds}{\mathscr{D}}\newcommand{\happy}{\huge\smiley}\newcommand{\sad}{\huge\frownie}\newcommand{\danger}{\begin{tikzpicture}[x=1.5pt,y=1.5pt,rotate=-14.2]
	\definecolor{myred}{rgb}{1,0.215686,0}
	\draw[line width=0.1pt,fill=myred] (13.074200,4.937500)--(5.085940,14.085900)..controls (5.085940,14.085900) and (4.070310,15.429700)..(3.636720,13.773400)
	..controls (3.203130,12.113300) and (0.917969,2.382810)..(0.917969,2.382810)
	..controls (0.917969,2.382810) and (0.621094,0.992188)..(2.097660,1.359380)
	..controls (3.574220,1.726560) and (12.468800,3.984380)..(12.468800,3.984380)
	..controls (12.468800,3.984380) and (13.437500,4.132810)..(13.074200,4.937500)
	--cycle;
	\draw[line width=0.1pt,fill=white] (11.078100,5.511720)--(5.406250,11.875000)..controls (5.406250,11.875000) and (4.683590,12.812500)..(4.367190,11.648400)
	..controls (4.050780,10.488300) and (2.375000,3.675780)..(2.375000,3.675780)
	..controls (2.375000,3.675780) and (2.156250,2.703130)..(3.214840,2.964840)
	..controls (4.273440,3.230470) and (10.640600,4.847660)..(10.640600,4.847660)
	..controls (10.640600,4.847660) and (11.332000,4.953130)..(11.078100,5.511720)
	--cycle;
	\fill (6.144520,8.839900)..controls (6.460940,7.558590) and (6.464840,6.457090)..(6.152340,6.378910)
	..controls (5.835930,6.300840) and (5.320300,7.277400)..(5.003900,8.554750)
	..controls (4.683590,9.835940) and (4.679690,10.941400)..(4.996090,11.019600)
	..controls (5.312490,11.097700) and (5.824210,10.121100)..(6.144520,8.839900)
	--cycle;
	\fill (7.292960,5.261780)..controls (7.382800,4.898500) and (7.128900,4.523500)..(6.730460,4.421880)
	..controls (6.328120,4.324220) and (5.929680,4.535220)..(5.835930,4.898500)
	..controls (5.746080,5.261780) and (5.999990,5.640630)..(6.402340,5.738340)
	..controls (6.804690,5.839840) and (7.203110,5.625060)..(7.292960,5.261780)
	--cycle;
	\end{tikzpicture}}\newcommand{\alors}{\Large\Rightarrow}\newcommand{\equi}{\Leftrightarrow}
\newcommand{\fonction}[5]{\begin{array}{l|rcl}
		#1: & #2 & \longrightarrow & #3 \\
		& #4 & \longmapsto & #5 \end{array}}


\definecolor{vert}{RGB}{11,160,78}
\definecolor{rouge}{RGB}{255,120,120}
\definecolor{bleu}{RGB}{15,5,107}


\pagestyle{fancy}
\lhead{Optimal Sup Spé, groupe IPESUP}\chead{Année~2022-2023}\rhead{Niveau: Première année de PCSI }\lfoot{M. Botcazou}\cfoot{\thepage}\rfoot{mail: i.botcazou@gmx.fr }\renewcommand{\headrulewidth}{0.4pt}\renewcommand{\footrulewidth}{0.4pt}

\begin{document}
	
	
	\begin{center}
		\Large \sc colle 7 = fonctions continues
	\end{center}
\raggedright

\section*{Connaître son cours:}
\begin{enumerate}
	\item Soit $f$ une fonction continue sur un intervalle $I $. Montrer que: si $f$ est injective, alors $f$ est strictement monotone. 
	\item Soit $f : \R \rightarrow \R$ continue en $0$ telle que pour chaque $x \in \R$,
	$f(x) = f(2x)$. Montrer que $f$ est constante.
	\item Soit $f:[0,1]\to[0,1]$ une fonction continue. Démontrer que $f$ admet toujours au moins un point fixe.
\end{enumerate}

\section*{Fonctions continues:}	


\begin{exo}\textit{Théorème des cordes universelles}\quad\\[0.25cm]
Soit $f : [0, 1] \longrightarrow \R$ une application continue telle que $f (0) = f (1)$ et $n \in \N^*$. Définissons
	$$g : \left\{\begin{array}{cll}
	\left[0, 1-\frac{1}{n}\right] &\longrightarrow & \R\\
	x& \longmapsto &f (x + \frac{1}{n} ) - f (x)
	\end{array}\right.$$
	
	\begin{enumerate}
		\item Calculer$\sum\limits_{k=0}^{n-1} g\left(\frac{k}{n}\right)$.
		\item En déduire qu'il existe $\alpha_n \in \left[0, 1 -\frac{1}{n}\right]$ tel que $f (\alpha_n + \frac{1}{n}) = f (\alpha_n )$.
		\item Expliquer à l'aider d'un schéma ce qui est décrit par ce théorème. 
	\end{enumerate}

	\centering
\rule{1\linewidth}{0.6pt}
\end{exo}
	
	

\begin{exo}\textit{}\quad\\[0.25cm]
	Démontrer que si une fonction $f:\mathbb R\to\mathbb R$ est continue en $x_0$, alors $|f|$ est continue en $x_0$. Démontrer que la réciproque est fausse.
	
	
	\centering
	\rule{1\linewidth}{0.6pt}
\end{exo}


\begin{exo}\textit{}\quad\\[0.25cm]

	\begin{defi}\textit{}\quad\\
		$\triangleright$ Un réel $c \in[a, b]$ est un point fixe de l'application $f:[a, b] \rightarrow[a, b]$ si $f(c)=c$.
		
		$\triangleright$ Un réel $c \in[a, b]$ est un 2-cycle de l'application $f:[a, b] \rightarrow[a, b]$ si $f(c) \neq c$ et $f \circ f(c)=c$.
	\end{defi}
 On note $f^{n}$ la composée $n^{\text {ième }}$ de l'application $f:[a, b] \rightarrow I$; par exemple, $f^{0}=$ Id, $f^{1}=f, f^{2}=f \circ f, f^{3}=f \circ f \circ f \ldots$
 
 \begin{enumerate}
	\item Montrer que toute application continue $f:[a, b] \rightarrow[a, b]$ admet un point fixe.
 	\item Montrer que tout point fixe d'une application $f:[a, b] \rightarrow[a, b]$ est un point fixe de $f^{n}$ pour $n>0$.
 	\item Donner un exemple d'application continue $f:[a, b] \rightarrow[a, b]$ qui admet un 2-cycle.
 	\item Donner un exemple d'application continue $f:[a, b] \rightarrow[a, b]$ qui n'admet pas de 2-cycle.
 	\item Déterminer, selon la valeur de $\lambda \in] 0,4]$, les points fixes et les 2-cycles de l'application
 
 \end{enumerate}
	
	\centering
	\rule{1\linewidth}{0.6pt}
\end{exo}

\begin{exo}\textit{}\quad\\[0.25cm]
	Soient $f,g:\mathbb R\to\mathbb R$ deux fonctions continues.
	Montrer que $\inf(f,g)$ et $\sup(f,g)$ sont continues de deux manières différentes.
	
	
	\centering
	\rule{1\linewidth}{0.6pt}
\end{exo}

\begin{exo}\textit{}\quad\\[0.25cm]
	
	Soit $F:\R^+ \to \R$ une application uniformément continue sur $\R^+$.\\[-0.5cm] 
	$$\forall\epsilon>0,\ \exists \alpha>0 , \ \forall(x,y) \in (\R^+)^2, \ |x-y|\leq\alpha \ \alors |f(x)-f(y)|\leq \epsilon $$ On se propose de démontrer qu'il existe
	deux réels $a$ et $b$ tels que, pour tout $x\in[0,+\infty[$, on ait $F(x)\leq ax+b$. Pour cela, on commence par fixer $\eta_1>0$
	tel que 
	$$\forall (x,y)\in([0,+\infty[)^2,\ \big(|x-y|<\eta_1\implies |F(x)-F(y)|\leq 1\big).$$
	On fixe également $x_0\in[0,+\infty[$.
	\begin{enumerate}
		\item Soit $n_0$ le plus petit entier tel que $\frac{x_0}{n_0}\leq \eta_1$; justifier l'existence de $n_0$ et démontrer que $n_0\leq \frac{x_0}{\eta_1}+1$.
		\item Montrer que 
		$$|F(x_0)-F(0)|\leq \sum_{k=0}^{n_0-1}\left|F\left(\frac{(k+1)x_0}{n_0}\right)-F\left(\frac{kx_0}{n_0}\right)\right|.$$
		\item Conclure.
		\item La fonction exponentielle est-elle uniformément continue sur $\R^+$?
	\end{enumerate}
	
	\centering
	\rule{1\linewidth}{0.6pt}
\end{exo}
\begin{exo}\textit{}\quad\\[0.25cm]%https://www.bibmath.net/ressources/index.php?action=affiche&quoi=mathsup/feuillesexo/limitecontinuite&type=fexo
	Soit $f:\mathbb R\to\mathbb R$ périodique et admettant une limite finie $l$ en $+\infty$. Montrer que $f$ est constante.
	
	
	\centering
	\rule{1\linewidth}{0.6pt}
\end{exo}


	
		

\end{document}

Soit $f,g:\mathbb R\to\mathbb R$ deux fonctions continues. On définit une fonction $h:\mathbb R\to\mathbb R$ par 
$$h(x)=\left\{
\begin{array}{ll}
f(x)&\textrm{ si }x\in\mathbb Q\\
g(x)&\textrm{ si }x\in\mathbb R\backslash \mathbb Q.
\end{array}\right.$$
Démontrer que $h$ est continue en $x_0\in\mathbb R$ si et seulement si $f(x_0)=g(x_0)$.