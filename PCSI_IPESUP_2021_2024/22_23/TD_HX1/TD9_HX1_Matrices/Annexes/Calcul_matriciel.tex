
 %Macros utilisées dans la base de données d'exercices 

\newcommand{\mtn}{\mathbb{N}}
\newcommand{\mtns}{\mathbb{N}^*}
\newcommand{\mtz}{\mathbb{Z}}
\newcommand{\mtr}{\mathbb{R}}
\newcommand{\mtk}{\mathbb{K}}
\newcommand{\mtq}{\mathbb{Q}}
\newcommand{\mtc}{\mathbb{C}}
\newcommand{\mch}{\mathcal{H}}
\newcommand{\mcp}{\mathcal{P}}
\newcommand{\mcb}{\mathcal{B}}
\newcommand{\mcl}{\mathcal{L}}
\newcommand{\mcm}{\mathcal{M}}
\newcommand{\mcc}{\mathcal{C}}
\newcommand{\mcmn}{\mathcal{M}}
\newcommand{\mcmnr}{\mathcal{M}_n(\mtr)}
\newcommand{\mcmnk}{\mathcal{M}_n(\mtk)}
\newcommand{\mcsn}{\mathcal{S}_n}
\newcommand{\mcs}{\mathcal{S}}
\newcommand{\mcd}{\mathcal{D}}
\newcommand{\mcsns}{\mathcal{S}_n^{++}}
\newcommand{\glnk}{GL_n(\mtk)}
\newcommand{\mnr}{\mathcal{M}_n(\mtr)}
\newcommand{\veps}{\varepsilon}
\newcommand{\mcu}{\mathcal{U}}
\newcommand{\mcun}{\mcu_n}
\newcommand{\dis}{\displaystyle}
\newcommand{\croouv}{[\![}
\newcommand{\crofer}{]\!]}
\newcommand{\rab}{\mathcal{R}(a,b)}
\newcommand{\pss}[2]{\langle #1,#2\rangle}

\DeclareMathOperator{\ch}{ch}
\DeclareMathOperator{\sh}{sh}
\DeclareMathOperator{\vect}{vect}
\DeclareMathOperator{\card}{card}
\DeclareMathOperator{\comat}{comat}
\DeclareMathOperator{\imv}{Im}
\DeclareMathOperator{\rang}{rg}
\DeclareMathOperator{\Fr}{Fr}
\DeclareMathOperator{\diam}{diam}
\DeclareMathOperator{\supp}{supp}
 %Document 

\begin{document} 

\begin{center}\textsc{{\huge Calcul matriciel}}\end{center}

%1 Exercice 3029 

\vskip0.3cm\noindent\textsc{Exercice 1} - Identité remarquable
\vskip0.2cm
Soit $A,\ B\in M_{2}(\mathbb R)$ les matrices définies par
\begin{equation*}
	 A=\left(\begin{array}{cc} 3 & -1\\-2&0 \end{array} \right)
	\quad \textrm{et} \quad B=\left(\begin{array}{cc} 0 & 1\\3&2 \end{array} \right).
\end{equation*}
Comparer les deux matrices $(A+B)^2$ et $A^2+2AB+B^2$. Puis comparer les deux matrices $(A+B)^2$ et $A^2+AB+BA+B^2$.


%2 Exercice 913


\vskip0.3cm\noindent\textsc{Exercice 2} - Produit non commutatif
\vskip0.2cm
Déterminer deux éléments $A$ et $B$ de
$\mathcal M_2({\mathbb R})$ tels que : $AB=0$ et $BA\not = 0$.


%3 Exercice 2372


\vskip0.3cm\noindent\textsc{Exercice 3} - Matrices stochastiques
\vskip0.2cm
Soit $A,B\in\mathcal M_n(\mathbb R)$ deux matrices telles que la somme des coefficients sur chaque colonne de $A$ et sur chaque colonne de $B$ vaut $1$
(on dit qu'une telle matrice est une matrice stochastique).
Montrer que la somme des coefficients sur chaque colonne de $AB$ vaut $1$.


%4 Exercice 914


\vskip0.3cm\noindent\textsc{Exercice 4} - Puissance $n$-ième - avec la formule du binôme
\vskip0.2cm
Soit $$A=\left(
\begin{array}{ccc}
1&1&0\\
0&1&1\\
0&0&1
\end{array}\right),\quad
I=\left(
\begin{array}{ccc}
1&0&0\\
0&1&0\\
0&0&1
\end{array}\right)\textrm{ et }
B=A-I.$$
Calculer $B^n$ pour tout $n\in\mathbb N$. En déduire $A^n$.


%5 Exercice 915


\vskip0.3cm\noindent\textsc{Exercice 5} - Puissance $n$-ième - avec un polynôme annulateur
\vskip0.2cm
\begin{enumerate}
\item Pour $n\geq 2$, déterminer le reste de la division euclidienne de $X^n$ par $X^2-3X+2$.
\item Soit  $A=\begin{pmatrix} 
0&1&-1\\
-1&2&-1\\
1&-1&2
\end{pmatrix}$. Déduire de la question précédente la valeur de $A^n$, pour $n\geq 2$.
\end{enumerate}


%6 Exercice 3117


\vskip0.3cm\noindent\textsc{Exercice 6} - Produit et somme de matrices nilpotentes
\vskip0.2cm
On dit qu'une matrice $A\in\mathcal M_n(\mathbb K)$ est nilpotente s'il existe $p\in\mathbb N$ tel que $A^p=0$. Démontrer que si $A,B\in\mathcal M_n(\mathbb K)$ sont deux matrices nilpotentes telles que $AB=BA$, alors $AB$ et $A+B$ sont nilpotentes.


%7 Exercice 928


\vskip0.3cm\noindent\textsc{Exercice 7} - Matrices symétriques et anti-symétriques
\vskip0.2cm
Montrer que l'ensemble des matrices symétriques ($A=\ ^t\!A$) et l'ensemble des matrices anti-symétriques ($A=-\ ^t\!A$) sont deux sous-espaces vectoriels supplémentaires de $\mathcal M_n(\mathbb R)$.



%8 Exercice 916


\vskip0.3cm\noindent\textsc{Exercice 8} - Produit et trace
\vskip0.2cm
Soient $A,B\in\mathcal M_n(\mathbb R)$. 
\begin{enumerate}
\item On suppose que $\textrm{tr}(AA^T)=0$. Que dire de la matrice $A$?
\item On suppose que, pour tout $X\in\mathcal M_n(\mathbb R)$, on a $\textrm{tr}(AX)=\textrm{tr}(BX)$. Démontrer que $A=B$.
\end{enumerate}


%9 Exercice 917


\vskip0.3cm\noindent\textsc{Exercice 9} - Centre de $\mathcal M_n(\mathbb R)$.
\vskip0.2cm
Déterminer le centre de $\mathcal M_n(\mathbb R)$, c'est-à-dire l'ensemble des matrices $A\in\mathcal M_n(\mathbb R)$ telle que, pour tout $M\in\mathcal M_n(\mathbb R)$, on a $AM=MA$.


%10 Exercice 922


\vskip0.3cm\noindent\textsc{Exercice 10} - Inverse avec calculs!
\vskip0.2cm
Dire si les matrices suivantes sont inversibles et, le
cas échéant, calculer leur inverse :
$$A=\left(
\begin{array}{rcl}
1&1&2\\
1&2&1\\
2&1&1
\end{array}
\right),\quad
B=\left(
\begin{array}{rcl}
0&1&2\\
1&1&2\\
0&2&3
\end{array}
\right),\quad 
C=\left(
\begin{array}{rcl}
1&4&7\\
2&5&8\\
3&6&9
\end{array}\right),\quad
I=\left(
\begin{array}{rcl}
i&-1&2i\\
2&0&2\\
-1&0&1
\end{array}\right).$$


%11 Exercice 527


\vskip0.3cm\noindent\textsc{Exercice 11} - Sans problèmes
\vskip0.2cm
Résoudre les systèmes linéaires suivants :
$$\left\{
\begin{array}{rcl}
x+y+2z&=&3\\
x+2y+z&=&1\\
2x+y+z&=&0
\end{array}\right.
\quad\quad\quad
\left\{
\begin{array}{rcl}
x+2z&=&1\\
-y+z&=&2\\
x-2y&=&1
\end{array}\right.$$


%12 Exercice 528


\vskip0.3cm\noindent\textsc{Exercice 12} - Trop d'inconnues ou d'équations
\vskip0.2cm
Résoudre les systèmes suivants :
\begin{eqnarray*}
\left\{
\begin{array}{rcl}
x+y+z-3t&=&1\\
2x+y-z+t&=&-1
\end{array}\right.
\quad\quad\quad
\left\{
\begin{array}{rcl}
x+2y-3z&=&4\\
x+3y-z&=&11\\
2x+5y-5z&=&13\\
x+4y+z&=&18
\end{array}\right.
\end{eqnarray*}


%13 Exercice 529


\vskip0.3cm\noindent\textsc{Exercice 13} - Paramètre dans le second membre
\vskip0.2cm
Discuter suivant la valeur du paramètre $m\in\mathbb R$ le système :$$\left\{
\begin{array}{rcl}
3x+y-z&=&1\\
x-2y+2z&=&m\\
x+y-z&=&1
\end{array}\right.$$



%14 Exercice 2872


\vskip0.3cm\noindent\textsc{Exercice 14} - Discussion suivant deux valeurs
\vskip0.2cm
Résoudre le système suivant, en discutant suivant la valeur du paramètre $m$.
$$\left\{
\begin{array}{rcl}
 x+y+mz&=&0\\
 x+my+z&=&0\\
 mx+y+z&=&0
\end{array}\right.$$


%15 Exercice 2643


\vskip0.3cm\noindent\textsc{Exercice 15} - Polynômes vérifiant certaines propriétés
\vskip0.2cm
Déterminer tous les triplets $(a,b,c)\in\mathbb R^3$ tels que le polynôme $P(x)=ax^2+bx+c$ vérifie
\begin{enumerate}
 \item $P(-1)=5$, $P(1)=1$ et $P(2)=2$;
 \item $P(-1)=4$ et $P(2)=1$.
\end{enumerate}




\vskip0.5cm
\noindent{\small Cette feuille d'exercices a été conçue à l'aide du site \textsf{http://www.bibmath.net}}

%Vous avez accès aux corrigés de cette feuille par l'url : https://www.bibmath.net/ressources/justeunefeuille.php?id=25821
\end{document}