\documentclass[a4paper,10pt]{article}



\usepackage{fancyhdr} % pour personnaliser les en-têtes
\usepackage[utf8]{inputenc}
\usepackage{lastpage}
\usepackage[frenchb]{babel}
\usepackage{amsfonts,amssymb}
\usepackage{amsmath,amsthm}
\usepackage{paralist}
\usepackage{xspace}
\usepackage{xcolor,multicol}
\usepackage{variations}
\usepackage{xypic}
\usepackage{eurosym}
\usepackage{graphicx}
\usepackage[np]{numprint}
\usepackage{hyperref} 
\usepackage{listings} % pour écrire des codes avec coloration syntaxique  

\usepackage{tikz}
\usetikzlibrary{calc, arrows, plotmarks,decorations.pathreplacing}
\usepackage{colortbl}
\usepackage{multirow}
\usepackage[top=2cm,bottom=1.5cm,right=2cm,left=1.5cm]{geometry}

\newtheorem{thm}{Théorème}
\newtheorem*{pro}{Propriété}
\newtheorem*{exemple}{Exemple}

\theoremstyle{definition}
\newtheorem*{remarque}{Remarque}
\theoremstyle{definition}
\newtheorem{exo}{Exercice}
\newtheorem{definition}{Définition}


\newcommand{\vtab}{\rule[-0.4em]{0pt}{1.2em}}
\newcommand{\V}{\overrightarrow}
\renewcommand{\thesection}{\Roman{section} }
\renewcommand{\thesubsection}{\arabic{subsection} }
\renewcommand{\thesubsubsection}{\alph{subsubsection} }

\newcommand{\C}{\mathbb{C}}
\newcommand{\R}{\mathbb{R}}
\newcommand{\Q}{\mathbb{Q}}
\newcommand{\Z}{\mathbb{Z}}
\newcommand{\N}{\mathbb{N}}


\definecolor{vert}{RGB}{11,160,78}
\definecolor{rouge}{RGB}{255,120,120}
% Set the beginning of a LaTeX document
\pagestyle{fancy}
\lhead{Optimal Sup Spé, groupe IPESUP}\chead{Année~2021-2022}\rhead{Niveau: Première année de PCSI }\lfoot{M. Botcazou}\cfoot{\thepage/2}\rfoot{mail: ibotca52@gmail.com }\renewcommand{\headrulewidth}{0.4pt}\renewcommand{\footrulewidth}{0.4pt}

\begin{document}
	
	
	\begin{center}
		\Large \sc colle 5 = suites numériques et fonctions continues
	\end{center}
	
\section *{Questions de cours:}
%\hyperref{http://www.normalesup.org/~sage/Enseignement/TSI/index.html}
\noindent Soient $(u_n)_{n\in\N}$ , $(v_n)_{n\in\N}$,  $(u'_n)_{n\in\N}$ et $(v'_n)_{n\in\N}$ trois suites réelles ou complexes et $\lambda \in \C $. 
\begin{enumerate} %[$\square$]
\item Démontrer les affirmations suivantes: \begin{enumerate}
\item Si $u_n= o\left(u'_n\right)$ et $v_n = o\left(u'_n\right)$ alors $u_n+v_n = o\left(u'_n\right)$
\item Si $u_n= o\left(u'_n\right)$ alors $\lambda u_n= o\left(u'_n\right)$ 
\item  Si $u_n= o\left(u'_n\right)$ alors $ u_nv_n= o\left(u'_nv_n\right)$ 
\end{enumerate}
\item Rappeler le théorème de Cesàro et donner sa démonstration
\item Démontrer les affirmations suivantes: \begin{enumerate}
\item Si $u_n \sim_{+\infty}u'_n$ alors $u'_n \sim_{+\infty}u_n$
\item  Si $u_n \sim_{+\infty}u'_n$ et $v_n \sim_{+\infty}v'_n$ alors $u_nv_n \sim_{+\infty}u'_nv'_n$ 
\item Si $u_n \sim_{+\infty}u'_n$ alors $\dfrac{1}{u_n} \sim_{+\infty}\dfrac{1}{u'_n}$
\item Si $u_n \sim_{+\infty}u'_n$ et $v_n \sim_{+\infty}v'_n$ alors $\dfrac{u_n}{v_n} \sim_{+\infty}\dfrac{u'_n}{v'_n}$ 
\end{enumerate}
\item Rappeler la formule de Stirling et donner un équivalent de $\left(\begin{matrix}
2n\\n
\end{matrix}\right)$
\item Démontrer la propriété suivante:
\begin{pro}\hfil\\
$f$ tend vers $l$  en $a$ si et seulement si  pour toute suite $(u_n)_{n\in\N}$ convergeant vers $a$, $(f(u_n))_{n\in\N}$ converge vers $l$
\end{pro} 
\end{enumerate}

\section*{Suites numériques :}

\begin{minipage}{1\linewidth}
\begin{minipage}[t]{0.48\linewidth}
\raggedright
\begin{exo}\quad\\
Soit $(u_n)_{n\in\N}$ une suite à termes réels strictement positifs telle que $\left(\dfrac{u_{n+1}}{u_n}\right)_{n\in\N}$ converge vers un réel $l\in\R^+$.
\begin{enumerate}
\item On suppose $l<1$ et on fixe $\epsilon>0$ tel que $l+\epsilon<1$. 
\begin{enumerate}
\item Démontrer qu'il existe un entier $n_0\in\N$ tel que, pour $n\geq n_0$, on a 
$$u_n\leq\left(l+\epsilon\right)^{n-n_0}u_{n_0}$$
\item En déduire que la suite $(u_n)_{n\in\N}$ est convergente et donner sa limite. 


\end{enumerate}
\item  On suppose $l>1$. Démontrer que $(u_n)$ diverge vers $+\infty$.
\item Étudier le cas $l=1$\\
(\textit{Indication: étudier les suites } $(n^\alpha)_{n\in\N^*}$)
\end{enumerate} 
\centering
\rule{1\linewidth}{0.6pt}
\end{exo}

\begin{exo}\quad\\

Soient $(u_n)$ et $(v_n)$ deux suites réelles convergeant respectivement vers $u$ et $v$. Montrer que la suite $w_n=\dfrac{u_0v_n+...+u_nv_0}{n+1}$ converge vers $uv$.\\\hfill\\
(\textit{Indication: on coupera ici la somme en 3 en isolant les bords})

\centering\rule{1\linewidth}{0.6pt}
\end{exo}

\end{minipage}	
\hfill\vrule\hfill
\begin{minipage}[t]{0.48\linewidth}
\raggedright

\begin{exo}\quad\\
Soit $(u_n)$ une suite de réels positifs vérifiant $$u_n\leq \dfrac{1}{k}+\dfrac{k}{n}$$ pour tous $(k,n)\in(\N^*)^2$.\\ Démontrer que $(u_n)$ tend vers $0$.

\centering
\rule{1\linewidth}{0.6pt}
\end{exo}

\begin{exo}\quad\\
Démontrer que  
\begin{enumerate}
\item$ \ln(n+ e^n) \sim_{+\infty} n$
\item $b^n-a^n   \sim_{+\infty} a^n+b^n$, \ $0<a<b$
\item $ 4\ln(1+\sqrt{n}) \sim_{+\infty} \ln(1+n^2)$
\end{enumerate}
\centering
\rule{1\linewidth}{0.6pt}
\end{exo}
\begin{exo}\quad\\
Montrer que  

$$\sum_{k=1}^{n-1}k! \ =_{+\infty} \  o\left(n!\right)$$

En déduire que
$$\sum_{k=1}^{n}k!  \sim_{+\infty} n!$$
\centering
\rule{1\linewidth}{0.6pt}
\end{exo}
\end{minipage}
\end{minipage}
\newpage
\section*{Fonctions continues:}
\begin{minipage}{1\linewidth}
\begin{minipage}[t]{0.48\linewidth}
\raggedright
\begin{exo}\quad\\
Soit $f: \R^*\rightarrow\R$ la fonction définie par $$f(x) = x\sqrt{1+\dfrac{1}{x^2}}$$
La fonction $f$ admet-elle un prolongement par continuité en $0$ ? 

\centering
\rule{1\linewidth}{0.6pt}
\end{exo}

\begin{exo}\quad\\
Donner si elles existe les limites suivantes:
\begin{multicols}{2}
\begin{enumerate}
\item $\lim\limits_{x\rightarrow+\infty} \dfrac{\lfloor 2x\rfloor}{\lfloor x\rfloor}$
\item $\lim\limits_{x\rightarrow 0}~ \left\lfloor\dfrac{ 1}{ x} \right\rfloor$
\item $\lim\limits_{x\rightarrow 0} ~x\left\lfloor\dfrac{ 1}{ x} \right\rfloor$
\item$\lim\limits_{x\rightarrow 0} ~x^2\left\lfloor\dfrac{ 1}{ x} \right\rfloor$
\end{enumerate}
\end{multicols}
\centering\rule{1\linewidth}{0.6pt}
\end{exo}

\begin{exo}\quad\\
Soit $f: \R\rightarrow\R$ la fonction définie par $$f(x) =\lfloor x\rfloor + \sqrt{x - \lfloor x\rfloor }$$
Montrer que la fonction $f$ est continue sur $\R$.\\
(\textit{Indication: Étudier }$f(x+1)$)

\centering
\rule{1\linewidth}{0.6pt}
\end{exo}

\end{minipage}	
\hfill\vrule\hfill
\begin{minipage}[t]{0.48\linewidth}
\raggedright

\begin{exo}\quad\\
Soit $f: \R\rightarrow\R$ la fonction définie par 
$$f(x) =\left\{\begin{array}{cl}
1 & \text{si} \  x \in\Q\\
0 & \text{sinon} 
\end{array}\right.$$
Montrer que la fonction $f$ est discontinue en tout point de $\R$.\\

\centering
\rule{1\linewidth}{0.6pt}
\end{exo}
\begin{exo}\quad\\
Soit $f: \R\rightarrow\R$ périodique et admettant une limite finie $l$ en $+\infty$. Montrer que $f$ est constante.

\centering
\rule{1\linewidth}{0.6pt}
\end{exo}

\begin{exo}\quad\\
Étudier les limites suivantes
\begin{multicols}{2}
\begin{enumerate}
\item $\lim\limits_{x\rightarrow+\infty} \dfrac{e^{3x}+2x+7}{e^x + e^{-x}}$
\item $\lim\limits_{x\rightarrow 0}~\dfrac{\sqrt{1+x} - \left(1+ \dfrac{x}{2}\right) }{x^2 } $

\end{enumerate}
\end{multicols}
\centering\rule{1\linewidth}{0.6pt}
\end{exo}
\begin{exo}\quad\\
Soit $f: \R\rightarrow\R$ la fonction définie par 
$$f(x) =\left\{\begin{array}{cl}
0  & \text{Si} \ x \   \text{est irrationnel ou} \  x=0. \\
\dfrac{1}{q} & \text{Si} \  x=\dfrac{p}{q},\ \text{avec} \  p\in\Z, q\geq 1 \ \text{et} \ pgcd(p,q) =1  
\end{array}\right.$$
Montrer que la fonction $f$ est continue sur $\R \backslash \Q$, discontinue sur $\Q^*$ \\

\centering
\rule{1\linewidth}{0.6pt}
\end{exo}


\end{minipage}
\end{minipage}

\end{document}