\documentclass[a4paper,10pt]{article}



\usepackage{fancyhdr} % pour personnaliser les en-têtes
\usepackage[utf8]{inputenc}
\usepackage[T1]{fontenc}
\usepackage{lastpage}
\usepackage[frenchb]{babel}
\usepackage{amsfonts,amssymb}
\usepackage{amsmath,amsthm,mathtools}
\usepackage{paralist}
\usepackage{xspace,xypic}
\usepackage{xcolor,multicol,tabularx}
\usepackage{variations}
\usepackage{xypic}
\usepackage{eurosym,multicol}
\usepackage{graphicx}
\usepackage{mathdots}%faire des points suspendus en diagonale
\usepackage[np]{numprint}
\usepackage{hyperref} 
\usepackage{listings} % pour écrire des codes avec coloration syntaxique  

\usepackage{tikz}
\usetikzlibrary{calc, arrows, plotmarks,decorations.pathreplacing}
\usepackage{colortbl}
\usepackage{multirow}
\usepackage[top=2cm,bottom=1.5cm,right=2cm,left=1.5cm]{geometry}

\newtheorem{thm}{Théorème}
\newtheorem*{pro}{Propriété}
\newtheorem*{exemple}{Exemple}

\theoremstyle{definition}
\newtheorem*{remarque}{Remarque}
\theoremstyle{definition}
\newtheorem{exo}{Exercice}
\newtheorem{definition}{Définition}


\newcommand{\vtab}{\rule[-0.4em]{0pt}{1.2em}}
\newcommand{\V}{\overrightarrow}
\renewcommand{\thesection}{\Roman{section} }
\renewcommand{\thesubsection}{\arabic{subsection} }
\renewcommand{\thesubsubsection}{\alph{subsubsection} }
\newcommand*{\transp}[2][-3mu]{\ensuremath{\mskip1mu\prescript{\smash{\mathrm t\mkern#1}}{}{\mathstrut#2}}}%

\newcommand{\K}{\mathbb{K}}
\newcommand{\C}{\mathbb{C}}
\newcommand{\R}{\mathbb{R}}
\newcommand{\Q}{\mathbb{Q}}
\newcommand{\Z}{\mathbb{Z}}
\newcommand{\N}{\mathbb{N}}
\newcommand{\p}{\mathbb{P}}

\renewcommand{\Im}{\mathop{\mathrm{Im}}\nolimits}



\definecolor{vert}{RGB}{11,160,78}
\definecolor{rouge}{RGB}{255,120,120}
% Set the beginning of a LaTeX document
\pagestyle{fancy}
\lhead{Optimal Sup Spé, groupe IPESUP}\chead{Année~2021-2022}\rhead{Niveau: Première année de PCSI }\lfoot{M. Botcazou}\cfoot{\thepage}\rfoot{mail: ibotca52@gmail.com }\renewcommand{\headrulewidth}{0.4pt}\renewcommand{\footrulewidth}{0.4pt}

\begin{document}
\medskip 	
	
	\begin{center}
		\Large \sc colle 20 = Développements limités,comportement asymptotique et Séries numériques
	\end{center}
\medskip 
\section*{Développements limités,comportement asymptotique:}\hfill\\%[-0.25cm]
\begin{minipage}{1\linewidth}
	\begin{minipage}[t]{0.48\linewidth}
		\raggedright
		
		
		
		\begin{exo}\quad\\[0.25cm]
			Etudier l'existence et la valeur éventuelle des limites suivantes
			\begin{enumerate}
				\item $\lim_{x\rightarrow \pi/2}(\sin x)^{1/(2x-\pi)}$
				\item $\lim_{n\rightarrow +\infty}\left(\cos(\frac{n\pi}{3n+1})+\sin(\frac{n\pi}{6n+1})\right)^n$
				\item $\lim_{x\rightarrow 0}(\cos x)^{\ln|x|}$
				\item $\lim_{x\rightarrow e,\;x<e}(\ln x)^{\ln(e-x)}$
				\item $\lim_{x\rightarrow +\infty}\left(\frac{\ln(x+1)}{\ln x}\right)^x$
			\end{enumerate}
			
			\centering
			\rule{1\linewidth}{0.6pt}
		\end{exo}
	
			\begin{exo}\quad\\[0.25cm]
			\begin{enumerate}
				\item  Développement asymptotique à la précision $x^2$ en $0$ de $\frac{1}{x(e^x-1)}-\frac{1}{x^2}$.
				\item  Développement asymptotique à la précision $\frac{1}{x^3}$ en $+\infty$ de $x\ln(x+1)-(x+1)\ln x$.
			\end{enumerate}	
		
		\centering
		\rule{1\linewidth}{0.6pt}
	\end{exo}	
	

		
		
		
		
	\end{minipage}	
	\hfill\vrule\hfill
	\begin{minipage}[t]{0.48\linewidth}
		\raggedright
		
		
		\begin{exo}\quad\\[0.25cm]
			\begin{enumerate}
				\item  Donner un équivalent simple en $+\infty$ et $-\infty$ de $\sqrt{x^2+3x+5}-x+1$.
				\item Donner un équivalent simple en $0$ de $(\sin x)^{x-x^2}-(x-x^2)^{\sin x}$.
				\item  Donner un équivalent simple en $+\infty$ de $x^{\tanh x}$.
			\end{enumerate}
			
			\rule{1\linewidth}{0.6pt}
		\end{exo}	

		\begin{exo}\quad\\[0.25cm]
		
		Soit $u_0\in]0,\frac{\pi}{2}]$. Pour $n\in\N$, on pose $u_{n+1}=\sin(u_n)$.
		\begin{enumerate}
			\item  Montrer brièvement que la suite $u$ est strictement positive et converge vers $0$.
			\item 
			\begin{enumerate}
				\item Déterminer un réel $\alpha$ tel que la suite $u_{n+1}^\alpha-u_n^\alpha$ ait une limite finie non nulle.
				\item En utilisant le lemme de \textsc{Cesaro}, déterminer un équivalent simple de $u_n$.
			\end{enumerate}
		\end{enumerate}
		\centering
		\rule{1\linewidth}{0.6pt}
		\end{exo}	
		
		
	\end{minipage}
\end{minipage}	

\bigskip 

\section*{Séries numériques:}\hfill\\%[-0.25cm]
\begin{minipage}{1\linewidth}
	\begin{minipage}[t]{0.48\linewidth}
		\raggedright
		
		
		
		\begin{exo}\quad\\[0.25cm]
			Donner un développement asymptotique à la précision $\frac{1}{n^3}$ de $u_n=\frac{1}{n!}\sum_{k=0}^{n}k!$.
			
			\centering
			\rule{1\linewidth}{0.6pt}
		\end{exo}
		
		\begin{exo}\quad\\[0.25cm]
			Donner la nature de la série de terme général:
			\begin{multicols}{2}
				\begin{enumerate}
					\item $\ln\left(\frac{n^2+n+1}{n^2+n-1}\right)$
					\item $\frac{1}{n+(-1)^n\sqrt{n}}$
					\item $\left(\frac{n+3}{2n+1}\right)^{\ln n}$
					\item $\frac{n^2}{(n-1)!}$
				\end{enumerate}
			\end{multicols}
			 
			
			\centering
			\rule{1\linewidth}{0.6pt}
		\end{exo}	
		
			\begin{exo}\quad\\[0.25cm]
			
			Trouver un développement limité à l'ordre $4$ quand $n$ tend vers l'infini de $\left(e-\sum_{k=0}^{n}\frac{1}{k!}\right)\times(n+1)!$.
			
			\centering
			\rule{1\linewidth}{0.6pt}
		\end{exo}	
		
					\begin{exo}\quad\\[0.25cm]
						Soit $(u_n)_{n\in\N}$ une suite de réels strictement positifs. Montrer que les séries de termes généraux $u_n$, $\frac{u_n}{1+u_n}$, $\ln(1+u_n)$ et $\int_{0}^{u_n}\frac{dx}{1+x^e}$  sont de mêmes natures.
			
			\centering
			\rule{1\linewidth}{0.6pt}
		\end{exo}
		
		
		
	\end{minipage}	
	\hfill\vrule\hfill
	\begin{minipage}[t]{0.48\linewidth}
		\raggedright
		
		
		\begin{exo}\quad\\[0.25cm]
			Soit $(u_n)_{n\in\N}$ une suite décroissante de nombres réels strictement positifs telle que la série de terme général $u_n$ converge.\\[0.2cm] Montrer que $u_n\underset{n\rightarrow+\infty}{=}o\left(\frac{1}{n}\right)$. Trouver un exemple de suite $(u_n)_{n\in\N}$ de réels strictement positifs telle que la série de terme général $u_n$ converge mais telle que la suite de terme général $nu_n$ ne tende pas vers $0$.
			
			\centering
			\rule{1\linewidth}{0.6pt}
		\end{exo}	
		
		\begin{exo}\quad\\[0.25cm]
			Donner la nature de la série de terme général $u_n=\sum_{k=1}^{n}\frac{1}{(n+k)^p}$, $p\in]0,+\infty[$.
			
			\centering
			\rule{1\linewidth}{0.6pt}
		\end{exo}
	
	
		\begin{exo}\quad\\[0.25cm]
		 Calculer $\sum_{n=0}^{+\infty}\frac{(-1)^n}{3n+1}$.
		
		\centering
		\rule{1\linewidth}{0.6pt}
		\end{exo}
	
	\begin{exo}\quad\\[0.25cm]
		Donner un développement limité à l'ordre $4$ de $\sum_{k=n+1}^{+\infty}\frac{1}{k^2}$ quand $n$ tend vers l'infini.
		
		\centering
		\rule{1\linewidth}{0.6pt}
	\end{exo}	
		
		
	\end{minipage}
\end{minipage}	



\end{document}