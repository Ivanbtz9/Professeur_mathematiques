\documentclass[11pt]{article}

 %Configuration de la feuille 
 
\usepackage{amsmath,amssymb,enumerate,graphicx,pgf,tikz,fancyhdr}
\usepackage[utf8]{inputenc}
\usetikzlibrary{arrows}
\usepackage{geometry}
\usepackage{tabvar}
\geometry{hmargin=2.2cm,vmargin=1.5cm}\pagestyle{fancy}
\lfoot{\bfseries http://www.bibmath.net}
\rfoot{\bfseries\thepage}
\cfoot{}
\renewcommand{\footrulewidth}{0.5pt} %Filet en bas de page

 %Macros utilisées dans la base de données d'exercices 

\newcommand{\mtn}{\mathbb{N}}
\newcommand{\mtns}{\mathbb{N}^*}
\newcommand{\mtz}{\mathbb{Z}}
\newcommand{\mtr}{\mathbb{R}}
\newcommand{\mtk}{\mathbb{K}}
\newcommand{\mtq}{\mathbb{Q}}
\newcommand{\mtc}{\mathbb{C}}
\newcommand{\mch}{\mathcal{H}}
\newcommand{\mcp}{\mathcal{P}}
\newcommand{\mcb}{\mathcal{B}}
\newcommand{\mcl}{\mathcal{L}}
\newcommand{\mcm}{\mathcal{M}}
\newcommand{\mcc}{\mathcal{C}}
\newcommand{\mcmn}{\mathcal{M}}
\newcommand{\mcmnr}{\mathcal{M}_n(\mtr)}
\newcommand{\mcmnk}{\mathcal{M}_n(\mtk)}
\newcommand{\mcsn}{\mathcal{S}_n}
\newcommand{\mcs}{\mathcal{S}}
\newcommand{\mcd}{\mathcal{D}}
\newcommand{\mcsns}{\mathcal{S}_n^{++}}
\newcommand{\glnk}{GL_n(\mtk)}
\newcommand{\mnr}{\mathcal{M}_n(\mtr)}
\DeclareMathOperator{\ch}{ch}
\DeclareMathOperator{\sh}{sh}
\DeclareMathOperator{\vect}{vect}
\DeclareMathOperator{\card}{card}
\DeclareMathOperator{\comat}{comat}
\DeclareMathOperator{\imv}{Im}
\DeclareMathOperator{\rang}{rg}
\DeclareMathOperator{\Fr}{Fr}
\DeclareMathOperator{\diam}{diam}
\DeclareMathOperator{\supp}{supp}
\newcommand{\veps}{\varepsilon}
\newcommand{\mcu}{\mathcal{U}}
\newcommand{\mcun}{\mcu_n}
\newcommand{\dis}{\displaystyle}
\newcommand{\croouv}{[\![}
\newcommand{\crofer}{]\!]}
\newcommand{\rab}{\mathcal{R}(a,b)}
\newcommand{\pss}[2]{\langle #1,#2\rangle}
 %Document 

\begin{document} 

\begin{center}\textsc{{\huge }}\end{center}

% Exercice 979


\vskip0.3cm\noindent\textsc{Exercice 1} - Pour s'échauffer...
\vskip0.2cm
\begin{enumerate}
\item Calculer le déterminant suivant :
$$\left|\begin{array}{cccc}
1&1&1&1\\
1&-1&1&1\\
1&1&-1&1\\
1&1&1&-1
\end{array}\right|.$$
\item Soit $E$ un $\mathbb R$-espace vectoriel et $f\in\mathcal L(E)$ tel que $f^2=-Id_E$. Que dire de la dimension de $E$?
\end{enumerate}


% Exercice 2931


\vskip0.3cm\noindent\textsc{Exercice 2} - Déterminant et matrice antisymétrique
\vskip0.2cm
Soit $A\in\mathcal M_{2n}(\mathbb R)$ une matrice antisymétrique et soit $J\in\mathcal M_{2n}(\mathbb R)$ dont tous les coefficients sont égaux à $1$. Démontrer que, pour tout $x\in\mathbb R$, $\det(A+xJ)=\det(A)$.


% Exercice 1003


\vskip0.3cm\noindent\textsc{Exercice 3} - Avec des coefficients binomiaux
\vskip0.2cm
Soient $n\geq 1$, $p\geq 0$. Calculer le déterminant suivant :
$$\left|
\begin{array}{cccc}
\binom{n}0&\binom n1&\dots&\binom np\\
\binom{n+1}0&\binom{n+1}1&\dots&\binom{n+1}p\\
\vdots&\vdots&&\vdots\\
\binom{n+p}0&\binom{n+p}1&\dots&\binom{n+p}p
\end{array}
\right|.
$$


% Exercice 1004


\vskip0.3cm\noindent\textsc{Exercice 4} - Sur des polynômes
\vskip0.2cm
Soit $u\in\mathcal L(\mathbb R_n[X])$. Calculer $\det(u)$ dans chacun des cas suivants :
\begin{enumerate}
\item $u(P)=P+P'$;
\item $u(P)=P(X+1)-P(X)$;
\item $u(P)=XP'+P(1)$.
\end{enumerate}


% Exercice 1007


\vskip0.3cm\noindent\textsc{Exercice 5} - Inversibilité dans $\mathcal M_n(\mathbb Z)$
\vskip0.2cm
Soit $M\in \mathcal M_n(\mathbb Z)$. Donner une condition nécessaire et suffisante pour
que $M$ soit inversible et que $M^{-1}$ soit dans $\mathcal M_n(\mathbb Z)$.


% Exercice 1008


\vskip0.3cm\noindent\textsc{Exercice 6} - Rang de la comatrice et applications
\vskip0.2cm
Soit $A\in\mcmnr$.
\begin{enumerate}
\item Discuter le rang de $\comat A$ en fonction du rang de $A$.
\item Résoudre, pour $n\geq 3$, l'équation $\comat A=A$.
\end{enumerate}


% Exercice 1013


\vskip0.3cm\noindent\textsc{Exercice 7} - Polynômes
\vskip0.2cm
Soient $(z_0,\dots,z_{n})$ des nombres complexes deux à deux distincts. Montrer que la famille
$$\big( (X-z_0)^n,(X-z_1)^n,\dots,(X-z_n)^n\big)$$
est une base de $\mathbb C_n[X]$.


% Exercice 1014


\vskip0.3cm\noindent\textsc{Exercice 8} - Similarité
\vskip0.2cm
Soit $A,B\in M_n(\mathbb R)$. On suppose que $A$ et $B$ sont semblables sur $\mathbb C$,
ie qu'il existe $P\in Gl_n(\mathbb C)$ tel que $A=PBP^{-1}$. Montrer que $A$ et $B$ sont semblables
sur $\mathbb R$.


% Exercice 1015


\vskip0.3cm\noindent\textsc{Exercice 9} - Densité des matrices inversibles
\vskip0.2cm
Soit $A$ une matrice carrée d'ordre $n$ à coefficients complexes. Montrer :
$$\exists \alpha>0,\ \forall\veps\in\mtr,\ 0<|\veps|<\alpha,\ A+\veps I_n \textrm{ est inversible.}$$


% Exercice 2669


\vskip0.3cm\noindent\textsc{Exercice 10} - Lien entre la trace et le déterminant
\vskip0.2cm
Soit $E$ un espace vectoriel de dimension $n$ dont une base est $\mathcal{B}$. Soient $(x_{1},\ldots,x_{n})\in E$ et $f\in L(E)$. 

Démontrer que $\sum_{k=1}^{n}\det_{\mathcal{B}}(x_{1},\ldots,f(x_{k}),\ldots,x_{n})=\textrm{Tr}(f)\det_{\mathcal{B}}(x_{1},\text{…},x_{n})$




\vskip0.5cm
\noindent{\small Cette feuille d'exercices a été conçue à l'aide du site \textsf{https://www.bibmath.net}}

%Vous avez accès aux corrigés de cette feuille par l'url : https://www.bibmath.net/ressources/justeunefeuille.php?id=28076
\end{document}