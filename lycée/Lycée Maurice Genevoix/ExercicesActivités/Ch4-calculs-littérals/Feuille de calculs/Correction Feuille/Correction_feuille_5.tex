\documentclass[10pt,a4paper]{article}


\usepackage[T1]{fontenc}
\usepackage[francais]{babel}
\usepackage{times}
\usepackage[utf8]{inputenc}
\usepackage{enumitem}
\usepackage{multicol}
\usepackage{fancyhdr}
\usepackage{tcolorbox}
\usepackage{tablists}
\usepackage[a4paper,bottom = 50pt]{geometry}
\usepackage{color}
\usepackage{amsmath,amssymb,amsthm, mathrsfs,pifont}
\usepackage{pgf,tikz,pgfplots,tkz-tab}
\usepackage{hyperref}
\usepackage{cancel}
\usepackage{array}
\usepackage{xcolor} 
\usepackage{amssymb}
\usepackage{amsthm}
\usepackage{graphicx}
\usepackage{pstricks}

\usepackage{geometry}
\geometry{top=1.5cm, bottom=2cm, left=2cm, right=1.5cm}


\newtheorem{thm}{Théorème}
\newtheorem*{pro}{Propriété}
\newtheorem*{exemple}{Exemple}

\theoremstyle{definition}
\newtheorem*{remarque}{Remarque}
\theoremstyle{definition}
\newtheorem{exo}{Exercice}
\newtheorem{definition}{Définition}

\renewcommand{\vec}{\overrightarrow}

\begin{document}
	
	%\leftline{\bfseries Lycée Le Corbusier \hfill Année~2020-2021}
	\leftline{\bfseries Niveau: Seconde  }
	\leftline{\bfseries M.Botcazou }
	\leftline{\bfseries mail: ibotca52@gmail.com}
	\rule[0.5ex]{\textwidth}{0.1mm}	
	
	\begin{center}
		\large \sc Correction Feuille 5 : Calcul littéral  	
	\end{center}
\section*{Puissances entières:}\quad\\

\begin{center}
	\begin{minipage}[c]{0.4\linewidth}
		\raggedright
		\begin{exo}\quad\\
			Calculer les puissances suivantes:\hfill\textbf{}\\
			\begin{multicols}{2}
				\begin{enumerate}
					\item $(-9)^0 \ = \ 1$
					\item $10^3 \ = \ 1000$
				
					\item $(-2)^3 \ = \ -8$
					\item $-7^2 \ = \ -49$
					
				\end{enumerate}
			\end{multicols}
		\end{exo}
		\begin{exo}\quad\hfill\textbf{}\\
		Calculer les puissances négatives suivantes:
		
			\begin{enumerate}
				\item $10^{-5} \ = \ \dfrac{1}{10^5} \ = \ 0,00001$
				\item $1^{-3} \ = \ \dfrac{1}{1^3} \ = \ \dfrac{1}{1} \ = \ 1$
				
				\item $(-2)^{-3} \ = \ \dfrac{1}{(-2)^{3}} \ = \ \dfrac{1}{-8} \ = \ -\dfrac{1}{8}$
				\item $-6^{-2} \ = \ \dfrac{1}{-6^2} \ = \ \dfrac{1}{-36} \ = \ -\dfrac{1}{36}$
			\end{enumerate}
	
	\end{exo}
	\begin{exo}\quad\hfill\textbf{}\\
		Exprimer sous la forme d'une seule puissance
		
			\begin{enumerate}
				\item $4^5 \times 4^7 \ = \ 4^{5+7}  \  =  \  4^{12}$
				\item $7^3 \times 7^{-2} \ = \ 7^{3-2} \ = \ 7^1 \ = \ 7 $
			
				\item $10^3 \times 10^{-4} \times 10^5 \ = \ 10^{3-4+5} \ = \ 10^{4}$
				\item $5^4\times (5^{-1})^2 \ = \ 5^4 \times 5^{-1\times2} \ = \ 5^{4-2} \ = \ 5^2$
			\end{enumerate}
	
	\end{exo}
\begin{exo}\quad\hfill\textbf{}\\
	Exprimer sous la forme d'une seule puissance
	
	\begin{enumerate}
		
		\item $\dfrac{5^4}{5^6} \ = \ 5^{4-6} \ = \ 5^{-2} \ = \ \dfrac{1}{5^2} \ = \ \dfrac{1}{25}$
		\item $\dfrac{4^3}{4^2}\times \dfrac{4^5}{4^6} \ = \ 4^{3-2}\times4^{5-6} $ \\[2mm ] $\ = \ 4^{3-2+5-6} \ = \ 4^{0} \ = \ 1$

	\end{enumerate}
	
\end{exo}
	\end{minipage}
	\hfill\vrule\hfill
	\begin{minipage}[c]{0.4\linewidth}
		\raggedright
		
		
				\begin{enumerate}
		
					\item[3.] $7^{-1}\times \dfrac{7^3}{7^4} \ = \ 7^{-1+3-4} \ = \ 7^{-2}\ = \ \dfrac{1}{7^2} $ \\[2mm] $\ = \ \dfrac{1}{49}$
					\item[4.] $\dfrac{3^2}{3^{-2}} \ = \ 3^{2-(-2)} \ = \ 3^{2+2} \ = \ 3^4$
				\end{enumerate}

	\begin{exo}\quad\\
		Calculer les puissances suivantes:\hfill\textbf{}\\
	
			\begin{enumerate}
				
				\item $\left(\dfrac{5}{3}\right)^2 \ = \ \dfrac{5^2}{3^2} \ = \ \dfrac{25}{9}$
				\item $\left(\dfrac{1}{4}\right)^{-2} \ = \ \left(\dfrac{4}{1}\right)^{2} \ = \ 4^2 \ = \ 16 $
				\item $\left(\dfrac{1}{2^2}\right)^3 \ = \ \dfrac{1^3}{(2^2)^3}$
				\item $\left(\dfrac{4}{3^2}\right)^{-2} \ = \ \left(\dfrac{3^2}{4}\right)^{2} \ = \  \dfrac{(3^2)^2}{4^2} \ = \ \dfrac{3^4}{4^2}$\\[2mm]
				$\ = \ \dfrac{81}{16}$
			\end{enumerate}

	\end{exo}
		\begin{exo}\quad\hfill\textbf{}\\
			Exprimer sous la forme d'une seule puissance: \\
	
				\begin{enumerate}
					
				\item $5^{6}\times3^{5}\times5^{4}\times3^{5} \ = \ 5^{6}\times5^{4}\times3^{5}\times3^{5} $ \\
				$\ = \ 5^{6+4}\times3^{5+5} \ = \ 5^{10}\times3^{10} $ $\ = \ (5\times3)^{10} $\\[5mm]
				
					
				\item $(12^{2})^{5}\times(12^3)^{-3} \ = \ 12^{2\times5}\times12^{3\times-3} $\\$\ = \ 12^{10}\times12^{-9}  \ = \ 12^{10-9} $
				\\ $ \ = \ 12^1 \ = \ 12$
				\end{enumerate} 

			
			
		\end{exo}
	\end{minipage}
\end{center}
\newpage
\quad\\
\section*{Racine carrée d'un nombre réel positif:}
\quad\\
\begin{center}
	\begin{minipage}[c]{0.4\linewidth}
		\raggedright
		\begin{exo}\quad\\
			Simplifier les racines carrées 
				\begin{enumerate}
					\item $\sqrt{1}\times\sqrt{0} \ = \ 1 \times0  \ = \ 0$
					\item $\sqrt{25} \times \sqrt{16} \ = \ 5\times4 \ = \ 20 $
					
					\item $\sqrt{121} \ = \ \sqrt{11\times11} \ = \ \sqrt{11^2} \ = \ 11$
					\item $\sqrt{144} \ = \ \sqrt{12\times12} \ = \ 12$
					
				\end{enumerate}

		\end{exo}
		\begin{exo}\quad\hfill\textbf{}\\
			Écrire plus simplement les nombres suivants:
	
				\begin{enumerate}
					\item $\sqrt{1.2^{2}} \ = \ 1.2 $
					\item $\sqrt{\pi^{2}} \ = \ \pi$
					
					\item $\sqrt{\left(-2.666\right)^{2}} \ = \ |-2.666~| \ = \ 2.666$
					\item $\left. \sqrt{7.89}\right.^2 \ = \ 7.89$
				\end{enumerate}

		\end{exo}
		\begin{exo}\quad\hfill\textbf{}\\
			Écrire plus simplement les nombres suivants:
			
				\begin{enumerate}
					\item $\sqrt{32}\times \sqrt{2} \ = \ \sqrt{2\times16}\times\sqrt{2}$ \\ $\ = \ \sqrt{2} \times \sqrt{16}\times\sqrt{2} \ = \ 4\times 2 \ = \ 8  $
					%\item $\sqrt{3} \times \sqrt{27}$
					\item $\sqrt{3}\times \sqrt{36}\times \sqrt{3} \ = \ 3\times6 \ = \ 18$
					
					\item $\dfrac{\sqrt{98}}{\sqrt{2}} \ = \ \sqrt{\dfrac{98}{2}}  \ = \ \sqrt{49} \ = \ 7$
					%\item $\frac{\sqrt{50}}{\sqrt{72}}$
					\item $(4\sqrt{5})^2 \ = \ 16\times5 \ = \ 80$
					 
				\end{enumerate}  
		
		\end{exo}
	\end{minipage}
	\hfill\vrule\hfill
	\begin{minipage}[c]{0.4\linewidth}
		\raggedright
		\begin{exo}\quad\hfill\textbf{}\\
			Écrire plus simplement les nombres suivants:
		
				\begin{enumerate}
					\item $\sqrt{\dfrac{32}{2}} \ = \ \sqrt{16} \ = \ 4$
					\item $\sqrt{3}^{~-4} \ = \ \dfrac{1}{\left(\sqrt{3}\right)^4} \ = \ \dfrac{1}{\left(\sqrt{3}^2\right)^2} \ = \ \dfrac{1}{9}$
				
					
					\item $\left(\dfrac{5}{\sqrt{2}}\right)^2 \ = \ \dfrac{5^2}{\left(\sqrt{2}\right)^2} \ = \ \dfrac{25}{2} $
					
					\item $\dfrac{1}{\sqrt{7}^{-2}} \ = \ \dfrac{1^{-2}}{\sqrt{7}^{-2}}  \ = \ \left(\dfrac{1}{\sqrt{7}}\right)^{-2}$\\ $ \ = \ \left(\sqrt{7}\right)^2 \ = \ 7$
					
				\end{enumerate}  

		\end{exo}
		\begin{exo}\quad\hfill\textbf{}\\
				Écrire sous la forme $a\sqrt b$, avec $a$ et $b$ entiers et $b$ étant le
				plus petit possible.
			
				\begin{enumerate}
					\item $\sqrt{8} \ = \ \sqrt{4\times2} \ =  \  2\sqrt{2}$  
					%\item $\sqrt{12}$ 
					\item $\sqrt{18} \ = \ \sqrt{9\times2} \ = \ 3\sqrt{2}$
					
					\item $\sqrt{45} \ = \ \sqrt{9\times5} \ =  \ 3\sqrt{5}$
					%\item $\sqrt{72}$
					%\item $\sqrt{125}$
					\item $\sqrt{150} \ = \  \sqrt{15\times  10}  \   =  \ \sqrt{ 3 \times 5 \times 2 \times 5}$\\
					$ \ = \  \sqrt{5\times5} \times\sqrt{2\times3}\ = \ 5\sqrt{6} $
					%\item $\sqrt{147}$ 
					
				\end{enumerate}  
	
		\end{exo}
		\begin{exo}\quad\hfill\textbf{}\\
			Écrire plus simplement les nombres suivants: \\
			
				\begin{enumerate}
				\item $4\sqrt{3}-2\sqrt{3}+6\sqrt{3} \ = \ \sqrt{3}\left(4-2+6\right)$\\$ \ = \ \sqrt{3}\times8 $
				%\item $7\sqrt{2}-3\sqrt{5}+8\sqrt{2}-\sqrt{5}$
				 
				\item $(3-2\sqrt{3})-(4-6\sqrt{3})$\\$ \ = \ 3-2\sqrt{3} - 4+6\sqrt{3}$
				\\$ \ = \ -1+4\sqrt{3}$
				\end{enumerate}			
		\end{exo}
	\end{minipage}
\end{center}

\newpage

\section*{Transformations d'expressions algébriques:}
\quad \\

\begin{center}
	\begin{minipage}[c]{0.4\linewidth}
		\raggedright
		\begin{exo}\quad\hfill\textbf{}\\
		Réduire et ordonner les expressions ci-dessous:
			\begin{enumerate}
				%\item $x - 3 + (5x+2)$
				\item $(2x + 4) + 3x \ = \ 5x + 4$
				%\item $(5 - x) - (9 + 9x)$
				\item $3 + (2 + 3x)-(x - 2)\ = \ 3 + 2 + 3x -x + 2 $ \\ $ \ = \ 2x + 7$
				
				%\item $(5x-2) -(4x+2)$
				\item $(2x-3)-(2x+3) \ = \ 2x-3-2x-3 \ = \ -6$ 
				%\item $-(3x-2)-(5-2)$
				%\item $2x+4-(-2x-3)$
			\end{enumerate}

		\end{exo}
		\begin{exo}\quad\hfill\textbf{}\\
		Développer, réduire et ordonner les expressions ci-dessous:

			\begin{enumerate}
				%\item $(6x + 1)( x - 1)$
				\item $2(1 + 6x) \ = \ 12x +2$
				\item $ (8 - x)(2 + x) \ = \ -x^2 + 6x + 16$
				%\item $(3 + 8x)(x - 8)$
				
				
				\item $(x-1)^2 \ = \ (x-1)\times(x-1) \ = \  x^2 -2x +1$
				%\item $(2x-1)(x+3)^2$
				%\item $(x-2)(x+3)$
				%\item $(2x-3)(3-x)$
				%\item $(3x+2)(5x-4)$
				%\item $(2x-1)(x-3)$
			\end{enumerate}

		\end{exo}
		\begin{exo}\quad\hfill\textbf{}\\
				Développer, réduire et ordonner les expressions ci-dessous:

				\begin{enumerate}
					\item $(x+2)(4x-3)-x(7-x)$ \\ $\ = \ 4x^2+5x-6-7x+x^2$\\
					$ \ = \ 5x^2-2x-6$  
					\item $(3x-1)(3x+1) \ = \ 9x^2-1$
					%\item $(x+3)-(9x-1)(1-7x)$
					
					%\item $(8x+2)(4x-1)-(2x+3)x$
					%\item $(1-2x)(3x-4)+(7x+1)(7x-1)$
					\item $(8x-1)(2x+1)+(16x-2)(3-x)$\\
					$ \ = \ (8x-1)(2x+1)+2(8x-1)(3-x)$\\ 
					$ \ = \ (8x-1)\left[2x+1+2(3-x)\right] $\\
					$ \ = \ (8x-1)\left[2x+1+6-2x\right]$\\
					$ \ = \ (8x-1)\times 7 \ = \ 56x - 7$ 
					
				\end{enumerate}

		\end{exo}
	\begin{exo}\quad\hfill\textbf{}\\
		Les trois identités remarquables que l'on vous demande de connaître par cœur sont:\\~\\
		Pour tout:\quad $\alpha, \beta\in\mathbb{R}$
		\begin{enumerate}
			\item $(\alpha + \beta)^2 \ = \  \alpha^2 + 2\alpha\beta + \beta^2$
			\item $(\alpha+\beta)(\alpha-\beta) \ = \ \alpha^2 - \beta^2 $
			\item $(\alpha - \beta)^2  \ = \ \alpha^2 - 2\alpha\beta + \beta^2$
		\end{enumerate}
	\end{exo}
	\end{minipage}
	\hfill\vrule\hfill
	\begin{minipage}[c]{0.4\linewidth}
		\raggedright
		Vous devez aussi les connaître dans l'autre sens de l'égalité avec d'autres lettres comme ci-dessous:\\~\\
		Pour tout:\quad $x, y\in\mathbb{R}$
		\begin{enumerate}
			\item $ x^2 + 2xy + y^2 \ = \ (x + y)^2  $
			\item $x^2 - y^2 \ = \ (x+y)(x-y)  $
			\item $x^2 - 2xy + y^2  \ = \ (x - y)^2$
		\end{enumerate}  
		\begin{exo}\quad\hfill\textbf{}\\
			 Factoriser les expressions suivantes.
	
				\begin{enumerate}
					\item $3(2 + 3x) - (5 + 2x)(2 + 3x)$\\
					$\ = \ (2 + 3x)\left[3 - (5 + 2x)\right]$\\
					$\ = \ (2 + 3x)(3 - 5 - 2x) $\\
					$  \  =  \  (2 + 3x)(-2x-2)  \  =  \ - (3x+2)(2x+2) $
					\item $(2 - 5x)2 - (2 - 5x)(1 + x)$\\
					$ \ = \ (2 - 5x)\left[(2 - 5x) - (1 + x)\right] $\\
					$ \ = \ (2 - 5x)(2 - 5x - 1 - x) $\\
					$ \ = \ (2 - 5x)(- 6x + 1) $
					\item $5(1 - 2x) - (4 + 3x)(1-2x)$\\
					$ \ = \ (1 - 2x)\left[ 5 - (4 + 3x)\right]$\\
					$ \ = \ (1 - 2x)(5 - 4 - 3x)$\\
					$ \ = \ (1 - 2x)( - 3x + 1)$
					%\item $3x^2 - x$
				\end{enumerate}
			
		\end{exo}
		\begin{exo}\quad\hfill\textbf{}\\
			Nous factorisons les expressions suivantes grâce aux identités remarquables.
		
				\begin{enumerate}
					\item $x^2-2x+1  \ = \ (x-1)^2$
					\item $16x^2+16x+4 \ = \ (4x+2)^2 $
					%\item $9x^2+6x+1 $
					
					%\item $49-42x +9x^2$
					\item $1-x^2  \ =  \ (1+x)(1-x)$
					%\item $9x^2-49$
				\end{enumerate}

		\end{exo}
		\begin{exo}\quad\hfill\textbf{}\\
			Utiliser les identités remarquables pour écrire les expressions suivantes
			sous la forme:\\
			 $a+b\sqrt c$,\quad où \quad $a, b, c\in \mathbb{Z}$ \hspace{3cm}

				\begin{enumerate}
					\item $(\sqrt{3}-4)^2 \ = \ 3 -8\sqrt{3} +16 \ = \ -8\sqrt{3}+19$
					\item $(3+\sqrt{5})^2 \ = \  9 + 6\sqrt{5} +5 \ = \ 6\sqrt{5} + 14$
					\item $(\sqrt{2}-\sqrt{5})(\sqrt{2}+\sqrt{5}) \ = \ 2 - 5 \ = \ -3$
					%\item $(3+\sqrt{3})(4-2\sqrt{3})$ 
				\end{enumerate}
	
		\end{exo}

		
	\end{minipage}
\end{center}

\quad\\
\newpage
\section*{Transformations d'expressions fractionnaires:}	
\quad\\	

\begin{center}
	\begin{minipage}[c]{0.4\linewidth}
		\raggedright
		
			\begin{exo}\hfill\textbf{}\\
			Pour chaque expression, donner pour quelle(s)  valeur(s) de $x$ les nombres suivants ne sont pas définis.  
			
			\begin{enumerate}
				\item $\dfrac{7x}{x-2}-\dfrac{5}{3-x }$ \\[2mm]
				n'est pas défini si: $x\in\{2;~3\}$
				\item $ 3+\dfrac{5x}{2x+1}$\\[2mm]
				n'est pas défini si: $x\in\{-\dfrac{1}{2}\}$
				%\item $\dfrac{3}{x+1} - \dfrac{2}{x-1}$
				%\item $\dfrac{7x+2}{2x-1}- \dfrac{8}{3x-6}$
				%\item $\dfrac{9x+2}{x} - \dfrac{3x+1}{2x+3}$
			\end{enumerate}
			
		\end{exo}
		\begin{exo}\hfill\textbf{}\\
		Réduire les expressions suivantes au même dénominateur.
		
			\begin{enumerate}
				\item $\dfrac{7x}{x-2}-\dfrac{5}{3-x } \ = \    \dfrac{7x(3-x) - 5(x-2)}{(x-2)(3-x)}$\\[2mm]
				$\ = \    \dfrac{21x -7x^2-5x+10}{(x-2)(3-x)} $\\[2mm]$\ = \ \dfrac{ -7x^2+16x+10}{(x-2)(3-x)}$\\[5mm]
				%\item $ 3+\dfrac{5x}{2x+1}$
				%\item $\dfrac{3}{x+1} - \dfrac{2}{x-1}$
				\item $\dfrac{7x+2}{2x-1}- \dfrac{8}{3x-6}$\\[2mm]$ \ = \ \dfrac{(7x+2)(3x-6)-8(2x-1)}{(2x-1)(3x-6)}$\\[2mm]
				$ \ = \ \dfrac{21x^2-42x+6x-12-16x+8}{(2x-1)(3x-6)}$
				\\[2mm]
				$ \ = \ \dfrac{21x^2-52x-4}{(2x-1)(3x-6)}$
				%\item $\dfrac{9x+2}{x} - \dfrac{3x+1}{2x+3}$
			\end{enumerate}
	\end{exo}

	\end{minipage}
	\hfill\vrule\hfill
	\begin{minipage}[c]{0.4\linewidth}
		\raggedright
		\begin{exo}\hfill\textbf{}\\
		Résoudre les équations suivantes:\\[3mm]
		\begin{enumerate}
			
			\item $ \quad\quad~3 \ = \ \dfrac{5x}{2x+1}$\\[3mm]
			$\Leftrightarrow\quad 3 - \dfrac{5x}{2x+1} \ = \ 0$\\[3mm]
			$\Leftrightarrow\quad \dfrac{3(2x+1)}{(2x+1)} - \dfrac{5x}{2x+1} \ = \ 0$\\[3mm]
			$\Leftrightarrow\quad \dfrac{6x+3-5x}{(2x+1)} \ = \ 0$\\[3mm]
			$\Leftrightarrow\quad \dfrac{x+3}{(2x+1)} \ = \ 0$\\[3mm]
			$ \Leftrightarrow\quad x+3\ = \ 0$\\[3mm]
			$ \Leftrightarrow\quad x \ = \ -3$\\[5mm]
			
			
			
			\item $\quad\quad~\dfrac{3}{x+1} \ = \  \dfrac{2}{x-1}$\\[3mm]
			$\Leftrightarrow\quad \dfrac{3}{x+1} -  \dfrac{2}{x-1} \ = \ 0$\\[3mm]
			$\Leftrightarrow\quad \dfrac{3(x-1) - 2(x+1)}{(x+1)(x-1)} \ = \ 0$\\[3mm]
			$\Leftrightarrow\quad \dfrac{3x-3 - 2x-2}{(x+1)(x-1)} \ = \ 0$\\[3mm]
			$\Leftrightarrow\quad \dfrac{x-5}{(x+1)(x-1)} \ = \ 0$\\[3mm]
			$ \Leftrightarrow\quad x-5\ = \ 0$\\[3mm]
			$ \Leftrightarrow\quad x \ = \ 5$\\[5mm]
			
			
			\item $\quad\quad~\dfrac{x+2}{x} \ = \  - \dfrac{1}{x^2}$\\[3mm]
			$\Leftrightarrow\quad x\times\dfrac{x+2}{x} \ = \  - \dfrac{1}{x^2}\times x$\\[3mm]
			$\Leftrightarrow\quad (x+2) \ = \  - \dfrac{1}{x}$\\[3mm]
			$\Leftrightarrow\quad (x+2)  + \dfrac{1}{x} \ = \ 0$\\[3mm]
			$\Leftrightarrow\quad \dfrac{(x+2)x}{x}  + \dfrac{1}{x} \ = \ 0$\\[3mm]
			$\Leftrightarrow\quad \dfrac{x^2+2x+1}{x}  \ = \ 0$\\[3mm]
			$\Leftrightarrow\quad x^2+2x+1  \ = \ 0$\\[3mm]
			$\Leftrightarrow\quad (x+1)^2 \ = \ 0$\\[3mm]
			$\Leftrightarrow\quad (x+1) \ = \ 0$\\[3mm]
			$\Leftrightarrow\quad x \ = \ -1$\\[3mm]
			
			
			
			
		\end{enumerate}
		\end{exo}
	\end{minipage}
\end{center}
	
\end{document}