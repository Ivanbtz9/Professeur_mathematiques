
%%%%%%%%%%%%%%%%%% PREAMBULE %%%%%%%%%%%%%%%%%%

\documentclass[11pt,a4paper]{article}

\usepackage{amsfonts,amsmath,amssymb,amsthm}
\usepackage[utf8]{inputenc}
\usepackage[T1]{fontenc}
\usepackage[francais]{babel}
\usepackage{mathptmx}
\usepackage{fancybox}
\usepackage{graphicx}
\usepackage{ifthen}

\usepackage{tikz}   

\usepackage{hyperref}
\hypersetup{colorlinks=true, linkcolor=blue, urlcolor=blue,
pdftitle={Exo7 - Exercices de mathématiques}, pdfauthor={Exo7}}

\usepackage{geometry}
\geometry{top=2cm, bottom=2cm, left=2cm, right=2cm}

%----- Ensembles : entiers, reels, complexes -----
\newcommand{\Nn}{\mathbb{N}} \newcommand{\N}{\mathbb{N}}
\newcommand{\Zz}{\mathbb{Z}} \newcommand{\Z}{\mathbb{Z}}
\newcommand{\Qq}{\mathbb{Q}} \newcommand{\Q}{\mathbb{Q}}
\newcommand{\Rr}{\mathbb{R}} \newcommand{\R}{\mathbb{R}}
\newcommand{\Cc}{\mathbb{C}} \newcommand{\C}{\mathbb{C}}
\newcommand{\Kk}{\mathbb{K}} \newcommand{\K}{\mathbb{K}}

%----- Modifications de symboles -----
\renewcommand{\epsilon}{\varepsilon}
\renewcommand{\Re}{\mathop{\mathrm{Re}}\nolimits}
\renewcommand{\Im}{\mathop{\mathrm{Im}}\nolimits}
\newcommand{\llbracket}{\left[\kern-0.15em\left[}
\newcommand{\rrbracket}{\right]\kern-0.15em\right]}
\renewcommand{\ge}{\geqslant} \renewcommand{\geq}{\geqslant}
\renewcommand{\le}{\leqslant} \renewcommand{\leq}{\leqslant}

%----- Fonctions usuelles -----
\newcommand{\ch}{\mathop{\mathrm{ch}}\nolimits}
\newcommand{\sh}{\mathop{\mathrm{sh}}\nolimits}
\renewcommand{\tanh}{\mathop{\mathrm{th}}\nolimits}
\newcommand{\cotan}{\mathop{\mathrm{cotan}}\nolimits}
\newcommand{\Arcsin}{\mathop{\mathrm{arcsin}}\nolimits}
\newcommand{\Arccos}{\mathop{\mathrm{arccos}}\nolimits}
\newcommand{\Arctan}{\mathop{\mathrm{arctan}}\nolimits}
\newcommand{\Argsh}{\mathop{\mathrm{argsh}}\nolimits}
\newcommand{\Argch}{\mathop{\mathrm{argch}}\nolimits}
\newcommand{\Argth}{\mathop{\mathrm{argth}}\nolimits}
\newcommand{\pgcd}{\mathop{\mathrm{pgcd}}\nolimits} 

%----- Structure des exercices ------

\newcommand{\exercice}[1]{\video{0}}
\newcommand{\finexercice}{}
\newcommand{\noindication}{}
\newcommand{\nocorrection}{}

\newcounter{exo}
\newcommand{\enonce}[2]{\refstepcounter{exo}\hypertarget{exo7:#1}{}\label{exo7:#1}{\bf Exercice \arabic{exo}}\ \  #2\vspace{1mm}\hrule\vspace{1mm}}

\newcommand{\finenonce}[1]{
\ifthenelse{\equal{\ref{ind7:#1}}{\ref{bidon}}\and\equal{\ref{cor7:#1}}{\ref{bidon}}}{}{\par{\footnotesize
\ifthenelse{\equal{\ref{ind7:#1}}{\ref{bidon}}}{}{\hyperlink{ind7:#1}{\texttt{Indication} $\blacktriangledown$}\qquad}
\ifthenelse{\equal{\ref{cor7:#1}}{\ref{bidon}}}{}{\hyperlink{cor7:#1}{\texttt{Correction} $\blacktriangledown$}}}}
\ifthenelse{\equal{\myvideo}{0}}{}{{\footnotesize\qquad\texttt{\href{http://www.youtube.com/watch?v=\myvideo}{Vidéo $\blacksquare$}}}}
\hfill{\scriptsize\texttt{[#1]}}\vspace{1mm}\hrule\vspace*{7mm}}

\newcommand{\indication}[1]{\hypertarget{ind7:#1}{}\label{ind7:#1}{\bf Indication pour \hyperlink{exo7:#1}{l'exercice \ref{exo7:#1} $\blacktriangle$}}\vspace{1mm}\hrule\vspace{1mm}}
\newcommand{\finindication}{\vspace{1mm}\hrule\vspace*{7mm}}
\newcommand{\correction}[1]{\hypertarget{cor7:#1}{}\label{cor7:#1}{\bf Correction de \hyperlink{exo7:#1}{l'exercice \ref{exo7:#1} $\blacktriangle$}}\vspace{1mm}\hrule\vspace{1mm}}
\newcommand{\fincorrection}{\vspace{1mm}\hrule\vspace*{7mm}}

\newcommand{\finenonces}{\newpage}
\newcommand{\finindications}{\newpage}


\newcommand{\fiche}[1]{} \newcommand{\finfiche}{}
%\newcommand{\titre}[1]{\centerline{\large \bf #1}}
\newcommand{\addcommand}[1]{}

% variable myvideo : 0 no video, otherwise youtube reference
\newcommand{\video}[1]{\def\myvideo{#1}}

%----- Presentation ------

\setlength{\parindent}{0cm}

\definecolor{myred}{rgb}{0.93,0.26,0}
\definecolor{myorange}{rgb}{0.97,0.58,0}
\definecolor{myyellow}{rgb}{1,0.86,0}

\newcommand{\LogoExoSept}[1]{  % input : echelle       %% NEW
{\usefont{U}{cmss}{bx}{n}
\begin{tikzpicture}[scale=0.1*#1,transform shape]
  \fill[color=myorange] (0,0)--(4,0)--(4,-4)--(0,-4)--cycle;
  \fill[color=myred] (0,0)--(0,3)--(-3,3)--(-3,0)--cycle;
  \fill[color=myyellow] (4,0)--(7,4)--(3,7)--(0,3)--cycle;
  \node[scale=5] at (3.5,3.5) {Exo7};
\end{tikzpicture}}
}


% titre
\newcommand{\titre}[1]{%
\vspace*{-4ex} \hfill \hspace*{1.5cm} \hypersetup{linkcolor=black, urlcolor=black} 
\href{http://exo7.emath.fr}{\LogoExoSept{3}} 
 \vspace*{-5.7ex}\newline 
\hypersetup{linkcolor=blue, urlcolor=blue}  {\Large \bf #1} \newline 
 \rule{12cm}{1mm} \vspace*{3ex}}

%----- Commandes supplementaires ------



\begin{document}

%%%%%%%%%%%%%%%%%% EXERCICES %%%%%%%%%%%%%%%%%%
\fiche{f00151, quinio, 2011/05/20}

Exercices : Martine Quinio

\titre{Probabilité conditionnelle}

\exercice{5992, quinio, 2011/05/11}

\enonce{005992}{}
Dans la salle des profs $60$\% sont des femmes ; une femme sur trois
porte des lunettes et un homme sur deux porte des lunettes : quelle est la
probabilité pour qu'un porteur de lunettes pris au hasard soit une femme?
\finenonce{005992}


\finexercice\exercice{5993, quinio, 2011/05/20}

\enonce{005993}{}
Une fête réunit $35$ hommes, $40$ femmes, $25$ enfants ; sur
une table, il y a $3$ urnes $H$, $F$, $E$ contenant des boules de couleurs dont
respectivement $10$\%, $40$\%, $80$\% de boules noires. Un présentateur
aux yeux bandés désigne une personne au hasard et lui demande de
tirer une boule dans l'urne $H$ si cette personne est un homme, dans l'urne $F$
si cette personne est une femme, dans l'urne $E$ si cette personne est un
enfant. La boule tirée est noire : quelle est la probabilité pour que
la boule ait été tirée par un homme? une femme? un enfant? Le présentateur n'est pas plus magicien que vous et moi et pronostique le
genre de la personne au hasard : que doit-il dire pour avoir le moins de
risque d'erreur?
\finenonce{005993}


\finexercice
\exercice{5994, quinio, 2011/05/20}

\enonce{005994}{}
Un fumeur, après avoir lu une série de statistiques
effrayantes sur les risques de cancer, problèmes cardio-vasculaires 
liés au tabac, décide d'arrêter de fumer; toujours d'après des
statistiques, on estime les probabilités suivantes : si cette personne
n'a pas fumé un jour $J_{n}$, alors la probabilité
pour qu'elle ne fume pas le jour suivant $J_{n+1}$ est $0.3$; 
mais si elle a fumé un jour $J_{n}$, alors la probabilité 
pour qu'elle ne fume pas le jour suivant $J_{n+1}$ est $0.9$; 
quelle est la probabilité $P_{n+1}$ pour qu'elle
fume le jour $J_{n+1}$ en fonction de la probabilité 
$P_{n}$ pour qu'elle fume le jour $J_{n}$ ? Quelle est la
limite de $P_{n}$ ? Va-t-il finir par s'arrêter?
\finenonce{005994}


\finexercice
\exercice{5995, quinio, 2011/05/20}

\enonce{005995}{}
Un professeur oublie fréquemment ses clés. Pour tout $n$, on note :
$E_n$ l'événement <<le jour $n$, le professeur oublie ses clés>>, 
$P_{n}=P(E_n)$, $Q_n=P(\overline{E_n})$.

On suppose que : $P_{1}=a$ est donné et que si le jour $n$ il oublie ses clés, 
le jour suivant il les oublie avec la probabilité $\frac{1}{10}$ ; 
si le jour $n$ il n'oublie pas ses clés, le jour suivant il les oublie
avec la probabilité $\frac{4}{10}$.

Montrer que $P_{n+1}=\frac{1}{10}P_{n}+\frac{4}{10}Q_{n}$.
En déduire une relation entre $P_{n+1}$ et $P_{n}$

Quelle est la probabilité de l'événement <<le jour $n$, le professeur oublie ses clés>> ?
\finenonce{005995}


\finexercice
\exercice{5996, quinio, 2011/05/20}

\enonce{005996}{}
Dans les barres de chocolat N., on trouve des images équitablement
réparties des cinq personnages du dernier Walt Disney, une image par
tablette. Ma fille veut avoir le héros Princecharmant : combien dois-je
acheter de barres pour que la probabilité d'avoir la figurine attendue dépasse $80$\%? 
Même question pour être sûr à $90$\%.
\finenonce{005996}


\finexercice
\exercice{5997, quinio, 2011/05/20}

\enonce{005997}{}
En cas de migraine trois patients sur cinq prennent de l'aspirine
(ou équivalent), deux sur cinq prennent un médicament M présentant des effets secondaires :

Avec l'aspirine, 75\% des patients sont soulagés.

Avec le médicament M, 90\% des patients sont soulagés.
\begin{enumerate}
\item Quel est le taux global de personnes soulagées?

\item Quel est la probabilité pour un patient d'avoir pris de l'aspirine
sachant qu'il est soulagé?
\end{enumerate}
\finenonce{005997}


\finexercice
\exercice{5998, quinio, 2011/05/20}

\enonce{005998}{}
Dans une population 40\% des individus ont les yeux bruns, 25\% des
individus ont les cheveux blonds, 15\% des individus ont les yeux bruns
et les cheveux blonds.

On choisit un individu au hasard. Calculez :
\begin{enumerate}
\item La probabilité de l'événement : si un individu a les yeux
bruns d'avoir les cheveux blonds.

\item La probabilité de l'événement : si un individu a les cheveux
blonds d'avoir les yeux bruns.

\item La probabilité de l'événement : si un individu a les cheveux
blonds, de ne pas avoir les yeux bruns.
\end{enumerate}
\finenonce{005998}


\finexercice
\exercice{5999, quinio, 2011/05/20}

\enonce{005999}{}
Un constructeur aéronautique équipe ses avions trimoteurs d'un
moteur central de type A et de deux moteurs, un par aile, de type B; chaque
moteur tombe en panne indépendamment d'un autre, et on estime à $p$ la
probabilité pour un moteur de type A de tomber en panne et à $q$ la
probabilité pour un moteur de type B de tomber en panne.

Le trimoteur peut voler si le moteur central \emph{ou} les deux moteurs d'ailes
fonctionnent : quelle est la probabilité pour l'avion de voler?
Application numérique : $p = 0.001\%$, $q = 0.02\%$.
\finenonce{005999}


\finexercice
\exercice{6000, quinio, 2011/05/20}

\enonce{006000}{}
On sait qu'à une date donnée, 3\% d'une population
est atteinte d'hépatite
On dispose de tests de dépistage de la maladie :
\begin{itemize}
\item Si la personne est malade, alors le test est positif avec une probabilité de 95\%.
\item Si la personne est saine, alors le test est positif avec une probabilité de 10\%.
\end{itemize}

\begin{enumerate}
\item Quelle est la probabilité pour une personne d'être malade
si son test est positif ?
\item Quelle est la probabilité pour une personne d'être saine si
son test est positif ?
\item Quelle est la probabilité pour une personne d'être malade
si son test est négatif ?
\item Quelle est la probabilité pour une personne d'être saine si
son test est négatif ?
\end{enumerate}
\finenonce{006000}


\finexercice
\exercice{6001, quinio, 2011/05/20}

\enonce{006001}{}
Dans mon trousseau de clés il y a $8$ clés; elles sont toutes
semblables. Pour rentrer chez moi je mets une clé au hasard; je
fais ainsi des essais jusqu'à ce que je trouve la bonne; j'écarte au
fur et à mesure les mauvaises clés. Quelle est la probabilité
pour que j'ouvre la porte : 
\begin{enumerate}
\item du premier coup ? 
\item au troisième essai ? 
\item au cinquième essai ? 
\item au huitième essai?
\end{enumerate}
\finenonce{006001}


\finexercice
\exercice{6002, quinio, 2011/05/20}

\enonce{006002}{}
Six couples sont réunis dans une soirée de réveillon. Une
fois les bises de bonne année échangées, on danse, de façon
conventionnelle: un homme avec une femme, mais pas forcément la sienne.

\begin{enumerate}
\item Quelle est la probabilité $P(A)$ pour que chacun des 6 hommes danse
avec son épouse légitime ?
\item Quelle est la probabilité $P(B)$ pour que André danse avec son 
épouse ?
\item Quelle est la probabilité $P(C)$ pour que André et René
dansent avec leur épouse ?
\item Quelle est la probabilité $P(D)$ pour que André ou René
danse(nt) avec leur épouse ?
\end{enumerate}
\finenonce{006002}


\finexercice
\exercice{6003, quinio, 2011/05/20}

\enonce{006003}{}
Dans l'ancienne formule du Loto il fallait choisir 6 numéros parmi 49.
\begin{enumerate}
  \item Combien y-a-t-il de grilles possibles ? En déduire la probabilité de
gagner en jouant une grille.

  \item Quelle est la probabilité que la grille gagnante comporte 2
nombres consécutifs?
\end{enumerate}
\finenonce{006003}


\finexercice
\exercice{6004, quinio, 2011/05/20}

\enonce{006004}{}
Un débutant à un jeu effectue plusieurs parties successives. Pour la
première partie, les probabilités de gagner ou perdre sont les mêmes; puis, on suppose que:
\begin{itemize}
  \item Si une partie est gagnée, la probabilité de gagner la suivante est $0.6$.

  \item Si une partie est perdue, la probabilité de perdre la suivante est $0.7$.
\end{itemize}
Soit $G_n$ l'événement <<Gagner la partie $n$>>, et $u_n=P(G_n)$.
On note $v_n = P(\overline{G_n})$.
\begin{enumerate}
  \item Ecrire 2 relations entre $u_n$, $u_{n+1}$, $v_n$, $v_{n+1}$.

  \item A l'aide de la matrice mise en évidence en déduire $u_n$ et $v_n$.
Faire un calcul direct à l'aide de $u_n+v_n$.
\end{enumerate}
\finenonce{006004}


\finexercice

\finfiche

 \finenonces 



 \finindications 

\noindication
\noindication
\noindication
\noindication
\noindication
\noindication
\noindication
\noindication
\noindication
\noindication
\noindication
\noindication
\noindication


\newpage

\correction{005992}
Notons les différents événements :
$Fe$ : <<être femme>>, $Lu$ : <<porter des lunettes>>, $H$ : <<être homme>>

Alors on a $P(Fe)=0.6,$ $P(Lu/Fe)=\frac{1}{3};$ il s'agit de la 
probabilité conditionnelle probabilité de 
<<porter des lunettes>> sachant que la personne est une femme.
De même, on a $P(Lu/H)=0.5$. On cherche la probabilité
conditionnelle $P(Fe/Lu)$.
D'après la formule des probabilités totales on a :
$P(Fe/Lu)P(Lu)=P(Lu/Fe)P(Fe)$ avec $P(Lu)=P(Lu/Fe)P(Fe)+P(Lu/H)P(H)$.

Application numérique : $P(Lu)=0.4$, donc  $P(Fe/Lu)=\frac{P(Lu/Fe)P(Fe)}{P(Lu)}=0.5$.
Remarque : on peut trouver les mêmes réponses par des raisonnements 
élémentaires.
\fincorrection
\correction{005993}
C'est évidemment le même que le précédent (exercice \ref{exo:quinio11}), seul le contexte
est différent : il suffit d'adapter les calculs faits.
En pronostiquant un enfant, le présentateur a une chance sur deux
environ de ne pas se tromper.
\fincorrection
\correction{005994}
Fumeurs

Définissons les événements: $F_{n}$ <<Fumer le $n$\up{ème} jour>>, et $\overline{F_{n}}$ 
l'événement complémentaire.
Alors $\{\overline{F_{n}},F_{n}\}$ constitue un système complet d'événements, 
$P_{n}=$ $P(F_{n})$; on peut donc écrire :
$P(\overline{F_{n+1}})=P(\overline{F_{n+1}}/F_{n})P(F_{n})
+P(\overline{F_{n+1}}/\overline{F_{n}})P(\overline{F_{n}})$.

Comme $P(\overline{F_{n+1}}/F_{n})=0.9$ et $P(\overline{F_{n+1}}/\overline{F_{n}})=0.3$
$1-P_{n+1}=0.9P_{n}+0.3(1-P_{n})$, soit $P_{n+1}=-0.6P_{n}+0.7$. Notons (R)
cette relation.

Pour connaître le comportement à long terme, il faut étudier cette
suite récurrente; il y a des techniques mathématiques pour ça,
c'est le moment de s'en servir.

Cherchons la solution de l'équation <<$\ell=-0.6\ell+0.7$>>, 
la limite éventuelle satisfait nécessairement cette équation : faire un passage à la limite dans la
relation (R), ou utiliser le théorème du point fixe.

On trouve $\ell=\frac{7}{16};$ alors, la suite $Q_{n}=(P_{n}- \ell)$ vérifie : 
$Q_{n+1}= - 0.6Q_{n}$, ce qui permet de 
conclure : $Q_{n+1}=(-0.6)^{n}Q_{1}$
et comme $((-0.6)^{n})$ est une suite qui tend vers $0$, on peut dire que la
suite $(Q_{n})$ tend vers $0$ et donc que la suite $(P_n)$ tend vers $\ell=\frac{7}{16}.$

Conclusion : la probabilité $P_{n}$ pour qu'elle fume le jour $J_{n}$
tend vers $\frac{7}{16} \simeq 0.4375$.
\fincorrection
\correction{005995}
 $P_{n+1}=P(E_{n+1})=
P(E_{n+1}/E_n)P(E_n)+P(E_{n+1}/\overline{E_n})P(\overline{E_n})=\frac{1}{10}P_{n}+\frac{4}{10}Q_{n}$.
Donc
$P_{n+1}=\frac{1}{10}P_{n}+\frac{4}{10}(1-P_{n})=\frac{4}{10}-\frac{3}{10}P_{n}$.

La suite ($P_{n}-\ell)$ est géométrique, où $\ell$ est solution
de $\frac{4}{10}-\frac{3}{10}\ell=\ell$ soit $\ell=\frac{4}{13}$.
Donc $P_{n}=\frac{4}{13}+a(-\frac{3}{10})^{n-1}$.
\fincorrection
\correction{005996}
La probabilité d'avoir Princecharmant dans la barre B est $\frac{1}{5}; $ si j'achète $n$ barres, la probabilité de n'avoir la
figurine dans aucune des $n$ barres est $(\frac{4}{5})^{n}$, puisqu'il
s'agit de $n$ événements indépendants de probabilité $\frac{4}{5}$.
Je cherche donc $n$ tel que : $1-(\frac{4}{5})^{n}\geq 0.8$. On a facilement : $n\geq 8$.

Puis, je cherche $m$ tel que : $1-(\frac{4}{5})^{m}\geq 0.9$ ; il
faut au moins $11$ barres pour que la probabilité dépasse $90$\%.
Pour la probabilité $99$\%, $n\geq 21$ .
\fincorrection
\correction{005997}
\begin{enumerate}
\item Le taux global de personnes soulagées :
$P(S)=\frac{3}{5}0.75+\frac{2}{5}0.90=0.81$.

\item Probabilité pour un patient d'avoir pris de l'aspirine sachant qu'il
est soulagé :
$P(A/S)=P(A\cap S)/P(S)=P(A)P(S/A)/P(S)=\frac{\frac{3}{5}0.75}{0.81}=55.6\%$.
\end{enumerate}
\fincorrection
\correction{005998}
\begin{enumerate}
\item Probabilité conditionnelle : si un individu a les yeux bruns d'avoir
les cheveux blonds. 
C'est $P(CB/YB)=P(YB/CB)P(CB)/P(YB)$=$P(YB\cap CB)/P(YB)=\frac{0.15}{0.4}=0.375$.

\item La probabilité de l'événement : si un individu a les cheveux
blonds d'avoir les yeux bruns.
C'est $P(YB/CB)=P(YB\cap CB)/P(CB)$=$\frac{0.15}{0.25}=0.6$.

\item La probabilité de l'événement : si un individu a les cheveux
blonds,de ne pas avoir les yeux bruns.
C'est $P(\text{non}YB/CB)=1-P(YB/CB)=0.4$.
\end{enumerate}
\fincorrection
\correction{005999}
On obtient par calcul direct ou par événement contraire
la probabilité de voler : $1-p+p(1-q)^{2}$.
\fincorrection
\correction{006000}
\begin{enumerate}
\item La probabilité pour une personne d'être malade si son test est
positif est $P(M/T^{+})=P(T^{+}/M)P(M)/P(T^{+})$
or $P(T^{+})=P(T^{+}/M)P(M)+P(T^{+}/S)P(S)=0.95\cdot 0.03+0.1\cdot
0.97=0.125\,5$. 
D'où : $P(M/T^{+})=22.7\%$.

\item La probabilité pour une personne d'être saine si son test est
positif est $P(S/T^{+})=1-P(M/T^{+})=77.3\%$.

\item La probabilité pour une personne d'être malade si son test est négatif 
est $P(M/T^{-})=0.0017$.

\item La probabilité pour une personne d'être saine si son test est négatif
est $1-P(M/T^{-})=0.998=99.8\%$.
\end{enumerate}
\fincorrection
\correction{006001}
Une manière de résoudre le problème est la suivante:
puisqu'il y a $8$ clés et que j'écarte une après l'autre les mauvaises clés, 
je considère comme ensemble de toutes les possibilités, toutes les permutations de ces huit clés : 
il y en a $8$!. Alors la solution de chaque question est basée sur le même principe:
\begin{enumerate}
\item Les permutations (fictives) qui traduisent le cas (1) sont celles qui
peuvent être représentées par une suite : 
$BMMMMMMM$, la lettre $B$ désigne la bonne, $M$ désigne une
mauvaise. Il y a $7!$ permutations de ce type.
Donc $P(A)=\frac{7!}{8!}=\frac{1}{8},$ on s'en doutait!
\item De même, les permutations (fictives) sont
celles qui peuvent être représentées par une suite:
$MBMMMMMM$: il y en a encore $7$!, et la probabilité est la même.
\item Le raisonnement permet en fait de conclure que la probabilité, avant
de commencer, d'ouvrir la porte est la même pour le premier, deuxième,..., huitième essai.
\end{enumerate}
\fincorrection
\correction{006002}
\begin{enumerate}
\item L'univers des possibles est l'ensemble des couples possibles:
il y en a $6!=720$ (imaginez les dames assises et les hommes choisissant
leur partenaire). La probabilité $P(A)$ pour que chacun des $6$ hommes
danse avec son épouse légitime est, si chacun choisit au hasard, $\frac{1}{6!}$.

\item André danse avec son épouse, les autres choisissent au hasard: il y
a $5!$ permutations pour ces derniers:
$P(B)=\frac{5!}{6!}=\frac{1}{6}$.

\item André et René dansent avec leur épouse, les $4$ autres
choisissent au hasard: il y a $4!$ permutations pour ces derniers:
$P(C)=\frac{4!}{6!}=\frac{1}{30}$.

\item André ou René dansent avec leur épouse, les $4$ autres font ce
qu'ils veulent. Considérons les événements $D_{1}:$ <<André
danse avec son épouse>> ; $D_{2}$ : <<René danse avec son épouse>>. 
Alors $D=D_{1}\cup D_{2}$ et
 $P(D_{1}\cup D_{2})=P(D_{1})+P(D_{2})-P(D_{1}\cap D_{2})=\frac{3}{10}$.
\end{enumerate}
\fincorrection
\correction{006003}
\begin{enumerate}
  \item Combien de grilles ?
Il y en a $\binom{49}{6}=13\,983\,816$

  \item Combien de grilles avec 2 nombres consécutifs ?
Ce problème peut être résolu par astuce: considérer les numéros gagnants 
comme 6 places à <<choisir>> parmi 49.
En considérant des cloisons matérialisant les numéros gagnants,
c'est un problème de points et cloisons
Par exemple:
$$\mid \bullet \bullet \left\vert {}\right\vert \bullet
\left\vert \bullet \bullet \bullet \mid \bullet \bullet \right\vert$$
les gagnants sont: 1; 4; 5; 7; 11; 14.
Dans notre cas on ne veut pas de cloisons consécutives.
Les cinq cloisons séparent les numéros en 7 bo\^{\i}tes.
Les 5 bo\^{\i}tes intérieures étant non vides, on y met 5 points,
puis $38(=49-5-6)$ dans 7 bo\^{\i}tes.
Il y a $\frac{(38-1+7)!}{38!6!}=7.\,059\,1\times
10^{6}$ séquences ne comportant pas 2 nombres consécutifs.

D'où la probabilité d'avoir une grille
comportant 2 nombres consécutifs: $0.4952$.
\end{enumerate}
\fincorrection
\correction{006004}
\begin{enumerate}
\item $u_{n+1}=P(G_{n+1})=P(G_{n+1}/Gn)P(Gn)+P(G_{n+1}/\overline{G_n})P(\overline{G_n})
=0.6u_{n}+0.3v_{n}$.  

$v_{n+1}=0.4u_{n}+0.7v_{n}$.

Donc $\left( 
\begin{array}{c}
u_{n+1} \\ 
v_{n+1}\end{array}\right) =\left( 
\begin{array}{cc}
0.6 & 0.3 \\ 
0.4 & 0.7\end{array}\right) \left( 
\begin{array}{c}
u_{n} \\ 
v_{n}\end{array}\right)$

Comme $u_{n}+v_{n}=1$, $u_{n+1}=0.6u_{n}+0.3(1-u_{n})=0.3+0.3u_{n}$.
La suite $(u_{n}-\ell)$ est géométrique, où $\ell$ est solution
de $0.3+0.3\ell=\ell$, donc $\ell=\frac{3}{7}$.
Donc $u_{n}=\frac{3}{7}+u_{1}(0.3)^{n-1}=\frac{3}{7}+0.5 (0.3)^{n-1}$.
\end{enumerate}
\fincorrection


\end{document}

