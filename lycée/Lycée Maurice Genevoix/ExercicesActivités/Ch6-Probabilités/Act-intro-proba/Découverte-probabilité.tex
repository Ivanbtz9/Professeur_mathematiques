\documentclass[a4paper,11pt]{article}
\usepackage[utf8]{inputenc}
\usepackage[T1]{fontenc}
\usepackage{fancyhdr} % pour personnaliser les en-têtes
\usepackage{lastpage}
\usepackage[frenchb]{babel}
\usepackage{amsfonts,amssymb}
\usepackage{amsmath,amsthm}
\usepackage{paralist}
\usepackage{xspace}
\usepackage{xcolor}
\usepackage{variations}
\usepackage{xypic}
\usepackage{eurosym,multicol}
\usepackage{graphicx}
\usepackage{diagbox}
\usepackage[np]{numprint}
\usepackage{hyperref} 
\usepackage{listings} % pour écrire des codes avec coloration syntaxique  

\usepackage{tikz}
\usetikzlibrary{calc, arrows, plotmarks,decorations.pathreplacing}
\usepackage{colortbl}
\usepackage{multirow}
\usepackage[top=1.5cm,bottom=1.5cm,right=1.5cm,left=1.5cm]{geometry}

\newtheorem{defi}{Définition}
\newtheorem{thm}{Théorème}
\newtheorem{thm-def}{Théorème/Définition}
\newtheorem{rmq}{Remarque}
\newtheorem{prop}{Propriété}
\newtheorem{cor}{Corollaire}
\newtheorem{lem}{Lemme}
\newtheorem{ex}{Exemple}
\newtheorem{cex}{Contre-exemple}
\newtheorem{prop-def}{Propriété-définition}
\newtheorem{exer}{Exercice}
\newtheorem{nota}{Notation}
\newtheorem{ax}{Axiome}
\newtheorem{appl}{Application}
\newtheorem{csq}{Conséquence}
\theoremstyle{definition}
\newtheorem{exo}{Exercice}


\newcommand{\vtab}{\rule[-0.4em]{0pt}{1.2em}}
\newcommand{\V}{\overrightarrow}
\renewcommand{\thesection}{\Roman{section} }
\renewcommand{\thesubsection}{\arabic{subsection} }
\renewcommand{\thesubsubsection}{\alph{subsubsection} }
\newcommand{\C}{\mathbb{C}}
\newcommand{\R}{\mathbb{R}}
\newcommand{\Q}{\mathbb{Q}}
\newcommand{\Z}{\mathbb{Z}}
\newcommand{\N}{\mathbb{N}}
\newcommand{\notni}{\not\owns}

\definecolor{vert}{RGB}{11,160,78}
\definecolor{rouge}{RGB}{255,120,120}
% Set the beginning of a LaTeX document
\pagestyle{fancy}



\begin{document}
	
\lhead{Lycée Maurice Genevoix}\chead{}\rhead{Année~2021-2022}\lfoot{M. Botcazou}\cfoot{\thepage/2}\rfoot{\textbf{Tourner la page S.V.P.}}\renewcommand{\headrulewidth}{0.4pt}\renewcommand{\footrulewidth}{0.4pt}

\hfill\\[-0.7cm]
$$	\fbox{\text{\Large{ \sc Activités d'introduction aux Probabilités}}}$$

\hfill\\[-0.5cm]

\begin{exo}\textbf{"Maximum de deux dés"}\hfill\\[0.25cm]
	On lance deux dés équilibrés à six faces numérotés de $1$ à $6$: un dé bleu et un dé rouge. On note le résultat du lancer sous la forme d'un couple.\\
	 Par exemple (2 ; 5) signifie que l'on a obtenu le $2$ avec le dé bleu et $5$ avec le dé rouge (\textit{mettre des couleurs si besoin}).\\[0.5cm]
	\textbf{QUESTIONS:}
	
	\begin{enumerate}
		\item 
		\begin{enumerate}
			\item Combien de couples différents peut-on obtenir ? 
			\item Quelle est la probabilité d'obtenir le couple (3 ; 5) et le couple (6 ; 6) ?
		\end{enumerate}
		\item On s'intéresse à l'expérience aléatoire suivante:\\[0.25cm]
		Après avoir lancé ces deux dés, on observe la valeur maximale des faces supérieures obtenues. Le couple (2 ; 5) donne ainsi un maximum de 5 et le couple (6 ; 6) donne ainsi un maximum de 6. \\[-0.25cm]
		\begin{enumerate}
			\item  Remplir en \textbf{Annexe 1} le tableau associé à cette expérience aléatoire. 
			\item Quelles sont les issues de cette expérience aléatoire, en déduire l'univers \large{$\Omega$} \normalsize associé à cette expérience aléatoire
			\item En utilisant les questions précédentes, associer à chacune des issues possibles la probabilité qui lui correspond. 
			%\item Donner un exemple d'un évènement certain pour cette expérience aléatoire. 
			%\item Donner un exemple d'un évènement impossible pour cette expérience aléatoire.
		\end{enumerate}
		
	\end{enumerate}
\bigskip	  
\end{exo}

\begin{exo}\textbf{"Notion d'évènements"}\hfill\\[0.25cm]
	Dans un verger, trois variétés de pommes sont cultivées: des Golden Délicious ($50 \%$ de la production), des Gala ($30 \%$ de la production) et des Granny Smith. Malheureusement, ces variétés sont sensibles à une maladie appelée Tavelure. La Tavelure affecte $6\%$ des pommiers Golden, $4\%$ des Gala et $7\%$ des Granny Smith. \\[0.1cm]
	
	On choisit un pommier et on note:\\
	\begin{itemize}[$\square$]
		\item D l'évènement \textit{"Le pommier est de variété Golden Délicious"}
		\item G l'évènement \textit{"Le pommier est de variété Gala"}
		\item S l'évènement \textit{"Le pommier est de variété Granny Smith"}
		\item U l'évènement \textit{"Le pommier n'est pas atteint de Tavelure"}
		\item T l'évènement \textit{"Le pommier est atteint de Tavelure"}
	\end{itemize}
	\hfill \\
	\textbf{QUESTIONS:}
	
	
	\begin{enumerate}
		\item Remplir le tableau présent en \textbf{Annexe 2}.
		\item À partir de l'évènement T, on peut définir l'évènement contraire de l'évènement T, on le note $\overline{T}$.\\ Ainsi $\overline{T} = U$. 
		\begin{enumerate}
			\item Donner les valeurs de $P(T)$ et de $P(\overline{T})$.  
			\item En déduire une relation entre $P(T)$ et de $P(\overline{T})$
		\end{enumerate}
		\item À partir du tableau en \textbf{Annexe 2} :
		\begin{enumerate}
			\item Déterminer les probabilités $P(T \cap D)$, $P(\overline{T} \cap D)$ et $P(\overline{D} \cap T)$. 
			\item Interpréter les trois probabilités précédentes dans le contexte. En déduire $P(T \cup D)$.
			\item Comparer $P(T) + P(D)$ et $P(T \cup D) + P(T \cap D)$. Que remarque-t-on?
		\end{enumerate}
		
	\end{enumerate}	  
\end{exo}

\newpage
\lhead{Lycée Le Maurice Genevoix}\chead{}\rhead{Année~2021-2022}\lfoot{M. Botcazou}\cfoot{\thepage/2}\rfoot{\textbf{FIN.}}\renewcommand{\headrulewidth}{0.4pt}\renewcommand{\footrulewidth}{0.4pt}
\subsection*{Annexe 1:}
\hfill\\
\begin{center}
		\setlength{\extrarowheight}{0.25cm}
	\begin{tabular*}{0.7\linewidth}{@{\extracolsep{\stretch{0.5}}}|c|c|c|c|c|c|c|}
		\hline
		\diagbox{D\'{e} rouge}{D\'{e} bleu} & 
		1 & 2 & 3 & 4 & 5 & 6 \\ 
		\hline
		1   & 1 &  & & &5 & \\[0.2cm] \hline
		2 &  2 &   &  & & & \\[0.2cm] \hline
		3    &  3 &  & & & &  \\[0.2cm] \hline
		4    &  &   &  & & &  \\[0.2cm] \hline
		5    &  &   &  & & &  \\[0.2cm] \hline
		6    &  & 6  &  & & &  \\[0.2cm] \hline
	\end{tabular*}
\end{center}
\hfill\\
\subsection*{Annexe 2:}
\hfill\\
\begin{center}
	\setlength{\extrarowheight}{0.25cm}
	\begin{tabular*}{0.7\linewidth}{@{\extracolsep{\stretch{0.5}}}|c|c|c|c|c|}
		\hline
		\diagbox{Maladie}{Vari\'{e}t\'{e}} & 
		D & G & S & TOTAL  \\ 
		\hline
		U   &  &  & & \\[0.2cm] \hline
		T &  0.03 &   &  & \\[0.2cm] \hline
		TOTAL    & 0.5  &  & &  1 \\[0.2cm] \hline
	
	\end{tabular*}
\end{center}
	

\end{document}