
%%%%%%%%%%%%%%%%%% PREAMBULE %%%%%%%%%%%%%%%%%%

\documentclass[11pt,a4paper]{article}

\usepackage{amsfonts,amsmath,amssymb,amsthm}
\usepackage[utf8]{inputenc}
\usepackage[T1]{fontenc}
\usepackage[francais]{babel}
\usepackage{mathptmx}
\usepackage{fancybox}
\usepackage{graphicx}
\usepackage{ifthen}

\usepackage{tikz}   

\usepackage{hyperref}
\hypersetup{colorlinks=true, linkcolor=blue, urlcolor=blue,
pdftitle={Exo7 - Exercices de mathématiques}, pdfauthor={Exo7}}

\usepackage{geometry}
\geometry{top=2cm, bottom=2cm, left=2cm, right=2cm}

%----- Ensembles : entiers, reels, complexes -----
\newcommand{\Nn}{\mathbb{N}} \newcommand{\N}{\mathbb{N}}
\newcommand{\Zz}{\mathbb{Z}} \newcommand{\Z}{\mathbb{Z}}
\newcommand{\Qq}{\mathbb{Q}} \newcommand{\Q}{\mathbb{Q}}
\newcommand{\Rr}{\mathbb{R}} \newcommand{\R}{\mathbb{R}}
\newcommand{\Cc}{\mathbb{C}} \newcommand{\C}{\mathbb{C}}
\newcommand{\Kk}{\mathbb{K}} \newcommand{\K}{\mathbb{K}}

%----- Modifications de symboles -----
\renewcommand{\epsilon}{\varepsilon}
\renewcommand{\Re}{\mathop{\mathrm{Re}}\nolimits}
\renewcommand{\Im}{\mathop{\mathrm{Im}}\nolimits}
\newcommand{\llbracket}{\left[\kern-0.15em\left[}
\newcommand{\rrbracket}{\right]\kern-0.15em\right]}
\renewcommand{\ge}{\geqslant} \renewcommand{\geq}{\geqslant}
\renewcommand{\le}{\leqslant} \renewcommand{\leq}{\leqslant}

%----- Fonctions usuelles -----
\newcommand{\ch}{\mathop{\mathrm{ch}}\nolimits}
\newcommand{\sh}{\mathop{\mathrm{sh}}\nolimits}
\renewcommand{\tanh}{\mathop{\mathrm{th}}\nolimits}
\newcommand{\cotan}{\mathop{\mathrm{cotan}}\nolimits}
\newcommand{\Arcsin}{\mathop{\mathrm{arcsin}}\nolimits}
\newcommand{\Arccos}{\mathop{\mathrm{arccos}}\nolimits}
\newcommand{\Arctan}{\mathop{\mathrm{arctan}}\nolimits}
\newcommand{\Argsh}{\mathop{\mathrm{argsh}}\nolimits}
\newcommand{\Argch}{\mathop{\mathrm{argch}}\nolimits}
\newcommand{\Argth}{\mathop{\mathrm{argth}}\nolimits}
\newcommand{\pgcd}{\mathop{\mathrm{pgcd}}\nolimits} 

%----- Structure des exercices ------

\newcommand{\exercice}[1]{\video{0}}
\newcommand{\finexercice}{}
\newcommand{\noindication}{}
\newcommand{\nocorrection}{}

\newcounter{exo}
\newcommand{\enonce}[2]{\refstepcounter{exo}\hypertarget{exo7:#1}{}\label{exo7:#1}{\bf Exercice \arabic{exo}}\ \  #2\vspace{1mm}\hrule\vspace{1mm}}

\newcommand{\finenonce}[1]{
\ifthenelse{\equal{\ref{ind7:#1}}{\ref{bidon}}\and\equal{\ref{cor7:#1}}{\ref{bidon}}}{}{\par{\footnotesize
\ifthenelse{\equal{\ref{ind7:#1}}{\ref{bidon}}}{}{\hyperlink{ind7:#1}{\texttt{Indication} $\blacktriangledown$}\qquad}
\ifthenelse{\equal{\ref{cor7:#1}}{\ref{bidon}}}{}{\hyperlink{cor7:#1}{\texttt{Correction} $\blacktriangledown$}}}}
\ifthenelse{\equal{\myvideo}{0}}{}{{\footnotesize\qquad\texttt{\href{http://www.youtube.com/watch?v=\myvideo}{Vidéo $\blacksquare$}}}}
\hfill{\scriptsize\texttt{[#1]}}\vspace{1mm}\hrule\vspace*{7mm}}

\newcommand{\indication}[1]{\hypertarget{ind7:#1}{}\label{ind7:#1}{\bf Indication pour \hyperlink{exo7:#1}{l'exercice \ref{exo7:#1} $\blacktriangle$}}\vspace{1mm}\hrule\vspace{1mm}}
\newcommand{\finindication}{\vspace{1mm}\hrule\vspace*{7mm}}
\newcommand{\correction}[1]{\hypertarget{cor7:#1}{}\label{cor7:#1}{\bf Correction de \hyperlink{exo7:#1}{l'exercice \ref{exo7:#1} $\blacktriangle$}}\vspace{1mm}\hrule\vspace{1mm}}
\newcommand{\fincorrection}{\vspace{1mm}\hrule\vspace*{7mm}}

\newcommand{\finenonces}{\newpage}
\newcommand{\finindications}{\newpage}


\newcommand{\fiche}[1]{} \newcommand{\finfiche}{}
%\newcommand{\titre}[1]{\centerline{\large \bf #1}}
\newcommand{\addcommand}[1]{}

% variable myvideo : 0 no video, otherwise youtube reference
\newcommand{\video}[1]{\def\myvideo{#1}}

%----- Presentation ------

\setlength{\parindent}{0cm}

\definecolor{myred}{rgb}{0.93,0.26,0}
\definecolor{myorange}{rgb}{0.97,0.58,0}
\definecolor{myyellow}{rgb}{1,0.86,0}

\newcommand{\LogoExoSept}[1]{  % input : echelle       %% NEW
{\usefont{U}{cmss}{bx}{n}
\begin{tikzpicture}[scale=0.1*#1,transform shape]
  \fill[color=myorange] (0,0)--(4,0)--(4,-4)--(0,-4)--cycle;
  \fill[color=myred] (0,0)--(0,3)--(-3,3)--(-3,0)--cycle;
  \fill[color=myyellow] (4,0)--(7,4)--(3,7)--(0,3)--cycle;
  \node[scale=5] at (3.5,3.5) {Exo7};
\end{tikzpicture}}
}


% titre
\newcommand{\titre}[1]{%
\vspace*{-4ex} \hfill \hspace*{1.5cm} \hypersetup{linkcolor=black, urlcolor=black} 
\href{http://exo7.emath.fr}{\LogoExoSept{3}} 
 \vspace*{-5.7ex}\newline 
\hypersetup{linkcolor=blue, urlcolor=blue}  {\Large \bf #1} \newline 
 \rule{12cm}{1mm} \vspace*{3ex}}

%----- Commandes supplementaires ------



\begin{document}

%%%%%%%%%%%%%%%%%% EXERCICES %%%%%%%%%%%%%%%%%%
\fiche{f00092, rouget, 2010/07/11}

\titre{Suites} 

Exercices de Jean-Louis Rouget.
Retrouver aussi cette fiche sur \texttt{\href{http://www.maths-france.fr}{www.maths-france.fr}}

\begin{center}
* très facile\quad** facile\quad*** difficulté moyenne\quad**** difficile\quad***** très difficile\\
I~:~Incontournable\quad T~:~pour travailler et mémoriser le cours
\end{center}


\exercice{5220, rouget, 2010/06/30}
\enonce{005220}{***IT}
Soient $(u_n)_{n\in\Nn}$ une suite réelle et $(v_n)_{n\in\Nn}$ la suite définie par~:~$\forall n\in\Nn,\;v_n=\frac{u_0+u_1+...+u_n}{n+1}$.

\begin{enumerate}
\item  Montrer que si la suite $(u_n)_{n\in\Nn}$ vers un réel $\ell$, la suite $(v_n)_{n\in\Nn}$ converge et a pour limite $\ell$. Réciproque~?
\item  Montrer que si la suite $(u_n)_{n\in\Nn}$ est bornée, la suite $(v_n)_{n\in\Nn}$ est bornée. Réciproque~?
\item  Montrer que si la suite $(u_n)_{n\in\Nn}$ est croissante alors la suite $(v_n)_{n\in\Nn}$ l'est aussi.
\end{enumerate}
\finenonce{005220}


\finexercice
\exercice{5221, rouget, 2010/06/30}
\enonce{005221}{***}
Soit $(u_n)_{n\in\Nn}$ une suite réelle. Montrer que si la suite $(u_n)_{n\in\Nn}$ converge au sens de \textsc{Césaro} et est monotone, alors la suite $(u_n)_{n\in\Nn}$ converge.
\finenonce{005221}


\finexercice
\exercice{5222, rouget, 2010/06/30}
\enonce{005222}{**IT}
\label{exo:suprou3bis}
Pour $n$ entier naturel non nul, on pose $H_n=\sum_{k=1}^{n}\frac{1}{k}$ (série harmonique).

\begin{enumerate}
\item  Montrer que~:~$\forall n\in\Nn^*,\;\ln(n+1)<H_n<1+\ln(n)$ et en déduire $\lim_{n\rightarrow +\infty}H_n$.
\item  Pour $n$ entier naturel non nul, on pose $u_n=H_n-\ln(n)$ et $v_n=H_n-\ln(n+1)$. Montrer que les suites $(u_n)$ et $(v_n)$ convergent vers un réel $\gamma\in\left[\frac{1}{2},1\right]$ ($\gamma$ est appelée la constante d'\textsc{Euler}). Donner une valeur approchée de $\gamma$ à $10^{-2}$ près.
\end{enumerate}
\finenonce{005222}


\finexercice
\exercice{5223, rouget, 2010/06/30}
\enonce{005223}{**}
Soit $(u_n)_{n\in\Nn}$ une suite arithmétique ne s'annulant pas. Montrer que pour tout entier naturel $n$, on a $\sum_{k=0}^{n}\frac{1}{u_ku_{k+1}}=\frac{n+1}{u_0u_{n+1}}$.
\finenonce{005223}


\finexercice
\exercice{5224, rouget, 2010/06/30}
\enonce{005224}{**}
Calculer $\lim_{n\rightarrow +\infty}\sum_{k=1}^{n}\frac{1}{1^2+2^2+...+k^2}$.
\finenonce{005224}


\finexercice\exercice{5225, rouget, 2010/06/30}
\enonce{005225}{***}
Soient $a$ et $b$ deux réels tels que $0<a<b$. On pose $u_0=a$ et $v_0=b$ puis, pour $n$ entier naturel donné, $u_{n+1}= \frac{u_n+v_n}{2}$ et $v_{n+1}=\sqrt{u_{n+1}v_n}$.
Montrer que les suites $(u_n)$ et $(v_n)$ sont adjacentes et que leur limite commune est égale à $\frac{b\sin(\Arccos(\frac{a}{b}))}{\Arccos(\frac{a}{b})}$.
\finenonce{005225}


\finexercice
\exercice{5226, rouget, 2010/06/30}
\enonce{005226}{**}
Limite quand $n$ tend vers $+\infty$ de 
\begin{enumerate}
 \item $\frac{\sin n}{n}$,
 \item $\left(1+\frac{1}{n}\right)^n$,
 \item $\frac{n!}{n^n}$,
 \item $\frac{E\left((n+\frac{1}{2})^2\right)}{E\left((n-\frac{1}{2})^2\right))}$,
 \item $\sqrt[n]{n^2}$,
 \item $\sqrt{n+1}-\sqrt{n}$,
 \item $\frac{\sum_{k=1}^{n}k^2}{n^3}$,
 \item $\prod_{k=1}^{n}2^{k/2^{2^k}}$.
\end{enumerate}

\finenonce{005226}


\finexercice
\exercice{5227, rouget, 2010/06/30}
\enonce{005227}{**}
Etudier la suite $(u_n)$ définie par $\sqrt{n+1}-\sqrt{n}=\frac{1}{2\sqrt{n+u_n}}$.
\finenonce{005227}


\finexercice
\exercice{5228, rouget, 2010/06/30}
\enonce{005228}{**T Récurrences homographiques}
Déterminer $u_n$ en fonction de $n$ quand la suite $u$ vérifie~:
\begin{enumerate}
 \item $\forall n\in\Nn,\;u_{n+1}=\frac{u_n}{3-2u_n}$,
 \item $\forall n\in\Nn,\;u_{n+1}=\frac{4(u_n-1)}{u_n}$ (ne pas se poser de questions d'existence).
\end{enumerate}
\finenonce{005228}


\finexercice
\exercice{5229, rouget, 2010/06/30}
\enonce{005229}{**}
Soient $(u_n)$ et $(v_n)$ les suites définies par la donnée de $u_0$ et $v_0$ et les relations de récurrence 

$$u_{n+1}=\frac{2u_n+v_n}{3}\;\mbox{et}\;v_{n+1}=\frac{u_n+2v_n}{3}.$$
Etudier les suites $u$ et $v$ puis déterminer $u_n$ et $v_n$ en fonction de $n$ en recherchant des combinaisons linéaires intéressantes de $u$ et $v$. En déduire $\lim_{n\rightarrow +\infty}u_n$ et $\lim_{n\rightarrow +\infty}v_n$.
\finenonce{005229}


\finexercice
\exercice{5230, rouget, 2010/06/30}
\enonce{005230}{**}
Soient $(u_n)$, $(v_n)$ et $(w_n)$ les suites définies par la donnée de $u_0$,
$v_0$ et $w_0$ et les relations de récurrence 
$$u_{n+1}=\frac{v_n+w_n}{2},\,v_{n+1}=\frac{u_n+w_n}{2}\;\mbox{et}\;
w_{n+1}=\frac{u_n+v_n}{2}.$$

Etudier les suites $u$, $v$ et $w$ puis déterminer $u_n$, $v_n$ et $w_n$ en fonction de $n$
en recherchant des combinaisons linéaires intéressantes de $u$, $v$ et $w$. En
déduire $\lim_{n\rightarrow +\infty}u_n$, $\lim_{n\rightarrow +\infty}v_n$et $\lim_{n\rightarrow +\infty}w_n$.

\finenonce{005230}


\finexercice
\exercice{5231, rouget, 2010/06/30}
\enonce{005231}{***}
Montrer que les suites définies par la donnée de $u_0$, $v_0$ et $w_0$ réels tels que $0<u_0<v_0<w_0$ et les relations de récurrence~:
 
$$\frac{3}{u_{n+1}}=\frac{1}{u_n}+\frac{1}{v_n}+\frac{1}{w_n}\;\mbox{et}\;v_{n+1}=\sqrt[3]{u_nv_nw_n}\;\mbox{et}\;w_{n+1}=\frac{u_n+v_n+w_n}{3},$$  
 
 
ont une limite commune que l'on ne cherchera pas à déterminer.

\finenonce{005231}


\finexercice\exercice{5232, rouget, 2010/06/30}
\enonce{005232}{***}
Soit $u$ une suite complexe et $v$ la suite définie par $v_n=|u_n|$. On suppose que la suite $(\sqrt[n]{v_n})$ converge vers un réel positif $l$. Montrer que si $0\leq\ell<1$, la suite $(u_n)$ converge vers $0$ et si $\ell>1$, la suite $(v_n)$ tend vers $+\infty$.
Montrer que si $\ell=1$, tout est possible.
\finenonce{005232}


\finexercice
\exercice{5233, rouget, 2010/06/30}
\enonce{005233}{***}
\begin{enumerate}
\item  Soit $u$ une suite de réels strictement positifs. Montrer que si la suite $(\frac{u_{n+1}}{u_n})$ converge vers un réel $\ell$, alors $(\sqrt[n]{u_n})$ converge et a même limite.
\item  Etudier la réciproque.
\item  Application~:~limites de 
  \begin{enumerate}
  \item $\sqrt[n]{C_{2n}^n}$,
  \item $\frac{n}{\sqrt[n]{n!}}$,
  \item $\frac{1}{n^2}\sqrt[n]{\frac{(3n)!}{n!}}$.
  \end{enumerate}
\end{enumerate}
\finenonce{005233}


\finexercice
\exercice{5234, rouget, 2010/06/30}
\enonce{005234}{*}
Soient $u$ et $v$ deux suites de réels de $[0,1]$ telles que $\lim_{n\rightarrow +\infty}u_nv_n=1$. Montrer que $(u_n)$ et $(v_n)$ convergent vers $1$.

\finenonce{005234}


\finexercice
\exercice{5235, rouget, 2010/06/30}
\enonce{005235}{**}
Montrer que si les suites $(u_n^2)$ et $(u_n^3)$ convergent alors $(u_n)$ converge.
\finenonce{005235}


\finexercice
\exercice{5236, rouget, 2010/06/30}
\enonce{005236}{***T}
Etudier les deux suites $u_n=\left(1+\frac{1}{n}\right)^n$  et $v_n=\left(1+\frac{1}{n}\right)^{n+1}$.
\finenonce{005236}


\finexercice
\exercice{5237, rouget, 2010/06/30}
\enonce{005237}{**T}
Etudier les deux suites $u_n=\sum_{k=0}^{n}\frac{1}{k!}$ et $v_n=u_n+\frac{1}{n.n!}$.

\finenonce{005237}


\finexercice
\exercice{5238, rouget, 2010/06/30}
\enonce{005238}{}
Etudier les deux suites $u_n=\left(\sum_{k=1}^{n}\frac{1}{\sqrt{k}}\right)-2\sqrt{n+1}$ et $v_n=\left(\sum_{k=1}^{n}\frac{1}{\sqrt{k}}\right)-2\sqrt{n}$.
\finenonce{005238}


\finexercice
\exercice{5239, rouget, 2010/06/30}
\enonce{005239}{**T}
Déterminer $u_n$ en fonction de $n$ et de ses premiers termes dans chacun des cas suivants~:

\begin{enumerate}
\item  $\forall n\in\Nn,\;4u_{n+2}=4u_{n+1}+3u_n$.
\item  $\forall n\in\Nn,\;4u_{n+2}=u_n$.
\item  $\forall n\in\Nn,\;4u_{n+2}=4u_{n+1}+3u_n+12$.
\item  $\forall n\in\Nn,\;\frac{2}{u_{n+2}}=\frac{1}{u_{n+1}}-\frac{1}{u_n}$.
\item  $\forall n\geq2,\;u_n= 3u_{n-1}-2u_{n-2}+n^3$.
\item  $\forall n\in\Nn,\;u_{n+3}-6u_{n+2}+11u_{n+1}-6u_n=0$.
\item  $\forall n\in\Nn,\;u_{n+4}-2u_{n+3}+2u_{n+2}-2u_{n+1}+u_n=n^5$.
\end{enumerate}

\finenonce{005239}


\finexercice
\exercice{5240, rouget, 2010/06/30}
\enonce{005240}{****}
On pose $u_1=1$ et, $\forall n\in\Nn^*,\;u_{n+1}=1+\frac{n}{u_n}$. Montrer que $\lim_{n\rightarrow +\infty}(u_n-\sqrt{n})=\frac{1}{2}$.
\finenonce{005240}


\finexercice
\exercice{5241, rouget, 2010/06/30}
\enonce{005241}{***}
Montrer que, pour $n\geq2$, 

\begin{center}
$\cos\left(\frac{\pi}{2^{n}}\right)=\frac{1}{2}\sqrt{2+\sqrt{2+...+\sqrt{2}}}$ ($n-1$ radicaux) et $\sin\left(\frac{\pi}{2^{n}}\right)=\frac{1}{2}\sqrt{2-\sqrt{2+...+\sqrt{2}}}$ ($n-1$ radicaux).
\end{center}
En déduire $\lim_{n\rightarrow +\infty}2^n\sqrt{2-\sqrt{2+...+\sqrt{2}}}$ ($n$ radicaux).
\finenonce{005241}


\finexercice
\exercice{5242, rouget, 2010/06/30}
\enonce{005242}{***}
\begin{enumerate}
\item  Montrer que pour $x$ réel strictement positif, on a~:~$\ln(1+x)<x<(1+x)\ln(1+x)$.
\item  Montrer que $\prod_{k=1}^{n}\left(1+\frac{1}{k}\right)^k<e^n<\prod_{k=1}^{n}\left(1+\frac{1}{k}\right)^{k+1}$ et en déduire la limite quand $n$ tend vers $+\infty$ de $\frac{\sqrt[n]{n!}}{n}$.
\end{enumerate}
\finenonce{005242}


\finexercice
\exercice{5243, rouget, 2010/06/30}
\enonce{005243}{****}
Soit $(u_n)=\left(\frac{p_n}{q_n}\right)$ avec $p_n\in\Zz$ et $q_n\in\Nn^*$, une suite de rationnels convergeant vers un irrationnel $x$. Montrer que les suites $(|p_n|)$ et $(q_n)$ tendent vers $+\infty$ quand $n$ tend vers $+\infty$.
\finenonce{005243}


\finexercice
\exercice{5244, rouget, 2010/06/30}
\enonce{005244}{**}
Donner un exemple de suite $(u_n)$ divergente, telle que $\forall k\in\Nn^*\setminus\{1\}$, la suite $(u_{kn})$ converge.
\finenonce{005244}


\finexercice
\exercice{5245, rouget, 2010/06/30}
\enonce{005245}{***I}
Soit $f$ une application injective de $\Nn$ dans $\Nn$. Montrer que $\lim_{n\rightarrow +\infty}f(n)=+\infty$.
\finenonce{005245}


\finexercice
\exercice{5246, rouget, 2010/06/30}
\enonce{005246}{***I}
Soit $u_n$ l'unique racine positive de l'équation $x^n+x-1=0$. Etudier la suite $(u_n)$.
\finenonce{005246}


\finexercice
\exercice{5247, rouget, 2010/07/04}
\enonce{005247}{****I}
Etude des suites $(u_n)=(\cos na)$ et $(v_n)=(\sin na)$ où $a$ est un réel donné.

\begin{enumerate}
\item  Montrer que si $\frac{a}{2\pi}$ est rationnel, les suites $u$ et $v$ sont périodiques et montrer dans ce cas que $(u_n)$ et $(v_n)$ convergent si et seulement si $a\in2\pi\Zz$.
\item  On suppose dans cette question que $\frac{a}{2\pi}$ est irrationnel .
\begin{enumerate}
\item Montrer que $(u_n)$ converge si et seulement si $(v_n)$ converge .
\item En utilisant différentes formules de trigonométrie fournissant des relations entre $u_n$ et $v_n$, montrer par l'absurde que $(u_n)$ et $(v_n)$ divergent.
\end{enumerate}
\item  On suppose toujours que $\frac{a}{2\pi}$ est irrationnel. On veut montrer que l'ensemble des valeurs de la suite $(u_n)$ (ou $(v_n)$) est dense dans $[-1,1]$, c'est-à-dire que $\forall x\in[-1,1],\;\forall\varepsilon>0,\;\exists n\in\Nn/\;|u_n-x|<\varepsilon$ (et de même pour $v$).
\begin{enumerate}
\item Montrer que le problème se ramène à démontrer que $\{na+2k\pi,\;n\in\Nn\;\mbox{et}\;k\in\Zz\}$ est dense dans $\Rr$.
\item Montrer que $E=\{na+2k\pi,\;n\in\Nn\;\mbox{et}\;k\in\Zz\}$ est dense dans $\Rr$ (par l'absurde en supposant que $\mbox{inf}(E\cap\Rr_+^*)>0$ pour en déduire que $\frac{a}{2\pi}\in\Qq$).
\item Conclure.
\end{enumerate}
\end{enumerate}
\finenonce{005247}


\finexercice
\exercice{5248, rouget, 2010/07/04}
\enonce{005248}{****}
Montrer que l'ensemble $E$ des réels de la forme $u_n=\sin(\ln(n))$, $n$ entier naturel non nul, est dense dans 
$[-1,1]$.
\finenonce{005248}


\finexercice
\exercice{5249, rouget, 2010/07/04}
\enonce{005249}{***}
Calculer $\mbox{inf}_{\alpha\in]0,\pi[}(\mbox{sup}_{n\in\Nn}(|\sin(n\alpha)|))$.
\finenonce{005249}


\finexercice
\exercice{5250, rouget, 2010/07/04}
\enonce{005250}{**I}
Soit $(u_n)$ une suite réelle non majorée. Montrer qu'il existe une suite extraite de $(u_n)$ tendant vers $+\infty$.

\finenonce{005250}


\finexercice
\exercice{5251, rouget, 2010/07/04}
\enonce{005251}{***}
Soit $(u_n)$ une suite de réels éléments de $]0,1[$ telle que $\forall n\in\Nn,\;(1-u_n)u_{n+1}>\frac{1}{4}$. Montrer que $(u_n)$ converge vers $\frac{1}{2}$.
\finenonce{005251}


\finexercice

\finfiche

 \finenonces 



 \finindications 

\noindication
\noindication
\noindication
\noindication
\noindication
\noindication
\noindication
\noindication
\noindication
\noindication
\noindication
\noindication
\noindication
\noindication
\noindication
\noindication
\noindication
\noindication
\noindication
\noindication
\noindication
\noindication
\noindication
\noindication
\noindication
\noindication
\noindication
\noindication
\noindication
\noindication
\noindication
\noindication


\newpage

\correction{005220}
\begin{enumerate}
 \item  Soit $\varepsilon>0$. Il existe un rang $n_0$ tel que, si $n\geq n_0$ alors $|u_n-\ell|<\frac{\varepsilon}{2}$. Soit $n$ un entier naturel strictement supérieur à $n_0$.

\begin{align*}
|v_n-\ell|&=\left|\frac{1}{n+1}\sum_{k=0}^{n}u_k-\ell\right|=\left|\frac{1}{n+1}\sum_{k=0}^{n}(u_k-\ell)\right|\\
 &\leq\frac{1}{n+1}\sum_{k=0}^{n}|u_k-\ell|=\frac{1}{n+1}\sum_{k=0}^{n_0}|u_k-\ell|+\frac{1}{n+1}\sum_{k=n_0+1}^{n}|u_k-\ell|\\
 &\leq\frac{1}{n+1}\sum_{k=0}^{n_0}|u_k-\ell|+\frac{1}{n+1}\sum_{k=n_0+1}^{n}\frac{\varepsilon}{2}\leq\frac{1}{n+1}\sum_{k=0}^{n_0}|u_k-\ell|+\frac{1}{n+1}\sum_{k=0}^{n}\frac{\varepsilon}{2}\\
 &=\frac{1}{n+1}\sum_{k=0}^{n_0}|u_k-\ell|+\frac{\varepsilon}{2}
\end{align*}
Maintenant, $\sum_{k=0}^{n_0}|u_k-\ell|$ est une expression constante quand $n$ varie et donc, $\lim_{n\rightarrow +\infty}\frac{1}{n+1}\sum_{k=0}^{n_0}|u_k-\ell|=0$. Par suite, il existe un entier $n_1\geq n_0$ tel que pour $n\geq n_1$, $\frac{1}{n+1}\sum_{k=0}^{n_0}|u_k-\ell|<\frac{\varepsilon}{2}$.
Pour $n\geq n_1$, on a alors $|v_n-\ell|<\frac{\varepsilon}{2}+\frac{\varepsilon}{2}=\varepsilon$.
On a montré que $\forall\varepsilon>0,\;\exists n_1\in\Nn/\;(\forall n\in\Nn)(n\geq n_1\Rightarrow|v_n-\ell|<\varepsilon)$. La suite $(v_n)$ est donc convergente et $\lim_{n\rightarrow +\infty}v_n=\ell$.

\begin{center}
\shadowbox{
Si la suite $u$ converge vers $\ell$ alors la suite $v$ converge vers $\ell$.
}
\end{center}
La réciproque est fausse. Pour $n$ dans $\Nn$, posons $u_n=(-1)^n$. La suite $(u_n)$ est divergente. D'autre part, pour $n$ dans $\Nn$, $\sum_{k=0}^{n}(-1)^k$ vaut $0$ ou $1$ suivant la parité de $n$ et donc, dans tous les cas, $|v_n|\leq\frac{1}{n+1}$. Par suite, la suite $(v_n)$ converge et $\lim_{n\rightarrow +\infty}v_n=0$.
 \item  Si $u$ est bornée, il existe un réel $M$ tel que, pour tout naturel $n$, $|u_n|\leq M$.
Pour $n$ entier naturel donné, on a alors

$$|v_n|\leq\frac{1}{n+1}\sum_{k=0}^{n}|u_k|\leq\frac{1}{n+1}\sum_{k=0}^{n}M=\frac{1}{n+1}(n+1)M=M.$$
La suite $v$ est donc bornée.

\begin{center}
\shadowbox{
Si la suite $u$ est bornée alors la suite $v$ est bornée.
}
\end{center}
La réciproque est fausse. Soit $u$ la suite définie par~:~$\forall n\in\Nn,\;u_n=(-1)^nE\left(\frac{n}{2}\right)=
\left\{
\begin{array}{l}
p\;\mbox{si}\;n=2p,\;p\in\Nn\\
-p\;\mbox{si}\;n=2p+1,\;p\in\Nn
\end{array}
\right.$.
$u$ n'est pas bornée car la suite extraite $(u_{2p})$ tend vers $+\infty$ quand $p$ tend vers $+\infty$. 
Mais, si $n$ est impair, $v_n=0$, et si $n$ est pair, $v_n=\frac{1}{n+1}\times u_n=\frac{n}{2(n+1)}$, et dans tous les cas $|v_n|\leq\frac{1}{n+1}\frac{n}{2}\leq\frac{1}{n+1}\frac{n+1}{2}=\frac{1}{2}$ et la suite $v$ est bornée.
 \item  Si $u$ est croissante, pour $n$ entier naturel donné on a~:

\begin{align*}
v_{n+1}-v_n&=\frac{1}{n+2}\sum_{k=0}^{n+1}u_k-\frac{1}{n+1}\sum_{k=0}^{n}u_k=\frac{1}{(n+1)(n+2)}\left((n+1)\sum_{k=0}^{n+1}u_k-(n+2)\sum_{k=0}^{n}u_k\right)\\
 &=\frac{1}{(n+1)(n+2)}\left((n+1)u_{n+1}-\sum_{k=0}^{n}u_k\right)=\frac{1}{(n+1)(n+2)}\sum_{k=0}^{n}(u_{n+1}-u_k)\geq0.
\end{align*}
La suite $v$ est donc croissante.

\begin{center}
\shadowbox{
Si la suite $u$ est croissante alors la suite $v$ est croissante.
}
\end{center}
\end{enumerate}
\fincorrection
\correction{005221}
Supposons sans perte de généralité $u$ croissante (quite à remplacer $u$ par $-u$).
Dans ce cas, ou bien $u$ converge, ou bien $u$ tend vers $+\infty$.
Supposons que $u$ tende vers $+\infty$, et montrons qu'il en est de même pour la suite $v$.
Soit $A\in\Rr$. Il existe un rang $n_0$ tel que pour n naturel supérieur ou égal à $n_0$, $u_n\geq2A$.
Pour $n\geq n_0+1$, on a alors,

\begin{align*}
v_n&=\frac{1}{n+1}\left(\sum_{k=0}^{n_0}u_k+\sum_{k=n_0+1}^{n}u_k\right)\geq \frac{1}{n+1}\sum_{k=0}^{n_0}u_k+\frac{(n-n_0)2A}{n+1}
\end{align*}
Maintenant, quand $n$ tend vers $+\infty$, $\frac{1}{n+1}\sum_{k=0}^{n_0}u_k+\frac{(n-n_0)2A}{n+1}$ tend vers $2A$ et donc, il existe un rang $n_1$ à partir duquel $v_n\geq\frac{1}{n+1}\sum_{k=0}^{n_0}u_k+\frac{(n-n_0)2A}{n+1}>A$.
On a montré que~:~$\forall n\in\Nn,\;\exists n_1\in\Nn/\;(\forall n\in\Nn),\;(n\geq n_1\Rightarrow v_n>A)$. Par suite, $\lim_{n\rightarrow +\infty}v_n=+\infty$. Par contraposition, si $v$ ne tend pas vers $+\infty$, la suite $u$ ne tend pas vers $+\infty$ et donc converge, d'après la remarque initiale.
\fincorrection
\correction{005222}
\begin{enumerate}
 \item  La fonction $x\mapsto\frac{1}{x}$ est continue et décroissante sur $]0,+\infty[$ et donc, pour $k\in\Nn^*$, on a~:
 
$$\frac{1}{k+1}=(k+1-k)\frac{1}{k+1}\leq\int_{k}^{k+1}\frac{1}{x}\;dx\leq(k+1-k)\frac{1}{k}=\frac{1}{k}.$$
Donc, pour $k\geq1$,  $\frac{1}{k}\geq\int_{k}^{k+1}\frac{1}{x}\;dx$ et, pour $k\geq2$, $\frac{1}{k}\leq\int_{k-1}^{k}\frac{1}{x}\;dx$.
En sommant ces inégalités, on obtient pour $n\geq1$,

$$H_n=\sum_{k=1}^{n}\frac{1}{k}\geq\sum_{k=1}^{n}\int_{k}^{k+1}\frac{1}{x}\;dx=\int_{1}^{n+1}\frac{1}{x}\;dx=\ln(n+1),$$ 
et pour $n\geq2$,

$$H_n=1+\sum_{k=2}^{n}\frac{1}{k}\leq1+\sum_{k=2}^{n}\int_{k-1}^{k}\frac{1}{x}\;dx=1+\int_{1}^{n}\frac{1}{x}\;dx=1+\ln n,$$
cette dernière inégalité restant vraie quand $n=1$. Donc,

\begin{center}
\shadowbox{
$\forall n\in\Nn^*,\;\ln(n+1)\leq H_n\leq1+\ln n.$
}
\end{center}
 \item  Soit $n$ un entier naturel non nul.
$$u_{n+1}-u_n=\frac{1}{n+1}-\ln(n+1)+\ln n=\frac{1}{n+1}-\int_{n}^{n+1}\frac{1}{x}\;dx=\int_{n}^{n+1}\left(\frac{1}{n+1}-\frac{1}{x}\right)\;dx\leq0$$
car la fonction $x\mapsto\frac{1}{x}$ décroit sur $[n,n+1]$. De même,

$$v_{n+1}-v_n=\frac{1}{n+1}-\ln(n+2)+\ln(n+1)=\frac{1}{n+1}-\int_{n+1}^{n+2}\frac{1}{x}\;dx=\int_{n+1}^{n+2}\left(\frac{1}{n+1}-\frac{1}{x}\right)\;dx\geq0$$
car la fonction $x\mapsto\frac{1}{x}$ décroit sur $[n+1,n+2]$. Enfin,

$$u_n-v_n=\ln(n+1)-\ln n=\ln\left(1+\frac{1}{n}\right)$$ 
et donc la suite $u-v$ tend vers 0 quand $n$ tend vers $+\infty$.
Finalement, la suite $u$ décroit, la suite $v$ croit et la suite $u-v$ tend vers $0$. On en déduit que les suites $u$ et $v$ sont adjacentes, et en particulier convergentes et de même limite. Notons $\gamma$ cette limite.
Pour tout entier naturel non nul $n$, on a $v_n\leq\gamma\leq u_n$, et en particulier, $v_3\leq\gamma\leq u_1$ avec $v_3=0,5...$ et $u_1=1$. Donc, $\gamma\in\left[\frac{1}{2},1\right]$.
Plus précisément, pour $n$ entier naturel non nul donné, on a

$$0\leq u_n-v_n\leq\frac{10^{-2}}{2}\Leftrightarrow\ln\left(1+\frac{1}{n}\right)\leq0,005\Leftrightarrow\frac{1}{n}\leq e^{0,005}-1\Leftrightarrow n\geq\frac{1}{e^{0,005}-1}=199,5...\Leftrightarrow n\geq200.$$
Donc $0\leq\gamma-v_{100}\leq\frac{10^{-2}}{2}$ et une valeur approchée de $v_{200}$ à $\frac{10^{-2}}{2}$ près (c'est-à-dire arrondie à la 3 ème décimale la plus proche) est une valeur approchée de $\gamma$ à $10^{-2}$ près. On trouve $\gamma=0,57$ à $10^{-2}$ près par défaut. Plus précisémént,

\begin{center}
\shadowbox{
$\gamma=0,5772156649...$ ($\gamma$ est la constante d'\textsc{Euler}).
}
\end{center}
\end{enumerate}
\fincorrection
\correction{005223}
Soit $r$ la raison de la suite $u$.
Pour tout entier naturel $k$, on a 

\begin{center}
$\frac{r}{u_ku_{k+1}}=\frac{u_{k+1}-u_k}{u_ku_{k+1}}=\frac{1}{u_k}-\frac{1}{u_{k+1}}$.
\end{center}
En sommant ces égalités, on obtient~:

$$r\sum_{k=0}^{n}\frac{1}{u_ku_{k+1}}=\sum_{k=0}^{n}\left(\frac{1}{u_k}-\frac{1}{u_{k+1}}\right)=\frac{1}{u_0}-\frac{1}{u_{n+1}}=\frac{u_{n+1}-u_0}{u_0u_{n+1}}=\frac{(n+1)r}{u_0u_{n+1}}.$$
Si $r\neq0$, on obtient $\sum_{k=0}^{n}\frac{1}{u_ku_{k+1}}=\frac{(n+1)}{u_0u_{n+1}}$, et si $r=0$ (et $u_0\neq0$), $u$ est constante et le résultat est immédiat.
\fincorrection
\correction{005224}
Soit $k$ un entier naturel non nul. On sait que $\sum_{i=1}^{k}i^2=\frac{k(k+1)(2k+1)}{6}$. Déterminons alors trois réels $a$, $b$ et $c$ tels que, pour entier naturel non nul $k$, 

$$\frac{6}{k(k+1)(2k+1)}=\frac{a}{k}+\frac{b}{k+1}+\frac{c}{2k+1}\;(*).$$
Pour $k$ entier naturel non nul donné,

$$\frac{a}{k}+\frac{b}{k+1}+\frac{c}{2k+1}=\frac{a(k+1)(2k+1)+bk(2k+1)+ck(k+1)}{k(k+1)(2k+1)}=
\frac{(2a+2b+c)k^2+(3a+b+c)k+a}{k(k+1)(2k+1)}.$$
Par suite,

$$(*)\Leftarrow\left\{
\begin{array}{l}
2a+2b+c=0\\
3a+b+c=0\\
a=6
\end{array}
\right.\Leftrightarrow\left\{
\begin{array}{l}
a=6\\
b=6\\
c=-24
\end{array}
\right.,$$
et donc,

$$\forall n\in\Nn^*,\;\sum_{k=1}^{n}\frac{6}{k(k+1)(2k+1)}=6\left(\sum_{k=1}^{n}\frac{1}{k}+\sum_{k=1}^{n}\frac{1}{k+1}-4\sum_{k=1}^{n}\frac{1}{2k+1}\right).$$
Ensuite, d'après l'exercice \ref{exo:suprou3bis}, quand $n$ tend vers $+\infty$, $\sum_{k=1}^{n}\frac{1}{k}=\ln n+\gamma+o(1)$ puis 

$$\sum_{k=1}^{n}\frac{1}{k+1}=\sum_{k=2}^{n+1}\frac{1}{k}=H_{n+1}-1=-1+\ln(n+1)+\gamma+o(1)=\ln n+\ln\left(1+\frac{1}{n}\right)+\gamma-1+o(1)=\ln n+\gamma-1+o(1).$$
Enfin,
  
\begin{align*}
\sum_{k=1}^{n}\frac{1}{2k+1}&=-1+\sum_{k=1}^{2n+1}\frac{1}{k}-\sum_{k=1}^{n}\frac{1}{2k}=-1+H_{2n+1}-\frac{1}{2}H_n\\
 &=\ln(2n+1)+\gamma-\frac{1}{2}(\ln n+\gamma)-1+o(1)=\ln2+\ln n+\ln\left(1+\frac{1}{2n}\right)+\gamma-\frac{1}{2}\ln n-\frac{1}{2}\gamma-1+o(1)\\
 &=\frac{1}{2}\ln n+\ln2+\frac{1}{2}\gamma-1+o(1)
\end{align*}
Finalement, quand $n$ tend vers $+\infty$, on a

$$\sum_{k=1}^{n}\frac{1}{1^2+2^2+...+k^2}=6\left(\ln n+\gamma+\ln n+\gamma-1-4\left(\frac{1}{2}\ln n+\ln2+\frac{1}{2}\gamma-1\right)\right)=6(3-4\ln2)+o(1).$$
Donc,

\begin{center}
\shadowbox{
$\lim_{n\rightarrow +\infty}\sum_{k=1}^{n}\frac{1}{1^2+2^2+...+k^2}=6(3-4\ln2)$.
}
\end{center}
\fincorrection
\correction{005225}
Posons $\alpha=\Arccos\frac{a}{b}$. $\alpha$ existe car $0<\frac{a}{b}<1$ et est élément de $\left]0,\frac{\pi}{2}\right[$. De plus, $a=b\cos\alpha$. Enfin, pour tout entier naturel $n$, $\frac{\alpha}{2^n}\in\left]0,\frac{\pi}{2}\right[$ et donc, $\cos\frac{\alpha}{2^n}>0$.
On a $u_0=b\cos\alpha$ et $v_0=b$ puis $u_1=\frac{1}{2}(u_0+v_0)=\frac{b}{2}(1+\cos\alpha)=b\cos^2\frac{\alpha}{2}$ et $v_1=\sqrt{u_1v_0}=\sqrt{b\cos^2\frac{\alpha}{2}\times b}=b\cos\frac{\alpha}{2}$ puis $u_2=\frac{b}{2}\cos\frac{\alpha}{2}(1+\cos\frac{\alpha}{2})=b\cos\frac{\alpha}{2}\cos^2\frac{\alpha}{2^2}$ et 
$v_2=\sqrt{b\cos\frac{\alpha}{2}\cos^2\frac{\alpha}{2^2}\times b\cos\frac{\alpha}{2}}=b\cos\frac{\alpha}{2}\cos\frac{\alpha}{2^2}$...
Montrons par récurrence que pour tout entier naturel non nul $n$, $v_n=b\prod_{k=1}^{n}\cos\frac{\alpha}{2^k}$ et $u_n=v_n\cos\frac{\alpha}{2^n}$.
C'est vrai pour $n=1$ et si pour $n\geq1$ donné, on a $v_n=b\prod_{k=1}^{n}\cos\frac{\alpha}{2^k}$ et $u_n=v_n\cos\frac{\alpha}{2^n}$ alors, 

$$u_{n+1}=\frac{1}{2}(v_n\cos\frac{\alpha}{2^n}+v_n)=v_n\cos^2\frac{\alpha}{2^{n+1}}$$ puis 

$$v_{n+1}=\sqrt{u_{n+1}v_n}=v_n\cos\frac{\alpha}{2^{n+1}}\;(\mbox{car}\;\cos\frac{\alpha}{2^{n+1}}>0),$$
et donc, $v_{n+1}=b\prod_{k=1}^{n+1}\cos\frac{\alpha}{2^k}$ puis $u_{n+1}=v_{n+1}\cos\frac{\alpha}{2^{n+1}}$.
On a montré par récurrence que

\begin{center}
\shadowbox{
$\forall n\in\Nn^*,\;v_n=b\prod_{k=1}^{n}\cos\frac{\alpha}{2^k}\;\mbox{et}\;u_n=v_n\cos\frac{\alpha}{2^n}.$
}
\end{center}
Pour tout entier naturel non nul $n$, on a $v_n>0$ et $\frac{v_{n+1}}{v_n}=\cos\frac{\alpha}{2^{n+1}}<1$. La suite $v$ est donc strictement décroissante. Ensuite, pour tout entier naturel non nul $n$, on a $u_n>0$ et
 
$$\frac{u_{n+1}}{u_n}=\frac{v_{n+1}}{v_n}\frac{\cos\frac{\alpha}{2^{n+1}}}{\cos\frac{\alpha}{2^n}}
=\frac{\cos^2\frac{\alpha}{2^{n+1}}}{\cos\frac{\alpha}{2^n}}
=\frac{1}{2}\left(1+\frac{1}{\cos\frac{\alpha}{2^n}}\right)>\frac{1}{2}(1+1)=1.
$$
La suite $u$ est strictement croissante. Maintenant, pour $n\in\Nn^*$,

\begin{align*}
v_n&=b\prod_{k=1}^{n}\cos\frac{\alpha}{2^k}=b\prod_{k=1}^{n}\frac{\sin\frac{\alpha}{2^{k-1}}}
{2\sin\frac{\alpha}{2^k}}\\
 &=\frac{\sin\alpha}{2^n\sin\frac{\alpha}{2^n}}
\end{align*}
Donc, quand $n$ tend vers $+\infty$, $v_n\sim\frac{\sin\alpha}{2^n\frac{\alpha}{2^n}}=\frac{\sin\alpha}{\alpha}$, puis $u_n=v_n\cos\frac{\alpha}{2^n}\sim v_n\sim\frac{\sin\alpha}{\alpha}$.
Ainsi, les suites $u$ et $v$ sont adjacentes de limite commune $b\frac{\sin\alpha}{\alpha}=\frac{\sqrt{b^2-a^2}}{\Arccos\left(\frac{a}{b}\right)}$.
\fincorrection
\correction{005226}
\begin{enumerate}
 \item  Pour $n\in\Nn^*$, $\left|\frac{\sin n}{n}\right|\leq\frac{1}{n}$. Comme $\frac{1}{n}\underset{n\rightarrow+\infty}{\longrightarrow}0$, $\frac{\sin n}{n}\underset{n\rightarrow+\infty}{\longrightarrow}0$.

\begin{center}
\shadowbox{
$\lim_{n\rightarrow +\infty}\frac{\sin n}{n}=0$.
}
\end{center}
 \item  Quand $n$ tend vers $+\infty$, $\ln\left(\left(1+\frac{1}{n}\right)^n\right)=n\ln\left(1+\frac{1}{n}\right)\sim n\times \frac{1}{n}=1$. Donc, $\ln\left(\left(1+\frac{1}{n}\right)^n\right)$ tend vers $1$ puis, $\left(1+\frac{1}{n}\right)^n=e^{n\ln(1+1/n)}$ tend vers $e^1=e$.

\begin{center}
\shadowbox{
$\lim_{n\rightarrow +\infty}\left(1+\frac{1}{n}\right)^n=e$.
}
\end{center}
 \item  Pour $n\in\Nn^*$, posons $u_n=\frac{n!}{n^n}$. Pour $n$ entier naturel non nul, on a

$$\frac{u_{n+1}}{u_n}=\frac{(n+1)!}{n!}\times\frac{n^n}{(n+1)^{n+1}}=\left(\frac{n}{n+1}\right)^n=\left(1+\frac{1}{n}\right)^{-n}.$$
Donc, quand $n$ tend vers $+\infty$, $\frac{u_{n+1}}{u_n}=e^{-n\ln(1+1/n)}=e^{-n(1/n+o(1/n))}=e^{-1+o(1)}$. Ainsi, $\frac{u_{n+1}}{u_n}$ tend vers $\frac{1}{e}=0.36...<1$. On sait alors que $\lim_{n\rightarrow +\infty}u_n=0$.

\begin{center}
\shadowbox{
$\lim_{n\rightarrow +\infty}\frac{n!}{n^n}=0$.
}
\end{center}

 \item  Pour $n\geq1$, $\frac{(n+\frac{1}{2})^2-1}{(n-\frac{1}{2})^2}\leq u_n\leq\frac{(n+\frac{1}{2})^2}{(n-\frac{1}{2})^2-1}$. Or, $\frac{(n+\frac{1}{2})^2-1}{(n-\frac{1}{2})^2}$ et $\frac{(n+\frac{1}{2})^2}{(n-\frac{1}{2})^2-1}$ tendent vers $1$ quand $n$ tend vers $+\infty$ et donc, d'après le théorème de la limite par encadrement, la suite $u$ converge et a pour limite $1$.

\begin{center}
\shadowbox{
$\lim_{n\rightarrow +\infty}\frac{E\left(\left(n+\frac{1}{2}\right)^2\right)}{E\left(\left(n-\frac{1}{2}\right)^2\right)}=1$.
}
\end{center}

 \item  Quand $n$ tend vers $+\infty$, $\sqrt[n]{n^2}=e^{\frac{1}{n}\ln(n^2)}=e^{2\ln n/n}=e^{o(1)}$, et donc $\sqrt[n]{n^2}$ tend vers $1$.

\begin{center}
\shadowbox{
$\lim_{n\rightarrow +\infty}\sqrt[n]{n^2}=1$.
}
\end{center}
 \item  $\sqrt{n+1}-\sqrt{n}=\frac{1}{\sqrt{n+1}+\sqrt{n}}\rightarrow0$.
 \item  $\frac{1}{n^3}\sum_{k=1}^{n}k^2=\frac{n(n+1)(2n+1)}{6n^3}\sim\frac{2n^3}{6n^3}=\frac{1}{6}$.
 \item  $\prod_{k=1}^{n}2^{k/2^k}=2^{\frac{1}{2}\sum_{k=1}^{n}\frac{k}{2^{k-1}}}$. Pour $x$ réel, posons $f(x)=\sum_{k=1}^{n}kx^{k-1}$. $f$ est dérivable sur $\Rr$ en tant que polynôme et pour tout réel $x$, 

$$f(x)=\left(\sum_{k=1}^{n}x^k\right)'(x)=\left(\sum_{k=0}^{n}x^k\right)'(x).$$
Pour $x\neq1$, on a donc 

$$f(x)=\left(\frac{x^{n+1}-1}{x-1}\right)'(x)=\frac{(n+1)x^n(x-1)-(x^{n+1}-1)}{(x-1)^2}=\frac{nx^{n+1}-(n+1)x^n+1}{(x-1)^2}.$$
En particulier, $\sum_{k=1}^{n}\frac{k}{2^{k-1}}=f\left(\frac{1}{2}\right)=\frac{\frac{n}{2^{n+1}}-\frac{n+1}{2^n}+1}{(\frac{1}{2}-1)^2}
\rightarrow4$ (d'après un théorème de croissances comparées). Finalement,

$$\prod_{k=1}^{n}2^{k/2^k}\rightarrow2^{4/2}=4.$$
\end{enumerate}
\fincorrection
\correction{005227}
Soit $n\in\Nn$.

\begin{align*}
\frac{1}{2\sqrt{n+u_n}}=\sqrt{n+1}-\sqrt{n}&\Leftrightarrow 2\sqrt{n+u_n}=\frac{1}{\sqrt{n+1}-\sqrt{n}}\Leftrightarrow2\sqrt{n+u_n}=\sqrt{n+1}+\sqrt{n}\\
 &4(n+u_n)=(\sqrt{n+1}+\sqrt{n})^2\Leftrightarrow u_n=-n+\frac{1}{4}(2n+1+2\sqrt{n(n+1)})\\
 &\Leftrightarrow u_n=\frac{1}{4}(-2n+1+2\sqrt{n(n+1)})
\end{align*}
Par suite, quand $n$ tend vers $+\infty$,

\begin{align*}
u_n&=-\frac{n}{2}+\frac{1}{4}+\frac{1}{2}\sqrt{n^2+n}=
\frac{1}{4}+\frac{n}{2}\left(\sqrt{1+\frac{1}{n}}-1\right)=\frac{1}{4}+\frac{n}{2}\frac{1/n}{\sqrt{1+\frac{1}{n}}+1}\\
 &=
\frac{1}{4}+\frac{1}{2}\frac{1}{\sqrt{1+\frac{1}{n}}+1}=\frac{1}{4}+\frac{1}{4}+o(1)=\frac{1}{2}+o(1).
\end{align*}
La suite $(u_n)$ converge et a pour limite $\frac{1}{2}$.

\fincorrection
\correction{005228}
\begin{enumerate}
 \item  Calcul formel de $u_n$.
Soit $x\in\Rr$. $\frac{x}{3-2x}=x\Leftrightarrow2x^2-2x=0\Leftrightarrow x=0\;\mbox{ou}\;x=1$. Pour $n$ entier naturel donné, on a alors

$$\frac{u_{n+1}-1}{u_{n+1}}=\frac{\frac{u_n}{3-2u_n}-1}{\frac{u_n}{3-2u_n}}=\frac{3u_n-3}{u_n}
=3\frac{u_n-1}{u_n}.$$
Par suite, $\frac{u_n-1}{u_n}=3^n\frac{u_0-1}{u_0}$, puis $u_n=\frac{u_0}{u_0-3^n(u_0-1)}$.
 \item   Calcul formel de $u_n$.
Soit $x\in\Rr$. $\frac{4(x-1)}{x}=x\Leftrightarrow x^2-4x+4=0\Leftrightarrow x=2$. Pour $n$ entier naturel donné, on a alors

$$\frac{1}{u_{n+1}-2}=\frac{1}{\frac{4(u_n-1)}{u_n}-2}=\frac{u_n}{2(u_n-2)}=\frac{u_n-2+2}{2(u_n-2)}=\frac{1}{2}+\frac{1}{u_n-2}.$$
Par suite, $\frac{1}{u_n-2}=\frac{n}{2}+\frac{1}{u_0-2}$, puis $u_n=2+\frac{2(u_0-2)}{(u_0-2)n+2}$.
\end{enumerate}
\fincorrection
\correction{005229}

Pour tout entier naturel $n$, on a $\left\{
\begin{array}{l}
u_{n+1}-u_n=\frac{1}{3}(v_n-u_n)\\
v_{n+1}-v_n=-\frac{1}{3}(v_n-u_n)\\
v_{n+1}-u_{n+1}=\frac{1}{3}(v_n-u_n)
\end{array}
\right.$.
La dernière relation montre que la suite $v-u$ garde un signe constant puis les deux premières relations montrent que pour tout entier naturel $n$, $\mbox{sgn}(u_{n+1}-u_n)=\mbox{sgn}(v_n-u_n)$ et 
$\mbox{sgn}(v_{n+1}-v_n)=-\mbox{sgn}(v_n-u_n)$. Les suites $u$ et $v$ sont donc monotones de sens de variation opposés.
Si par exemple $u_0\leq v_0$, alors, pour tout naturel $n$, on a~:

$$u_0\leq u_n\leq u_{n+1}\leq v_{n+1}\leq v_n\leq v_0.$$
Dans ce cas, la suite $u$ est croissante et majorée par $v_0$ et donc converge vers un certain réel $\ell$. De même, la suite $v$ est décroissante et minorée par $u_0$ et donc converge vers un certain réel $\ell'$. Enfin, puisque pour tout entier naturel $n$, on a $u_{n+1}=\frac{2u_n+v_n}{3}$, on obtient par passage à la limite quand $n$ tend vers l'infini,  $\ell=\frac{2\ell+\ell'}{3}$ et donc $\ell=\ell'$. Les suites $u$ et $v$ sont donc adjacentes. Si $u_0>v_0$, il suffit d'échanger les rôles de $u$ et $v$.
\textbf{Calcul des suites $u$ et $v$.}
Pour $n$ entier naturel donné, on a $v_{n+1}-u_{n+1}=\frac{1}{3}(v_n-u_n)$. La suite $v-u$ est géométrique de raison $\frac{1}{3}$. Pour tout naturel $n$, on a donc $v_n-u_n=\frac{1}{3^n}(v_0-u_0)$.
D'autre part, pour $n$ entier naturel donné, $v_{n+1}+u_{n+1}=v_n+u_n$. La suite $v+u$ est constante et donc, pour tout entier naturel $n$, on a $v_n+u_n=v_0+u_0$.
En additionnant et en retranchant les deux égalités précédentes, on obtient pour tout entier naturel $n$~:

$$u_n=\frac{1}{2}\left(v_0+u_0+\frac{1}{3^n}(v_0-u_0)\right)\;\mbox{et}\;v_n=\frac{1}{2}\left(v_0+u_0-\frac{1}{3^n}(v_0-u_0)\right).$$
En particulier, $\ell=\ell'=\frac{u_0+v_0}{2}$.
\fincorrection
\correction{005230}
Pour tout entier naturel $n$, on a $u_{n+1}-v_{n+1}=-\frac{1}{2}(u_n-v_n)$  et donc, $u_n-v_n=\left(-\frac{1}{2}\right)^n(u_0-v_0)$.
De même, en échangeant les rôles de $u$, $v$ et $w$, $v_n-w_n=\left(-\frac{1}{2}\right)^n(v_0-w_0)$ et $w_n-u_n=\left(-\frac{1}{2}\right)^n(w_0-v_0)$ (attention, cette dernière égalité n'est autre que la somme des deux premières et il manque encore une équation).
On a aussi, $u_{n+1}+v_{n+1}+w_{n+1}=u_n+v_n+w_n$ et donc, pour tout naturel $n$, $u_n+v_n+w_n=u_0+v_0+w_0$.
Ainsi, $u_n$, $v_n$ et $w_n$ sont solutions du système

$$\left\{
\begin{array}{l}
v_n-u_n=\left(-\frac{1}{2}\right)^n(v_0-u_0)\\
\rule{0mm}{7mm}w_n-u_n=\left(-\frac{1}{2}\right)^n(w_0-u_0)\\
\rule{0mm}{4mm}u_n+v_n+w_n=u_0+v_0+w_0
\end{array}
\right..$$
Par suite, pour tout entier naturel $n$, on a

$$\left\{
\begin{array}{l}
u_n=\frac{1}{3}\left((u_0+v_0+w_0)+\left(-\frac{1}{2}\right)^n(2u_0-v_0-w_0)\right)\\
v_n=\frac{1}{3}\left((u_0+v_0+w_0)+\left(-\frac{1}{2}\right)^n(-u_0+2v_0-w_0)\right)\\
w_n=\frac{1}{3}\left((u_0+v_0+w_0)+\left(-\frac{1}{2}\right)^n(-u_0-v_0+2w_0)\right)
\end{array}
\right..$$
Les suites $u$, $v$ et $w$ convergent vers $\frac{u_0+v_0+w_0}{3}$.
\fincorrection
\correction{005231}
Montrons tout d'abord que~:

$$\forall(x,y,z)\in]0,+\infty[^3,\;(x\leq y\leq z\Rightarrow\frac{3}{\frac{1}{x}+\frac{1}{y}+\frac{1}{z}}\leq\sqrt[3]{xyz}\leq\frac{x+y+z}{3}).$$
Posons $m=\frac{x+y+z}{3}$, $g=\sqrt[3]{xyz}$ et $h=\frac{3}{\frac{1}{x}+\frac{1}{y}+\frac{1}{z}}$.
Soient $y$ et $z$ deux réels strictement positifs tels que $y\leq z$. Pour $x\in]0,y]$, posons

\begin{center}
$u(x)=\ln m-\ln g=\ln\left(\frac{x+y+z}{3}\right)-\frac{1}{3}\left(\ln x+\ln y+\ln z\right)$.
\end{center}
$u$ est dérivable sur $]0,y]$ et pour $x\in]0,y]$, 

$$u'(x)=\frac{1}{x+y+z}-\frac{1}{3x}\leq\frac{1}{x+x+x}-\frac{1}{3x}=0.$$
$u$ est donc décroissante sur $]0,y]$ et pour $x$ dans $]0,y]$, $u(x)\geq u(y)=\ln\left(\frac{2y+z}{3}\right)-\frac{1}{3}(2\ln y+\ln z)$.
Soit $z$ un réel strictement positif fixé. Pour $y\in]0,z]$, posons $v(y)=\ln\left(\frac{2y+z}{3}\right)-\frac{1}{3}(2\ln y+\ln z)$. $v$ est dérivable sur $]0,z]$ et pour $y\in]0,z]$, 

$$v'(y)=\frac{2}{2y+z}-\frac{2}{3z}\leq\frac{2}{3z}-\frac{2}{3z}=0.$$
$v$ est donc décroissante sur $]0,z]$ et pour $y$ dans $]0,z]$, on a $v(y)\geq v(z)=0$. On vient de montrer que $g\leq m$.
En appliquant ce résultat à $\frac{1}{x}$, $\frac{1}{y}$ et $\frac{1}{z}$, on obtient $\frac{1}{g}\leq\frac{1}{h}$ et donc $h\leq g$.
Enfin, $m\leq\frac{z+z+z}{3}=z$ et $h\geq\frac{3}{\frac{1}{x}+\frac{1}{x}+\frac{1}{x}}=x$. Finalement,

\begin{center}
\shadowbox{
$x\leq h\leq g\leq m\leq z.$
}
\end{center}
Ce résultat préliminaire étant établi, puisque $0<u_0<v_0<w_0$, par récurrence, les suites $u$, $v$ et $w$ sont définies puis, pour tout naturel $n$, on a $u_n\leq v_n\leq w_n$, et de plus $u_0\leq u_n\leq u_{n+1}\leq w_{n+1}\leq w_n\leq w_0$.
La suite $u$ est croissante et majorée par $w_0$ et donc converge. La suite $w$ est décroissante et minorée par $u_0$ et donc converge. Enfin, puisque pour tout entier naturel $n$, $v_n=3w_{n+1}-u_n-w_n$, la suite $v$ converge.
Soient alors $a$, $b$ et $c$ les limites respectives des suites $u$, $v$ et $w$.
Puisque pour tout entier naturel $n$, on a $0<u_0\leq u_n\leq v_n\leq w_n$, on a déjà par passage à la limite $0<u_0\leq a\leq b\leq c$.
Toujours par passage à la limite quand $n$ tend vers $+\infty$~:

$$
\left\{
\begin{array}{l}
\frac{3}{a}=\frac{1}{a}+\frac{1}{b}+\frac{1}{c}\\
\rule{0mm}{5mm}b=\sqrt[3]{abc}\\
c=\frac{a+b+c}{3}
\end{array}
\right.
\Leftrightarrow
\left\{
\begin{array}{l}
2bc=ab+ac\\
b^2=ac\\
a+b=2c
\end{array}
\right.
\Leftrightarrow
\left\{
\begin{array}{l}
b=2c-a\\
a^2-5ac+4c^2=0\\
\end{array}
\right.
\Leftrightarrow(a=c\;\mbox{et}\;b=c)\;\mbox{ou}\;(a=4c\;\mbox{et}\;b=-2c).$$
$b=-2c$ est impossible car $b$ et $c$ sont strictement positifs et donc, $a=b=c$.
Les suites $u$, $v$ et $w$ convergent vers une limite commune.
\fincorrection
\correction{005232}
Supposons que la suite $(\sqrt[n]{v_n})$ tende vers le réel positif $\ell$.

\begin{itemize}
\item[\textbullet] Supposons que $0\leq\ell<1$. Soit $\varepsilon=\frac{1-\ell}{2}$.

$\varepsilon$ est un réel strictement positif et donc, $\exists n_0\in\Nn/\;\forall n\in\Nn,(n\geq n_0\Rightarrow\sqrt[n]{v_n}<\ell+\frac{1-\ell}{2}=\frac{1+\ell}{2})$.

Pour $n\geq n_0$, par croissance de la fonction $t\mapsto t^n$ sur $\Rr^+$, on obtient $|u_n|<\left(\frac{1+\ell}{2}\right)^n$. Or, $0<\frac{1+\ell}{2}<\frac{1+1}{2}=1$ et donc 
$\left(\frac{1+\ell}{2}\right)^n$ tend vers $0$ quand $n$ tend vers $+\infty$. Il en résulte que $u_n$ tend vers  $0$ quand $n$ tend vers $+\infty$.

\item[\textbullet] Supposons que $\ell>1$. $\exists n_0\in\Nn/\;\forall n\in\Nn,\;(n\geq n_0\Rightarrow\sqrt[n]{v_n}>\ell-\frac{\ell-1}{2}=\frac{1+\ell}{2})$. Mais alors, pour $n\geq n_0$, $|u_n|>\left(\frac{1+\ell}{2}\right)^n$. Or, $\frac{1+\ell}{2}>\frac{1+1}{2}=1$, et donc $\left(\frac{1+\ell}{2}\right)^n$ tend vers $+\infty$ quand $n$ tend vers $+\infty$. Il en résulte que $|u_n|$ tend vers $+\infty$ quand $n$ tend vers $+\infty$.
\end{itemize}
Soit, pour $\alpha$ réel et $n$ entier naturel non nul, $u_n=n^\alpha$. $\sqrt[n]{u_n}=e^{\alpha\frac{\ln n}{n}}$ tend vers $1$ quand $n$ tend vers $+\infty$, et ceci pour toute valeur de $\alpha$. Mais, si $\alpha<0$, $u_n$ tend vers $0$, si $\alpha=0$, $u_n$ tend vers $1$ et si $\alpha>0$, $u_n$ tend vers $+\infty$. Donc, si $\ell=1$, on ne peut rien conclure.
\fincorrection
\correction{005233}
\begin{enumerate}
 \item  Supposons $\ell>0$. Soit $\varepsilon$ un réel strictement positif, élément de $]0,\ell[$.
$\exists n_0\in\Nn/\;\forall n\in\Nn,\;(n\geq n_0\Rightarrow0< \ell-\frac{\varepsilon}{2}<\frac{u_{n+1}}{u_n}<\ell+\frac{\varepsilon}{2})$.
Pour $n>n_0$, puisque $u_n=\frac{u_n}{u_{n-1}}\frac{u_{n-1}}{u_{n-2}}\frac{u_{n-2}}{u_{n-3}}...\frac{u_{n_0+1}}{u_{n_0}}u_{n_0}$, on a 
$u_{n_0}\left(\ell-\frac{\varepsilon}{2}\right)^{n-n_0}\leq  u_n\leq u_{n_0}\left(\ell+\frac{\varepsilon}{2}\right)^{n-n_0}$, et donc 

$$(u_{n_0})^{1/n}\left(\ell-\frac{\varepsilon}{2}\right)^{-n_0/n}\left(\ell-\frac{\varepsilon}{2}\right)\leq \sqrt[n]{u_n}\leq(u_{n_0})^{1/n}\left(\ell+\frac{\varepsilon}{2}\right)^{-n_0/n}\left(\ell+\frac{\varepsilon}{2}\right).$$
Maintenant, le membre de gauche de cet encadrement tend vers $\ell-\frac{\varepsilon}{2}$, et le membre de droite rend vers $\ell+\frac{\varepsilon}{2}$. Par suite, on peut trouver un entier naturel $n_1\geq n_0$ tel que, pour $n\geq n_1$, $(u_{n_0})^{1/n}\left(\ell-\frac{\varepsilon}{2}\right)^{-n_0/n}\left(\ell-\frac{\varepsilon}{2}\right)>\ell-\varepsilon$, et $(u_{n_0})^{1/n}\left(\ell+\frac{\varepsilon}{2}\right)^{-n_0/n}\left(\ell+\frac{\varepsilon}{2}\right)<\ell+\varepsilon$. Pour $n\geq n_1$, on a alors $\ell-\varepsilon<\sqrt[n]{u_n}<\ell+\varepsilon$.
On a montré que $\forall\varepsilon>0,\;\exists n_1\in\Nn/\;(\forall n\in\Nn),\;(n\geq n_1\Rightarrow\ell-\varepsilon<\sqrt[n]{u_n}<\ell+\varepsilon)$. Donc, $\sqrt[n]{u_n}$ tend vers $\ell$.
On traite de façon analogue le cas $\ell=0$.
 \item  Soient $a$ et $b$ deux réels tels que $0<a<b$. Soit $u$ la suite définie par 

$$\forall p\in\Nn,\;u_{2p}=a^pb^p\;\mbox{et}\;u_{2p+1}=a^{p+1}b^p.$$
(on part de $1$ puis on multiplie alternativement par $a$ ou $b$).
Alors, $\sqrt[2p]{u_{2p}}=\sqrt{ab}$ et $\sqrt[2p+1]{u_{2p+1}}=a^{\frac{p+1}{2p+1}}b^{\frac{p}{2p+1}}\rightarrow\sqrt{ab}$. Donc, $\sqrt[n]{u_n}$ tend vers $\sqrt{ab}$ (et en particulier converge).
On a bien sûr $\frac{u_{2p+1}}{u_{2p}}=a$ et $\frac{u_{2p+2}}{u_{2p+1}}=b$. La suite $\left(\frac{u_{n+1}}{u_n}\right)$ admet donc deux suites extraites convergentes de limites distinctes et est ainsi divergente. La réciproque du 1) est donc fausse.
 \item  
  \begin{enumerate}
  \item  Pour $n$ entier naturel donné, posons $u_n=\dbinom{2n}{n}$.
 
$$\frac{u_{n+1}}{u_n}=\frac{(2n+2)!}{(2n)!}\frac{n!^2}{(n+1)!^2}=\frac{(2n+2)(2n+1)}{(n+1)^2}=\frac{4n+2}{n+1}.$$
Ainsi, $\frac{u_{n+1}}{u_n}$ tend vers $4$ quand $n$ tend vers $+\infty$, et donc $\sqrt[n]{\dbinom{2n}{n}}$ tend vers $4$ quand $n$ tend vers $+\infty$.
  \item Pour $n$ entier naturel donné, posons $u_n=\frac{n^n}{n!}$.
 
$$\frac{u_{n+1}}{u_n}=\frac{(n+1)^{n+1}}{n^n}\frac{n!}{(n+1)!}=\left(1+\frac{1}{n}\right)^n.$$
Ainsi, $\frac{u_{n+1}}{u_n}$ tend vers $e$ quand $n$ tend vers $+\infty$, et donc $\sqrt[n]{u_n}=\frac{n}{\sqrt[n]{n!}}$ tend vers $e$ quand $n$ tend vers $+\infty$.
  \item Pour $n$ entier naturel donné, posons $u_n=\frac{(3n)!}{n^{2n}n!}$.

\begin{align*}
\frac{u_{n+1}}{u_n}&=\frac{(3n+3)!}{(3n)!}\frac{n^{2n}}{(n+1)^{2n+2}}\frac{n!}{(n+1)!}
=\frac{(3n+3)(3n+2)(3n+1)}{(n+1)^2(n+1)}\left(\frac{n}{n+1}\right)^{2n}\\
 &=
\frac{3(3n+2)(3n+1)}{(n+1)^2}\left(1+\frac{1}{n}\right)^{-2n}.
\end{align*}
Maintenant, $\left(1+\frac{1}{n}\right)^{-2n}=e^{-2n\ln(1+1/n)}=e^{-2n(\frac{1}{n}+o(\frac{1}{n}))}=e^{-2+o(1)}$, et donc $\frac{u_{n+1}}{u_n}$ tend vers $27e^{-2}$. Par suite, $\frac{1}{n^2}\sqrt[n]{\frac{(3n)!}{n!}}$ tend vers $\frac{27}{e^2}$.
  \end{enumerate}
\end{enumerate}
\fincorrection
\correction{005234}
D'après le théorème de la limite par encadrement~:

$$0\leq u_nv_n\leq u_n\leq1\Rightarrow u\;\mbox{converge et tend vers}\;1.$$
Il en est de même pour $v$ en échangeant les rôles de $u$ et $v$.
\fincorrection
\correction{005235}
Si $u_n^2\rightarrow0$, alors $|u_n|=\sqrt{|u_n^2|}\rightarrow0$ et donc $u_n\rightarrow0$.
Si $u_n^2\rightarrow\ell\neq0$, alors $(u_n)=(\frac{u_n^3}{u_n^2})$ converge.
(L'exercice n'a d'intérêt que si la suite $u$ est une suite complexe, car si $u$ est une suite réelle, on écrit immédiatement $u_n=\sqrt[3]{u_n^3}$ 
(et non pas $u_n=\sqrt{u_n^2}$)).
\fincorrection
\correction{005236}
Les suites $u$ et $v$ sont définies à partir du rang $1$ et strictement positives.
Pour tout naturel non nul $n$, on a~:

$$\frac{u_{n+1}}{u_n}=\left(\frac{n+2}{n+1}\right)^{n+1}\left(\frac{n}{n+1}\right)^n=e^{(n+1)\ln(n+2)+n\ln n-(2n+1)\ln(n+1)}.$$
Pour $x$ réel strictement positif, posons alors $f(x)=(x+1)\ln(x+2)+x\ln x-(2x+1)\ln(x+1)$.
$f$ est dérivable sur $]0,+\infty[$ et pour $x>0$,

\begin{align*}
f'(x)&=\frac{x+1}{x+2}+\ln(x+2)+1+\ln x-\frac{2x+1}{x+1}-2\ln(x+1)\\
 &=\frac{x+2-1}{x+2}+\ln(x+2)+1+\ln x-\frac{2x+2-1}{x+1}-2\ln(x+1)\\
 &=-\frac{1}{x+2}+\frac{1}{x+1}+\ln x+\ln(x+2)-2\ln(x+1).
\end{align*}
De même, $f'$ est dérivable sur $]0,+\infty[$ et pour $x>0$,

\begin{align*}
f''(x)&=\frac{1}{(x+2)^2}-\frac{1}{(x+1)^2}+\frac{1}{x}+\frac{1}{x+2}-\frac{2}{x+1}\\ 
 &=\frac{x(x+1)^2-x(x+2)^2+(x+1)^2(x+2)^2+x(x+1)^2(x+2)-2x(x+1)(x+2)^2}{x(x+1)^2(x+2)^2}\\
 &=\frac{-2x^2-3x+(x^2+2x+1)(x^2+4x+4)+(x^2+2x)(x^2+2x+1)-2(x^2+x)(x^2+4x+4)}{x(x+1)^2(x+2)^2}\\
 &=\frac{3x+4}{x(x+1)^2(x+2)^2}>0.
\end{align*}
$f'$ est strictement croissante sur $]0,+\infty[$ et donc, pour $x>0$, 

$$f'(x)<\lim_{t\rightarrow +\infty}f'(t)=\lim_{t\rightarrow +\infty}\left(-\frac{1}{t+2}+\frac{1}{t+1}+\ln\frac{t(t+2)}{(t+1)^2}\right)=0.$$
Donc, $f$ est strictement décroissante sur $]0,+\infty[$. Or, pour $x>0$,

\begin{align*}
f(x)&=(x+1)\ln(x+2)+x\ln x-(2x+1)\ln(x+1)\\
 &=(x+(x+1)-(2x+1))\ln x+(x+1)\ln\left(1+\frac{2}{x}\right)-(2x+1)\ln\left(1+\frac{1}{x}\right)\\
 &=\ln\left(1+\frac{2}{x}\right)-\ln\left(1+\frac{1}{x}\right)+2\frac{\ln\left(1+\frac{2}{x}\right)}{\frac{2}{x}}-2\frac{\ln\left(1+\frac{1}{x}\right)}{\frac{1}{x}}.
\end{align*}
On sait que $\lim_{u\rightarrow 0}\frac{\ln(1+u)}{u}=1$, et donc, quand $x$ tend vers $+\infty$, $f(x)$ tend vers $0+0+2-2=0$. Comme $f$ est strictement décroissante sur $]0,+\infty[$, pour tout réel $x>0$, on a $f(x)>\lim_{t\rightarrow +\infty}f(t)=0$.
f est donc strictement positive sur $]0,+\infty[$. Ainsi, $\forall n\in\Nn^*,\;f(n)>0$ et donc $\frac{u_{n+1}}{u_n}=e^{f(n)}>1$. La suite $u$ est strictement croissante.
(Remarque. On pouvait aussi étudier directement la fonction $x\mapsto\left(1+\frac{1}{x}\right)^x$ sur $]0,+\infty[$.)
On montre de manière analogue que la suite $v$ est strictement décroissante. Enfin, puisque $u_n$ tend vers $e$, et que $v_n=\left(1+\frac{1}{n}\right)u_n$ tend vers $e$, les suites $u$ et $v$ sont adjacentes.
(Remarque. En conséquence, pour tout entier naturel non nul $n$, $\left(1+\frac{1}{n}\right)^n<e<\left(1+\frac{1}{n}\right)^{n+1}$. Par exemple, pour $n=10$, on obtient $\left(\frac{11}{10}\right)^{10}<e<\left(\frac{11}{10}\right)^{11}$ et donc, $2,59...<e<2,85...$ et pour $n=100$, on obtient $1,01^{100}<e<1,01^{101}$ et donc $2,70...<e<2,73...$ Ces deux suites convergent vers $e$ lentement).
\fincorrection
\correction{005237}
Il est immédiat que $u$ croit strictement et que $v-u$ est strictement positive et tend vers $0$.
De plus, pour $n$ entier naturel donné, 

$$v_{n+1}-v_n=\frac{1}{(n+1)!}+\frac{1}{(n+1)\times(n+1)!}-\frac{1}{n\times n!}=\frac{n(n+1)+n-(n+1)^2}{n(n+1)\times(n+1)!}=\frac{-1}{n(n+1)\times(n+1)!}< 0,$$
et la suite $v$ est strictement décroissante. Les suites $u$ et $v$ sont donc adjacentes et convergent vers une limite commune (à savoir $e$).
 

(Remarque. Dans ce cas, la convergence est très rapide. On a pour tout entier naturel non nul $n$,  $\sum_{k=0}^{n}\frac{1}{k!}<e<\sum_{k=0}^{n}\frac{1}{k!}+\frac{1}{n\times n!}$ et $n=5$ fournit par exemple $2,716...<e<2,718...$).
\fincorrection
\correction{005238}
Pour $n$ entier naturel non nul donné, on a
$$u_{n+1}-u_n=\frac{1}{\sqrt{n+1}}-2\sqrt{n+2}+2\sqrt{n+1}=\frac{1}{\sqrt{n+1}}-\frac{2}{\sqrt{n+1}+\sqrt{n+2}}>\frac{1}{\sqrt{n+1}}-\frac{2}{\sqrt{n+1}+\sqrt{n+1}}=0.$$
De même,

$$v_{n+1}-v_n=\frac{1}{\sqrt{n+1}}-2\sqrt{n+1}+2\sqrt{n}=\frac{1}{\sqrt{n+1}}-\frac{2}{\sqrt{n+1}+\sqrt{n}}<\frac{1}{\sqrt{n+1}}-\frac{2}{\sqrt{n+1}+\sqrt{n+1}}=0.$$
La suite $u$ est strictement croissante et la suite $v$ est strictement décroissante. Enfin, 

$$v_n-u_n=2\sqrt{n+1}-2\sqrt{n}=\frac{2}{\sqrt{n}+\sqrt{n+1}},$$
et la suite $v-u$ converge vers $0$. Les suites $u$ et $v$ sont ainsi adjacentes et donc convergentes, de même limite.
\fincorrection
\correction{005239}
\begin{enumerate}
 \item  L'équation caractéristique est $4z^2-4z-3=0$. Ses solutions sont $-\frac{1}{2}$ et $\frac{3}{2}$. Les suites cherchées sont les suites de la forme $(u_n)=\left(\lambda\left(-\frac{1}{2}\right)^n+\mu\left(\frac{3}{2}\right)^n\right)$ où $\lambda$ et $\mu$ sont deux réels (ou deux complexes si on cherche toutes les suites complexes). Si $u_0$ et $u_1$ sont les deux premiers termes de la suite $u$, $\lambda$ et $\mu$ sont les solutions du système $\left\{
\begin{array}{l}
\lambda+\mu=u_0\\
-\frac{\lambda}{2}+\frac{3\mu}{2}=u_1
\end{array}
\right.$ et donc $\lambda=\frac{1}{4}(3u_0-2u_1)$ et $\mu=\frac{1}{4}(u_0+2u_1)$.

\begin{center}
\shadowbox{
$\forall n\in\Nn,\;u_n=\frac{1}{4}(3u_0-2u_1)\left(-\frac{1}{2}\right)^n+\frac{1}{4}(u_0+2u_1)\left(\frac{3}{2}\right)^n.$
}
\end{center}
 \item  Clairement $u_{2n}=\frac{1}{4^n}u_0$ et $u_{2n+1}=\frac{1}{4^n}u_1$ et donc $u_n=\frac{1}{2}\left(\frac{1}{2^n}(1+(-1)^n)u_0+2\times\frac{1}{2^n}(1-(-1)^n)u_1\right)$.

\begin{center}
\shadowbox{
$\forall n\in\Nn$, $u_n=\frac{1}{2^{n+1}}\left((1+(-1)^n)u_0+2(1-(-1)^n)u_1\right)$.
}
\end{center}
 \item  Les solutions de l'équation homogène associée sont les suites de la forme $\lambda\left(-\frac{1}{2}\right)^n+\mu\left(\frac{3}{2}\right)^n$.
Une solution particulière de l'équation proposée est une constante $a$ telle que $4a=4a+3a+12$ et donc $a=-4$.
Les solutions de l'équation proposée sont donc les suites de la forme $\left(-4+\lambda\left(-\frac{1}{2}\right)^n+\mu\left(\frac{3}{2}\right)^n\right)$ où $\lambda$ et $\mu$ sont les solutions du système $\left\{
\begin{array}{l}
\lambda+\mu=4+u_0\\
-\frac{\lambda}{2}+\frac{3\mu}{2}=4+u_1
\end{array}
\right.$ et donc $\lambda=\frac{1}{4}(4+3u_0-2u_1)$ et $\mu=\frac{1}{4}(12+u_0+2u_1)$.

\begin{center}
\shadowbox{
$\forall n\in\Nn,\;u_n=-4+\frac{1}{4}(4+3u_0-2u_1)\left(-\frac{1}{2}\right)^n+\frac{1}{4}(12+u_0+2u_1)\left(\frac{3}{2}\right)^n.$
}
\end{center}

 \item  La suite $v=\frac{1}{u}$ est solution de la récurrence $2v_{n+2}=v_{n+1}-v_n$ et donc,
$(v_n)$ est de la forme $\left(\lambda\left(\frac{1+i\sqrt{7}}{4}\right)^n+\mu\left(\frac{1-i\sqrt{7}}{4}\right)^n\right)$ et donc $u_n=\frac{1}{\lambda\left(\frac{1+i\sqrt{7}}{4}\right)^n+\mu\left(\frac{1-i\sqrt{7}}{4}\right)^n}$.
 \item  Les solutions de l'équation homogène associée sont les suites de la forme $(\lambda+\mu2^n)$.
$1$ est racine simple de l'équation caractéristique et donc il existe une solution particulière de l'équation proposée de la forme $u_n=an^4+bn^3+cn^2+dn$. Pour $n\geq2$, on a

\begin{align*}
u_n-3u_{n-1}+2u_{n-2}&=(an^4+bn^3+cn^2+dn)-3(a(n-1)^4+b(n-1)^3+c(n-1)^2+d(n-1))\\
 &\;+2(a(n-2)^4+b(n-2)^3+c(n-2)^2+d(n-2))\\
 &= a(n^4-3(n-1)^4+2(n-2)^4)+b(n^3-3(n-1)^3+2(n-2)^3)\\
 &\;+c(n^2-3(n-1)^2+2(n-2)^2)+d(n-3(n-1)+2(n-2))\\
  &= a(-4n^3+30n^2-52n+29)+b(-3n^2+15n-13)+c(-2n+5)+d(-1)\\
 &=n^3(-4a)+n^2(30a-3b)+n(-52a+15b-2c)+29a-13b+5c-d.
\end{align*}

\begin{align*}
u\;\mbox{est solution}&\Leftrightarrow-4a=1\;\mbox{et}\;30a-3b=0\;\mbox{et}\;-52a+15b-2c=0\;\mbox{et}\;29a-13b+5c-d=0\\
 &\Leftrightarrow a=-\frac{1}{4},\;b=-\frac{5}{2},\;c=-\frac{49}{4},\;d=-36.
\end{align*}
Les suites cherchées sont les suites de la forme $\left(-\frac{1}{4}(n^3+10n^2+49n+144)+\lambda+\mu2^n\right)$.
 \item  Pour tout complexe $z$, $z^3-6z^2+11z-6=(z-1)(z-2)(z-3)$ et les suites solutions sont les suites de la forme 
$(\alpha+\beta2^n+\gamma3^n)$.
 \item  Pour tout complexe $z$, $z^4-2z^3+2z^2-2z+1=(z^2+1)^2-2z(z^2+1)=(z-1)^2(z^2+1)$.
Les solutions de l'équation homogène associée sont les suites de la forme $\alpha+\beta n+\gamma i^n+\delta(-i)^n$.
$1$ est racine double de l'équation caractéristique et donc l'équation proposée admet une solution particulière de la forme $u_n=an^7+bn^6+cn^5+dn^4+en^3+fn^2$. Pour tout entier naturel $n$, on a

\begin{align*}
u_{n+4}-2u_{n+3}&+2u_{n+2}-2u_{n+1}+u_n=a((n+4)^7-2(n+3)^7+2(n+2)^7-2(n+1)^7+n^7)\\
 &+b((n+4)^6-2(n+3)^6+2(n+2)^6-2(n+1)^6+n^6)\\
 &+c((n+4)^5-2(n+3)^5+2(n+2)^5-2(n+1)^5+n^5)\\
 &+d((n+4)^4-2(n+3)^4+2(n+2)^4-2(n+1)^4+n^4)\\
 &+e((n+4)^3-2(n+3)^3+2(n+2)^3-2(n+1)^3+n^3)\\
 &+f((n+4)^2-2(n+3)^2+2(n+2)^2-2(n+1)^2+n^2)\\
 &=a(84n^5+840n^4+4340n^3+12600n^2+19348n+12264)\\
 &+b(60n^4+480n^3+1860n^2+3600n+2764)\\
 &+c(40n^3+240n^2+620n+600)+d(24n^2+96n+124)+e(12n+24)+4f\\
 &=n^5(84a)+n^4(840a+60b)+n^3(4340a+480b+40c)+n^2(12600a+1860b+240c+24d)\\
 &+n(19348a+3600b+620c+96d+12e)+(12264a+2764b+600c+124d+24e+4f)
\end{align*}
$u$ est solution si et seulement si $84a=1$ et donc $a=\frac{1}{84}$, puis $840a+60b=0$ et donc $b=-\frac{1}{6}$,
puis $4340a+480b+40c=0$ et donc $c=\frac{17}{24}$, puis $12600a+1860b+240c+24d=0$ et donc $d=-\frac{5}{12}$
puis $19348a+3600b+620c+96d+12e=0$ et donc $e=-\frac{59}{24}$ puis $12264a+2764b+600c+124d+24e+4f=0$ et donc $f=\frac{1}{12}$.
La solution générale de l'équation avec second membre est donc~:

$$\forall n\in\Nn,\;u_n=\frac{1}{168}(2n^7-28n^6+119n^5-70n^4-413n^3+14n^2)+\alpha+\beta n+\gamma i^n+\delta(-i)^n,\; (\alpha,\beta,\gamma,\delta)\in\Cc^4.$$
\end{enumerate}
\fincorrection
\correction{005240}
Tout d'abord , on montre facilement par récurrence que, pour tout entier naturel non nul $n$, $u_n$ existe et $u_n\geq1$.
Mais alors, pour tout entier naturel non nul $n$, $1\leq u_{n+1}=1+\frac{n}{u_n}\leq1+n$. Par suite, pour $n\geq2$, $1\leq u_n\leq n$, ce qui reste vrai pour $n=1$.

$$\forall n\in\Nn^*,\;1\leq u_n\leq n.$$
Supposons momentanément que la suite $(u_n-\sqrt{n})_{n\geq1}$ converge vers un réel $\ell$. Dans ce cas :

$$1+\frac{n}{u_n}=1+\frac{n}{\sqrt{n}+\ell+o(1)}=1+\sqrt{n}\frac{1}{1+\frac{\ell}{\sqrt{n}}+o\left(\frac{1}{\sqrt{n}}\right)}= 1+\sqrt{n}\left(1-\frac{\ell}{\sqrt{n}}+o\left(\frac{1}{\sqrt{n}}\right)\right)=\sqrt{n}+1-\ell+o(1).$$
D'autre part,

$$u_{n+1}=\sqrt{n+1}+\ell+o(1)=\sqrt{n}\left(1+\frac{1}{n}\right)^{1/2}+\ell+o(1)=\sqrt{n}+\ell+o(1),$$
et donc $\ell-(1-\ell)=o(1)$ ou encore $2\ell-1=0$. Donc, si la suite $(u_n-\sqrt{n})_{n\geq1}$ converge vers un réel $\ell$, alors $\ell=\frac{1}{2}$.
Il reste à démontrer que la suite $(u_n-\sqrt{n})_{n\geq1}$ converge.
On note que pour tout entier naturel non nul,

$$u_{n+1}-u_n=\frac{1}{u_n}(-u_n^2+u_n+n)=\frac{1}{u_n}\left(\frac{1}{2}(1+\sqrt{4n+1})-u_n\right)\left(u_n-\frac{1}{2}
(1-\sqrt{4n+1})\right).$$
Montrons par récurrence que pour $n\geq1$, $\frac{1}{2}(1+\sqrt{4n-3})\leq u_n\leq\frac{1}{2}(1+\sqrt{4n+1})$. Posons $v_n=\frac{1}{2}(1+\sqrt{4n-3})$ et $w_n=\frac{1}{2}(1+\sqrt{4n+1})$.

Si $n=1$, $v_1=1\leq u_1=1\leq\frac{1}{2}(1+\sqrt{5})=w_1$.

Soit $n\geq1$. Supposons que $v_n\leq u_n\leq w_n$. Alors, 

$$1+\frac{2n}{\sqrt{4n+1}+1}\leq u_{n+1}=1+\frac{n}{u_n}\leq1+\frac{2n}{\sqrt{4n-3}+1}.$$

Mais, pour $n\geq1$,

\begin{align*}
\mbox{sgn}(\frac{1}{2}(1+\sqrt{4n+5})-&(1+\frac{2n}{\sqrt{4n-3}+1}))=\mbox{sgn}((1+\sqrt{4n+5})(1+\sqrt{4n-3})-2(2n+1+\sqrt{4n-3}))\\
 &=\mbox{sgn}(\sqrt{4n+5}(1+\sqrt{4n-3})-(4n+1+\sqrt{4n-3}))\\
 &=\mbox{sgn}((4n+5)(1+\sqrt{4n-3})^2-(4n+1+\sqrt{4n-3})^2)\;(\mbox{par croissance de}\;x\mapsto x^2\;\mbox{sur}\;[0,+\infty[)\\
 &=\mbox{sgn}((4n+5)(4n-2+2\sqrt{4n-3})-((4n+1)^2+2(4n+1)\sqrt{4n-3}+4n-3))\\
 &=\mbox{sgn}(-8+8\sqrt{4n-3}) =\mbox{sgn}(\sqrt{4n-3}-1) =\mbox{sgn}((4n-3)-1)=\mbox{sgn}(n-1)=+
\end{align*}

Donc, $u_{n+1}\leq 1+1+\frac{2n}{\sqrt{4n-3}+1}\leq w_{n+1}$.

D'autre part,

$$1+\frac{2n}{\sqrt{4n+1}+1}=\frac{2n+1+\sqrt{4n+1}}{\sqrt{4n+1}+1}=\frac{(\sqrt{4n+1}+1)^2}{2(\sqrt{4n+1}+1)}
=\frac{1}{2}(1+\sqrt{4n+1})=v_{n+1},$$ 

et donc $v_{n+1}\leq u_{n+1}\leq w_{n+1}$.

On a montré par récurrence que 

$$\forall n\in\Nn^*,\;\frac{1}{2}(1+\sqrt{4n-3})\leq u_n\leq\frac{1}{2}(1+\sqrt{4n+1}),$$

(ce qui montre au passage que $u$ est croissante).

Donc, pour $n\geq1$,

$$\frac{1}{2}+\sqrt{n-\frac{3}{4}}-\sqrt{n}\leq u_n-\sqrt{n}\leq\frac{1}{2}+\sqrt{n+\frac{1}{4}}-\sqrt{n},$$

ou encore, pour tout $n\geq1$,
 
$$\frac{1}{2}-\frac{3}{4}\frac{1}{\sqrt{n-\frac{3}{4}}+\sqrt{n}}\leq u_n-\sqrt{n}\leq\frac{1}{2}+\frac{1}{4}\frac{1}{\sqrt{n+\frac{1}{4}}+\sqrt{n}}.$$

Maintenant, comme les deux suites 
$(\frac{1}{2}-\frac{3}{4}\frac{1}{\sqrt{n-\frac{3}{4}}+\sqrt{n}})$ et $(\frac{1}{2}+\frac{1}{4}\frac{1}{\sqrt{n+\frac{1}{4}}+\sqrt{n}})$ convergent toutes deux vers $\frac{1}{2}$, d'après le théorème de la limite par encadrements, la suite $(u_n-\sqrt{n})_{n\geq1}$ converge vers $\frac{1}{2}$.
\fincorrection
\correction{005241}
L'égalité proposée est vraie pour $n=2$ car $\cos\frac{\pi}{2^2}=\cos\frac{\pi}{4}=\frac{\sqrt{2}}{2}$.

Soit $n\geq2$. Supposons que $\cos(\frac{\pi}{2^{n}})=\frac{1}{2}\sqrt{2+\sqrt{2+...\sqrt{2}}}$ ($n-1$ radicaux).

Alors, puisque $\cos(\frac{\pi}{2^{n+1}})>0$ (car $\frac{\pi}{2^{n+1}}$ est dans $]0,\frac{\pi}{2}[$), 

$$\cos(\frac{\pi}{2^{n+1}})=\sqrt{\frac{1+\cos(\frac{\pi}{2^{n}})}{2}}=\sqrt{\frac{1}{2}(1+\frac{1}{2}\sqrt{2+\sqrt{2+...\sqrt{2}}})}=\frac{1}{2}\sqrt{2+\sqrt{2+...\sqrt{2}}},\;(n\;\mbox{radicaux}).$$

On a montré par récurrence que, pour $n\geq2$, $\cos(\frac{\pi}{2^{n}})=\frac{1}{2}\sqrt{2+\sqrt{2+...\sqrt{2}}}$ ($n-1$ radicaux).
 
Ensuite, pour $n\geq2$, 

$$\sin(\frac{\pi}{2^{n}})=\sqrt{\frac{1}{2}(1-\cos(\frac{\pi}{2^{n-1}})}=\frac{1}{2}\sqrt{2-\sqrt{2+...\sqrt{2}}}\;(n-1\;\mbox{radicaux})$$

Enfin, 

$$2^n\sqrt{2-\sqrt{2+...\sqrt{2}}}=2^n.2\sin\frac{\pi}{2^{n+1}}\sim2^{n+1}\frac{\pi}{2^{n+1}}=\pi.$$

Donc, $\lim_{n\rightarrow +\infty}2^n\sqrt{2-\sqrt{2+...\sqrt{2}}}=\pi$.
\fincorrection
\correction{005242}
\begin{enumerate}
\item  Pour $x$ réel positif, posons $f(x)=x-\ln(1+x)$ et $g(x)=(x+1)\ln(x+1)-x$.
$f$ et $g$ sont dérivables sur $[0,+\infty[$ et pour $x>0$, on a

$$f'(x)=1-\frac{1}{x+1}=\frac{x}{x+1}>0,$$

et

$$g'(x)=\ln(x+1)+1-1=\ln(x+1)>0.$$

$f$ et $g$ sont donc strictement croissantes sur $[0,+\infty[$ et en particulier, pour $x>0$, $f(x)>f(0)=0$ et de même, $g(x)>g(0)=0$. Finalement, $f$ et $g$ sont strictement positives sur $]0,+\infty[$ ou encore,

$$\forall x>0,\;\ln(1+x)<x<(1+x)\ln(1+x).$$

\item  Soit $k$ un entier naturel non nul.

D'après 1), $\ln(1+\frac{1}{k})<\frac{1}{k}<(1+\frac{1}{k})\ln(1+\frac{1}{k})$, ce qui fournit $k\ln(1+\frac{1}{k})<1<(k+1)Ln(1+\frac{1}{k})$, puis, par stricte croissance de la fonction exponentielle sur $\Rr$, 

$$\forall k\in\Nn^*,\;0<(1+\frac{1}{k})^k<e<(1+\frac{1}{k})^{k+1}.$$

En multipliant membre à membre ces encadrements, on obtient pour tout naturel non nul $n$~:

$$\prod_{k=1}^{n}(1+\frac{1}{k})^k<e^n<\prod_{k=1}^{n}(1+\frac{1}{k})^{k+1}.$$

Maintenant, 

$$\prod_{k=1}^{n}(1+\frac{1}{k})^k=\prod_{k=1}^{n}\left(\frac{k+1}{k}\right)^k=\frac{\prod_{k=2}^{n+1}k^{k-1}}{\prod_{k=1}^{n}k^k}=\frac{(n+1)^n}{n!}.$$

De même,
$$\prod_{k=1}^{n}(1+\frac{1}{k})^{k+1}=\frac{\prod_{k=2}^{n+1}k^{k}}{\prod_{k=1}^{n}k^{k+1}}=\frac{(n+1)^{n+1}}{n!}.$$

On a montré que $\forall n\in\Nn^*,\;\frac{(n+1)^n}{n!}<e^n<\frac{(n+1)^{n+1}}{n!}$ et donc 
 
$$\forall n\in\Nn^*,\;\frac{1}{e}\frac{n+1}{n}<\frac{\sqrt[n]{n!}}{n}<\frac{1}{e}\frac{n+1}{n}(n+1)^{1/n}.$$  

D'après le théorème de la limite par encadrements, comme $\frac{n+1}{n}$ tend vers 1 quand $n$ tend vers l'infini de même que $(n+1)^{1/n}=e^{\ln(n+1)/n}$, on a montré que $\frac{\sqrt[n]{n!}}{n}$ tend vers $\frac{1}{e}$ quand $n$ tend vers $+\infty$. 
\end{enumerate}
\fincorrection
\correction{005243}
Soit $x$ un irrationnel et $(\frac{p_n}{q_n})_{n\in\Nn}$ une suite de rationnels tendant vers $x$ ($p_n$ entier relatif et $q_n$ entier naturel non nul, la fraction $\frac{p_n}{q_n}$ n'étant pas nécessairement irréductible). Supposons que la suite $(q_n)_{n\in\Nn}$ ne tende pas vers $+\infty$. Donc~:

$$\exists A>0/\;(\forall n_0\in\Nn)(\exists n\geq n_0/\;q_n\geq A)$$ 

ou encore, il existe une suite extraite $(q_{\varphi}(n))_{n\in\Nn}$ de la suite $(q_n)_{n\in\Nn}$ qui est bornée.

La suite $(q_{\varphi}(n))_{n\in\Nn}$ est une suite d'entiers naturels qui est bornée, et donc cette suite ne prend qu'un nombre fini de valeurs. Mais alors, on peut extraire de la suite $(q_{\varphi}(n))_{n\in\Nn}$ et donc de la suite $(q_n)_{n\in\Nn}$ une suite $(q_{\psi(n)})_{n\in\Nn}$ qui est constante et en particulier convergente.

La suite $(p_{\psi(n)})_{n\in\Nn}=(\frac{p_{\psi(n)}}{q_{\psi(n)}})_{n\in\Nn}(q_{\psi(n)})_{n\in\Nn}$ est aussi une suite d'entiers relatifs convergente et est donc constante à partir d'un certain rang.

Ainsi, on peut extraire de la suite $(p_{\psi(n)})_{n\in\Nn}$ et donc de la suite $(p_n)_{n\in\Nn}$ une suite $(p_{\sigma(n)})_{n\in\Nn}$ constante.
La suite $((q_{\sigma(n)})_{n\in\Nn}$ est également constante car extraite de la suite constante $(q_{\psi(n)})_{n\in\Nn}$ et finalement, on a extrait de la suite $(\frac{p_n}{q_n})_{n\in\Nn}$  une sous suite $(\frac{p_{\sigma(n)}}{q_{\sigma(n)}})_{n\in\Nn}$ constante.

Mais la suite $(\frac{p_n}{q_n})_{n\in\Nn}$ tend vers $x$ et donc la suite extraite $(\frac{p_{\sigma(n)}}{q_{\sigma(n)}})_{n\in\Nn}$ tend vers $x$. Puisque $(\frac{p_{\sigma(n)}}{q_{\sigma(n)}})_{n\in\Nn}$ est constante, on a $\forall n\in\Nn,\;\frac{p_{\sigma(n)}}{q_{\sigma(n)}}=x$ et donc $x$ est rationnel. Contradiction .

Donc la suite $(q_n)_{n\in\Nn}$ tend vers $+\infty$. Enfin si $(|p_n|)_{n\in\Nn}$ ne tend pas vers $+\infty$, on peut extraire de $(p_n)_{n\in\Nn}$ une sous-suite bornée $(p_{\varphi}(n))_{n\in\Nn}$. Mais alors, la suite $(\frac{p_{\varphi(n)}}{q_{\varphi(n)}})_{n\in\Nn}$ tend vers $x=0$ contredisant l'irrationnalité de $x$. Donc, la suite $(|p_n|)_{n\in\Nn}$ tend vers $+_infty$.
\fincorrection
\correction{005244}
On pose $u_0=0$, $u_1=0$, $u_2=1$, $u_3=1$, $u_4=0$, $u_5=1$,... c'est-à-dire 

$$\forall n\in\Nn,\;u_n=\left\{
\begin{array}{l}
0\;\mbox{si}\;n\;\mbox{n'est pas premier}\\
1\;\mbox{si}\;n\;\mbox{est premier}
\end{array}
\right..$$
 
Soit $k$ un entier naturel supérieur ou égal à $2$. Pour $n\geq2$, l'entier $kn$ est composé et donc, pour $n\geq 2$, $u_{kn}=0$. En particulier, la suite $(u_{kn})_{n\in\Nn}$ converge et a pour limite $0$. Maintenant, l'ensemble des nombres premiers est infini et si $p_n$ est le $n$-ième nombre premier, la suite $(p_n)_{n\in\Nn}$ est strictement croissante. La suite $(u_{p_n})_{n\in\Nn}$ est extraite de $(u_n)_{n\in\Nn}$ et est constante égale à $1$. En particulier, la suite $(u_{p_n})_{n\in\Nn}$ tend vers $1$. Ainsi la suite $(u_n)_{n\in\Nn}$ admet au moins deux suites extraites convergentes de limites distinctes et donc la suite $(u_n)_{n\in\Nn}$ diverge bien que toutes les suites $(u_{kn})_{n\in\Nn}$ convergent vers $0$ pour $k\geq2$.
\fincorrection
\correction{005245}
Soit $f$ une application de $\Nn$ dans lui-même, injective. Montrons que $\lim_{n\rightarrow +\infty}f(n)=+\infty$.

Soient $A$ un réel puis $m=\mbox{Max}(0,1+E(A))$.

Puisque $f$ est injective, on a $\mbox{card}(f^{-1}(\{0,1,...,m\})\geq m+1$. En particulier, $f^{-1}(\{0,1,...,m\})$ est fini (éventuellement vide).

Posons $n_0=1+\left\{
\begin{array}{l}
0\;\mbox{si}\;f^{-1}(\{0,1,...,m\})=\emptyset\\
\mbox{Max}f^{-1}(\{0,1,...,m\})\;\mbox{sinon}
\end{array}
\right.$.

Par définition de $n_0$, si $n\geq n_0$, $n$ n'est pas élément de $f^{-1}(\{0,1,...,m\})$ et donc $f(n)>m>A$.

On a montré que $\forall A\in\Rr,\;\exists n_0\in\Nn/\;(\forall n\in\Nn),\;(n\geq n_0\Rightarrow f(n)>A)$ ou encore 
$\lim_{n\rightarrow +\infty}f(n)=+\infty$.

\fincorrection
\correction{005246}
Pour $n$ naturel non nul et $x$ réel positif, posons $f_n(x)=x^n+x-1$.

Pour $x\geq0$, $f_1(x)=0\Leftrightarrow x=\frac{1}{2}$ et donc $u_1=\frac{1}{2}$.

Pour $n\geq2$, $f_n$ est dérivable sur $\Rr^+$ et pour $x\geq 0$, $f_n'(x)=nx^{n-1}+1>0$.

$f_n$ est ainsi continue et strictemnt croissante sur $\Rr^+$ et donc bijective de $\Rr^+$ sur $f_n(\Rr^+)=[f(0),\lim_{x\rightarrow +\infty}f_n(x)[=[-1,+\infty[$, et en particulier,

$$\exists!x\in[0,+\infty[/\;f_n(x)=0.$$

Soit $u_n$ ce nombre. Puisque $f_n(0)=-1<0$ et que $f_n(1)=1>0$, par stricte croissance de $f_n$ sur $[0,+\infty[$, on a~:

$$\forall n\in\Nn,\;0<u_n<1.$$
 
La suite $u$ est donc bornée.

Ensuite, pour $n$ entier naturel donné et puisque $0<u_n<1$~:

$$f_{n+1}(u_n)=u_n^{n+1}+u_n-1<u_n^n+u_n-1=f_n(u_n)=0=f_{n+1}(u_{n+1}),$$

et donc $f_{n+1}(u_n)<f_{n+1}(u_{n+1})$ puis, par stricte croissance de $f_{n+1}$ sur $\Rr^+$, on obtient~:

$$\forall n\in\Nn,\;u_n<u_{n+1}.$$

La suite $u$ est bornée et strictement croissante. Donc, la suite $u$ converge vers un réel $\ell$, élément de $[0,1]$.

Si $0\leq\ell<1$, il existe un rang $n_0$ tel que pour $n\geq n_0$, on a~:~$u_n\leq\ell+\frac{1-\ell}{2}=\frac{1+\ell}{2}$. Mais alors, pour $n\geq n_0$, on a $1-u_n=u_n^n\leq(\frac{1+\ell}{2})^n$ et quand $n$ tend vers vers $+\infty$, on obtient $1-\ell\leq0$ ce qui est en contradiction avec $0\leq\ell<1$. Donc, $\ell=1$.

\fincorrection
\correction{005247}
\begin{enumerate}
\item  Posons $a=\frac{2p\pi}{q}$ où $p\in\Zz$, $q\in\Nn^*$ et $\mbox{PGCD}(p,q)=1$. Pour tout entier naturel $n$, on a

$$u_{n+q}=\cos\left((n+q)\frac{2p\pi}{q}\right)=\cos\left(n\frac{2p\pi}{q}+2p\pi\right)=\cos(na)=u_n.$$

La suite $u$ est donc $q$-périodique et de même la suite $v$ est $q$-périodique. Maintenant, une suite périodique converge si et seulement si elle est constante (en effet, soient $T$ une période strictement positive de $u$ et $\ell$ la limite de $u$. Soit $k\in\{0,...,T-1\}$. $|u_k-u_0|=|u_{k+nT}-u_{nT}|\rightarrow|\ell-\ell|=0$ quand $n$ tend vers l'infini).

Or, si $a=\frac{2p\pi}{q}$ où $p\in\Zz$, $q\in\Nn^*$, $\mbox{PGCD}(p,q)=1$ et $\frac{p}{q}\in\Zz$, alors $u_1\neq u_0$ et la suite $u$ n'est pas constante et donc diverge, et si $a\in2\pi\Zz$, la suite $u$ est constante et donc converge.

\item  (a) et b)) Pour tout entier naturel $n$, 

$$v_{n+1}=\sin((n+1)a)=\sin(na)\cos a+\cos(na)\sin a=u_n\sin a+v_n\cos a.$$

Puisque $\frac{a}{2\pi}\notin\Zz$, $\sin a\neq0$ et donc $u_n=\frac{v_{n+1}-v_n\cos a}{\sin a}$. Par suite, si $v$ converge alors $u$ converge. De même, à partir de $\cos((n+1)a)=\cos(na)\cos a-\sin(na)\sin a$, on voit que si $u$ converge alors $v$ converge. Les suites $u$ et $v$ sont donc simultanément convergentes ou divergentes.

Supposons que la suite $u$ converge, alors la suite $v$ converge. Soient $\ell$ et $\ell'$ les limites respectives de $u$ et $v$. D'après ce qui précède, $\ell$ et $\ell'$ sont solutions du système~:
 
$$\left\{
\begin{array}{l}
\ell\sin a+\ell'\cos a=\ell'\\
\ell\cos a-\ell'\sin a=\ell.
\end{array}
\right.\Leftrightarrow\left\{
\begin{array}{l}
\ell\sin a+\ell'(\cos a-1)=0\\
\ell(\cos a-1)-\ell'\sin a=0.
\end{array}
\right..$$

Le déterminant de ce système vaut $-\sin^2a-(\cos a-1)^2<0$ car $a\notin2\pi\Zz$. Ce système admet donc l'unique solution $\ell=\ell'=0$ ce qui contredit l'égalité $\ell^2+{\ell'}^2=1$. Donc, les suites $u$ et $v$ divergent.

\item 
\begin{enumerate}
\item Soit $E'=\{na+2k\pi,\;n\in\Nn,\;k\in\Zz\}$. Supposons que $E'$ est dense dans $\Rr$ et montrons que $\{u_n,\;n\in\Nn\}$ et $\{v_n,\;n\in\Nn\}$ sont dense dans $[-1,1]$.

Soient $x$ un réel de $[-1,1]$ et $b=\Arccos x$, de sorte que $b\in[0,\pi]$ et que $x=\cos b$.

Soit $\varepsilon>0$. Pour $n$ entier naturel et $k$ entier relatif donnés, on a~:

\begin{align*}\ensuremath
|u_n-x|&=|\cos(na)-\cos b|=|\cos(na+2k\pi)-\cos b|=2|\sin(\frac{na+2k\pi-b}{2})\sin(\frac{na+2k\pi+b}{2})|\\
 &\leq2\left|\frac{na+2k\pi-b}{2}\right|\;(\mbox{l'inégalité}\;|\sin x|\leq|x|\;\mbox{valable pour tout réel}\;x\;\mbox{est classique})\\
 &=|na+2k\pi-b|
\end{align*}

En résumé, $\forall k\in\Zz,\;\forall n\in\Nn,\;|u_n-x|\leq|na+2k\pi-b|$. Maintenant, si $E'$ est dense dans $\Rr$, on peut trouver $n\in\Nn$ et $k\in\Zz|$ tels que $|na+2k\pi-b|<\varepsilon$ et donc $|u_n-x|<\varepsilon$.

Finalement, $\{u_n,\;n\in\Nn\}$ est dense dans $[-1,1]$. De même, on montre que $\{v_n,\;n\in\Nn\}$ est dense dans $[-1,1]$.

Il reste donc à démontrer que $E'$ est dense dans $\Rr$.

\item Soit $E=\{na+2k\pi,\;n\in\Zz,\;k\in\Zz\}$. $E$ est un sous groupe non nul de $(\Rr,+)$ et donc est soit de la forme $\alpha\Zz$ avec $\alpha=\mbox{inf}(E\cap]0,+\infty[)>0$, soit dense dans $\Rr$ si $\mbox{inf}(E\cap]0,+\infty[)=0$.

Supposons par l'absurde que $\mbox{inf}(E\cap]0,+\infty[)>0$. Puisque $E=\alpha\Zz$ et que $2\pi$ est dans $E$, il existe un entier naturel non nul $q$ tel que $2\pi=q\alpha$, et donc tel que $\alpha=\frac{2\pi}{q}$.

Mais alors, $a$ étant aussi dans $E$, il existe un entier relatif $p$ tel que $a=p\alpha=\frac{2p\pi}{q}\in2\pi\Qq$. Ceci est exclu et donc, $E$ est dense dans $\Rr$.

\item Soit $x$ dans $[-1,1]$. D'après ce qui précède, pour $\varepsilon>0$ donné, il existe $n\in\Zz$ tel que $|\cos(na)-x|<\varepsilon$ et donc $|u_{|n|}-x|<\varepsilon$, ce qui montre que $\{u_n,\;n\in\Nn\}$ est dense dans $[-1,1]$. De même, $\{v_n,\;n\in\Nn\}$ est dense dans $[-1,1]$.
\end{enumerate}
\end{enumerate}
\fincorrection
\correction{005248}
Soit $x$ dans $[-1,1]$ et $\varepsilon>0$.

Soit $\theta=\Arcsin x$ (donc $\theta$ est élément de $[-\frac{\pi}{2},\frac{\pi}{2}]$ et $x=\sin\theta$). Pour $k$ entier naturel non nul donné, il existe un entier $n_k$ tel que $\ln(n_k)\leq \theta+2k\pi<\ln(n_k+1)$ à savoir $n_k=E(e^{\theta+2k\pi})$.

Mais,

$$0<\ln(n_k+1)-\ln(n_k)=\ln(1+\frac{1}{n_k})<\frac{1}{n_k}$$

(d'après l'inégalité classique $\ln(1+x)<x$ pour $x>0$, obtenue par exemple par l'étude de la fonction $f~:~x\mapsto\ln(1+x)-x$).
Donc,

$$0\leq\theta+2k\pi-\ln(n_k)<\ln(n_k+1)-\ln(n_k)<\frac{1}{n_k},$$

puis 

\begin{align*}\ensuremath
|\sin(\theta)-\sin(\ln(n_k))|&=2|\sin(\frac{\theta+2k\pi-\ln(n_k)}{2})\cos(\frac{\theta+2k\pi+\ln(n_k)}{2})|\\
 &\leq2\left|\frac{\theta+2k\pi-\ln(n_k)}{2}\right|=|\theta+2k\pi-\ln(n_k)|<\frac{1}{n_k}.
\end{align*}

Soit alors $\varepsilon$ un réel strictement positif.

Puisque $n_k=E(e^{\theta+2k\pi})$ tend vers $+\infty$ quand $k$ tend vers $+\infty$, on peut trouver un entier $k$ tel que  $\frac{1}{n_k}<\varepsilon$ et pour cet entier $k$, on a $|\sin\theta-\sin(\ln(n_k))|<\varepsilon$.

On a montré que $\forall x\in[-1,1],\;\forall\varepsilon>0,\;\exists n\in\Nn^*/\;|x-\sin(\ln n)|<\varepsilon$, et donc $\{\sin(\ln n),\;n\in\Nn^*\}$ est dense dans $[-1,1]$.
\fincorrection
\correction{005249}
Pour $\alpha\in]0,\pi[$, posons $f(\alpha)=\mbox{sup}_{n\in\Nn}(|\sin(n\alpha)|)$. $\{(\sin(n\alpha),\;n\in\Nn\}$ est une partie non vide et majorée (par $1$) de $\Rr$. Donc, pour tout réel $\alpha$ de $]0,\pi[$, $f(\alpha)$ existe dans $\Rr$.

Si $\alpha$ est dans $[\frac{\pi}{3},\frac{2\pi}{3}]$,

$$f(\alpha)=\mbox{sup}_{n\in\Nn}(|\sin(n\alpha)|)\geq\sin\alpha\geq\frac{\sqrt{3}}{2}=f(\frac{\pi}{3}).$$

Si $\alpha$ est dans $]0,\frac{\pi}{3}]$. Soit $n_0$ l'entier naturel tel que $(n_0-1)\alpha<\frac{\pi}{3}\leq n_0\alpha$ ($n_0$ existe car la suite $(n\alpha)_{n\in\Nn}$ est strictement croissante). Alors, 
 
$$\frac{\pi}{3}\leq n_0\alpha=(n_0-1)\alpha+\alpha<\frac{\pi}{3}+\alpha\leq\frac{\pi}{3}+\frac{\pi}{3}=\frac{2\pi}{3}.$$

Mais alors,

$$f(\alpha)=\mbox{sup}_{n\in\Nn}(|\sin(n\alpha)|)\geq|\sin(n_0\alpha)|\geq\frac{\sqrt{3}}{2}=f(\frac{\pi}{3}).$$

Si $\alpha$ est dans $[\frac{2\pi}{3},\pi[$, on note que 
$$f(\alpha)=\mbox{sup}_{n\in\Nn}(|\sin(n\alpha)|)=\mbox{sup}_{n\in\Nn}(|\sin(n(\pi-\alpha)|)=f(\pi-\alpha)\geq f(\frac{\pi}{3}),$$

car $\pi-\alpha$ est dans $]0,\frac{\pi}{3}]$.

On a montré que $\forall\alpha\in]0,\pi[,\;f(\alpha)\geq f(\frac{\pi}{3})=\frac{\sqrt{3}}{2}$. Donc, $\mbox{inf}_{\alpha\in]0,\pi[}(\mbox{sup}_{n\in\Nn}(|\sin(n\alpha)|))$ existe dans $\Rr$ et 

$$\mbox{inf}_{\alpha\in]0,\pi[}(\mbox{sup}_{n\in\Nn}(|\sin(n\alpha)|))= \mbox{Min}_{\alpha\in]0,\pi[}(\mbox{sup}_{n\in\Nn}(|\sin(n\alpha)|))=f(\frac{\pi}{3})=\frac{\sqrt{3}}{2}.$$
\fincorrection
\correction{005250}
La suite $u$ n'est pas majorée. Donc, $\forall M\in\Rr,\;\exists n\in\Nn/\;u_n> M$. En particulier, $\exists n_0\in\Nn/\;u_{n_0}\geq0$.

Soit $k=0$. Supposons avoir construit des entiers $n_0$, $n_1$,..., $n_k$ tels que $n_0<n_1<...<n_k$ et $\forall i\in\{0,...,k\},\;u_{n_i}\geq i$.

On ne peut avoir~:~$\forall n>n_k,\;u_n<k+1$ car sinon la suite $u$ est majorée par le nombre 
$\mbox{Max}\{u_0,u_1,...,u_{n_k},k+1\})$. Par suite, $\exists n_{k+1}>n_k/\;u_{n_{k+1}}\geq k+1$.

On vient de construire par récurrence une suite $(u_{n_k})_{k\in\Nn}$ extraite de la suite $u$ telle que $\forall k\in\Nn,\;u_{n_k}\geq k$ et en particulier telle que $\lim_{k\rightarrow +\infty}u_{n_k}=+\infty$.

\fincorrection
\correction{005251}
Si $u$ converge vers un réel $\ell$, alors $\ell\in[0,1]$ puis, par passage à la limite quand $n$ tend vers $+\infty$,  $\ell(1-\ell)\geq\frac{1}{4}$, et donc $(\ell-\frac{1}{2})^2\leq0$ et finalement $\ell=\frac{1}{2}$. Par suite, si $u$ converge, $\lim_{n\rightarrow +\infty}u_n=\frac{1}{2}$.

De plus, puisque la suite $u$ est à valeurs dans $]0,1[$, pour $n$ naturel donné, on a~:

$$u_n(1-u_n)=\frac{1}{4}-(\frac{1}{2}-u_n)^2\leq\frac{1}{4}<u_{n+1}(1-u_n),$$

et puisque $1-u_n>0$, on a donc $\forall n\in\Nn,\;u_n<u_{n+1}$.

$u$ est croissante et majorée. Donc $u$ converge et $\lim_{n\rightarrow +\infty}u_n=\frac{1}{2}$ (amusant).
\fincorrection


\end{document}

