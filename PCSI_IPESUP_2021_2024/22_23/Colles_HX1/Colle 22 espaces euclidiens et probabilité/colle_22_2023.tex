\documentclass[a4paper,11pt]{article}

\usepackage{inputenc}
\usepackage[T1]{fontenc}
\usepackage[frenchb]{babel}
\usepackage{fancyhdr,fancybox} % pour personnaliser les en-têtes
\usepackage{lastpage,setspace}
\usepackage{amsfonts,amssymb,amsmath,amsthm,mathrsfs}
\usepackage{relsize,exscale,bbold}
\usepackage{paralist}
\usepackage{xspace,multicol,diagbox,array}
\usepackage{xcolor}
\usepackage{variations}
\usepackage{xypic}
\usepackage{eurosym,stmaryrd}
\usepackage{graphicx}
\usepackage[np]{numprint}
\usepackage{hyperref} 
\usepackage{tikz}
\usepackage{colortbl}
\usepackage{multirow}
\usepackage{MnSymbol,wasysym}
\usepackage[top=1.5cm,bottom=1.5cm,right=1.2cm,left=1.5cm]{geometry}
\usetikzlibrary{calc, arrows, plotmarks, babel,decorations.pathreplacing}
\setstretch{1.25}
%\usepackage{lipsum} %\usepackage{enumitem} %\setlist[enumerate]{itemsep=1mm} bug avec enumerate



\newtheorem{thm}{Théorème}
\newtheorem{rmq}{Remarque}
\newtheorem{prop}{Propriété}
\newtheorem{cor}{Corollaire}
\newtheorem{lem}{Lemme}
\newtheorem{prop-def}{Propriété-définition}

\theoremstyle{definition}

\newtheorem{defi}{Définition}
\newtheorem{ex}{Exemple}
\newtheorem*{rap}{Rappel}
\newtheorem{cex}{Contre-exemple}
\newtheorem{exo}{Exercice} % \large {\fontfamily{ptm}\selectfont EXERCICE}
\newtheorem{nota}{Notation}
\newtheorem{ax}{Axiome}
\newtheorem{appl}{Application}
\newtheorem{csq}{Conséquence}
\def\di{\displaystyle}



\renewcommand{\thesection}{\Roman{section}}\renewcommand{\thesubsection}{\arabic{subsection} }\renewcommand{\thesubsubsection}{\alph{subsubsection} }


\newcommand{\bas}{~\backslash}\newcommand{\ba}{\backslash}
\newcommand{\C}{\mathbb{C}}\newcommand{\R}{\mathbb{R}}\newcommand{\K}{\mathbb{K}}\newcommand{\Q}{\mathbb{Q}}\newcommand{\Z}{\mathbb{Z}}\newcommand{\N}{\mathbb{N}}\newcommand{\V}{\overrightarrow}\newcommand{\Cs}{\mathscr{C}}\newcommand{\Ps}{\mathscr{P}}\newcommand{\Rs}{\mathscr{R}}\newcommand{\Gs}{\mathscr{G}}\newcommand{\Ds}{\mathscr{D}}\newcommand{\happy}{\huge\smiley}\newcommand{\sad}{\huge\frownie}\newcommand{\danger}{\begin{tikzpicture}[x=1.5pt,y=1.5pt,rotate=-14.2]
	\definecolor{myred}{rgb}{1,0.215686,0}
	\draw[line width=0.1pt,fill=myred] (13.074200,4.937500)--(5.085940,14.085900)..controls (5.085940,14.085900) and (4.070310,15.429700)..(3.636720,13.773400)
	..controls (3.203130,12.113300) and (0.917969,2.382810)..(0.917969,2.382810)
	..controls (0.917969,2.382810) and (0.621094,0.992188)..(2.097660,1.359380)
	..controls (3.574220,1.726560) and (12.468800,3.984380)..(12.468800,3.984380)
	..controls (12.468800,3.984380) and (13.437500,4.132810)..(13.074200,4.937500)
	--cycle;
	\draw[line width=0.1pt,fill=white] (11.078100,5.511720)--(5.406250,11.875000)..controls (5.406250,11.875000) and (4.683590,12.812500)..(4.367190,11.648400)
	..controls (4.050780,10.488300) and (2.375000,3.675780)..(2.375000,3.675780)
	..controls (2.375000,3.675780) and (2.156250,2.703130)..(3.214840,2.964840)
	..controls (4.273440,3.230470) and (10.640600,4.847660)..(10.640600,4.847660)
	..controls (10.640600,4.847660) and (11.332000,4.953130)..(11.078100,5.511720)
	--cycle;
	\fill (6.144520,8.839900)..controls (6.460940,7.558590) and (6.464840,6.457090)..(6.152340,6.378910)
	..controls (5.835930,6.300840) and (5.320300,7.277400)..(5.003900,8.554750)
	..controls (4.683590,9.835940) and (4.679690,10.941400)..(4.996090,11.019600)
	..controls (5.312490,11.097700) and (5.824210,10.121100)..(6.144520,8.839900)
	--cycle;
	\fill (7.292960,5.261780)..controls (7.382800,4.898500) and (7.128900,4.523500)..(6.730460,4.421880)
	..controls (6.328120,4.324220) and (5.929680,4.535220)..(5.835930,4.898500)
	..controls (5.746080,5.261780) and (5.999990,5.640630)..(6.402340,5.738340)
	..controls (6.804690,5.839840) and (7.203110,5.625060)..(7.292960,5.261780)
	--cycle;
	\end{tikzpicture}}\newcommand{\alors}{\Large\Rightarrow}\newcommand{\equi}{\Leftrightarrow}
\newcommand{\fonction}[5]{\begin{array}{l|rcl}
		#1: & #2 & \longrightarrow & #3 \\
		& #4 & \longmapsto & #5 \end{array}}


\definecolor{vert}{RGB}{11,160,78}
\definecolor{rouge}{RGB}{255,120,120}
\definecolor{bleu}{RGB}{15,5,107}


\pagestyle{fancy}
\lhead{Optimal Sup Spé, groupe IPESUP}\chead{Année~2022-2023}\rhead{Niveau: Première année de PCSI }\lfoot{M. Botcazou}\cfoot{\thepage}\rfoot{mail: i.botcazou@gmx.fr }\renewcommand{\headrulewidth}{0.4pt}\renewcommand{\footrulewidth}{0.4pt}

\begin{document}
	
	
	\begin{center}
		\Large \sc colle 22 = Espaces euclidiens et probabilité
	\end{center}
\raggedright


\section*{Connaître son cours:}
\begin{enumerate}
\item Citer l'identité du parallélogramme et donner une démonstration de celle-ci dans un espace préhilbertien. 
\item Soit $n\geq 1$ et soit $a_0,\dots,a_n$ des réels distincts deux à deux. Montrer que l'application $\varphi:\mathbb R_n[X]\times\mathbb R_n[X]\to\mathbb R$
définie par $\displaystyle \varphi(P,Q)=\sum_{i=0}^n P(a_i)Q(a_i)$ définit un produit scalaire sur $\mathbb R_n[X]$.
\item Montrer que l’application qui à deux matrices $A,B \in \mathcal{M}_n (\R)$ associe le réel tr$(A^T B )$ définit un produit scalaire sur $\mathcal{M}_n (\R)$.
\end{enumerate}

\section*{Exercices:} 	

	
\begin{exo}\textbf{(*)}\quad\\[0.25cm]
	Soient $x_{1}, \ldots, x_{n}>0$ tels que $x_{1}+\cdots+x_{n}=1$. Montrer que
	
	$$
	\sum_{k=1}^{n} \frac{1}{x_{k}} \geq n^{2}
	$$
	
	Préciser les cas d'égalité.
	
	
	\centering
	\rule{1\linewidth}{0.6pt}
\end{exo}

\begin{exo}\textbf{(**)}\quad\\[0.25cm]
	Donner un exemple de deux variables aléatoires $X$ et $Y$ indépendantes telles que
	 $X + Y$ et $X - Y$ ne sont pas indépendantes ?
		
	\centering
	\rule{1\linewidth}{0.6pt}
\end{exo}


	


\begin{exo}\textbf{(*)}\quad\\[0.25cm]%
	\noindent Soient $X$ et $Y$ deux variables aléatoires prenant pour valeurs $a_1 ,\dots  , a_n $ avec
	$$P(X = a_i ) = P(Y = a_i ) = p_i $$
	\noindent On suppose que les variables $X$ et $Y$ sont indépendantes.
	
	\noindent Montrer que
	$$P(X \neq Y ) = \sum_{i=1}^{n}p_i(1-p_i)$$

	\centering
\rule{1\linewidth}{0.6pt}
\end{exo}

\begin{exo}\textbf{(**)}\quad\\[0.25cm]%
	Soit $x,y,z$ trois réels tels que $2x^2+y^2+5z^2\leq 1$. Démontrer que $$(x+y+z)^2\leq\frac {17}{10}$$
	
	\centering
	\rule{1\linewidth}{0.6pt}
\end{exo}
\newpage 

\begin{exo}\textbf{(*)}\quad\\[0.25cm]%
	Pour $A,B\in\mathcal M_n(\mathbb R)$, on munit $\mathcal M_n(\mathbb R)$ du produit scalaire usuel : $ \displaystyle \langle A,B\rangle=\textrm{tr}(A^T B).$
	\begin{enumerate}
		\item Montrer que pour tous $A,B\in\mathcal S_n(\mathbb R)$, on a
		$$\big(\textrm{tr}(AB)\big)^2\leq \textrm{tr}(A^2)\textrm{tr}(B^2).$$
		\item Montrer que pour $A\in \mathcal M_n(\mathbb R)$, on a:
		
		$$\textrm{tr}(A^2) = \textrm{tr}(A^TA)  \equi A\in\mathcal S_n(\mathbb R)$$
	\end{enumerate}
	
	\centering
	\rule{1\linewidth}{0.6pt}
\end{exo}


\begin{exo}\textbf{(**)}\quad\\[0.25cm]%
	Soit $X$ et $Y$ deux variables aléatoires indépendantes suivant des lois de
	Bernoulli de paramètres $p$ et $q $.
	\begin{enumerate}
		\item Déterminer la loi de la variable
		$Z = \text{max}(X, Y )$.
		\item Deux archers tirent indépendamment sur n cibles. À chaque tir, le premier archer
		a la probabilité p de toucher, le second la probabilité q.
		\begin{enumerate}
			\item Quelle est la loi suivie par le nombre de cibles touchées au moins une fois ?
			\item Quelle est la loi suivie par le nombre de cibles épargnées ?
			
		\end{enumerate}
	\end{enumerate} 

	
	\centering
	\rule{1\linewidth}{0.6pt}
\end{exo}


	



\end{document}
