
%%%%%%%%%%%%%%%%%% PREAMBULE %%%%%%%%%%%%%%%%%%

\documentclass[11pt,a4paper]{article}

\usepackage{amsfonts,amsmath,amssymb,amsthm}
\usepackage[utf8]{inputenc}
\usepackage[T1]{fontenc}
\usepackage[francais]{babel}
\usepackage{mathptmx}
\usepackage{fancybox}
\usepackage{graphicx}
\usepackage{ifthen}

\usepackage{tikz}   

\usepackage{hyperref}
\hypersetup{colorlinks=true, linkcolor=blue, urlcolor=blue,
pdftitle={Exo7 - Exercices de mathématiques}, pdfauthor={Exo7}}

\usepackage{geometry}
\geometry{top=2cm, bottom=2cm, left=2cm, right=2cm}

%----- Ensembles : entiers, reels, complexes -----
\newcommand{\Nn}{\mathbb{N}} \newcommand{\N}{\mathbb{N}}
\newcommand{\Zz}{\mathbb{Z}} \newcommand{\Z}{\mathbb{Z}}
\newcommand{\Qq}{\mathbb{Q}} \newcommand{\Q}{\mathbb{Q}}
\newcommand{\Rr}{\mathbb{R}} \newcommand{\R}{\mathbb{R}}
\newcommand{\Cc}{\mathbb{C}} \newcommand{\C}{\mathbb{C}}
\newcommand{\Kk}{\mathbb{K}} \newcommand{\K}{\mathbb{K}}

%----- Modifications de symboles -----
\renewcommand{\epsilon}{\varepsilon}
\renewcommand{\Re}{\mathop{\mathrm{Re}}\nolimits}
\renewcommand{\Im}{\mathop{\mathrm{Im}}\nolimits}
\newcommand{\llbracket}{\left[\kern-0.15em\left[}
\newcommand{\rrbracket}{\right]\kern-0.15em\right]}
\renewcommand{\ge}{\geqslant} \renewcommand{\geq}{\geqslant}
\renewcommand{\le}{\leqslant} \renewcommand{\leq}{\leqslant}

%----- Fonctions usuelles -----
\newcommand{\ch}{\mathop{\mathrm{ch}}\nolimits}
\newcommand{\sh}{\mathop{\mathrm{sh}}\nolimits}
\renewcommand{\tanh}{\mathop{\mathrm{th}}\nolimits}
\newcommand{\cotan}{\mathop{\mathrm{cotan}}\nolimits}
\newcommand{\Arcsin}{\mathop{\mathrm{arcsin}}\nolimits}
\newcommand{\Arccos}{\mathop{\mathrm{arccos}}\nolimits}
\newcommand{\Arctan}{\mathop{\mathrm{arctan}}\nolimits}
\newcommand{\Argsh}{\mathop{\mathrm{argsh}}\nolimits}
\newcommand{\Argch}{\mathop{\mathrm{argch}}\nolimits}
\newcommand{\Argth}{\mathop{\mathrm{argth}}\nolimits}
\newcommand{\pgcd}{\mathop{\mathrm{pgcd}}\nolimits} 

%----- Structure des exercices ------

\newcommand{\exercice}[1]{\video{0}}
\newcommand{\finexercice}{}
\newcommand{\noindication}{}
\newcommand{\nocorrection}{}

\newcounter{exo}
\newcommand{\enonce}[2]{\refstepcounter{exo}\hypertarget{exo7:#1}{}\label{exo7:#1}{\bf Exercice \arabic{exo}}\ \  #2\vspace{1mm}\hrule\vspace{1mm}}

\newcommand{\finenonce}[1]{
\ifthenelse{\equal{\ref{ind7:#1}}{\ref{bidon}}\and\equal{\ref{cor7:#1}}{\ref{bidon}}}{}{\par{\footnotesize
\ifthenelse{\equal{\ref{ind7:#1}}{\ref{bidon}}}{}{\hyperlink{ind7:#1}{\texttt{Indication} $\blacktriangledown$}\qquad}
\ifthenelse{\equal{\ref{cor7:#1}}{\ref{bidon}}}{}{\hyperlink{cor7:#1}{\texttt{Correction} $\blacktriangledown$}}}}
\ifthenelse{\equal{\myvideo}{0}}{}{{\footnotesize\qquad\texttt{\href{http://www.youtube.com/watch?v=\myvideo}{Vidéo $\blacksquare$}}}}
\hfill{\scriptsize\texttt{[#1]}}\vspace{1mm}\hrule\vspace*{7mm}}

\newcommand{\indication}[1]{\hypertarget{ind7:#1}{}\label{ind7:#1}{\bf Indication pour \hyperlink{exo7:#1}{l'exercice \ref{exo7:#1} $\blacktriangle$}}\vspace{1mm}\hrule\vspace{1mm}}
\newcommand{\finindication}{\vspace{1mm}\hrule\vspace*{7mm}}
\newcommand{\correction}[1]{\hypertarget{cor7:#1}{}\label{cor7:#1}{\bf Correction de \hyperlink{exo7:#1}{l'exercice \ref{exo7:#1} $\blacktriangle$}}\vspace{1mm}\hrule\vspace{1mm}}
\newcommand{\fincorrection}{\vspace{1mm}\hrule\vspace*{7mm}}

\newcommand{\finenonces}{\newpage}
\newcommand{\finindications}{\newpage}


\newcommand{\fiche}[1]{} \newcommand{\finfiche}{}
%\newcommand{\titre}[1]{\centerline{\large \bf #1}}
\newcommand{\addcommand}[1]{}

% variable myvideo : 0 no video, otherwise youtube reference
\newcommand{\video}[1]{\def\myvideo{#1}}

%----- Presentation ------

\setlength{\parindent}{0cm}

\definecolor{myred}{rgb}{0.93,0.26,0}
\definecolor{myorange}{rgb}{0.97,0.58,0}
\definecolor{myyellow}{rgb}{1,0.86,0}

\newcommand{\LogoExoSept}[1]{  % input : echelle       %% NEW
{\usefont{U}{cmss}{bx}{n}
\begin{tikzpicture}[scale=0.1*#1,transform shape]
  \fill[color=myorange] (0,0)--(4,0)--(4,-4)--(0,-4)--cycle;
  \fill[color=myred] (0,0)--(0,3)--(-3,3)--(-3,0)--cycle;
  \fill[color=myyellow] (4,0)--(7,4)--(3,7)--(0,3)--cycle;
  \node[scale=5] at (3.5,3.5) {Exo7};
\end{tikzpicture}}
}


% titre
\newcommand{\titre}[1]{%
\vspace*{-4ex} \hfill \hspace*{1.5cm} \hypersetup{linkcolor=black, urlcolor=black} 
\href{http://exo7.emath.fr}{\LogoExoSept{3}} 
 \vspace*{-5.7ex}\newline 
\hypersetup{linkcolor=blue, urlcolor=blue}  {\Large \bf #1} \newline 
 \rule{12cm}{1mm} \vspace*{3ex}}

%----- Commandes supplementaires ------



\begin{document}

%%%%%%%%%%%%%%%%%% EXERCICES %%%%%%%%%%%%%%%%%%
\fiche{f00006, bodin, 2007/09/01} 

\titre{Arithmétique dans $\Zz$} 

\section{Divisibilité, division euclidienne}
\exercice{251, bodin, 1998/09/01}
\video{NL9J5okSGCw}
\enonce{000251}{}

Sachant que l'on a $96842=256\times 375 + 842$, d\'eterminer, sans faire
la division, le reste de la division du nombre $96842$ par chacun des nombres
$256$ et $375$.
\finenonce{000251} 


\finexercice
\exercice{257, bodin, 1998/09/01}
\video{wEt81lOn-LA}
\enonce{000257}{}

 Montrer que $\forall n\in \Nn$ :
$$n(n+1)(n+2)(n+3)  \text {  est divisible par }24,$$
$$n(n+1)(n+2)(n+3)(n+4) \text{  est divisible par }120.$$
\finenonce{000257} 


\finexercice
\exercice{267, cousquer, 2003/10/01}
\video{0ClhoFc-5jQ}
\enonce{000267}{}

 Montrer que si $n$ est un entier naturel somme de deux carr\'es d'entiers 
alors le reste de la division euclidienne de $n$ par $4$ n'est jamais \'egal \`a $3$.

\finenonce{000267} 


\finexercice
\exercice{254, bodin, 1998/09/01}
\video{I_npnj8Lb0M}
\enonce{000254}{}
 D\'emontrer que le nombre $7^n+1$ est divisible par
$8$ si $n$ est impair ; dans le cas $n$ pair, donner le reste de
sa division par $8$.

\finenonce{000254} 


\finexercice
\exercice{250, bodin, 1998/09/01}
\video{6MywGfpcPic}
\enonce{000250}{}

Trouver le reste de la division par $13$ du nombre $100^{1000}$.
\finenonce{000250} 


\finexercice
\exercice{285, gourio, 2001/09/01}
\video{goNvkUJGM88}
\enonce{000285}{}
\begin{enumerate}
\item Montrer que le reste de la division euclidienne par $8$ du carr\'{e} de tout
nombre impair est $1$.
\item Montrer de m\^{e}me que tout nombre pair v\'{e}rifie $x^{2}=0 \pmod{8} $ ou
$x^{2}=4 \pmod{8}.$
\item Soient $a,b,c$ trois entiers impairs. D\'{e}terminer le reste modulo $8$ de
$a^{2}+b^{2}+c^{2}$ et celui de $2(ab+bc+ca).$
\item En d\'{e}duire que ces deux nombres ne sont pas des carr\'{e}s puis que
$ab+bc+ca$ non plus.
\end{enumerate}

\finenonce{000285} 


\finexercice

\section{pgcd, ppcm, algorithme d'Euclide}
\exercice{290, bodin, 1998/09/01}
\video{uMmuJ-TCa6U}
\enonce{000290}{}

Calculer le pgcd des nombres suivants :
\begin{enumerate}
    \item 126, 230.
    \item 390, 720, 450.
    \item 180, 606, 750.
\end{enumerate}
\finenonce{000290} 


\finexercice
\exercice{292, bodin, 1998/09/01}
\video{haNOPZqNwMY}
\enonce{000292}{}
 D\'eterminer les couples d'entiers naturels de pgcd
18 et de somme 360. De m\^eme avec pgcd 18 et produit 6480.
\finenonce{000292} 


\finexercice
\exercice{296, bodin, 1998/09/01}
\video{uekZIfTccPg}
\enonce{000296}{}

Calculer par l'algorithme d'Euclide : $\pgcd(18480,9828)$.
En d\'eduire une \'ecriture de $84$ comme combinaison lin\'eaire de $18480$ et
$9828$.
\finenonce{000296} 


\finexercice
\exercice{303, bodin, 1998/09/01}
\video{atC2uUlIQ64}
\enonce{000303}{}
Notons $a=1\;111\;111\;111$ et $b=123\;456\;789$.
\begin{enumerate}
    \item Calculer le quotient et le reste de la division euclidienne de $a$ par $b$.
    \item Calculer $p=\, \text{pgcd}(a,b)$.
    \item D\'eterminer deux entiers relatifs $u$ et $v$ tels que $au+bv=p$.
\end{enumerate}

\finenonce{000303} 


\finexercice
\exercice{305, gourio, 2001/09/01}
\video{IXE-IGeo-ts}
\enonce{000305}{}

R\'{e}soudre dans ${\Zz}:1665x+1035y=45.$
\finenonce{000305} 



\section{Nombres premiers, nombres premiers entre eux}
\exercice{249, bodin, 1998/09/01}
\video{akuNAnpeHZM}
\enonce{000249}{}
Combien $15!$ admet-il de diviseurs ?

\finenonce{000249} 


\finexercice
\exercice{337, bodin, 1998/09/01}
\video{POYqKNGJr44}
\enonce{000337}{}
D\'emontrer que, si $a$ et $b$ sont des entiers
premiers entre eux, il en est de m\^eme des entiers $a+b$ et $ab$.

\finenonce{000337} 


\finexercice
\exercice{336, bodin, 1998/09/01}
\video{xFq39ocyF_c}
\enonce{000336}{}
 Soient $a,b$ des entiers sup\'erieurs ou \'egaux
\`a $1$. Montrer :
\begin{enumerate}
    \item $(2^a-1) | (2^{ab}-1)$ ;
    \item $2^p-1 \text{ premier}\ \   \Rightarrow \ \  p \text { premier }$ ;
    \item $\pgcd(2^a-1,2^b-1) =  2^{\pgcd(a,b)}-1$.
\end{enumerate}
\finenonce{000336} 


\finexercice\exercice{349, gourio, 2001/09/01}
\video{HuY48EkNke0}
\enonce{000349}{}
Soit $a\in \Nn  $ tel que $a^{n}+1 $ soit premier, montrer que $\exists
 k\in \Nn,n=2^{k}.$
Que penser de la conjecture : $\forall  n\in \Nn,2^{2^{n}}+1$ est
premier ?

\finenonce{000349} 


\finexercice
\exercice{339, bodin, 1998/09/01}
\video{RR5gG5ZLCGs}
\enonce{000339}{}
Soit $p$ un nombre premier.
\begin{enumerate}
    \item Montrer que $\forall i\in \Nn, 0< i < p$ on a : $$C_p^i\text{  est divisible par }p .$$
    \item Montrer par r\'ecurence que : $$\forall p \text{ premier}, \forall a\in \Nn^* ,\text{ on a } a^p-a \text{ est divisible par } p.$$
\end{enumerate}

\finenonce{000339} 


\finexercice
\exercice{341, bodin, 1998/09/01}
\video{I-7jctXUUZk}
\enonce{000341}{}
\begin{enumerate}
\item
Montrer par r\'ecurrence que $\forall n\in \Nn ,\forall k\geqslant 1$ on a :
$$2^{2^{n+k}}-1=\left( 2^{2^n}-1 \right) \times \prod_{i=0}^{k-1}(2^{2^{n+i}}+1).$$
\item
On pose $F_n=2^{2^n}+1$. Montrer que pour $m\not= n$,
 $F_n$ et $F_m$ sont premiers entre eux.
\item En d\'eduire qu'il y a une infinit\'e de nombres premiers.
\end{enumerate}


\finenonce{000341} 


\finexercice
\exercice{348, ridde, 1999/11/01}
\video{x73t94ZLjSU}
\enonce{000348}{}
Soit $X$ l'ensemble des nombres premiers de la forme $4k + 3$ avec $k \in \Nn$.
\begin{enumerate}
\item Montrer que $X$ est non vide.
\item Montrer que le produit de nombres de la forme $4k + 1$ est encore de cette forme.
\item On suppose que $X$ est fini et on l'\'ecrit alors $X = \left\{
p_1, \ldots, p_n\right\}$.\\  Soit $a = 4p_1 p_2 \ldots p_n  - 1$. Montrer par l'absurde
que $a$ admet un diviseur premier de la forme $4k + 3$.
\item Montrer que ceci est impossible et donc que $X$ est infini.
\end{enumerate}

\finenonce{000348} 


\finexercice

\finfiche

 \finenonces 



 \finindications 

\indication{000251}
Attention le reste d'une division euclidienne est plus petit que le quotient !
\finindication
\noindication
\noindication
\indication{000254}
Utiliser les modulos (ici modulo $8$),
un entier est divisible par $8$ si et seulement si
il est \'equivalent \`a $0$ modulo $8$.
Ici vous pouvez commencer par calculer $7^n \pmod{8}$.
\finindication
\indication{000250}
Il faut travailler modulo $13$, tout d'abord r\'eduire $100$ modulo $13$.
Se souvenir que si $a\equiv b \pmod{13}$ alors $a^k\equiv b^k \pmod{13}$.
Enfin calculer ce que cela donne pour les exposants $k=1,2,3,\ldots$
en essayant de trouver une r\`egle g\'en\'erale.
\finindication
\indication{000285}
\begin{enumerate}
  \item \'Ecrire $n=2p+1$.
  \item  \'Ecrire $n=2p$ et discuter selon que $p$ est pair ou impair.
  \item Utiliser la premi\`ere question.
  \item Par l'absurde supposer que cela s'\'ecrive comme un carr\'e, par exemple $a^2+b^2+c^2=n^2$
puis discuter selon que $n$ est pair ou impair.
\end{enumerate}
\finindication
\noindication
\noindication
\noindication
\noindication
\indication{000305}
Commencer par simplifier l'équation !
Ensuite trouver une solution particulière $(x_0,y_0)$
à l'aide de l'algorithme d'Euclide par exemple. Ensuite trouver
un expression pour une solution générale.
\finindication
\indication{000249}
Il ne faut surtout pas chercher \`a calculer $15!=1\times2\times3\times4\times\cdots\times15$, mais profiter du fait
qu'il est d\'ej\`a ``presque'' factoris\'e.
\finindication
\indication{000337}
Raisonner par l'absurde et utiliser le lemme de Gauss.
\finindication
\indication{000336}
Pour 1. utiliser l'\'egalit\'e 
$$x^b-1 = (x-1)(x^{b-1}+\cdots+x+1).$$

Pour 2. raisonner par contraposition et utiliser la question 1.

La question 3. est difficile ! Supposer $a\ge b$.
Commencer par 
montrer que $\pgcd(2^a-1,2^b-1) = \pgcd(2^a-2^b,2^b-1) = \pgcd(2^{a-b}-1,2^b - 1)$.
Cela vour permettra de comparer l'agorithme d'Euclide pour le calcul de $\pgcd(a,b)$ avec
l'algorithme d'Euclide pour le calcul de  $\pgcd(2^a-1,2^b-1)$.
\finindication
\indication{000349}
Raisonner par contraposition (ou par l'absurde) : supposer que $n$ n'est pas de la forme $2^k$,
alors $n$ admet un facteur irr\'eductible $p>2$.
Utiliser aussi $x^p+1 = (x+1)(1-x+x^2-x^3+\ldots+x^{p-1})$ avec $x$ bien choisi.
\finindication
\indication{000339}
\begin{enumerate}
  \item \'Ecrire $$C_p^i = \frac{p(p-1)(p-2)\ldots(p-(i+1))}{i!}$$
et utiliser le lemme de Gauss ou le lemme d'Euclide.
  \item Raisonner avec les modulos, c'est-\`a-dire prouver $a^p \equiv a \pmod{p}$.

\end{enumerate}
\finindication
\indication{000341}
\begin{enumerate}
  \item Il faut \^etre tr\`es soigneux : $n$ est fix\'e une fois pour toute, la r\'ecurrence se fait sur $k \ge 1$.
  \item Utiliser la question pr\'ec\'edente avec $m=n+k$.
  \item Par l'absurde, supposer qu'il y a seulement $N$ nombres premiers, consid\'erer
$N+1$ nombres du type $F_i$. Appliquer le ``principe du tiroir'' : \emph{si vous avez $N+1$ chaussettes rang\'ees dans $N$ tiroirs alors il existe (au moins) un tiroir contenant (plus de) deux chaussettes.}
\end{enumerate}
\finindication
\noindication


\newpage

\correction{000251}
La seule chose \`a voir est que pour une division euclidienne le reste doit \^etre plus petit que le quotient.
Donc les divisions euclidiennes s'\'ecrivent :
$96842 = 256 \times 378 + 74$ et $96842 = 258 \times 375 + 92$.
\fincorrection
\correction{000257}
Il suffit de constater que pour $4$ nombres cons\'ecutifs il y a
n\'ecessairement : un multiple de $2$, un multiple de $3$, un
multiple de $4$ (distinct du mutliple de $2$). Donc le produit de $4$ nombres
cons\'ecutifs est divisible par $2\times 3\times 4 = 24$.
\fincorrection
\correction{000267}
Ecrire $n=p^2+q^2$ et \'etudier le reste de la division euclidienne de
$n$ par $4$ en distinguant les diff\'erents cas de parit\'e de $p$ et $q$. 
\fincorrection
\correction{000254}
Raisonnons modulo $8$ :
$$7 \equiv -1 \pmod{8}.$$
Donc
$$7^n +1 \equiv (-1)^n + 1 \pmod{8}.$$

Le reste de la division euclidienne de $7^n+1$ par $8$ est donc
$(-1)^n+1$ donc Si $n$ est impair alors $7^n+1$ est divisible par
$8$. Et si $n$ est pair $7^n+1$ n'est pas divisible par $8$.
\fincorrection
\correction{000250}
Il sagit de calculer $100^{1000}$ modulo $13$.
Tout d'abord  $100 \equiv 9 \pmod{13}$ donc  $100^{1000} \equiv 9^{1000} \pmod{13}$.
Or  $9^{2} \equiv 81 \equiv 3 \pmod{13}$, $9^{3} \equiv 9^2.9 \equiv 3.9 \equiv 1 \pmod{13}$, Or 
 $9^{4} \equiv 9^3.9 \equiv 9 \pmod{13}$, $9^{5} \equiv 9^4.9 \equiv 9.9 \equiv 3 \pmod{13}$. 
Donc 
$100^{1000} \equiv 9^{1000} \equiv 9^{3.333+1} \equiv (9^3)^{333}.9  \equiv 1^{333}.9 \equiv 9 \pmod{13}$.
\fincorrection
\correction{000285}
\begin{enumerate}
  \item Soit $n$ un nombre impair, alors il s'\'ecrit $n=2p+1$ avec $p\in \Nn$.
Maintenant $n^2 = (2p+1)^2 = 4p^2+4p+1 = 4p(p+1) + 1$. Donc $n^2 \equiv 1 \pmod{8}$.
  \item Si $n$ est pair alors il existe $p\in \Nn$ tel que $n=2p$. Et $n^2 = 4p^2$.
Si $p$ est pair alors $p^2$ est pair et donc $n^2 = 4p^2$ est divisible par $8$, donc
$n^2 \equiv 0 \pmod{8}$. Si $p$ est impair alors $p^2$ est impair et donc $n^2 = 4p^2$ est divisible par $4$ mais pas par $8$, donc
$n^2 \equiv 4 \pmod{8}$. 
  \item Comme $a$ est impair alors d'apr\`es la premi\`ere question $a^2 \equiv 1 \pmod{8}$, et de m\^eme
$c^2 \equiv 1 \pmod{8}$, $c^2 \equiv 1 \pmod{8}$. Donc $a^2+b^2+c^2 \equiv 1+1+1 \equiv 3 \pmod{8}$. 
Pour l'autre reste, \'ecrivons $a = 2p+1$ et $b=2q+1$, $c=2r+1$, alors $2ab = 2(2p+1)(2q+1) = 8pq + 4(p+q)+2$.
Alors $2(ab+bc+ca)= 8pq+8qr+8pr + 8(p+q+r)+6$, donc $2(ab+bc+ca) \equiv 6 \pmod{8}$.
  \item Montrons par l'absurde que le nombre $a^2+b^2+c^2$ n'est pas le carr\'e d'un nombre entier.
Supposons qu'il existe $n\in \Nn$ tel que $a^2+b^2+c^2=n^2$. Nous savons que 
$a^2+b^2+c^2 \equiv 3 \pmod{8}$. Si $n$ est impair alors $n^2 \equiv 1 \pmod{8}$ et si $n$ est pair alors
$n^2 \equiv 0 \pmod{8}$ ou $n^2 \equiv 4 \pmod{8}$. Dans tous les cas $n^2$ n'est pas congru \`a $3$ modulo $8$.
Donc il y a une contradiction. La conclusion est que l'hypoth\`ese de d\'epart est fausse donc 
$a^2+b^2+c^2$ n'est pas un carr\'e.
Le m\^eme type de raisonnement est valide pour $2(ab+bc+ca)$.

Pour $ab+bc+ca$ l'argument est similaire : 
d'une part $2(ab+bc+ca)\equiv 6 \pmod{8}$
et d'autre part si, par l'absurde, on suppose $ab+bc+ca=n^2$ alors
selon la parit\'e de $n$ nous avons $2(ab+bc+ca)\equiv 2n^2 \equiv 2 \pmod{8}$
ou \`a $0 \pmod{8}$. Dans les deux cas cela aboutit \`a une contradiction. Nous avons montrer que
$ab+bc+ca$ n'est pas un carr\'e.
\end{enumerate}
\fincorrection
\correction{000290}
Il s'agit ici d'utiliser la d\'ecomposition des nombres en facteurs premiers.
\begin{enumerate}
  \item $126 = 2.3^2.7$ et $230 = 2.5.23$ donc le pgcd de $126$ et $230$ est $2$.
  \item $390 = 2.3.5.13$, $720 = 2^4.3^2.5$, $450 = 2.3^2.5^2$ et donc le pgcd de ces trois nombres est
$2.3.5=30$.
  \item $\pgcd(180,606,750) = 6$.
\end{enumerate}
\fincorrection
\correction{000292}
Soient $a,b$ deux entiers de pgcd $18$ et de somme $360$. Soit
$a',b'$ tel que $a = 18a'$ et $b=18b'$. Alors $a'$ et $b'$ sont
premiers entre eux, et leur somme est $360/18 = 20$.

Nous pouvons facilement \'enum\'erer tous les couples d'entiers
naturels $(a',b')$ ($a'\leqslant b'$) qui v\'erifient cette condition,
ce sont les couples  :
$$(1,19),(3,17),(7,13),(9,11).$$

Pour obtenir les couples $(a,b)$ recherch\'es ($a\leqslant b$), il
suffit de multiplier les couples pr\'ec\'edents par $18$ :
$$(18,342),(54,306),(126,234),(162,198).$$
\fincorrection
\correction{000296}
\begin{enumerate}
  \item $\pgcd(18480,9828) = 84$;
  \item $25 \times 18480 + (-47) \times 9828 = 84$.
\end{enumerate}
\fincorrection
\correction{000303}
\begin{enumerate}
  \item $a=9b+10$.
  \item Calculons le pgcd par l'algorithme d'Euclide. 
$a = 9b+10$, $b = 12345678 \times 10 + 9$, $10 = 1 \times 9 +1$.
Donc le pgcd vaut $1$; 
  \item Nous reprenons les \'equations pr\'ec\'edentes en partant de la fin:
$1 = 10 - 9$, puis nous rempla\c{c}ons $9$ gr\^{a}ce \`a la deuxi\`eme \'equation 
de l'algorithme d'Euclide:
$1 = 10 - (b - 12345678 \times 10) = -b + 1234679 \times 10$. Maintenant nous 
rempla\c{c}ons $10$ gr\^{a}ce \`a la premi\`ere \'equation:
$1 = -b+12345679 (a-9b) = 12345679a - 111111112b$.
\end{enumerate}
\fincorrection
\correction{000305}
En divisant par $45$ (qui est le pgcd de $1665, 1035, 45$) nous obtenons l'\'equation \'equivalente :
$$37x+23y=1 \qquad (E)$$
Comme le pgcd de $37$ et $23$ est $1$, alors d'apr\`es le th\'eor\`eme de B\'ezout
cette \'equation $(E)$ a des solutions.

L'algorithme d'Euclide pour le calcul du pgcd de $37$ et $23$ fourni
les coefficients de Bézout: $37\times 5 + 23 \times (-8) = 1$.
Une solution particuli\`ere de $(E)$ est donc 
$(x_0,y_0) = (5,-8)$.

Nous allons maintenant trouver l'expression générale pour les solutions de l'équation $(E)$.
Soient $(x,y)$ une solution de l'équation $37x+23y=1$.
Comme $(x_0,y_0)$ est aussi solution, nous avons $37x_0+23y_0=1$.
Faisons la différence de ces deux égalités pour obtenir $37(x-x_0)+23(y-y_0)=0$.
Autrement dit 
$$37(x-x_0)=-23(y-y_0) \quad (*)$$
On en déduit que $37 | 23 (y-y_0)$, or $\pgcd(23,37)=1$ donc par le lemme de Gauss,
$37 | (y -y_0)$. (C'est ici qu'il est important d'avoir divisé par $45$ dès le début !)
Cela nous permet d'écrire $y-y_0 = 37 k$ pour un $k \in \Zz$.

Repartant de l'égalité $(*)$ : nous obtenons $37(x-x_0)=-23 \times 37 \times k$.
Ce qui donne $x-x_0 = -23 k$.
Donc si $(x,y)$ est solution de $(E)$ alors elle est de la forme :
$(x,y)= (x_0 - 23k,y_0+37k)$, avec $k \in \Zz$.

Réciproquement pour chaque  $k\in\Zz$, si $(x,y)$ est de cette forme alors c'est une solution de $(E)$
(vérifiez-le !).

Conclusion : les solutions sont 
 $$\big\lbrace (5 - 23k,-8+37k) \mid k\in \Zz\big\rbrace.$$
\fincorrection
\correction{000249}
\'Ecrivons la d\'ecomposition de  $15 !=1.2.3.4\ldots15$ en facteurs premiers. $15 !  = 2^{11}.3^6.5^3.7^2 .11.13$.
Un diviseur de $15 !$ s'\'ecrit $d = 2^{\alpha}.3^\beta.5^\gamma.7^\delta .11^\epsilon.13^\eta$
avec $0 \leq \alpha \leq 11$, $0 \leq \beta \leq 6$, $0 \leq \gamma \leq 3$, $0 \leq \delta \leq 2$,
$0 \leq \epsilon \leq 1$, $0 \leq \eta \leq 1$. De plus tout nombre $d$ de cette forme est un diviseur de $15 !$.
Le nombre de diviseurs est donc $(11+1)(6+1)(3+1)(2+1)(1+1)(1+1) = 4032$.
\fincorrection
\correction{000337}
Soit $a$ et $b$ des entiers premiers entre eux. Raisonnons par
l'absurde et supposons que $ab$ et $a+b$ ne sont pas premiers
entre eux. Il existe alors $p$ un nombre premier divisant
$ab$ et $a+b$. Par le lemme d'Euclide comme $p|ab$ alors
$p|a$ ou $p|b$. Par exemple supposons que $p|a$.
Comme $p|a+b$ alors $p$ divise aussi $(a+b)-a$, donc $p|b$.
$\delta$ ne divise pas $b$ cela implique que $\delta$ et $b$ sont
premiers entre eux.

D'apr\`es le lemme de Gauss, comme $\delta$ divise $ab$ et
 $\delta$ premier avec $b$ alors $\delta$ divise $a$.
Donc $p$ est un facteur premier de $a$ et de
$b$ ce qui est absurde.
\fincorrection
\correction{000336}

\begin{enumerate}
 \item Nous savons que
$$x^b-1 = (x-1)(x^{b-1}+\cdots+x+1),$$
pour $x=2^a$ nous obtenons :
$$2^{ab}-1 = {(2^{a})}^b -1
     = (2^a-1)\left( 2^{a(b-1)}+\cdots+2^{a}+1 \right).$$
Donc $(2^a-1) | (2^{ab}-1)$.

 \item Montrons la contraposée.
Supposons que $p$ ne soit pas premier. Donc $p=ab$ avec $1<p,q<a$.
Par la question précédente $2^a-1$ divise $2^p-1$
(et $1 < 2^a-1 < 2^p-1)$. Donc $2^p - 1$ n'est pas un nombre premier.

 
 \item  Nous supposons $a \ge b$.
Nous allons montrer que faire l'algorithme d'Euclide pour le couple $(2^a-1,2^b-1)$
revient à faire l'algorithme d'Euclide pour $(a,b)$.
Tout d'abord rappellons la formule qui est à la base de l'algorithme d'Euclide :
$\pgcd(a,b) = \pgcd(a-b,b)$. 
Appliqué à $2^a-1$ et $2^b-1$ cela donne directement $\pgcd(2^a-1,2^b-1) = \pgcd(2^a-2^b,2^b-1)$.
Mais $2^a-2^b = 2^b(2^{a-b}-1)$ d'où
 $\pgcd(2^a-1,2^b-1) = \pgcd(2^b(2^{a-b}-1),2^b-1) = \pgcd(2^{a-b}-1,2^b-1)$.
La dernière égalité vient du fait $2^b$ et $2^b-1$ sont premiers entre eux (deux entiers consécutifs
sont toujours premiers entre eux).

Nous avons montrer : $\pgcd(2^a-1,2^b-1) =\pgcd(2^{a-b}-1,2^b-1)$.
Cette formule est à mettre en parallèle de $\pgcd(a,b) = \pgcd(a-b,b)$.
En itérant cette formule nous obtenons que si $a=bq+r$ alors :
 $\pgcd(2^a-1,2^b-1) = \pgcd(2^{a-bq}-1,2^b-1) = \pgcd(2^r-1,2^b-1)$
à comparer avec 
$\pgcd(a,b)=\pgcd(a-bq,b)=\pgcd(r,b)$.
Nous avons notre première étape de l'algorithme d'Euclide.
En itérant l'algorithme d'Euclide pour $(a,b)$,
nous nous arêtons au dernier reste non nul:
$\pgcd(a,b) = \pgcd(b,r) = \cdots = \pgcd(r_n,0)=r_n$.
Ce qui va donner pour nous
$\pgcd(2^a-1,2^b-1) = \pgcd(2^b-1,2^r-1) = \cdots = \pgcd(2^{r_n}-1,2^0-1) = 2^{r_n}-1$.

Bilan : $\pgcd(2^a-1,2^b-1) =  2^{\pgcd(a,b)}-1$.

\end{enumerate}

\fincorrection
\correction{000349}
\begin{enumerate}
  \item Supposons que $a^n + 1$ est premier. Nous allons montrer la contrapos\'ee. Supposons
que $n$ n'est pas de la forme $2^k$, c'est-\`a-dire que $n=p\times q$ avec
$p$ un nombre premier $>2$ et $q\in\Nn$.
Nous utilisons la formule
$$x^p+1 = (x+1)(1-x+x^2-x^3+\ldots+x^{p-1})$$
avec $x = a^q$ :
$$a^n+1 = a^{pq}+1 = (a^q)^p+1 = (a^q+1)(1-a^q+(a^q)^2+\cdots+(a^q)^{p-1}).$$
Donc $a^q+1$ divise $a^n+1$ et comme $1 < a^q+1 < a^{n}+1$ alors $a^n+1$ n'est pas premier.
Par contraposition si $a^n+1$ est premier alor $n = 2^k$.
  \item Cette conjecture est fausse, mais pas facile \`a v\'erifier
sans une bonne calculette ! En effet pour $n=5$ nous obtenons :
$$2^{2^5} + 1 = 4294967297 = 641 \times 6700417.$$
\end{enumerate}
\fincorrection
\correction{000339}
\begin{enumerate}
  \item \'Etant donn\'e $0< i < p$, nous avons 
$$C_p^i = \frac{p!}{i!(p-i)!} = \frac{p(p-1)(p-2)\ldots(p-(i+1))}{i!}$$
Comme $C_p^i$ est un entier alors $i!$ divise $ p(p-1)\ldots(p-(i+1))$.
Mais $i!$ et $p$ sont premiers entre eux (en utilisant l'hypoth\`ese $0 < i < p$).
Donc d'apr\`es le th\'eor\`eme de Gauss: $i!$ divise $(p-1)\ldots(p-(i+1))$, autrement dit
il existe $k\in\Zz$ tel que $k i! = (p-1)\ldots(p-(i+1))$. Maintenant nous avons
$C_p^i = p k$ donc $p$ divise $C_p^i$.
  \item Il s'agit de montrer le petit th\'eor\`eme de Fermat: pour $p$ premier et $a\in\Nn^*$, alors
$a^p \equiv a \pmod{p}$. Fixons $p$. Soit l'assertion
$$(\mathcal{H}_a) \ \ \ a^p \equiv a \pmod{p}.$$
Pour $a=1$ cette assertion est vraie !
\'Etant donn\'e $a \geq 1$ supposons que $\mathcal{H}_a$ soit vraie.
Alors 
$$(a+1)^p = \sum_{i=0}^p {C_p^i}a^i.$$
Mais d'apr\`es la question pr\'ec\'edente pour $0 < i < p$, $p$ divise $C_p^i$.
En termes de modulo nous obtenons:
$$ (a+1)^p \equiv C_p^0 a^0 + C_p^pa^p \equiv 1+a^p \pmod{p}.$$
Par l'hypoth\`ese de r\'ecurrence nous savons que $a^p \equiv a \pmod{p}$, donc
$$(a+1)^p \equiv a+1 \pmod{p}.$$ Nous venons de prouver que $\mathcal{H}_{a+1}$ est vraie.
Par le principe de r\'ecurrence alors quelque soit $a\in \Nn^*$ nous avons:
$$a^p \equiv a \pmod{p}.$$
\end{enumerate}
\fincorrection
\correction{000341}
\begin{enumerate}
  \item Fixons $n$ et montrons la r\'ecurrence sur $k \ge 1$.
La formule est vraie pour $k=1$.
Supposons la formule vraie au rang $k$.
Alors
\begin{align*}
(2^{2^n}-1) \times \prod_{i=0}^{k}{(2^{2^{n+i}}+1)} 
&= (2^{2^n}-1) \times \prod_{i=0}^{k-1}{(2^{2^{n+i}}+1)} \times (2^{2^{n+k}}+1) \\
&= (2^{2^{n+k}}-1)\times (2^{2^{n+k}}+1) = (2^{2^{n+k}})^2-1
= 2^{2^{n+k+1}}-1.\\
\end{align*}
Nous avons utiliser l'hypoth\`ese de r\'ecurrence dans ces \'egalit\'es.
Nous avons ainsi montrer la formule au rang $k+1$. Et donc par
le principe de r\'ecurrence elle est vraie.
  \item \'Ecrivons $m=n+k$, alors l'\'egalit\'e pr\'ec\'edente devient:
$$F_m+2 = (2^{2^n}-1) \times \prod_{i=n}^{m-1} {F_i}.$$
Soit encore :
$$F_n \times (2^{2^n}-1) \times \prod_{i=n+1}^{m-1} {F_i} \ \ \  - \ \ F_m = 2.$$
Si $d$ est un diviseur de $F_n$ et $F_m$ alors $d$ divise $2$
(ou alors on peut utiliser le th\'eor\`eme de B\'ezout). En cons\'equent $d=1$ ou $d=2$. Mais $F_n$ est impair donc $d=1$. Nous avons montrer
que tous diviseurs de $F_n$ et $F_m$ est $1$, cela signifie que
$F_n$ et $F_m$ sont premiers entre eux.
  \item Supposons qu'il y a un nombre fini de nombres premiers.
Nous les notons alors $\{p_1,\ldots,p_N\}$. Prenons alors $N+1$
nombres de la famille $F_i$, par exemple $\{F_1,\ldots,F_{N+1}\}$.
Chaque $F_i$, $i=1,\ldots,N+1$ est divisible par (au moins) un facteur 
premier $p_j$, $j=1,\ldots,N$. Nous avons $N+1$ nombres $F_i$ et seulement $N$ facteurs premiers $p_j$. Donc par le principe des tiroirs
il existe deux nombres distincts $F_k$ et $F_{k'}$ 
(avec $1 \leq k,k' \leq N+1$) qui ont un facteur premier en commun.
En cons\'equent $F_k$ et $F_{k'}$ ne sont pas premiers entre eux. Ce qui contredit la question pr\'ec\'edente. Il existe donc une infinit\'e de nombres premiers.
\end{enumerate}
\fincorrection
\correction{000348}
\begin{enumerate}
  \item $X$ est non vide car, par exemple pour $k=2$, $4k+3=11$ est premier.
  \item $(4k+1)(4\ell+1) = 16k\ell + 4(k+\ell)+1 = 4(4k\ell+k+\ell)+1$.
Si l'on note l'entier $k' = 4k\ell+k+\ell$ alors $(4k+1)(4\ell+1) = 4k'+1$, 
ce qui est bien de la forme voulue.
  \item Remarquons que $2$ est le seul nombre premier pair, les autres sont de la forme
$4k+1$ ou $4k+3$. Ici $a$ n'est pas divisible par $2$, supposons --par l'absurde-- 
que $a$ n'a pas de diviseur de la forme $4k+3$, alors 
tous les diviseurs de $a$ sont de la forme $4k+1$. C'est-\`a-dire que $a$ s'\'ecrit comme produit
de nombre de la forme $4k+1$, et par la question pr\'ec\'edente $a$ peut s'\'ecrire $a=4k'+1$.
Donc $a \equiv 1 \pmod{4}$. Mais comme $a = 4p_1p_2\ldots p_n -1$, $a \equiv -1 \equiv 3 \pmod{4}$.
Nous obtenons une contradiction. Donc $a$ admet une diviseur premier $p$ de la forme $p=4\ell+3$. 
  \item Dans l'ensemble $X = \{p_1,\ldots,p_n\}$ il y a tous les nombres premiers de la formes $4k+3$.
Le nombre $p$ est premier et s'\'ecrit $p = 4\ell+3$ donc $p$ est un \'el\'ement de $X$, donc
il existe $i\in \{1,\ldots,n\}$ tel que $p=p_i$.
Raisonnons modulo $p=p_i$: $a \equiv 0 \pmod{p}$ car $p$ divise $a$.
D'autre part $a=4p_1\ldots p_n - 1$ donc $a\equiv -1 \pmod{p}$. (car $p_i$ divise $p_1\ldots p_n$).
Nous obtenons une contradiction, donc $X$ est infini: il existe une infinit\'e de nombre 
premier de la forme $4k+3$.
Petite remarque, tous les nombres de la forme $4k+3$ ne sont pas des nombres premiers, par
exemple pour $k=3$, $4k+3 = 15$ n'est pas premier.
\end{enumerate}
\fincorrection


\end{document}

