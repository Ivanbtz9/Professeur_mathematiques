\documentclass[a4paper,11pt]{article}

\usepackage{inputenc}
\usepackage[T1]{fontenc}
\usepackage[frenchb]{babel}
\usepackage{fancyhdr,fancybox} % pour personnaliser les en-têtes
\usepackage{lastpage,setspace}
\usepackage{amsfonts,amssymb,amsmath,amsthm,mathrsfs}
\usepackage{relsize,exscale,bbold}
\usepackage{paralist}
\usepackage{xspace,multicol,diagbox,array}
\usepackage{xcolor}
\usepackage{variations}
\usepackage{xypic}
\usepackage{eurosym,stmaryrd}
\usepackage{graphicx}
\usepackage[np]{numprint}
\usepackage{hyperref} 
\usepackage{tikz}
\usepackage{colortbl}
\usepackage{multirow}
\usepackage{MnSymbol,wasysym}
\usepackage[top=1.5cm,bottom=1.5cm,right=1.2cm,left=1.5cm]{geometry}
\usetikzlibrary{calc, arrows, plotmarks, babel,decorations.pathreplacing}
\setstretch{1.25}
%\usepackage{lipsum} %\usepackage{enumitem} %\setlist[enumerate]{itemsep=1mm} bug avec enumerate



\newtheorem{thm}{Théorème}
\newtheorem{rmq}{Remarque}
\newtheorem{prop}{Propriété}
\newtheorem{cor}{Corollaire}
\newtheorem{lem}{Lemme}
\newtheorem{prop-def}{Propriété-définition}

\theoremstyle{definition}

\newtheorem{defi}{Définition}
\newtheorem{ex}{Exemple}
\newtheorem*{rap}{Rappel}
\newtheorem{cex}{Contre-exemple}
\newtheorem{exo}{Exercice} % \large {\fontfamily{ptm}\selectfont EXERCICE}
\newtheorem{nota}{Notation}
\newtheorem{ax}{Axiome}
\newtheorem{appl}{Application}
\newtheorem{csq}{Conséquence}
\def\di{\displaystyle}



\renewcommand{\thesection}{\Roman{section}}\renewcommand{\thesubsection}{\arabic{subsection} }\renewcommand{\thesubsubsection}{\alph{subsubsection} }


\newcommand{\bas}{~\backslash}\newcommand{\ba}{\backslash}
\newcommand{\C}{\mathbb{C}}\newcommand{\R}{\mathbb{R}}\newcommand{\Q}{\mathbb{Q}}\newcommand{\Z}{\mathbb{Z}}\newcommand{\N}{\mathbb{N}}\newcommand{\V}{\overrightarrow}\newcommand{\Cs}{\mathscr{C}}\newcommand{\Ps}{\mathscr{P}}\newcommand{\Rs}{\mathscr{R}}\newcommand{\Gs}{\mathscr{G}}\newcommand{\Ds}{\mathscr{D}}\newcommand{\happy}{\huge\smiley}\newcommand{\sad}{\huge\frownie}\newcommand{\danger}{\begin{tikzpicture}[x=1.5pt,y=1.5pt,rotate=-14.2]
	\definecolor{myred}{rgb}{1,0.215686,0}
	\draw[line width=0.1pt,fill=myred] (13.074200,4.937500)--(5.085940,14.085900)..controls (5.085940,14.085900) and (4.070310,15.429700)..(3.636720,13.773400)
	..controls (3.203130,12.113300) and (0.917969,2.382810)..(0.917969,2.382810)
	..controls (0.917969,2.382810) and (0.621094,0.992188)..(2.097660,1.359380)
	..controls (3.574220,1.726560) and (12.468800,3.984380)..(12.468800,3.984380)
	..controls (12.468800,3.984380) and (13.437500,4.132810)..(13.074200,4.937500)
	--cycle;
	\draw[line width=0.1pt,fill=white] (11.078100,5.511720)--(5.406250,11.875000)..controls (5.406250,11.875000) and (4.683590,12.812500)..(4.367190,11.648400)
	..controls (4.050780,10.488300) and (2.375000,3.675780)..(2.375000,3.675780)
	..controls (2.375000,3.675780) and (2.156250,2.703130)..(3.214840,2.964840)
	..controls (4.273440,3.230470) and (10.640600,4.847660)..(10.640600,4.847660)
	..controls (10.640600,4.847660) and (11.332000,4.953130)..(11.078100,5.511720)
	--cycle;
	\fill (6.144520,8.839900)..controls (6.460940,7.558590) and (6.464840,6.457090)..(6.152340,6.378910)
	..controls (5.835930,6.300840) and (5.320300,7.277400)..(5.003900,8.554750)
	..controls (4.683590,9.835940) and (4.679690,10.941400)..(4.996090,11.019600)
	..controls (5.312490,11.097700) and (5.824210,10.121100)..(6.144520,8.839900)
	--cycle;
	\fill (7.292960,5.261780)..controls (7.382800,4.898500) and (7.128900,4.523500)..(6.730460,4.421880)
	..controls (6.328120,4.324220) and (5.929680,4.535220)..(5.835930,4.898500)
	..controls (5.746080,5.261780) and (5.999990,5.640630)..(6.402340,5.738340)
	..controls (6.804690,5.839840) and (7.203110,5.625060)..(7.292960,5.261780)
	--cycle;
	\end{tikzpicture}}\newcommand{\alors}{\Large\Rightarrow}\newcommand{\equi}{\Leftrightarrow}
\newcommand{\fonction}[5]{\begin{array}{l|rcl}
		#1: & #2 & \longrightarrow & #3 \\
		& #4 & \longmapsto & #5 \end{array}}


\definecolor{vert}{RGB}{11,160,78}
\definecolor{rouge}{RGB}{255,120,120}
\definecolor{bleu}{RGB}{15,5,107}


\pagestyle{fancy}
\lhead{Optimal Sup Spé, groupe IPESUP}\chead{Année~2022-2023}\rhead{Niveau: Première année de PCSI }\lfoot{M. Botcazou}\cfoot{\thepage}\rfoot{mail: i.botcazou@gmx.fr }\renewcommand{\headrulewidth}{0.4pt}\renewcommand{\footrulewidth}{0.4pt}

\begin{document}
	%https://www.bibmath.net/ressources/index.php?action=affiche&quoi=bde/analyse/suitesseries/suitenum_prat&type=fexo
	
	
	\begin{center}
		\Large \sc colle 5 = suites numériques
	\end{center}

\section*{Connaître son cours:}
\begin{enumerate}
	\item On suppose que $(u_{2n})_n$, $(u_{2n+1})_n$ convergent vers une même limite. En déduire que $(u_{n})_n$ converge.
	\item Montrer que: si $u_n \sim_{+\infty} v_n$ alors $v_n \sim_{+\infty} u_n$.
	\item Énoncer la proposition de la moyenne de $Ces\acute{a}ro$ et donner la preuve de celle-ci dans le cas d'une suite convergente vers une limite finie.  
\end{enumerate}


	\section*{Suites numériques:}
	
		\begin{exo}%\textit{}\quad\\[0.25cm]
		\begin{prop}
			\textit{Tous les sous groupe de} $(\R, +)$ \textit{sont dense dans $\R$, ou bien de la forme} $a\Z$ \textit{pour} $a\in \R$.\\
		\end{prop}
		
		Soit $G$ une partie non vide de $\R$ stable par addition et différence (un sous-groupe de $(\R, +)$) .\hfil\\[-0.5cm] $$\forall a,b,\in G \quad a+b \in G \ \  \text{et} \ \ a-b \in G$$
		\begin{enumerate}
			\item Montrer que $0\in G$ et que pour tout $x\in G$ alors $-x\in G$.
			\item Si $G = \{0\}$, montrer que $G$ est un sous groupe $(\R, +)$ vérifiant la propriété énoncée.
			
			On suppose que $G \neq \{0\}$ et on pose $a = \text{inf } G\cap \R^{+*}$.
			\item Si $a>0$.
			\begin{enumerate}
				\item Montrer que $a$ est le plus petit élément de $G$. 
				
				(\textit{Revenir à la propriété de la borne inférieure en étudiant l'intervalle} $]a,2a[$)
				\item Montrer par double-inclusion que $G = a\Z$.
			\end{enumerate}
			\item Si $a=0$. Montrer que $G$ est dense dans $\R$.
			\item Application:
			\begin{enumerate}
				\item Montrer que l'ensemble $\Z + 2\Z$ est un sous-groupe de $\R$
				\item Montrer que $\Z + 2\Z$ n'est pas de la forme $a\R$.
				\item Conclure 
			\end{enumerate}
			
		\end{enumerate}
		\centering
		\rule{1\linewidth}{0.6pt}
	\end{exo}

	\begin{exo}\textit{}\quad\\[0.25cm]
	Considérons la suite définie, pour tout $n \in \N$, par \quad $u_n = \sqrt{n}-\lfloor\sqrt{n}\rfloor$.
	
	\begin{enumerate}
		\item Soit $p \in \N$. Montrer que $(u_n )_n$ est croissante pour $p^2 < n < (p + 1)^2 $.
		\item Exhiber des suites extraites de $(u_n)_n$ de limite $0$ et $1$.
		\item Montrer que tout réel $x \in]0, 1[$ est une valeur d'adhérence de $(u_n )_n$.
	\end{enumerate}
	\centering
	\rule{1\linewidth}{0.6pt}
	
	\end{exo}

	\begin{exo}\textit{}\quad\\[0.25cm]
	Soit $(u_n )_n$ une suite telle que $u_{n+1} - u_n \rightarrow 0$. 
	
	\noindent Alors, l'ensemble des valeurs d'adhérence est un intervalle (éventuellement vide).

	\centering
	\rule{1\linewidth}{0.6pt}
	\end{exo}
		
	
	%https://www.bibmath.net/ressources/index.php?action=affiche&quoi=bde/analyse/equadiff/eqlineairessecordre&type=fexo
	\begin{exo}\textit{}\quad\\[0.25cm]
	Posons $  u_2=1-\frac{1}{2^2}$ et pour tout entier $n\geq 3$,
	\[
	u_n=\left(1-\frac{1}{2^2}\right)\left(1-\frac{1}{3^2}\right)\cdots\left(1-\frac{1}{n^2}\right).\]
	Calculer $u_n$ et en déduire sa limite. %En d{\'e}duire que l'on a $\lim{u_n}=\dfrac{1}{2}$.
		
		
		\centering
		\rule{1\linewidth}{0.6pt}
	\end{exo}
	
	
	
	\begin{exo}\textit{}\quad\\
	Soit $u$ une suite complexe et $v$ la suite définie par $v_n=|u_n|$. On suppose que la suite $(\sqrt[n]{v_n})$ converge vers un réel positif $l$. Montrer que si $0\leq\ell<1$, la suite $(u_n)$ converge vers $0$ et si $\ell>1$, la suite $(v_n)$ tend vers $+\infty$.
	Montrer que si $\ell=1$, tout est possible.
		
		\centering
		\rule{1\linewidth}{0.6pt}
	\end{exo}
	
	\begin{exo}\quad\\
		On considère la suite $$u_n=\left(2\sin\big(\dfrac{1}{n}\big)+\dfrac{3}{4}\cos(n)\right)^n$$
		\begin{enumerate}
			\item Justifier qu'il existe $l\in \ ]0,1[$ et $N\in\N$
			tels pour tout $n\in\N$,\
			 $n\geq N \ \alors |u_n|\leq l$
			\item Quelle est la nature de la suite $u_n$ ?
		\end{enumerate}
		
		\centering
		\rule{1\linewidth}{0.6pt}
	\end{exo}
		


		
	
		
		

\end{document}