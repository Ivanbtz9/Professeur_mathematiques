\documentclass[a4paper,11pt]{article}

\usepackage{inputenc}
\usepackage[T1]{fontenc}
\usepackage[frenchb]{babel}
\usepackage{fancyhdr,fancybox} % pour personnaliser les en-têtes
\usepackage{lastpage,setspace}
\usepackage{amsfonts,amssymb,amsmath,amsthm,mathrsfs}
\usepackage{relsize,exscale,bbold}
\usepackage{paralist}
\usepackage{xspace,multicol,diagbox,array}
\usepackage{xcolor}
\usepackage{variations}
\usepackage{xypic}
\usepackage{eurosym,stmaryrd}
\usepackage{graphicx}
\usepackage[np]{numprint}
\usepackage{hyperref} 
\usepackage{tikz}
\usepackage{colortbl}
\usepackage{multirow}
\usepackage{MnSymbol,wasysym}
\usepackage[top=1.5cm,bottom=1.5cm,right=1.2cm,left=1.5cm]{geometry}
\usetikzlibrary{calc, arrows, plotmarks, babel,decorations.pathreplacing}
\setstretch{1.25}
%\usepackage{lipsum} %\usepackage{enumitem} %\setlist[enumerate]{itemsep=1mm} bug avec enumerate



\newtheorem{thm}{Théorème}
\newtheorem{rmq}{Remarque}
\newtheorem{prop}{Propriété}
\newtheorem{cor}{Corollaire}
\newtheorem{lem}{Lemme}
\newtheorem{prop-def}{Propriété-définition}

\theoremstyle{definition}

\newtheorem{defi}{Définition}
\newtheorem{ex}{Exemple}
\newtheorem*{rap}{Rappel}
\newtheorem{cex}{Contre-exemple}
\newtheorem{exo}{Exercice} % \large {\fontfamily{ptm}\selectfont EXERCICE}
\newtheorem{nota}{Notation}
\newtheorem{ax}{Axiome}
\newtheorem{appl}{Application}
\newtheorem{csq}{Conséquence}
\def\di{\displaystyle}



\renewcommand{\thesection}{\Roman{section}}\renewcommand{\thesubsection}{\arabic{subsection} }\renewcommand{\thesubsubsection}{\alph{subsubsection} }


\newcommand{\bas}{~\backslash}\newcommand{\ba}{\backslash}
\newcommand{\C}{\mathbb{C}}\newcommand{\R}{\mathbb{R}}\newcommand{\K}{\mathbb{K}}\newcommand{\Q}{\mathbb{Q}}\newcommand{\Z}{\mathbb{Z}}\newcommand{\N}{\mathbb{N}}\newcommand{\V}{\overrightarrow}\newcommand{\Cs}{\mathscr{C}}\newcommand{\Ps}{\mathscr{P}}\newcommand{\Rs}{\mathscr{R}}\newcommand{\Gs}{\mathscr{G}}\newcommand{\Ds}{\mathscr{D}}\newcommand{\happy}{\huge\smiley}\newcommand{\sad}{\huge\frownie}\newcommand{\danger}{\begin{tikzpicture}[x=1.5pt,y=1.5pt,rotate=-14.2]
	\definecolor{myred}{rgb}{1,0.215686,0}
	\draw[line width=0.1pt,fill=myred] (13.074200,4.937500)--(5.085940,14.085900)..controls (5.085940,14.085900) and (4.070310,15.429700)..(3.636720,13.773400)
	..controls (3.203130,12.113300) and (0.917969,2.382810)..(0.917969,2.382810)
	..controls (0.917969,2.382810) and (0.621094,0.992188)..(2.097660,1.359380)
	..controls (3.574220,1.726560) and (12.468800,3.984380)..(12.468800,3.984380)
	..controls (12.468800,3.984380) and (13.437500,4.132810)..(13.074200,4.937500)
	--cycle;
	\draw[line width=0.1pt,fill=white] (11.078100,5.511720)--(5.406250,11.875000)..controls (5.406250,11.875000) and (4.683590,12.812500)..(4.367190,11.648400)
	..controls (4.050780,10.488300) and (2.375000,3.675780)..(2.375000,3.675780)
	..controls (2.375000,3.675780) and (2.156250,2.703130)..(3.214840,2.964840)
	..controls (4.273440,3.230470) and (10.640600,4.847660)..(10.640600,4.847660)
	..controls (10.640600,4.847660) and (11.332000,4.953130)..(11.078100,5.511720)
	--cycle;
	\fill (6.144520,8.839900)..controls (6.460940,7.558590) and (6.464840,6.457090)..(6.152340,6.378910)
	..controls (5.835930,6.300840) and (5.320300,7.277400)..(5.003900,8.554750)
	..controls (4.683590,9.835940) and (4.679690,10.941400)..(4.996090,11.019600)
	..controls (5.312490,11.097700) and (5.824210,10.121100)..(6.144520,8.839900)
	--cycle;
	\fill (7.292960,5.261780)..controls (7.382800,4.898500) and (7.128900,4.523500)..(6.730460,4.421880)
	..controls (6.328120,4.324220) and (5.929680,4.535220)..(5.835930,4.898500)
	..controls (5.746080,5.261780) and (5.999990,5.640630)..(6.402340,5.738340)
	..controls (6.804690,5.839840) and (7.203110,5.625060)..(7.292960,5.261780)
	--cycle;
	\end{tikzpicture}}\newcommand{\alors}{\Large\Rightarrow}\newcommand{\equi}{\Leftrightarrow}
\newcommand{\fonction}[5]{\begin{array}{l|rcl}
		#1: & #2 & \longrightarrow & #3 \\
		& #4 & \longmapsto & #5 \end{array}}


\definecolor{vert}{RGB}{11,160,78}
\definecolor{rouge}{RGB}{255,120,120}
\definecolor{bleu}{RGB}{15,5,107}


\pagestyle{fancy}
\lhead{Optimal Sup Spé, groupe IPESUP}\chead{Année~2022-2023}\rhead{Niveau: Première année de PCSI }\lfoot{M. Botcazou}\cfoot{\thepage}\rfoot{mail: i.botcazou@gmx.fr }\renewcommand{\headrulewidth}{0.4pt}\renewcommand{\footrulewidth}{0.4pt}

\begin{document}
	
	
	\begin{center}
		\Large \sc colle 19 = déterminants
	\end{center}
\raggedright


\section*{Connaître son cours:}
\begin{enumerate}
	\item Soit $\beta$ une base d’un $\K $-espace vectoriel $E$ de dimension $n $. Monter  que $n$ vecteurs $x_1 ,\dots , x_n \in E$ forme une base de $E$ si, et seulement si,
	$\text{det}_\beta \ (x_1 , \dots , x_n ) \neq 0$.
	\item Soit $\Phi : M \longmapsto M^T$, calculer det($\Phi$).
	\item 	Soient $A=(a_{i,j})_{1\leqslant i,j\leqslant n}$ une matrice carrée et $B= (b_{i,j})_{1\leqslant i,j\leqslant n}$ où $b_{i,j}=(-1)^{i+j}a_{i,j}$. Calculer $\text{det}(B)$ en fonction de $\text{det}(A)$. 
\end{enumerate}

\section*{Exercices:} 

%Livre Mansuy p.751 exo 2,3 et 4 

% https://www.bibmath.net/ressources/index.php?action=affiche&quoi=bde/algebrelineaire/determinant&type=fexo	

\begin{exo}\textbf{(**)}\quad\\[0.25cm]
Soit $a, b \in \mathbb{C}$. Donner une forme factorisée au déterminant suivant.

$$
\left|\begin{array}{lll}
1 & a & b \\
a & 1 & b \\
b & a & 1
\end{array}\right|
$$
	
	\centering
	\rule{1\linewidth}{0.6pt}
\end{exo}
	
\begin{exo}\textbf{(***)}\quad\\[0.25cm]
	Soit $x\in\mathbb R$. Calculer
	$$\left|
	\begin{array}{ccccc}
	1+x^2&-x&0&\dots&0\\
	-x&1+x^2&-x&\ddots&\vdots\\
	0&\ddots&\ddots&\ddots&0\\
	\vdots&\ddots&-x&1+x^2&-x\\
	0&\dots&0&-x&1+x^2
	\end{array}
	\right|.
	$$
	\centering
	\rule{1\linewidth}{0.6pt}
\end{exo}

	


\begin{exo}\textbf{(**)}\quad\\[0.25cm]
Notons, pour tout $k \in \llbracket 1, n \rrbracket, S_{k}=\sum_{i=1}^{k} i$. Calculer le déterminant

$$
\left|\begin{array}{ccccc}
S_{1} & S_{1} & S_{1} & \cdots & S_{1} \\
S_{1} & S_{2} & S_{2} & \cdots & S_{2} \\
S_{1} & S_{2} & S_{3} & \cdots & S_{3} \\
\vdots & \vdots & \vdots & \ddots & \vdots \\
S_{1} & S_{2} & S_{3} & \cdots & S_{n}
\end{array}\right| .
$$
	\centering
\rule{1\linewidth}{0.6pt}
\end{exo}

\begin{exo}\textbf{(***)}\quad\\[0.25cm]%An115

	Soit $n \geq 2$ et $P \in \mathbb{R}_{n-2}[X]$. Calculer le déterminant

$$\left|\begin{array}{cccc}P(1) & P(2) & \cdots & P(n) \\ P(2) & P(3) & \cdots & P(n+1) \\ \vdots & \vdots & & \vdots \\ P(n) & P(n+1) & \cdots & P(2 n-1)\end{array}\right|$$
	
	\centering
	%\rule{1\linewidth}{0.6pt}
\end{exo}


\begin{exo}\textbf{(**)}\quad\\[0.25cm]%An115
Calculer le déterminant de la matrice $M_{n} \in \mathscr{M}_{n}(\mathbb{R})$ de coefficients $m_{i, j}$ égaux à 1 si $i=j, i=1$ ou $j=1$, nuls sinon.

$$
\left|\begin{array}{cccc}
1 & \ldots & \ldots & 1 \\
\vdots & \ddots & & \\
\vdots & & \ddots & \\
1 & & & 1
\end{array}\right| .
$$
	
	\centering
	\rule{1\linewidth}{0.6pt}
\end{exo}




\begin{exo}\textbf{(***)}\quad\\[0.25cm]%An115
	Soient $a_1,\dots,a_n$ des nombres complexes, $\omega=e^{2i\pi/n}$, et $A$ et $M$ les matrices suivantes :
	$$A=\left(
	\begin{array}{ccccc}
	a_1&a_2&a_3&\dots&a_n\\
	a_n&a_1&a_2&\dots&a_{n-1}\\
	\vdots&\vdots&\vdots&\vdots&\vdots\\
	a_2&a_3&\dots&\dots&a_{1}
	\end{array}\right),$$
	$$M=\left(
	\begin{array}{ccccc}
	1&1&\dots&\dots&1\\
	1&\omega&\omega^2&\dots&\omega^{n-1}\\
	\vdots&\vdots&\vdots&\vdots\\
	1&\omega^{n-1}&\omega^{2(n-1)}&\dots&\omega^{(n-1)(n-1)}
	\end{array}
	\right).$$
	Calculer $\det(AM)$ et en déduire $\det(A)$.
	
	\centering
	\rule{1\linewidth}{0.6pt}
\end{exo}





\end{document}
