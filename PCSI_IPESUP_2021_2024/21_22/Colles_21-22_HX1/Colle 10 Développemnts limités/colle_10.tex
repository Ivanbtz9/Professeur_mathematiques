\documentclass[a4paper,10pt]{article}



\usepackage{fancyhdr} % pour personnaliser les en-têtes
\usepackage[utf8]{inputenc}
\usepackage[T1]{fontenc}
\usepackage{lastpage}
\usepackage[frenchb]{babel}
\usepackage{amsfonts,amssymb}
\usepackage{amsmath,amsthm,mathtools}
\usepackage{paralist}
\usepackage{xspace}
\usepackage{xcolor,multicol}
\usepackage{variations}
\usepackage{xypic}
\usepackage{eurosym}
\usepackage{graphicx}
\usepackage[np]{numprint}
\usepackage{hyperref} 
\usepackage{listings} % pour écrire des codes avec coloration syntaxique  

\usepackage{tikz}
\usetikzlibrary{calc, arrows, plotmarks,decorations.pathreplacing}
\usepackage{colortbl}
\usepackage{multirow}
\usepackage[top=2cm,bottom=1.5cm,right=2cm,left=1.5cm]{geometry}

\newtheorem{thm}{Théorème}
\newtheorem*{pro}{Propriété}
\newtheorem*{exemple}{Exemple}

\theoremstyle{definition}
\newtheorem*{remarque}{Remarque}
\theoremstyle{definition}
\newtheorem{exo}{Exercice}
\newtheorem{definition}{Définition}


\newcommand{\vtab}{\rule[-0.4em]{0pt}{1.2em}}
\newcommand{\V}{\overrightarrow}
\renewcommand{\thesection}{\Roman{section} }
\renewcommand{\thesubsection}{\arabic{subsection} }
\renewcommand{\thesubsubsection}{\alph{subsubsection} }
\newcommand*{\transp}[2][-3mu]{\ensuremath{\mskip1mu\prescript{\smash{\mathrm t\mkern#1}}{}{\mathstrut#2}}}%

\newcommand{\C}{\mathbb{C}}
\newcommand{\R}{\mathbb{R}}
\newcommand{\Q}{\mathbb{Q}}
\newcommand{\Z}{\mathbb{Z}}
\newcommand{\N}{\mathbb{N}}



\definecolor{vert}{RGB}{11,160,78}
\definecolor{rouge}{RGB}{255,120,120}
% Set the beginning of a LaTeX document
\pagestyle{fancy}
\lhead{Optimal Sup Spé, groupe IPESUP}\chead{Année~2021-2022}\rhead{Niveau: Première année de PCSI }\lfoot{M. Botcazou}\cfoot{\thepage/1}\rfoot{mail: ibotca52@gmail.com }\renewcommand{\headrulewidth}{0.4pt}\renewcommand{\footrulewidth}{0.4pt}

\begin{document}
	
	
	\begin{center}
		\Large \sc colle 10 = Développements limités et espaces vectoriels
	\end{center}
	
%\section *{Questions de cours:}

%\begin{enumerate} 

%\item Pour tout $n\in\N$ montrer que:
%$\dfrac{1}{1-x} \underset{x\rightarrow 0}{=} \  \sum\limits_{k=0}^{n} x^k \ + \  o(x^n)$

%\item Démontrer le théorème suivant:
%\begin{thm}(\textbf{Unicité des coefficients d’un développement limité})\hfil\\ 
%	Soient $f : D \rightarrow \R$ une fonction et $a \in D$. 
%	En cas d’existence d'un développement limité de $f$ en $a$ à l'ordre $n\in\N$, la liste des coefficients de ce développement limité est unique.
%\end{thm}

%\item Soit $f$ une fonction de $\mathcal{C}^n\left(I,\R\right)$ et $a\in I$. Donner la formule de Taylor-Young correspondant au développement limité d'ordre $n$ de $f$  au voisinage de $a$. Calculer les développements limités suivants à l’ordre 2 :
%\begin{multicols}{2}
%\begin{enumerate}[$\square$]
%\item $\dfrac{\ln(x)}{x}$ en 3.
%\item $xe^x$ en 1.
%\end{enumerate}
%\end{multicols}
%\end{enumerate}



\section*{Études locales et asymptotiques}
\begin{minipage}{1\linewidth}
\begin{minipage}[t]{0.48\linewidth}
\raggedright



\begin{exo}\quad\\
Soient $I$ un intervalle, $g \in\mathcal{D}(I ,\R)$ , $a \in I$  et
$n \in\N$.\\
\begin{enumerate}
	\item  Montrer que: si $g'(x)\underset{x\rightarrow a}{=} o\left((x-a)^n\right)$\\ alors: $g(x)\underset{x\rightarrow a}{=} g(a) + o\left((x-a)^{n+1}\right)$
	\item Soit $f \in\mathcal{D}(I ,\R)$.Montrer que:\\Si $f'$ possède
	un D.L. en $a$ à l'ordre $n$ tel que:
	$$f'(x) \underset{x\rightarrow a}{=} \sum\limits_{k=0}^{n}a_k(x-a)^k \ + o\left((x-a)^n\right)$$
	Alors $f$ possède
	un D.L. en $a$ à l'ordre $n+1$ tel que:
	$$f(x) \underset{x\rightarrow a}{=} f(a) \ + \  \sum\limits_{k=0}^{n}a_k\dfrac{(x-a)^{k+1} }{k+1}\ + o\left((x-a)^{n+1}\right)$$
	\item En déduire un D.L. à l'ordre $n$ des fonctions suivantes en $0$:
	\begin{enumerate}[$\square$]
		\item $ x \longmapsto \ln(1-x)$
		\item $ x \longmapsto \ln(1+x)$
	\end{enumerate}
\item En déduire un D.L. à l'ordre $2n+1$ de la fonction suivante en $0$:
$ x \longmapsto \arctan(x)$
\end{enumerate}

\centering
\rule{1\linewidth}{0.6pt}
\end{exo}

\begin{exo}\quad\\
	Calculer les limites suivantes
	$$\lim_{x\rightarrow 0}\frac{\ln (1+x)-\sin x}{x}
	\quad\quad \lim_{x\rightarrow 0}\frac{\cos x-\sqrt{1-x^2}}{x^4}$$
	$$\lim_{x\rightarrow 0}\frac{e^{x^2}-\cos x}{x^2}
	\quad\quad$$ 
	\centering\rule{1\linewidth}{0.6pt}
\end{exo}


\end{minipage}	
\hfill\vrule\hfill
\begin{minipage}[t]{0.48\linewidth}
\raggedright

\begin{exo}\quad\\
	
	\begin{enumerate}
		\item Développement limité en $1$ à l'ordre $3$ de $f(x)=\sqrt{x}$.
		
		\item Développement limité en $1$ à l'ordre $3$ de $g(x)= e^{\sqrt{x}}$.
		
		\item Développement limité à l'ordre $3$ en $\frac\pi 3$ de $h(x)=\ln (\sin x)$.
	\end{enumerate}
	\centering
	\rule{1\linewidth}{0.6pt}
\end{exo}

\begin{exo}\quad\\
	Donner un développement limité à l'ordre $2$ de $f(x)=
	\displaystyle{\frac{\sqrt{1+x^2}}{1+x+\sqrt{1+x^2}}}$ en $0$.
	En déduire un développement à l'ordre $2$ en $+\infty$.
	Calculer un développement à l'ordre $1$ en $-\infty$.
	\centering\rule{1\linewidth}{0.6pt}
\end{exo}



\begin{exo}\quad\\
\begin{enumerate}
  \item  Montrer que l'équation $\tan x = x$ possède une unique solution
   $x_n$ dans
   $\left]n\pi-\frac \pi 2, n\pi+\frac \pi 2\right[$ $(n\in \N)$.
  \item  Quelle relation lie $x_n$ et $\arctan(x_n)$ ? \label{relation}
  \item  Donner un DL de $x_n$ en fonction de $n$ à l'ordre $0$ pour $n\to\infty$.
  \item  En reportant dans la relation trouvée en \ref{relation},
     obtenir un DL de $x_n$ à l'ordre 2.
\end{enumerate}
\centering\rule{1\linewidth}{0.6pt}
\end{exo}
\begin{exo}\quad\\
\'Etudier la position du graphe de l'application $x\mapsto \ln(1+x+x^2)$ par rapport 
à sa tangente en $0$ et $1$.
	\centering\rule{1\linewidth}{0.6pt}
\end{exo}

\end{minipage}
\end{minipage}

\hfill\\

\section*{Espaces vectoriels}
\begin{minipage}{1\linewidth}
	\begin{minipage}[t]{0.48\linewidth}
		\raggedright
		
		
		
		\begin{exo}\quad\\
		Parmi les ensembles suivants reconna\^\i tre ceux qui sont des
		sous-espaces vectoriels.(Justifier)
		
		$ E_1 =\left\{ (x,y,z)\in \R^3 \mid x+y+a=0 \hbox{ et }  x +3az =0\right\}$
		
		$ E_2 =\left\{f \in {\mathcal F}(\R,\R) \mid f(1)=0\right\}$
		
		$ E_3 =\left\{f \in {\mathcal F}(\R,\R) \mid  f(0)=1\right\}$
		
		$E_4 =\left\{(x,y)\in \R^2 \mid x + \alpha y +1 \geqslant 0\right\}$	
			
			\centering
			\rule{1\linewidth}{0.6pt}
		\end{exo}
	
	\begin{exo}\quad\\
		Soit $\alpha \in \R$ et $f_\alpha : \R \to \R$, $x\mapsto e^{\alpha x}$.
		Montrer que la famille $(f_\alpha)_{\alpha \in \R}$  est libre.
		
		\centering
		\rule{1\linewidth}{0.6pt}
	\end{exo}
		
		
		
		
	\end{minipage}	
	\hfill\vrule\hfill
	\begin{minipage}[t]{0.48\linewidth}
		\raggedright
		
	\begin{exo}\quad\\
		\begin{enumerate}
			\item Soient $v_1=(2,1,4)$, $v_2=(1,-1,2)$ et $v_3=(3,3,6)$ des vecteurs de $\R^3$, 
			trouver trois r\'eels non tous nuls $\alpha,\beta,\gamma$ tels que $\alpha v_1+ \beta v_2 + \gamma v_3=0$.
			
			\item On considère deux plans vectoriels
			$$P_1=\{(x,y,z) \in \R^3 \mid x-y+z=0\}$$
			$$P_2=\{(x,y,z) \in \R^3 \mid x-y=0\}$$
			trouver un vecteur directeur de la droite $D=P_1\cap P_2$ ainsi qu'une \'equation param\'etr\'ee.
		\end{enumerate}
		
		\centering
		\rule{1\linewidth}{0.6pt}
	\end{exo}

	
	\end{minipage}
\end{minipage}
\end{document}