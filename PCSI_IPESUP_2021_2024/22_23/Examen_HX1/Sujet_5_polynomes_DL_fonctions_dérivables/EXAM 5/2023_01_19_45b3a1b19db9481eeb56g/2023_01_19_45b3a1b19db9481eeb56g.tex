\documentclass[10pt]{article}
\usepackage[utf8]{inputenc}
\usepackage[T1]{fontenc}
\usepackage{graphicx}
\usepackage[export]{adjustbox}
\graphicspath{ {./images/} }
\usepackage{amsmath}
\usepackage{amsfonts}
\usepackage{amssymb}
\usepackage[version=4]{mhchem}
\usepackage{stmaryrd}
\usepackage{bbold}

\title{Mathématiques 2 }

\author{}
\date{}


\begin{document}
\maketitle
\begin{center}
\includegraphics[max width=\textwidth]{2023_01_19_45b3a1b19db9481eeb56g-1}
\end{center}

CONCOURS CENTRALE·SUPÉLEC

MP

4 heures Calculatrice autorisée

\section{Inégalités de Bernstein}
Le but de ce problème est d'étudier les inégalités dites de Bernstein dans deux cadres différents.

La première partie s'intéresse à la démonstration de l'inégalité de Bernstein pour les polynômes et à certaines applications. La deuxième partie introduit la notion de transformée de Fourier et permet d'établir une inégalité de Bernstein pour des fonctions dont la transformée de Fourier vérifie certaines propriétés.

Les deux parties de ce sujet sont complètement indépendantes et peuvent être traitées dans l'ordre désiré.

\section{Inégalité polynomiale de Bernstein et applications}
Dans cette partie,

\begin{itemize}
  \item si $n \in \mathbb{N}$, on note $\mathbb{C}_{n}[X]$ le $\mathbb{C}$-espace vectoriel des polynômes à coefficients complexes de degré inférieur ou égal à $n$;

  \item si $n \in \mathbb{N}^{*}$, on note $\mathcal{S}_{n}$ le $\mathbb{C}$-espace vectoriel des fonctions $f: \mathbb{R} \rightarrow \mathbb{C}$ vérifiant

\end{itemize}

$$
\exists\left(a_{0}, \ldots, a_{n}\right) \in \mathbb{C}^{n+1}, \quad \exists\left(b_{1}, \ldots, b_{n}\right) \in \mathbb{C}^{n}, \quad \forall t \in \mathbb{R}, \quad f(t)=a_{0}+\sum_{k=1}^{n}\left(a_{k} \cos (k t)+b_{k} \sin (k t)\right) .
$$

On remarque que les éléments de $\mathcal{S}_{n}$ sont des fonctions bornées ;

\begin{itemize}
  \item si $I$ est un intervalle non vide de $\mathbb{R}$ et si $f$ est une fonction bornée de $I$ dans $\mathbb{C}$, on note
\end{itemize}

$$
\|f\|_{L^{\infty(I)}}=\sup _{x \in I}|f(x)| .
$$

On admet que $f \mapsto\|f\|_{L^{\infty}(I)}$ définit une norme sur le $\mathbb{C}$-espace vectoriel des fonctions bornées de $I$ dans $\mathbb{C}$.

\section{I.A - Polynômes de Tchebychev}
On définit la suite de polynômes $\left(T_{n}\right)_{n \in \mathbb{N}}$ par $T_{0}=1, T_{1}=X$ et $\forall n \in \mathbb{N}, T_{n+2}=2 X T_{n+1}-T_{n}$.

Q 1. Pour tout $n$ dans $\mathbb{N}$, déterminer le degré de $T_{n}$, puis montrer que $\left(T_{k}\right)_{0 \leqslant k \leqslant n}$ est une base de $\mathbb{C}_{n}[X]$.

Q 2. Montrer que, pour tous $n \in \mathbb{N}$ et $\theta \in \mathbb{R}, T_{n}(\cos \theta)=\cos (n \theta)$.

Q 3. En déduire que, pour tous $n \in \mathbb{N}$ et $P \in \mathbb{C}_{n}[X]$, la fonction de $\mathbb{R}$ dans $\mathbb{C}, \theta \mapsto P(\cos \theta)$ est dans $\mathcal{S}_{n}$.

Q 4. Pour $n \in \mathbb{N}$, calculer $\left\|T_{n}\right\|_{L^{\infty}([-1,1])}$.

Q 5. Montrer que, pour tout $n \in \mathbb{N},\left\|T_{n}^{\prime}\right\|_{L^{\infty}([-1,1])}=n^{2}$.

On pourra commencer par établir que, pour tous $n \in \mathbb{N}$ et $\theta \in \mathbb{R},|\sin (n \theta)| \leqslant n|\sin \theta|$.

\section{I.B - Inégalité de Bernstein}
Soit $n$ un entier naturel non nul.

Q 6. Soit $A \in \mathbb{C}_{2 n}[X]$, scindé à racines simples, et $\left(\alpha_{1}, \ldots, \alpha_{2 n}\right)$ ses racines. Montrer que

$$
\forall B \in \mathbb{C}_{2 n-1}[X], \quad B(X)=\sum_{k=1}^{2 n} B\left(\alpha_{k}\right) \frac{A(X)}{\left(X-\alpha_{k}\right) A^{\prime}\left(\alpha_{k}\right)} .
$$

Soit $P$ dans $\mathbb{C}_{2 n}[X]$, et, pour tout $\lambda \in \mathbb{C}, P_{\lambda}(X)=P(\lambda X)-P(\lambda)$.

Q 7. Si $\lambda \in \mathbb{C}$, vérifier que $X-1$ divise $P_{\lambda}$.

Pour tout $\lambda$ dans $\mathbb{C}$, on note $Q_{\lambda}$ le quotient de $P_{\lambda} \operatorname{par} X-1$ :

$$
Q_{\lambda}(X)=\frac{P(\lambda X)-P(\lambda)}{X-1} \in \mathbb{C}_{2 n-1}[X] .
$$

Q 8. Montrer que, pour tout $\lambda$ dans $\mathbb{C}, Q_{\lambda}(1)=\lambda P^{\prime}(\lambda)$.

On considère le polynôme $R(X)=X^{2 n}+1$. Pour $k$ dans $\llbracket 1,2 n \rrbracket$, on note $\varphi_{k}=\frac{\pi}{2 n}+\frac{k \pi}{n}$ et $\omega_{k}=\mathrm{e}^{\mathrm{i} \varphi_{k}}$.

Q 9. Montrer que

$$
R(X)=\prod_{k=1}^{2 n}\left(X-\omega_{k}\right) .
$$

Q 10. À l'aide de la formule (I.1), montrer que

$$
\forall \lambda \in \mathbb{C}, \quad Q_{\lambda}(X)=-\frac{1}{2 n} \sum_{k=1}^{2 n} \frac{P\left(\lambda \omega_{k}\right)-P(\lambda)}{\omega_{k}-1} \frac{X^{2 n}+1}{X-\omega_{k}} \omega_{k}
$$

puis en déduire que

$$
\forall \lambda \in \mathbb{C}, \quad \lambda P^{\prime}(\lambda)=\frac{1}{2 n} \sum_{k=1}^{2 n} P\left(\lambda \omega_{k}\right) \frac{2 \omega_{k}}{\left(1-\omega_{k}\right)^{2}}-\frac{P(\lambda)}{2 n} \sum_{k=1}^{2 n} \frac{2 \omega_{k}}{\left(1-\omega_{k}\right)^{2}} .
$$

Q 11. Montrer que

$$
\forall \lambda \in \mathbb{C}, \quad \lambda P^{\prime}(\lambda)=\frac{1}{2 n} \sum_{k=1}^{2 n} P\left(\lambda \omega_{k}\right) \frac{2 \omega_{k}}{\left(1-\omega_{k}\right)^{2}}+n P(\lambda) .
$$

On pourra appliquer l'égalité (I.2) au polynôme $X^{2 n}$.

Soit maintenant $f$ dans $\mathcal{S}_{n}$.

Q 12. Montrer qu'il existe $U \in \mathbb{C}_{2 n}[X]$ tel que, pour tout $\theta \in \mathbb{R}, f(\theta)=\mathrm{e}^{-\mathrm{i} n \theta} U\left(\mathrm{e}^{\mathrm{i} \theta}\right)$.

Q 13. Vérifier que, pour tout $k \in \llbracket 1,2 n \rrbracket, \frac{2 \omega_{k}}{\left(1-\omega_{k}\right)^{2}}=\frac{-1}{2 \sin \left(\varphi_{k} / 2\right)^{2}}$ et déduire des questions 11 et 12 que

$$
\forall \theta \in \mathbb{R}, \quad f^{\prime}(\theta)=\frac{1}{2 n} \sum_{k=1}^{2 n} f\left(\theta+\varphi_{k}\right) \frac{(-1)^{k}}{2 \sin \left(\varphi_{k} / 2\right)^{2}} .
$$

Q 14. En déduire que

$$
\forall \theta \in \mathbb{R}, \quad\left|f^{\prime}(\theta)\right| \leqslant n\|f\|_{L^{\infty}(\mathbb{R})} .
$$

\section{C - Quelques conséquences de l'inégalité (I.4)}
Soit $n$ un entier naturel non nul.

Q 15. Déduire des questions 3 et 14 que

$$
\forall P \in \mathbb{C}_{n}[X], \quad \forall x \in[-1,1], \quad\left|P^{\prime}(x) \sqrt{1-x^{2}}\right| \leqslant n\|P\|_{L^{\infty}([-1,1])} .
$$

Q 16. Montrer que

$$
\forall Q \in \mathbb{C}_{n-1}[X], \quad|Q(1)| \leqslant n \sup _{-1 \leqslant x \leqslant 1}\left|Q(x) \sqrt{1-x^{2}}\right| .
$$

On pourra considérer $f: \theta \mapsto Q(\cos \theta) \sin \theta$ et vérifier que $f \in \mathcal{S}_{n}$.

Q 17. Soit $R \in \mathbb{C}_{n-1}[X]$ et $t \in[-1,1]$. Montrer que

$$
|R(t)| \leqslant n \sup _{-1 \leqslant x \leqslant 1}\left|R(x) \sqrt{1-x^{2}}\right| .
$$

On pourra considérer le polynôme $S_{t}(X)=R(t X)$.

Q 18. En déduire que, pour tout $P$ dans $\mathbb{C}_{n}[X]$

$$
\left\|P^{\prime}\right\|_{L^{\infty}([-1,1])} \leqslant n^{2}\|P\|_{L^{\infty}([-1,1])} .
$$

Q 19. Peut-il y avoir égalité dans l'inégalité précédente?

\section{Inégalités de Bernstein et transformée de Fourier}
Dans cette partie,

\begin{itemize}
  \item pour $k \in \mathbb{N}$, on dit qu'une fonction $f$ de $\mathbb{R}$ dans $\mathbb{C}$ est de classe $\mathcal{C}^{k}$ si elle est $k$ fois dérivable sur $\mathbb{R}$, de dérivée $k$-ième continue sur $\mathbb{R}$ (si $k=0, f$ est continue) ; on dit que $f$ est $\mathcal{C}^{\infty}$ si $f$ est $\mathcal{C}^{k}$ pour tout $k \in \mathbb{N}$. On note $\mathcal{C}^{k}(\mathbb{R})$ (respectivement $\mathcal{C}^{\infty}(\mathbb{R})$ ) l'ensemble des fonctions de classe $\mathcal{C}^{k}$ (respectivement $\mathcal{C}^{\infty}$ ) sur $\mathbb{R}$;

  \item on note $L^{1}(\mathbb{R})$ l'ensemble des fonctions de $\mathbb{R}$ dans $\mathbb{C}$ continues et intégrables sur $\mathbb{R}$;

  \item pour $f \in L^{1}(\mathbb{R})$, on note $\|f\|_{1}=\int_{-\infty}^{+\infty}|f(t)| \mathrm{d} t$;

  \item on note $L^{\infty}(\mathbb{R})$ l'ensemble des fonctions de $\mathbb{R}$ dans $\mathbb{C}$ continues et bornées sur $\mathbb{R}$;

  \item pour $f \in L^{\infty}(\mathbb{R})$, on note $\|f\|_{\infty}=\sup _{x \in \mathbb{R}}|f(x)|$.

\end{itemize}

On admet que $L^{1}(\mathbb{R}), L^{\infty}(\mathbb{R})$ et $\mathcal{C}^{k}(\mathbb{R})(k \in \mathbb{N})$ sont des sous-espaces vectoriels de $\mathbb{C}^{\mathbb{R}}$. On admet également que $f \mapsto\|f\|_{1}$ définit une norme sur $L^{1}(\mathbb{R})$ et que $f \mapsto\|f\|_{\infty}$ définit une norme sur $L^{\infty}(\mathbb{R})$. On dispose ainsi des espaces vectoriels normés $\left(L^{1}(\mathbb{R}),\|\cdot\|_{1}\right)$ et $\left(L^{\infty}(\mathbb{R}),\|\cdot\|_{\infty}\right)$.

\section{II.A - Transformée de Fourier d'une fonction}
Soit $f \in L^{1}(\mathbb{R})$. On appelle transformée de Fourier de $f$ et on note $\hat{f}$ la fonction de $\mathbb{R}$ dans $\mathbb{C}$ telle que

$$
\forall \xi \in \mathbb{R}, \quad \hat{f}(\xi)=\int_{-\infty}^{+\infty} f(x) \mathrm{e}^{-\mathrm{i} x \xi} \mathrm{d} x .
$$

Q 20. Montrer que, pour toute fonction $f \in L^{1}(\mathbb{R}), \hat{f}$ est définie et continue sur $\mathbb{R}$.

Q 21. Montrer que l'application $f \mapsto \hat{f}$ est une application linéaire continue de l'espace vectoriel normé $\left(L^{1}(\mathbb{R}),\|\cdot\|_{1}\right)$ dans l'espace vectoriel normé $\left(L^{\infty}(\mathbb{R}),\|\cdot\|_{\infty}\right)$.

Q 22. Soit $f \in L^{1}(\mathbb{R}), \lambda \in \mathbb{R}_{+}^{*}$ et soit $g$ la fonction de $\mathbb{R}$ dans $\mathbb{C}$ telle que $g(x)=f(\lambda x)$ pour tout réel $x$. Montrer que $g \in L^{1}(\mathbb{R})$ et, pour tout réel $\xi$, exprimer $\hat{g}(\xi)$ à l'aide de $\hat{f}$, de $\xi$ et de $\lambda$.

\section{II.B - Produit de convolution}
Si $f$ et $g$ sont deux fonctions continues de $\mathbb{R}$ dans $\mathbb{C}$ telles que, pour tout $x \in \mathbb{R}$, la fonction $t \mapsto f(t) g(x-t)$ soit intégrable sur $\mathbb{R}$, on appelle produit de convolution de $f$ et $g$, et on note $f * g$, la fonction de $\mathbb{R}$ dans $\mathbb{C}$ telle que

$$
\forall x \in \mathbb{R}, \quad(f * g)(x)=\int_{-\infty}^{+\infty} f(t) g(x-t) \mathrm{d} t .
$$

On suppose désormais et jusqu'à la fin de la sous-partie II.B que $f \in L^{1}(\mathbb{R})$ et $g \in L^{\infty}(\mathbb{R})$.

Q 23. Montrer que $f * g$ est définie sur $\mathbb{R}$ et que

$$
\forall x \in \mathbb{R}, \quad(f * g)(x)=\int_{-\infty}^{+\infty} f(x-t) g(t) \mathrm{d} t=(g * f)(x) .
$$

Q 24. Montrer que $f * g$ est bornée et que $\|f * g\|_{\infty} \leqslant\|f\|_{1}\|g\|_{\infty}$.

Q 25. Soit $k \in \mathbb{N}$. Montrer que, si $g$ est de classe $\mathcal{C}^{k}$ et si les fonctions $g^{(j)}$ sont bornées pour $j \in \llbracket 0, k \rrbracket$, alors $f * g$ est de classe $\mathcal{C}^{k}$ et $(f * g)^{(k)}=f *\left(g^{(k)}\right)$.

Q 26. On suppose toujours que $f \in L^{1}(\mathbb{R})$ et $g \in L^{\infty}(\mathbb{R})$ et on suppose de plus que $g \in L^{1}(\mathbb{R})$ et $f * g \in L^{1}(\mathbb{R})$. En admettant que, pour tout $\xi$ réel,

$$
\int_{-\infty}^{+\infty}\left(\int_{-\infty}^{+\infty} \mathrm{e}^{-\mathrm{i} x \xi} f(t) g(x-t) \mathrm{d} t\right) \mathrm{d} x \quad \text { et } \int_{-\infty}^{+\infty}\left(\int_{-\infty}^{+\infty} \mathrm{e}^{-\mathrm{i} x \xi} f(t) g(x-t) \mathrm{d} x\right) \mathrm{d} t
$$

existent et sont égales, montrer que $\widehat{f * g}=\hat{f} \hat{g}$.

\section{II.C - Introduction d'une fonction plateau}
On cherche dans cette sous-partie à construire une fonction réelle positive $\rho$, définie et de classe $\mathcal{C}^{\infty}$ sur $\mathbb{R}$, telle que $\rho(t)=1$ pour tout $t \in[-1,1]$ et $\rho(t)=0$ pour tout $t \in \mathbb{R} \backslash[-2,2]$.

Soit $\varphi$ la fonction définie sur $\mathbb{R}$ par

$$
\forall t \in \mathbb{R}, \quad \varphi(t)= \begin{cases}0 & \text { si } t \leqslant 0 \\ \mathrm{e}^{-1 / t} & \text { sinon. }\end{cases}
$$

Q 27. Montrer que $\varphi$ est de classe $\mathcal{C}^{\infty}$ sur $\mathbb{R}$.

On pourra montrer que : $\forall k \in \mathbb{N}, \exists P_{k} \in \mathbb{R}[X], \forall t>0, \varphi^{(k)}(t)=P_{k}(1 / t) \mathrm{e}^{-1 / t}$.

Soit $\psi$ la fonction définie sur $\mathbb{R}$ par

$$
\forall t \in \mathbb{R}, \quad \psi(t)= \begin{cases}0 & \text { si } t \notin]-1,1[, \\ \mathrm{e}^{1 /\left(t^{2}-1\right)} & \text { sinon. }\end{cases}
$$

Q 28. Montrer, en l'exprimant à l'aide de $\varphi$, que $\psi$ est de classe $\mathcal{C}^{\infty}$.

Q 29. Soit $\theta$ l'unique primitive de $\psi$ s'annulant en 0 . Montrer que $\theta$ est de classe $\mathcal{C}^{\infty}$, constante sur $\left.]-\infty,-1\right]$ (on note $A$ cette constante) et constante sur $[1,+\infty[$ (on note $B$ cette constante). Vérifier que $A \neq B$.

Q 30. Construire alors une fonction $\rho \in \mathcal{C}^{\infty}(\mathbb{R})$, constante égale à 1 sur $[-1,1]$ et constante égale à 0 sur $\mathbb{R} \backslash[-2,2]$.

\section{II.D - Inégalités de Bernstein}
On admet les formules suivantes, dites formules d'inversion de Fourier :

\begin{itemize}
  \item si $f \in L^{1}(\mathbb{R})$ et si $\hat{f} \in L^{1}(\mathbb{R})$, alors, pour tout $x \in \mathbb{R}, f(x)=\frac{1}{2 \pi} \int_{-\infty}^{+\infty} \mathrm{e}^{\mathrm{i} x \xi} \hat{f}(\xi) \mathrm{d} \xi$;

  \item si $\alpha \in L^{1}(\mathbb{R})$, si $a$ est la fonction de $\mathbb{R}$ dans $\mathbb{C}: x \mapsto \frac{1}{2 \pi} \int_{-\infty}^{+\infty} \mathrm{e}^{\mathrm{i} x \xi} \alpha(\xi) \mathrm{d} \xi$, et si $a \in L^{1}(\mathbb{R})$, alors $\alpha=\hat{a}$.

\end{itemize}

On remarque que ces résultats permettent d'affirmer que, si $f$ et $g$ sont deux fonctions continues telles que $f, g$, $\hat{f}$ et $\hat{g}$ sont intégrables et si $\hat{f}=\hat{g}$, alors $f=g$.

On considère toujours la fonction $\rho$ définie à la question 30 .

Soit $r$ la fonction de $\mathbb{R}$ dans $\mathbb{C}$ telle que, pour tout réel $x$,

$$
r(x)=\frac{1}{2 \pi} \int_{-\infty}^{+\infty} \mathrm{e}^{\mathrm{i} x \xi} \rho(\xi) \mathrm{d} \xi .
$$

Q 31. Montrer que $r$ est dérivable sur $\mathbb{R}$ et donner une expression de sa fonction dérivée (faisant éventuellement intervenir une intégrale).

Q 32. Montrer que $x \mapsto x^{2} r(x)$ est bornée sur $\mathbb{R}$ et en déduire que $r$ est intégrable et bornée sur $\mathbb{R}$.

On admet qu'en utilisant la même méthode, on montre que $r^{\prime}$ est intégrable et bornée sur $\mathbb{R}$.

Soit $\lambda>0$ et soit $f \in L^{1}(\mathbb{R}) \cap \mathcal{C}^{1}(\mathbb{R})$ telle que $\hat{f} \in L^{1}(\mathbb{R})$ et telle que $\hat{f}$ soit nulle en dehors du segment $[-\lambda, \lambda]$. On note $r_{\lambda}$ la fonction de $\mathbb{R}$ dans $\mathbb{C}$ telle que $r_{\lambda}(x)=r(\lambda x)$ pour tout réel $x$.

Q 33. On admet que $f * r_{\lambda}$ est intégrable. Montrer que $f=\lambda f * r_{\lambda}$.

Q 34. En déduire que, si $f \in L^{\infty}(\mathbb{R})$, il existe une constante $C \in \mathbb{R}_{+}^{\star}$, indépendante de $\lambda$ et de $f$, telle que

$$
\left\|f^{\prime}\right\|_{\infty} \leqslant C \lambda\|f\|_{\infty} .
$$


\end{document}