\documentclass[a4paper,10pt]{article}



\usepackage{fancyhdr} % pour personnaliser les en-têtes
\usepackage[utf8]{inputenc}
\usepackage[T1]{fontenc}
\usepackage{lastpage}
\usepackage[frenchb]{babel}
\usepackage{amsfonts,amssymb}
\usepackage{amsmath,amsthm,mathtools}
\usepackage{paralist}
\usepackage{xspace}
\usepackage{xcolor,multicol}
\usepackage{variations}
\usepackage{xypic}
\usepackage{eurosym}
\usepackage{graphicx}
\usepackage[np]{numprint}
\usepackage{hyperref} 
\usepackage{listings} % pour écrire des codes avec coloration syntaxique  

\usepackage{tikz}
\usetikzlibrary{calc, arrows, plotmarks,decorations.pathreplacing}
\usepackage{colortbl}
\usepackage{multirow}
\usepackage[top=2cm,bottom=1.5cm,right=2cm,left=1.5cm]{geometry}

\newtheorem{thm}{Théorème}
\newtheorem*{pro}{Propriété}
\newtheorem*{exemple}{Exemple}

\theoremstyle{definition}
\newtheorem*{remarque}{Remarque}
\theoremstyle{definition}
\newtheorem{exo}{Exercice}
\newtheorem{definition}{Définition}


\newcommand{\vtab}{\rule[-0.4em]{0pt}{1.2em}}
\newcommand{\V}{\overrightarrow}
\renewcommand{\thesection}{\Roman{section} }
\renewcommand{\thesubsection}{\arabic{subsection} }
\renewcommand{\thesubsubsection}{\alph{subsubsection} }
\newcommand*{\transp}[2][-3mu]{\ensuremath{\mskip1mu\prescript{\smash{\mathrm t\mkern#1}}{}{\mathstrut#2}}}%

\newcommand{\C}{\mathbb{C}}
\newcommand{\R}{\mathbb{R}}
\newcommand{\Q}{\mathbb{Q}}
\newcommand{\Z}{\mathbb{Z}}
\newcommand{\N}{\mathbb{N}}



\definecolor{vert}{RGB}{11,160,78}
\definecolor{rouge}{RGB}{255,120,120}
% Set the beginning of a LaTeX document
\pagestyle{fancy}
\lhead{Optimal Sup Spé, groupe IPESUP}\chead{Année~2021-2022}\rhead{Niveau: Première année de PCSI }\lfoot{M. Botcazou}\cfoot{\thepage/2}\rfoot{mail: ibotca52@gmail.com }\renewcommand{\headrulewidth}{0.4pt}\renewcommand{\footrulewidth}{0.4pt}

\begin{document}
	
	
	\begin{center}
		\Large \sc colle 8 =   fonctions dérivables et polynômes
	\end{center}
	
\section *{Questions de cours:}

\begin{enumerate} 

\item Soient $k\in\N^*$ et $f,g\in\mathbb{C}^k\left(I,\R\right)$, exprimer $(fg)^{(k)}$. \\En déduire les dérivées successives de la fonction $x\mapsto x^3e^x$

\item Démontrer la propriété suivante:
\begin{pro}\hfil\\
Soient $I\subset\R$, $f:I\rightarrow\C$ une fonction dérivable sur $I$ et $a\in I$. \\Si $f(a)$ est un extremum local de $f$ alors $f'(a)=0$. 
\end{pro}
\item Rappeler le théorème de Rolle et donner sa démonstration.
\item Rappeler le théorème des acroissements finis et démontrer la propriété suivante:
\begin{pro}\hfil\\
Soient $I\subset\R$ et $f:I\rightarrow\C$ une fonction dérivable sur $I$. \\
$f$ est constante sur $I$ si et seulement si $f'$ est nulle sur $I$.
\end{pro}
\item Soit $P$ un polynôme différent de $X$. Montrer que $P(X)-X$ divise $P(P(X))-X$.
\item Démontrer la propriété suivante:
\begin{pro}\hfil\\
Soient $P\in K[X]$ et $a\in K$.\\
$a$ est racine de $P$ si et seulement si $(X-a)$ divise $P$
\end{pro}
\item Soient $x_1,x_2,...,x_n\in K$ distincts et $y_1,y_2,...,y_n\in K$ quelconques. Donner l'espression du seul et unique polynôme $P$ de degré $n-1$ pour lequel $P(x_i)=y_i$ pour tout $i\in [\![1;n]\!] $ 
\end{enumerate}


\section*{Fonctions dérivables:}
\begin{minipage}{1\linewidth}
\begin{minipage}[t]{0.48\linewidth}
\raggedright



\begin{exo}\quad\\
\noindent Soit $f\in\mathcal{C}^1\left(\R,\R\right)$. On fait l'hypothèse que:
 $$\forall x\in\R : f\circ f\left(x\right) = \dfrac{x}{4} +1$$
 \begin{enumerate}
\item Montrer que : $f'(x) = f'\left( \dfrac{x}{4} +1\right)  $ pour tout $x\in\R$.
\item En déduire de $f'$ est une fonction constante sur $\R$
\item Déterminer les fonctions $f\in\mathcal{C}^1\left(\R,\R\right)$ telles que $f\circ f\left(x\right) = \dfrac{x}{4} +1$ pour tout $x\in\R$.
 \end{enumerate}
\centering
\rule{1\linewidth}{0.6pt}
\end{exo}

\begin{exo}
Soient $x$ et $y$ r\'eels avec $0<x<y$.
\begin{enumerate}
    \item Montrer que
$$ x < \frac{y-x}{\ln y - \ln x} < y.$$
    \item On consid\`ere la fonction $f$ d\'efinie sur
$[0,1]$ par
$$\alpha \mapsto f(\alpha) = \ln (\alpha x +(1-\alpha)y)-\alpha
\ln x -(1-\alpha)\ln y.$$
De l'\'etude de $f$ d\'eduire que pour tout $\alpha$ de $]0,1[$
$$ \alpha
\ln x +(1-\alpha)\ln y < \ln (\alpha x +(1-\alpha)y) .$$
Interpr\'etation g\'eom\'etrique ?
\end{enumerate}


\centering
\rule{1\linewidth}{0.6pt}
\end{exo}








\end{minipage}	
\hfill\vrule\hfill
\begin{minipage}[t]{0.48\linewidth}
\raggedright

\begin{exo}\quad\\
\'Etudier  la d\'erivabilit\'e des fonctions suivantes :
$$f_1(x)=x^2\cos \frac{1}{x}, \text{\ \  si }x\not=0 \qquad ; \qquad f_1(0)=0 ;$$

$$f_2(x)= \sin x \cdot \sin \frac{1}{x}, \text{\ \  si }x\not=0 \qquad ; \qquad f_2(0)=0 ;$$

$$f_3(x) = \frac{|x|\sqrt{x^2-2x+1}}{x-1}, \text{\ \  si } x\not= 1 \qquad ; \qquad f_3(1)=1.$$
\centering\rule{1\linewidth}{0.6pt}
\end{exo}

\begin{exo}\quad\\
Montrer que le polyn\^ome $  X^n+aX+b  $,  ($  a  $ et $  b  $ r\' eels) admet au plus trois racines r\' eelles.

\centering\rule{1\linewidth}{0.6pt}
\end{exo}

\begin{exo}\quad\\
Pour tous $n\in\N^*$, calculer la dérivée $n^{\text{ème}}$ \\de $x \mapsto  x^{n-1}\ln(1+x)$ sur $]-1,+\infty[$.

\centering\rule{1\linewidth}{0.6pt}
\end{exo}

\begin{exo}\quad\\
Montrer que pour tous $n\in\N*$ et $x\in\R*$:
$$\dfrac{d^n}{dx^n}\left(x^{n-1} \exp\left(\tiny\dfrac{1}{x}\right)\right) = \dfrac{(-1)^n}{x^{n+1}} \exp\left(\tiny\dfrac{1}{x}\right)$$

\centering\rule{1\linewidth}{0.6pt}
\end{exo}
\end{minipage}
\end{minipage}


\section*{Polynômes:}
\begin{minipage}{1\linewidth}
\begin{minipage}[t]{0.48\linewidth}
\raggedright



\begin{exo}\quad\\
Soit $P$ un polynôme à coefficients réels tel que $\forall x\in\R,\;P(x)\geq 0$. Montrer qu'il existe deux polynômes $R$ et $S$ à coefficients réels tels que $P=R^2+S^2$.

\centering
\rule{1\linewidth}{0.6pt}
\end{exo}


\begin{exo}\quad\\
Montrer que pour tout $n\in\N$:
$$\sum_{k=0}^{n}\begin{pmatrix}
n\\k
\end{pmatrix}^2  =  \begin{pmatrix}
2n\\n
\end{pmatrix}$$
(\textit{Indication: étudier le polynôme } $(X+1)^{2n} $ )

\centering\rule{1\linewidth}{0.6pt}
\end{exo}

\begin{exo}\quad\\
Trouver les polynômes $P$ de $\R[X]$ vérifiant $P(X^2)=P(X)P(X+1)$ (penser aux racines de $P$).

\centering
\rule{1\linewidth}{0.6pt}
\end{exo}






\end{minipage}	
\hfill\vrule\hfill
\begin{minipage}[t]{0.48\linewidth}
\raggedright



\begin{exo}\quad\\
Soit $P$ un polynôme à coefficients entiers relatifs de degré supérieur ou égal à $1$. Soit $n$ un entier relatif 
et $m=P(n)$.
\begin{enumerate}
\item  Montrer que $\forall k\in\Z,\;P(n+km)$ est un entier divisible par $m$.
\item  Montrer qu'il n'existe pas de polynômes non constants à coefficients entiers tels que $P(n)$ soit premier pour tout entier $n$.
\end{enumerate}
\centering\rule{1\linewidth}{0.6pt}
\end{exo}

\begin{exo}\quad\\
Trouver tous les polynômes divisibles par leur dérivée.

\centering\rule{1\linewidth}{0.6pt}
\end{exo}

\begin{exo}\quad\\

Déterminer deux polynômes $U$ et $V$ vérifiant $UX^n+V(1-X)^m=1$ et $\mbox{deg}(U)<m$ et $\mbox{deg}(V)<n$.

\centering\rule{1\linewidth}{0.6pt}
\end{exo}







\end{minipage}
\end{minipage}
\end{document}