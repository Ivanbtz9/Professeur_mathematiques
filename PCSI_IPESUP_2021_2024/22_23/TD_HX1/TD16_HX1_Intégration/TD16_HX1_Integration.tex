\documentclass[a4paper,11pt]{article}

\usepackage{inputenc}
\usepackage[T1]{fontenc}
\usepackage[frenchb]{babel}
\usepackage{fancyhdr,fancybox} % pour personnaliser les en-têtes
\usepackage{lastpage,setspace}
\usepackage{amsfonts,amssymb,amsmath,amsthm,mathrsfs}
\usepackage{mathdots}
\usepackage{relsize,exscale,bbold}
\usepackage{paralist}
\usepackage{xspace,multicol,diagbox,array}
\usepackage{xcolor}
\usepackage{variations}
\usepackage{xypic}
\usepackage{eurosym,stmaryrd}
\usepackage{graphicx}
\usepackage[np]{numprint}
\usepackage{hyperref} 
\usepackage{tikz}
\usepackage{colortbl}
\usepackage{multirow}
\usepackage{MnSymbol,wasysym}
\usepackage[top=1.5cm,bottom=1.5cm,right=1.2cm,left=1.5cm]{geometry}
\usetikzlibrary{calc, arrows, plotmarks, babel,decorations.pathreplacing}
\setstretch{1.25}
%\usepackage{lipsum} %\usepackage{enumitem} %\setlist[enumerate]{itemsep=1mm} bug avec enumerate



\newtheorem{thm}{Théorème}
\newtheorem{rmq}{Remarque}
\newtheorem{prop}{Propriété}
\newtheorem{cor}{Corollaire}
\newtheorem{lem}{Lemme}
\newtheorem{prop-def}{Propriété-définition}

\theoremstyle{definition}

\newtheorem{defi}{Définition}
\newtheorem{ex}{Exemple}
\newtheorem*{rap}{Rappel}
\newtheorem{cex}{Contre-exemple}
\newtheorem{exo}{Exercice} % \large {\fontfamily{ptm}\selectfont EXERCICE}
\newtheorem{nota}{Notation}
\newtheorem{ax}{Axiome}
\newtheorem{appl}{Application}
\newtheorem{csq}{Conséquence}
\def\di{\displaystyle}



\renewcommand{\thesection}{\Roman{section}}\renewcommand{\thesubsection}{\arabic{subsection} }\renewcommand{\thesubsubsection}{\alph{subsubsection} }


\newcommand{\bas}{~\backslash}\newcommand{\ba}{\backslash}
\newcommand{\C}{\mathbb{C}}\newcommand{\K}{\mathbb{K}}\newcommand{\R}{\mathbb{R}}\newcommand{\Q}{\mathbb{Q}}\newcommand{\Z}{\mathbb{Z}}\newcommand{\N}{\mathbb{N}}\newcommand{\V}{\overrightarrow}\newcommand{\Cs}{\mathscr{C}}\newcommand{\Ps}{\mathscr{P}}\newcommand{\Rs}{\mathscr{R}}\newcommand{\Gs}{\mathscr{G}}\newcommand{\Ds}{\mathscr{D}}\newcommand{\happy}{\huge\smiley}\newcommand{\sad}{\huge\frownie}\newcommand{\danger}{\begin{tikzpicture}[x=1.5pt,y=1.5pt,rotate=-14.2]
	\definecolor{myred}{rgb}{1,0.215686,0}
	\draw[line width=0.1pt,fill=myred] (13.074200,4.937500)--(5.085940,14.085900)..controls (5.085940,14.085900) and (4.070310,15.429700)..(3.636720,13.773400)
	..controls (3.203130,12.113300) and (0.917969,2.382810)..(0.917969,2.382810)
	..controls (0.917969,2.382810) and (0.621094,0.992188)..(2.097660,1.359380)
	..controls (3.574220,1.726560) and (12.468800,3.984380)..(12.468800,3.984380)
	..controls (12.468800,3.984380) and (13.437500,4.132810)..(13.074200,4.937500)
	--cycle;
	\draw[line width=0.1pt,fill=white] (11.078100,5.511720)--(5.406250,11.875000)..controls (5.406250,11.875000) and (4.683590,12.812500)..(4.367190,11.648400)
	..controls (4.050780,10.488300) and (2.375000,3.675780)..(2.375000,3.675780)
	..controls (2.375000,3.675780) and (2.156250,2.703130)..(3.214840,2.964840)
	..controls (4.273440,3.230470) and (10.640600,4.847660)..(10.640600,4.847660)
	..controls (10.640600,4.847660) and (11.332000,4.953130)..(11.078100,5.511720)
	--cycle;
	\fill (6.144520,8.839900)..controls (6.460940,7.558590) and (6.464840,6.457090)..(6.152340,6.378910)
	..controls (5.835930,6.300840) and (5.320300,7.277400)..(5.003900,8.554750)
	..controls (4.683590,9.835940) and (4.679690,10.941400)..(4.996090,11.019600)
	..controls (5.312490,11.097700) and (5.824210,10.121100)..(6.144520,8.839900)
	--cycle;
	\fill (7.292960,5.261780)..controls (7.382800,4.898500) and (7.128900,4.523500)..(6.730460,4.421880)
	..controls (6.328120,4.324220) and (5.929680,4.535220)..(5.835930,4.898500)
	..controls (5.746080,5.261780) and (5.999990,5.640630)..(6.402340,5.738340)
	..controls (6.804690,5.839840) and (7.203110,5.625060)..(7.292960,5.261780)
	--cycle;
	\end{tikzpicture}}\newcommand{\alors}{\Large\Rightarrow}\newcommand{\equi}{\Leftrightarrow}
\newcommand{\fonction}[5]{\begin{array}{l|rcl}
		#1: & #2 & \longrightarrow & #3 \\
		& #4 & \longmapsto & #5 \end{array}}


\definecolor{vert}{RGB}{11,160,78}
\definecolor{rouge}{RGB}{255,120,120}
\definecolor{bleu}{RGB}{15,5,107}



\pagestyle{fancy}
\lhead{Groupe IPESUP}\chead{}\rhead{Année~2022-2023}\lfoot{M. Botcazou \& M.Dupré}\cfoot{\thepage/3}\rfoot{PCSI }\renewcommand{\headrulewidth}{0.4pt}\renewcommand{\footrulewidth}{0.4pt}


\begin{document}
 %%%%BIBMATH%%%%
 
 %(1) https://www.bibmath.net/ressources/index.php?action=affiche&quoi=mpsi/feuillesexo/integration&type=fexo

\noindent\shadowbox{
	\begin{minipage}{1\linewidth}
		\centering
		\huge{\textbf{ TD 16 : Intégration }}
	\end{minipage}}

\smallskip
\section*{Connaître son cours:}
\begin{itemize}[$\bullet$]
	\item Soit $f$ une fonction continue sur $[a , b ] $, positive et non nulle en au moins un point de $[a , b ] $.
	
	Alors\quad  $\displaystyle\int_{a}^{b}f(t) ~ dt > 0$
	\item Soit $f$ et $g$ deux fonctions continues par morceaux sur $[a , b ] $.
	 
	Alors, \quad  $\displaystyle\left|\int_{a}^{b}f(t)g(t) ~ dt\right| \leq  \left(\int_{a}^{b}f(t)^2 ~ dt\right)^{\frac{1}{2}} \times \left(\int_{a}^{b}g(t)^2 ~ dt\right)^{\frac{1}{2}}$
	\item Soit $b > a $. Calculer $\displaystyle\int_{a}^{b} e^t ~ dt$ avec les sommes de Riemann.
	\item Montrer que l’intégrale d’une fonction impaire sur un segment symétrique par rapport à $0$ est nulle.
	\item Soit $f$ une fonction de classe $\mathcal C^{n+1}$ sur $I$ un intervalle réel et $a \in I $. Donner la formule de Taylor avec reste intégral en $a$. 
	

\end{itemize}
\raggedright

\section*{Propriétés de l’intégrale:}\hfill\\%[-0.25cm]

   
\begin{minipage}{1\linewidth}\begin{minipage}[t]{0.48\linewidth}\raggedright
	
\begin{exo}\textbf{(*)}\quad\\[0.2cm]
Calculer les primitives des fonctions suivantes en précisant le ou les intervalles considérés~:

\begin{multicols}{2}
\begin{enumerate}
\item $\dfrac{1}{x^3+1}$
\item $\dfrac{x^2}{x^3+1}$
\end{enumerate}
\end{multicols}
\quad \quad \quad \quad \quad \quad \quad $3. \ \ \dfrac{1}{x(x^2+1)^2}$\vspace{0.25cm}
	
\centering\rule{1\linewidth}{0.6pt}\end{exo}



\begin{exo}\textbf{(*)}\quad\\[0.2cm]
	Calculer les primitives des fonctions suivantes en précisant le ou les intervalles considérés~:

	
	\begin{multicols}{2}
		\begin{enumerate}
			\item $\dfrac{1}{\cos x}$
			\item $\dfrac{\sin^2(x/2)}{x-\sin x}$
			\item $\dfrac{1}{\cos^4x+\sin^4x}$
			\item $\dfrac{\sin x}{\cos(3x)}$
		\end{enumerate}
	\end{multicols}

	\centering\rule{1\linewidth}{0.6pt}\end{exo}

\begin{exo}\textbf{(**)}\quad\\[0.2cm]
	Faire une étude de la fonction $$f(x)=\int_{-1}^{1}\frac{\sin x}{1-2t\cos x+t^2}\;dt$$ 
	\centering\rule{1\linewidth}{0.6pt}\end{exo}



%%%%%%%%%%%%%%%%%%%%%%%%%%%%%%%%%%%%%%%%%%%%%%%%%%%%%%%%%%%%%%%%%%%%%%%%%%%%%%%%%%%%%%%%%%
\end{minipage}\hfill\vrule\hfill\begin{minipage}[t]{0.48\linewidth}\raggedright
%%%%%%%%%%%%%%%%%%%%%%%%%%%%%%%%%%%%%%%%%%%%%%%%%%%%%%%%%%%%%%%%%%%%%%%%%%%%%%%%%%%%%%%%%%

\begin{exo}\textbf{(**)}\quad\\[0.2cm]
Faire une étude de la fonction

$f(x)=\int_{0}^{1}\mbox{Max}(x,t)\;dt$
\centering\rule{1\linewidth}{0.6pt}\end{exo}


\begin{exo}\textbf{(**)}\quad\\[0.2cm]
\begin{enumerate}
	\item  Soit $f$ une application de classe $C^1$ sur $[0,1]$ telle que $f(1)\neq0$.
	
	Pour $n\in\N$, on pose $u_n=\int_{0}^{1}t^nf(t)\;dt$. 
	
	Montrer que $\lim\limits_{n\rightarrow +\infty}u_n=0$ \ puis déterminer un équivalent simple de $u_n$ quand $n$ tend vers $+\infty$ (étudier $\lim\limits_{n\rightarrow +\infty}nu_n$).
	\item  Mêmes questions en supposant que $f$ est de classe $C^2$ sur $[0,1]$ et que $f(1)=0$ et $f'(1)\neq0$.
\end{enumerate}

\centering\rule{1\linewidth}{0.6pt}\end{exo}

	\begin{exo}\textbf{(**)/(***)}\quad (\textit{Lemme de \textsc{Lebesgue}})\\[0.2cm]
	\begin{enumerate}
		\item  On suppose que $f$ est une fonction de classe $C^1$ sur $[a,b]$. Montrer que $\lim_{\lambda\rightarrow +\infty}\int_{a}^{b}\sin(\lambda t)f(t)\;dt=0$.
		\item Redémontrer le même résultat en supposant simplement que $f$ est continue par morceaux sur $[a,b]$ (commencer par le cas des fonctions en escaliers).
	\end{enumerate}
	
	
	\centering\rule{1\linewidth}{0.6pt}\end{exo}


\end{minipage}\end{minipage} \newpage
\begin{minipage}{1\linewidth}\begin{minipage}[t]{0.48\linewidth}\raggedright
		

\begin{exo}\textbf{(*)}\quad\\[0.2cm]
	Soit $f,g\in\mathcal{C}^0_m([0,1],\R^+), \forall x\in[0,1],\;f(x)g(x)\geq1$. 
	
	Montrer que $\left(\int_{0}^{1}f(t)\;dt\right)\left(\int_{0}^{1}g(t)\;dt\right)\geq1$.
	
	\centering\rule{1\linewidth}{0.6pt}\end{exo}

\begin{exo}\textbf{(**)}\quad\\[0.2cm]
Soit $E$ l'ensemble des fonctions continues strictement positives sur $[a,b]$.

Soit $\begin{array}[t]{cccl}
\varphi~:&E&\rightarrow&\R\\
&f&\mapsto&\displaystyle\left(\int_{a}^{b}f(t)\;dt\right)\left(\int_{a}^{b}\frac{1}{f(t)}\;dt\right)
\end{array}$.

\begin{enumerate}
	\item  Montrer que $\varphi(E)$ n'est pas majoré.
	\item  Montrer que $\varphi(E)$ est minoré. Trouver $m=\mbox{Inf}\{\varphi(f),\;f\in E\}$. Montrer que cette borne infèrieure est atteinte et trouver toutes les $f$ de $E$ telles que $\varphi(f)=m$.
\end{enumerate}	
	
	
	\centering\rule{1\linewidth}{0.6pt}\end{exo}



\begin{exo}\textbf{(**)}\quad\\[0.2cm]
Soit $f$ une fonction de classe $C^1$ sur $[0,1]$ telle que $f(0)=0$. Montrer que $\displaystyle2\int_{0}^{1}f^2(t)\;dt\leq\int_{0}^{1}{f'}^2(t)\;dt$. 
	
	\centering\rule{1\linewidth}{0.6pt}\end{exo}


\begin{exo}\textbf{(**)}\quad\\[0.2cm]
	Soit $f$ continue sur $[0,1]$ telle que $\displaystyle\int_{0}^{1}f(t)\;dt=\frac{1}{2}$. Montrer que $f$ admet un point fixe.
	
	\centering\rule{1\linewidth}{0.6pt}\end{exo}


\begin{exo}\textbf{(**)}\quad\\[0.2cm]
	Déterminer les limites quand $n$ tend vers $+\infty$ de
	\begin{enumerate}
		\item $u_n = \displaystyle\frac{1}{n!}\int_{0}^{1}\text{Arcsin}^n(x)\;dx$
		\item $v_n = \displaystyle\int_{0}^{1}\frac{x^n}{1+x}\;dx$
		
	\end{enumerate}
	

	
	\centering\rule{1\linewidth}{0.6pt}\end{exo}


\begin{exo}\textbf{(**)}\quad\\[0.2cm]
	\begin{enumerate}
		\item Démontrer que la fonction $\sin$ est lipschitzienne sur $\mathbb R$.
		\item Soit $f:[a,b]\to\mathbb R$ continue. 
		
		Démontrer que la fonction $F:\mathbb R\to\mathbb R$ définie par 
		
		\centering$F(x)=\displaystyle\int_a^b f(t)\sin(xt)dt$\\
		
		
		\raggedright
		est lipschitzienne.
	\end{enumerate}	
	
	
	\centering\rule{1\linewidth}{0.6pt}\end{exo}




%%%%%%%%%%%%%%%%%%%%%%%%%%%%%%%%%%%%%%%%%%%%%%%%%%%%%%%%%%%%%%%%%%%%%%%%%%%%%%%%%%%%%%%%%%
\end{minipage}\hfill\vrule\hfill\begin{minipage}[t]{0.48\linewidth}\raggedright
%%%%%%%%%%%%%%%%%%%%%%%%%%%%%%%%%%%%%%%%%%%%%%%%%%%%%%%%%%%%%%%%%%%%%%%%%%%%%%%%%%%%%%%%%%

\begin{exo}\textbf{(**)}\quad\\[0.2cm]
	Soit $f$ une fonction de classe $C^1$ sur $[a,b]$ telle que $f(a)=f(b)=0$ et soit $\displaystyle M=\mbox{sup}\{|f '(x)|,\;x\in[a,b]\}$. Montrer que $\displaystyle\left|\int_{a}^{b}f(x)\;dx\right|\leq M\frac{(b-a)^2}{4}$. 	
	
	\centering\rule{1\linewidth}{0.6pt}\end{exo}


\begin{exo}\textbf{(**)}\quad\\[0.2cm]
	Déterminer les fonctions $f$ continues sur $[0,1]$ vérifiant $\displaystyle\left|\int_{0}^{1}f(t)\;dt\right|=\int_{0}^{1}|f(t)|\;dt$.	
	
	\centering\rule{1\linewidth}{0.6pt}\end{exo}




\begin{exo}\textbf{(**)}\quad\\[0.2cm]
	\begin{itemize}
		\item  Déterminer $$\displaystyle\lim_{x\rightarrow 1}\int_{x}^{x^2}\frac{dt}{\ln t}$$ 
		\item  Faire une étude complète de $$\displaystyle F(x)=\int_{x}^{x^2}\frac{dt}{\ln t}$$
	\end{itemize}
	\centering\rule{1\linewidth}{0.6pt}\end{exo}

\begin{exo}\textbf{(**)}\quad\\[0.2cm]
	Soit $f:[a,b]\to\R$ continue telle que, pour tout couple $(\alpha,\beta)\in[a,b]^2$, on a $$\displaystyle \int_{\alpha}^{\beta}f(x)dx=0$$ Montrer que $f=0$.
	
	
	
\centering\rule{1\linewidth}{0.6pt}\end{exo}


\begin{exo}\textbf{(***)}\quad(\textit{\textsc{Cesaro} pour les intégrales})\\[0.2cm]
	Soit $f:[0,+\infty[\to\mathbb R$ une fonction continue admettant une limite finie $a$ en $+\infty$.
	
	Montrer que
	
	$$\displaystyle\dfrac 1x\int_0^x f(t)dt\to a\quad\textrm{  quand  }\quad x\to+\infty.$$
	
	
	\centering\rule{1\linewidth}{0.6pt}\end{exo}

\begin{exo}\textbf{(***)}\quad\\[0.2cm]
	Soient $f$ et $g$ deux fonctions continues et strictement positives sur $[a,b]$. Pour $n$ entier naturel non nul donné, on pose $u_n=\left(\int_{a}^{b}(f(x))^ng(x)\;dx\right)^{1/n}$.
	
	Montrer que la suite $(u_n)$ converge et déterminer sa limite (commencer par le cas $g=1$).
	
	\centering\rule{1\linewidth}{0.6pt}\end{exo}










\end{minipage}\end{minipage} \newpage

\section*{Sommes de Riemann:}\hfill\\%[-0.25cm]


\begin{minipage}{1\linewidth}\begin{minipage}[t]{0.48\linewidth}\raggedright
	
\begin{exo}\textbf{(**)}\quad\\[0.2cm]
Calculer la limite des suites suivantes :
\begin{enumerate}
\item $\displaystyle u_n=\frac 1n\left(\sin\left(\frac{\pi}{n}\right)+\sin\left(\frac{2\pi}{n}\right)+\dots+\sin\left(\frac{n\pi}{n}\right)\right)$
\item $\displaystyle  u_n=\sqrt[n]{\left(1+\left(\frac{1}{n}\right)^2\right)\left(1+\left(\frac{2}{n}\right)^2\right)\dots\left(1+\left(\frac{n}{n}\right)^2\right)}$
\end{enumerate}		

\centering\rule{1\linewidth}{0.6pt}\end{exo}



\begin{exo}\textbf{(**)}\quad\\[0.2cm]
Déterminer la limite de \quad  $\displaystyle v_n=\frac1n\prod_{k=1}^n (k+n)^{1/n}.$

\centering\rule{1\linewidth}{0.6pt}\end{exo}

\begin{exo}\textbf{(**)}\quad\\[0.2cm]
Donner les limites de 
\begin{multicols}{2}
\begin{enumerate}
	\item $\displaystyle \frac{1}{n^3}\sum_{k=1}^{n}k^2\sin\frac{k\pi}{n}$
	\item $\displaystyle \sum_{k=1}^{n}\frac{n+k}{n^2+k}$
	\item $ \displaystyle \frac{1}{n\sqrt{n}}\sum_{k=1}^{n}E(\sqrt{k})$
\end{enumerate}
\end{multicols}

	
\centering\rule{1\linewidth}{0.6pt}\end{exo}


\begin{exo}\textbf{(***)}\quad(\textit{Inégalité de \textsc{Jensen}})\\[0.2cm]
Soit $f:[a,b]\to\mathbb R$ continue et $g:\mathbb R\to\mathbb R$ continue et convexe. Démontrer que 
$$g\left(\frac{1}{b-a}\int_a^b f(t)dt \right)\leq \frac{1}{b-a}\int_a^b g(f(t))dt.$$

\centering\rule{1\linewidth}{0.6pt}\end{exo}

		
		
		
%%%%%%%%%%%%%%%%%%%%%%%%%%%%%%%%%%%%%%%%%%%%%%%%%%%%%%%%%%%%%%%%%%%%%%%%%%%%%%%%%%%%%%%%%%
\end{minipage}\hfill\vrule\hfill\begin{minipage}[t]{0.48\linewidth}\raggedright
%%%%%%%%%%%%%%%%%%%%%%%%%%%%%%%%%%%%%%%%%%%%%%%%%%%%%%%%%%%%%%%%%%%%%%%%%%%%%%%%%%%%%%%%%%

\begin{exo}\textbf{(**)}\quad\\[0.2cm]
Soit $$\displaystyle I_{n} = \int_0^1 \frac{x^n}{1 + x}d x$$
\begin{enumerate}
	\item En majorant la fonction int\'egr\'ee, 
	
	montrer que $\displaystyle \lim_{n\to +\infty} I_{n}=0$.
	\item  Calculer $\displaystyle I_n + I_{n + 1}$.
	\item D\'eterminer $$\displaystyle \lim_{n \rightarrow  + \infty} \left(\sum_{k = 1}^n \frac{(-1)^{k + 1}}k\right)$$
\end{enumerate}	
	
	\centering\rule{1\linewidth}{0.6pt}\end{exo}


\begin{exo}\textbf{(**)}\quad\\[0.2cm]
	Étudier la suite $$u_n = \sum_{k=1}^{n}\sin\frac{k}{n}$$
	
\centering\rule{1\linewidth}{0.6pt}\end{exo}


\begin{exo}\textbf{(***)}\quad\\[0.2cm]
	Partie principale quand $n$ tend vers $+\infty$ de $$u_n=\sum_{k=1}^{n}\sin\frac{1}{(n+k)^2}$$
	
	\centering\rule{1\linewidth}{0.6pt}\end{exo}		
		
\end{minipage}\end{minipage} 

\end{document}