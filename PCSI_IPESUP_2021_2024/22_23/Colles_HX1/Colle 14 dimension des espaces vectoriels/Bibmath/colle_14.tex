\label{key}\documentclass[11pt]{article}

 %Configuration de la feuille 
 
\usepackage{amsmath,amssymb,enumerate,graphicx,pgf,tikz,fancyhdr}
\usepackage[utf8]{inputenc}
\usetikzlibrary{arrows}
\usepackage{geometry}
\usepackage{tabvar}
\geometry{hmargin=2.2cm,vmargin=1.5cm}\pagestyle{fancy}
\lfoot{\bfseries http://www.bibmath.net}
\rfoot{\bfseries\thepage}
\cfoot{}
\renewcommand{\footrulewidth}{0.5pt} %Filet en bas de page

 %Macros utilisées dans la base de données d'exercices 

\newcommand{\mtn}{\mathbb{N}}
\newcommand{\mtns}{\mathbb{N}^*}
\newcommand{\mtz}{\mathbb{Z}}
\newcommand{\mtr}{\mathbb{R}}
\newcommand{\mtk}{\mathbb{K}}
\newcommand{\mtq}{\mathbb{Q}}
\newcommand{\mtc}{\mathbb{C}}
\newcommand{\mch}{\mathcal{H}}
\newcommand{\mcp}{\mathcal{P}}
\newcommand{\mcb}{\mathcal{B}}
\newcommand{\mcl}{\mathcal{L}}
\newcommand{\mcm}{\mathcal{M}}
\newcommand{\mcc}{\mathcal{C}}
\newcommand{\mcmn}{\mathcal{M}}
\newcommand{\mcmnr}{\mathcal{M}_n(\mtr)}
\newcommand{\mcmnk}{\mathcal{M}_n(\mtk)}
\newcommand{\mcsn}{\mathcal{S}_n}
\newcommand{\mcs}{\mathcal{S}}
\newcommand{\mcd}{\mathcal{D}}
\newcommand{\mcsns}{\mathcal{S}_n^{++}}
\newcommand{\glnk}{GL_n(\mtk)}
\newcommand{\mnr}{\mathcal{M}_n(\mtr)}
\DeclareMathOperator{\ch}{ch}
\DeclareMathOperator{\sh}{sh}
\DeclareMathOperator{\vect}{vect}
\DeclareMathOperator{\card}{card}
\DeclareMathOperator{\comat}{comat}
\DeclareMathOperator{\imv}{Im}
\DeclareMathOperator{\rang}{rg}
\DeclareMathOperator{\Fr}{Fr}
\DeclareMathOperator{\diam}{diam}
\DeclareMathOperator{\supp}{supp}
\newcommand{\veps}{\varepsilon}
\newcommand{\mcu}{\mathcal{U}}
\newcommand{\mcun}{\mcu_n}
\newcommand{\dis}{\displaystyle}
\newcommand{\croouv}{[\![}
\newcommand{\crofer}{]\!]}
\newcommand{\rab}{\mathcal{R}(a,b)}
\newcommand{\pss}[2]{\langle #1,#2\rangle}
 %Document 

\begin{document} 

\begin{center}\textsc{{\huge }}\end{center}

% Exercice 3095


\vskip0.3cm\noindent\textsc{Exercice 1} - Eléments caractéristiques d'une projection
\vskip0.2cm
Soit $f$ l'endomorphisme de $\mathbb R^3$ tel que $f(x,y,z)=(-3x+2y-4z,2x+2z,4x-2y+5z)$. Montrer que $f$ est la projection sur un plan $P$ parallèlement à une droite $D$. Donner une équation cartésienne du plan $P$ et un vecteur directeur de $D$.


% Exercice 855


\vskip0.3cm\noindent\textsc{Exercice 2} - Endomorphismes annulant un polynôme de degré 2
\vskip0.2cm
Soit $f\in\mathcal L(E)$ et soient $\alpha,\beta$ deux réels distincts.
\begin{enumerate}
\item Démontrer que $E=\textrm{Im}(f-\alpha Id_E)+\textrm{Im}(f-\beta Id_E)$.\newline
On suppose de plus que $\alpha$ et $\beta$ sont non nuls et que $$(f-\alpha Id_E)\circ (f-\beta Id_E)=0.$$
\item Démontrer que $f$ est inversible, et calculer $f^{-1}$.
\item Démontrer que $E=\ker(f-\alpha Id_E)\oplus \ker(f-\beta Id_E)$.
\item Exprimer en fonction de $f$ le projecteur $p$ sur $\ker(f-\alpha Id_E)$ parallèlement
à $\ker(f-\beta Id_E)$.
\end{enumerate}


% Exercice 905


\vskip0.3cm\noindent\textsc{Exercice 3} - Noyaux itérés
\vskip0.2cm
Soit $E$ un espace vectoriel de dimension finie $n$ et soit $f\in\mathcal L(E)$.
\begin{enumerate}
\item Soit $k\geq 1$. Démontrer que $\ker(f^{k})\subset \ker(f^{k+1})$ 
et $\textrm{Im}(f^{k+1})\subset \textrm{Im}(f^k).$
\item \begin{enumerate}
\item Démontrer que si $\ker(f^k)=\ker(f^{k+1})$, alors $\ker(f^{k+1})= \ker(f^{k+2})$.
\item Démontrer qu'il existe $p\in\mathbb N$ tel que
\begin{itemize}
\item si $k<p$, alors $\ker(f^k)\neq \ker(f^{k+1})$;
\item si $k\geq p$, alors $\ker(f^k)= \ker(f^{k+1})$.
\end{itemize}
\item Démontrer que $p\leq n$;
\end{enumerate}
\item Démontrer que si $k<p$, alors $\textrm{Im}(f^k)\neq \textrm{Im}(f^{k+1})$ et
si $k\geq p$, alors $\textrm{Im}(f^k)=\textrm{Im}(f^{k+1})$.
\item Démontrer que $\ker(f^p)$ et $\textrm{Im}(f^p)$ sont supplémentaires.
\item Démontrer qu'il existe deux sous-espaces $F$ et $G$ de $E$ tels que $F$ et $G$ sont supplémentaires, $f_{|F}$ est nilpotent et $f_{|G}$ induit un automorphisme de $G$.
\item Soit $d_k=\dim\big(\textrm{Im}(f^k)\big)$. Montrer que la suite $(d_k-d_{k+1})$ est décroissante.
\end{enumerate}




\vskip0.5cm
\noindent{\small Cette feuille d'exercices a été conçue à l'aide du site \textsf{https://www.bibmath.net}}

%Vous avez accès aux corrigés de cette feuille par l'url : https://www.bibmath.net/ressources/justeunefeuille.php?id=27491
\end{document}