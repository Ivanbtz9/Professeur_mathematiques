\documentclass[11pt]{article}

 %Configuration de la feuille 
 
\usepackage{amsmath,amssymb,enumerate,graphicx,pgf,tikz,fancyhdr}
\usepackage[utf8]{inputenc}
\usetikzlibrary{arrows}
\usepackage{geometry}
\usepackage{tabvar}
\geometry{hmargin=2.2cm,vmargin=1.5cm}\pagestyle{fancy}
\lfoot{\bfseries http://www.bibmath.net}
\rfoot{\bfseries\thepage}
\cfoot{}
\renewcommand{\footrulewidth}{0.5pt} %Filet en bas de page

 %Macros utilisées dans la base de données d'exercices 

\newcommand{\mtn}{\mathbb{N}}
\newcommand{\mtns}{\mathbb{N}^*}
\newcommand{\mtz}{\mathbb{Z}}
\newcommand{\mtr}{\mathbb{R}}
\newcommand{\mtk}{\mathbb{K}}
\newcommand{\mtq}{\mathbb{Q}}
\newcommand{\mtc}{\mathbb{C}}
\newcommand{\mch}{\mathcal{H}}
\newcommand{\mcp}{\mathcal{P}}
\newcommand{\mcb}{\mathcal{B}}
\newcommand{\mcl}{\mathcal{L}}
\newcommand{\mcm}{\mathcal{M}}
\newcommand{\mcc}{\mathcal{C}}
\newcommand{\mcmn}{\mathcal{M}}
\newcommand{\mcmnr}{\mathcal{M}_n(\mtr)}
\newcommand{\mcmnk}{\mathcal{M}_n(\mtk)}
\newcommand{\mcsn}{\mathcal{S}_n}
\newcommand{\mcs}{\mathcal{S}}
\newcommand{\mcd}{\mathcal{D}}
\newcommand{\mcsns}{\mathcal{S}_n^{++}}
\newcommand{\glnk}{GL_n(\mtk)}
\newcommand{\mnr}{\mathcal{M}_n(\mtr)}
\DeclareMathOperator{\ch}{ch}
\DeclareMathOperator{\sh}{sh}
\DeclareMathOperator{\vect}{vect}
\DeclareMathOperator{\card}{card}
\DeclareMathOperator{\comat}{comat}
\DeclareMathOperator{\imv}{Im}
\DeclareMathOperator{\rang}{rg}
\DeclareMathOperator{\Fr}{Fr}
\DeclareMathOperator{\diam}{diam}
\DeclareMathOperator{\supp}{supp}
\newcommand{\veps}{\varepsilon}
\newcommand{\mcu}{\mathcal{U}}
\newcommand{\mcun}{\mcu_n}
\newcommand{\dis}{\displaystyle}
\newcommand{\croouv}{[\![}
\newcommand{\crofer}{]\!]}
\newcommand{\rab}{\mathcal{R}(a,b)}
\newcommand{\pss}[2]{\langle #1,#2\rangle}
 %Document 

\begin{document} 

\begin{center}\textsc{{\huge }}\end{center}

% Exercice 589


\vskip0.3cm\noindent\textsc{Exercice 1} - Exponentielle et équivalent
\vskip0.2cm
Soient $f$ et $g$ deux fonctions définies au voisinage d'un réel $a$ ou de $a=\pm\infty$. Montrer que
$e^f\sim_a e^g\iff \lim_a(f-g)=0$.
A-t-on $f\sim_a g\implies e^f\sim_a e^g$?


% Exercice 564


\vskip0.3cm\noindent\textsc{Exercice 2} - Quotient de DLs
\vskip0.2cm
Déterminer les développements limités des fonctions suivantes :
$$\begin{array}{lcl}
\displaystyle \mathbf 1.\ \frac{1}{1+x+x^2}\textrm{ à l'ordre 4 en 0}&&\displaystyle \mathbf 2.\ \tan(x)\textrm{ à l'ordre 5 en 0}\\
\displaystyle \mathbf 3.\ \frac{\sin x-1}{\cos x+1}\textrm{ à l'ordre 2 en 0}&&\displaystyle \mathbf 4.\ \frac{\ln(1+x)}{\sin x}\textrm{ à l'ordre 3 en 0}.
\end{array}$$


% Exercice 563


\vskip0.3cm\noindent\textsc{Exercice 3} - Composition de DLs
\vskip0.2cm
Calculer les développements limités suivants :
$$\begin{array}{lcl}
\displaystyle \mathbf 1.\ \ln\left(\frac{\sin x}{x}\right)\textrm{ à l'ordre 4 en 0}&&
\displaystyle \mathbf 2.\ \exp(\sin x)\textrm{ à l'ordre 4 en 0}\\
\displaystyle \mathbf 3.\ (\cos x)^{\sin x}\textrm{ à l'ordre 5 en 0}&&
\displaystyle \mathbf 4.\ x\big(\cosh x\big)^{\frac 1x}\textrm{ à l'ordre 4 en 0}.
\end{array}$$


% Exercice 807


\vskip0.3cm\noindent\textsc{Exercice 4} - Intégration de DLs
\vskip0.2cm
Calculer les développements limités suivants :
$$\begin{array}{lcl}
\displaystyle \mathbf 1.\ \arccos x\textrm{ à l'ordre 5 en 0}&&
\displaystyle \mathbf 2.\ \int_0^x e^{t^2}dt\textrm{ à l'ordre 5 en 0}.
\end{array}
$$


% Exercice 565


\vskip0.3cm\noindent\textsc{Exercice 5} - DLs pas en 0!
\vskip0.2cm
Calculer les développements limités suivants :
$$\begin{array}{lcl}
\mathbf 1. \frac 1x\textrm{ à l'ordre 3 en }2&&\displaystyle \mathbf 2. \ln(x)\textrm{ à l'ordre 3 en }2\\
\displaystyle \mathbf 3. e^x\textrm{ à l'ordre 3 en }1&&\displaystyle \mathbf 4. \cos(x)\textrm{ à l'ordre 3 en }\frac{\pi}3\\
\displaystyle \mathbf 5. \sqrt x\textrm{ à l'ordre 3 en 2}
\end{array}$$


% Exercice 806


\vskip0.3cm\noindent\textsc{Exercice 6} - DL en l'infini
\vskip0.2cm
Calculer les développements limités suivants :
$$\begin{array}{lcl}
\mathbf 1. \frac{\sqrt{x+2}}{\sqrt x}\textrm{ à l'ordre 3 en }+\infty&&
\displaystyle \mathbf 2. \ln\left(x+\sqrt {1+x^2}\right)-\ln x\textrm{ à l'ordre 4 en }+\infty
\end{array}$$


% Exercice 567


\vskip0.3cm\noindent\textsc{Exercice 7} - Astucieux!
\vskip0.2cm
Calculer, à l'ordre 100, le développement limité en 0 de $\ln\left(\sum_{k=0}^{99}\frac{x^k}{k!}\right)$.


% Exercice 808


\vskip0.3cm\noindent\textsc{Exercice 8} - Développement limité d'une fonction réciproque
\vskip0.2cm
Pour $x\in\mathbb R$, on pose $f(x)=x\exp(x^2)$.
\begin{enumerate}
\item Démontrer que $f$ réalise une bijection de $\mathbb R$ sur $\mathbb R$.
\item Justifier que $f^{-1}$ admet un développement limité à l'ordre $4$ en $0$.
\item Donner ce développement limité.
\end{enumerate}




\vskip0.5cm
\noindent{\small Cette feuille d'exercices a été conçue à l'aide du site \textsf{https://www.bibmath.net}}

%Vous avez accès aux corrigés de cette feuille par l'url : https://www.bibmath.net/ressources/justeunefeuille.php?id=26649
\end{document}