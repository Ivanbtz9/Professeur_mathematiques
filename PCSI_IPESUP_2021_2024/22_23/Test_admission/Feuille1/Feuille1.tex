\documentclass[a4paper,11pt]{article}

\usepackage{inputenc}
\usepackage[T1]{fontenc}
\usepackage[frenchb]{babel}
\usepackage{fancyhdr,fancybox} % pour personnaliser les en-têtes
\usepackage{lastpage,setspace}
\usepackage{amsfonts,amssymb,amsmath,amsthm,mathrsfs}
\usepackage{relsize,exscale,bbold}
\usepackage{paralist}
\usepackage{xspace,multicol,diagbox,array}
\usepackage{xcolor}
\usepackage{variations}
\usepackage{xypic}
\usepackage{eurosym,stmaryrd}
\usepackage{graphicx}
\usepackage[np]{numprint}
\usepackage{hyperref} 
\usepackage{tikz}
\usepackage{colortbl}
\usepackage{multirow}
\usepackage{MnSymbol,wasysym}
\usepackage[top=1.5cm,bottom=1.5cm,right=1.2cm,left=1.5cm]{geometry}
\usetikzlibrary{calc, arrows, plotmarks, babel,decorations.pathreplacing}
\setstretch{1.25}
%\usepackage{lipsum} %\usepackage{enumitem} %\setlist[enumerate]{itemsep=1mm} bug avec enumerate



\newtheorem{thm}{Théorème}
\newtheorem{rmq}{Remarque}
\newtheorem{prop}{Propriété}
\newtheorem{cor}{Corollaire}
\newtheorem{lem}{Lemme}
\newtheorem{prop-def}{Propriété-définition}

\theoremstyle{definition}

\newtheorem{defi}{Définition}
\newtheorem{ex}{Exemple}
\newtheorem*{rap}{Rappel}
\newtheorem{cex}{Contre-exemple}
\newtheorem{exo}{Exercice} % \large {\fontfamily{ptm}\selectfont EXERCICE}
\newtheorem{nota}{Notation}
\newtheorem{ax}{Axiome}
\newtheorem{appl}{Application}
\newtheorem{csq}{Conséquence}
\def\di{\displaystyle}



\renewcommand{\thesection}{\Roman{section}}\renewcommand{\thesubsection}{\arabic{subsection} }\renewcommand{\thesubsubsection}{\alph{subsubsection} }


\newcommand{\bas}{~\backslash}\newcommand{\ba}{\backslash}
\newcommand{\C}{\mathbb{C}}\newcommand{\R}{\mathbb{R}}\newcommand{\Q}{\mathbb{Q}}\newcommand{\Z}{\mathbb{Z}}\newcommand{\N}{\mathbb{N}}\newcommand{\V}{\overrightarrow}\newcommand{\Cs}{\mathscr{C}}\newcommand{\Ps}{\mathscr{P}}\newcommand{\Rs}{\mathscr{R}}\newcommand{\Gs}{\mathscr{G}}\newcommand{\Ds}{\mathscr{D}}\newcommand{\happy}{\huge\smiley}\newcommand{\sad}{\huge\frownie}\newcommand{\danger}{\begin{tikzpicture}[x=1.5pt,y=1.5pt,rotate=-14.2]
	\definecolor{myred}{rgb}{1,0.215686,0}
	\draw[line width=0.1pt,fill=myred] (13.074200,4.937500)--(5.085940,14.085900)..controls (5.085940,14.085900) and (4.070310,15.429700)..(3.636720,13.773400)
	..controls (3.203130,12.113300) and (0.917969,2.382810)..(0.917969,2.382810)
	..controls (0.917969,2.382810) and (0.621094,0.992188)..(2.097660,1.359380)
	..controls (3.574220,1.726560) and (12.468800,3.984380)..(12.468800,3.984380)
	..controls (12.468800,3.984380) and (13.437500,4.132810)..(13.074200,4.937500)
	--cycle;
	\draw[line width=0.1pt,fill=white] (11.078100,5.511720)--(5.406250,11.875000)..controls (5.406250,11.875000) and (4.683590,12.812500)..(4.367190,11.648400)
	..controls (4.050780,10.488300) and (2.375000,3.675780)..(2.375000,3.675780)
	..controls (2.375000,3.675780) and (2.156250,2.703130)..(3.214840,2.964840)
	..controls (4.273440,3.230470) and (10.640600,4.847660)..(10.640600,4.847660)
	..controls (10.640600,4.847660) and (11.332000,4.953130)..(11.078100,5.511720)
	--cycle;
	\fill (6.144520,8.839900)..controls (6.460940,7.558590) and (6.464840,6.457090)..(6.152340,6.378910)
	..controls (5.835930,6.300840) and (5.320300,7.277400)..(5.003900,8.554750)
	..controls (4.683590,9.835940) and (4.679690,10.941400)..(4.996090,11.019600)
	..controls (5.312490,11.097700) and (5.824210,10.121100)..(6.144520,8.839900)
	--cycle;
	\fill (7.292960,5.261780)..controls (7.382800,4.898500) and (7.128900,4.523500)..(6.730460,4.421880)
	..controls (6.328120,4.324220) and (5.929680,4.535220)..(5.835930,4.898500)
	..controls (5.746080,5.261780) and (5.999990,5.640630)..(6.402340,5.738340)
	..controls (6.804690,5.839840) and (7.203110,5.625060)..(7.292960,5.261780)
	--cycle;
	\end{tikzpicture}}\newcommand{\alors}{\Large\Rightarrow}\newcommand{\equi}{\Leftrightarrow}
\newcommand{\fonction}[5]{\begin{array}{l|rcl}
		#1: & #2 & \longrightarrow & #3 \\
		& #4 & \longmapsto & #5 \end{array}}


\definecolor{vert}{RGB}{11,160,78}
\definecolor{rouge}{RGB}{255,120,120}
\definecolor{bleu}{RGB}{15,5,107}



\pagestyle{fancy}
\lhead{Groupe IPESUP}\chead{}\rhead{Année~2022-2023}\lfoot{M. Botcazou}\cfoot{\thepage/4}\rfoot{PCSI }\renewcommand{\headrulewidth}{0.4pt}\renewcommand{\footrulewidth}{0.4pt}


\begin{document}
 	
	

\noindent\shadowbox{
	\begin{minipage}{1\linewidth}
		\huge{\textbf{ Feuille 1 : Évaluation de niveau sortie terminale }}
	\end{minipage}
}
\bigskip


\raggedright

\section*{Rédaction, types de raisonnements et vocabulaire ensembliste:}%\hfill\\%[-0.25cm]

		
		
\begin{exo}\textbf{(*)}\quad\\[0.2cm]
	Étudier les inclusions $A \subset B$ et $B \subset A$ pour :\\[0.2cm]
	$A = \left\{\dfrac{\epsilon}{k(k+1)} \ | \ k\in\N^*,\ \epsilon\in\{\pm1\}\right\}$\\[0.2cm]
	$B = \left\{\dfrac{1}{p} - \dfrac{1}{q}\ | \ p,q\in\N^*\right\}$
	
	\centering
	\rule{1\linewidth}{0.6pt}
\end{exo}	

\begin{exo}\textbf{(**)}\quad\\[0.2cm]
	Soit $E$ un ensemble, $A$, $B$ et $C$ des parties de $E$ telles que $A \cup B = A \cup C$ et $A \cap B = A \cap C$. 
	
	\noindent Montrer que $B = C$.
	
	\centering
	\rule{1\linewidth}{0.6pt}
\end{exo}

\begin{exo}\textbf{(*)}\quad\\[0.2cm]	
	
	\begin{enumerate}
		\item Montrer que pour tout $n$ dans $\N^*$, la somme des $n$ premiers entiers au carré est donnée par la formule:
		$$\displaystyle \sum_{k=1}^{n}k^2 = 
		\dfrac{n(n + 1)(2n+1)
		}{6}$$ 
		\item Montrer :
		$$\forall n \in \N^*, \ 1+ \dfrac{1}{2^2}+\dfrac{1}{3^2} + ...+ \dfrac{1}{n^2} \leq 2 - \dfrac{1}{n}$$
	\end{enumerate}
	
	\centering
	\rule{1\linewidth}{0.6pt}
\end{exo}


\begin{exo}\textbf{(**)}\quad\\[0.2cm]
Soit $(u_n)_{n\in\N^*} $la suite définie par $u_1=3$ et pour tout \ $n\geq1$, $u_{n+1} = \frac{2}{n}\sum_{k=1}^{n}u_k$.

\noindent Donner l'expression fonctionnelle de la suite $(u_n)_{n\in\N^*}$. 

%Démontrer que, pour tout $n\in\N^*$, on a $u_n=3n$.

\centering
\rule{1\linewidth}{0.6pt}
\end{exo}	

\begin{exo}\textbf{(*)}\quad\\[0.2cm]
	Montrer que $\dfrac{\ln(7)}{\ln(2)}$ est un irrationnel. 
	
	\centering
	\rule{1\linewidth}{0.6pt}
\end{exo}

\begin{exo}\textbf{(**)}\quad\\[0.2cm]
Montrer qu'il existe une infinité de nombres premiers.

	\centering
	\rule{1\linewidth}{0.6pt}
\end{exo}		
		
	
\newpage

\begin{exo}\textbf{(*)}\quad\\[0.2cm]
	Soit $f : \mathbb{R} \longrightarrow \mathbb{R}$ une application vérifiant :
	$$ \forall (x,y,z) \in \mathbb{R}^{3},\  \dfrac{f(x)-f(y)}{x-y} = \dfrac{f(x)-f(z)}{x-z}.$$
	
	\noindent Montrer qu'il existe un unique couple $(a,b) \in \mathbb{R}^{2}$ tel que pour tout $x\in\R$, \ $f(x) = ax+b$.
	
	
	\centering
	\rule{1\linewidth}{0.6pt}
\end{exo}

\begin{exo} \textbf{(**)}\quad\\[0.2cm]
	Soit $f$ une fonction de $\R$ dans $\R$. Montrer qu'existe un unique couple $(p, i)$ de fonctions de $\R$ dans $\R$ vérifiant les conditions suivantes:
	\begin{itemize}
		\item $p$ est paire, $i$ est impaire.
		\item $f = p + i$.
	\end{itemize}
	
	\centering
	\rule{1\linewidth}{0.6pt}
\end{exo}


\begin{exo}\textbf{(***)}\quad\\[0.2cm]
	\begin{enumerate}
		\item D\'emontrer que si $r \in \Q$ et $ x \notin \Q $ alors $ r+x
		\notin \Q $ et si $r\not= 0$ alors $ r\times x \notin \Q $.
		\item Montrer que $\sqrt 2 \not\in\Q$,
		\item En d\'eduire : entre deux nombres rationnels distincts il y a toujours un nombre irrationnel.
		\item En déduire que l'ensemble des nombres irrationnels est dense dans l'ensemble des nombres réels. 
	\end{enumerate}
	\centering
	\rule{1\linewidth}{0.6pt}
\end{exo}
\section*{Sommes, produits et trigonométrie}

\begin{exo}\textbf{(**)}\quad\\[0.2cm]
	Soit $z$ un nombre complexe de module~$\rho$, d'argument~$\theta$. Calculer \ $(z+\bar{z})(z^2+\bar{z}^2) \cdots (z^n+\bar{z}^n)
	$
	
	\raggedright en fonction de~$\rho$ et de~$\theta$.
	
	
	
	\centering
	\rule{1\linewidth}{0.6pt}
\end{exo}

\begin{exo}\textbf{(**)}\quad\\[0.2cm]
	Calculer les sommes suivantes pour $n\in\N$ à l'aide d'un télescopage.
	\begin{multicols}{2}
		\begin{enumerate}
			\item$ \quad\mathlarger\sum\limits_{k=0}^{n}k(3k+1)$
			\item$ \quad\mathlarger\sum\limits_{k=0}^{n} \frac{2^{k-1}}{3^{k+1}}$
			\item$ \quad\mathlarger\sum\limits_{k=0}^{n}(-1)^kk$
			\item$ \quad\mathlarger\sum\limits_{k=0}^{n}k(k+1)(k-1)$
		\end{enumerate}
		
	\end{multicols}
	
	\centering
	\rule{1\linewidth}{0.6pt}
\end{exo}




\begin{exo}\textbf{(**)}\quad\\[0.2cm]
	Soit $x\in\R, n \in \mathbb{N}^{*}$,
	calculer $\displaystyle\sum_{k=0}^{n}{\sin(kx)}$.
	
	\centering
	\rule{1\linewidth}{0.6pt}
\end{exo}
\newpage


\begin{exo}\textbf{(**)}\quad\\[0.2cm]
	
	Calculer les sommes suivantes 
	\begin{enumerate}
		\item $ \displaystyle\sum\limits_{k=0}^{n}{\dfrac{k}{(k+1)!}}$
		\item $ \displaystyle\sum\limits_{k=1}^{n}{\dfrac{1}{\sqrt{k+1} + \sqrt{k}}}$ 
	\end{enumerate}
	
	
	\centering
	\rule{1\linewidth}{0.6pt}
\end{exo}

\begin{exo}\textbf{(**)}\quad\\[0.2cm]
	Déterminer trois réels $a, b, c$ tels que :\quad\\[0.25cm]
	
$$\forall x\in\R\backslash\{0,-1,-2\},\ \dfrac{1}{x(x+1)(x+2)} = \dfrac{a}{x} + \dfrac{b}{x+1} + \dfrac{c}{x+2}$$
	
	Donner pour $n$ dans $\N^*$, une expression simple de\quad\\[0.25cm]
	
	\centering$ U_n = \displaystyle\sum\limits_{k=1}^{n}{\dfrac{1}{k(k+1)(k+2)}}$.\quad\\[0.25cm]
	
	\raggedright Donner la limite de $(U_n)_{n\geq 1}$ lorsque $n$ tend vers $+\infty$?
	
	\centering
	\rule{1\linewidth}{0.6pt}
\end{exo}


\begin{exo}\textbf{(**)}\quad\\[0.2cm]
	
	\noindent Soit $x$ un nombre réel non multiple
	entier de $\pi$. En remarquant que :\quad\\[0.25cm]
	
	\centering
	
	$\forall y\in\R, \ \sin(2y)= 2\sin(y)\cos(y)$\quad\\[0.25cm]
	
	
	
	\raggedright simplifier pour $n$ dans $\N^*$ le produit :\quad\\[0.25cm]
	
	\centering$ P_n(x) = \displaystyle\prod\limits_{k=1}^{n}{\cos\left(\dfrac{x}{2^k}\right)}$.\quad\\[0.25cm]
	
	\raggedright En utilisant la relation suivante après l’avoir justifiée \quad\\[0.25cm]
	
	\centering$ \dfrac{\sin u}{u} \underset{u \rightarrow 0}{\longrightarrow} 1$\quad\\[0.25cm]
	
	\raggedright donner la limite de $P_n(x)$ lorsque $n$ tend vers $+\infty$.
	
	\centering
	\rule{1\linewidth}{0.6pt}
\end{exo}

	\begin{exo}\textbf{(***)}\quad\\[0.2cm]
	Soit $n\in\N^*$, on pose $\displaystyle H_n = \sum_{i=1}^{n}\dfrac{1}{i}$.

	\noindent Montrer pour tout entier $n \geq 2$ que :
	$$\sum_{k=1}^{n-1}{H_k}=n H_n - n$$
	\centering
	\rule{1\linewidth}{0.6pt}
\end{exo}

\newpage

\section{Primitives et équations différentielles et étude de fonctions}

\begin{exo}\textbf{(**)}\quad\\[0.2cm]
	Calculer les primitives des fonctions suivantes
	\begin{multicols}{2}
		\begin{enumerate}
			\item $x\mapsto e^x \cos x $
			\item $x\mapsto \sqrt{e^x -1} $
			\item $x\mapsto x \sqrt[3]{1 + x } $
			\item $x \mapsto e^{ax} \sin b x $
		\end{enumerate}
	\end{multicols}
	
	\centering
	\rule{1\linewidth}{0.6pt}
\end{exo}

	\begin{exo}\textbf{(*)}\quad\\[0.2cm]
	La fonction $x \mapsto \arccos x$ est-elle solution de
	$$y'+ \dfrac{1}{\sqrt{1-x^2}}y = 0\ $$
	\centering
	\rule{1\linewidth}{0.6pt}
\end{exo}


\begin{exo}\textbf{(**)}\quad\\[0.2cm]
	Résoudre $y' (x ) - y (x ) = x^2 - 1$ avec la condition initiale $y (0) = 1 $ en cherchant une solution
	polynomiale $a x^2 + bx + c$.
	
	\centering
	\rule{1\linewidth}{0.6pt}
\end{exo}

\begin{exo}\textbf{(**)}\quad\\[0.2cm]
	Soit $n \geq 2$ un entier fix\'e et $f:\R^{+} = [0,
	+ \infty[ \longrightarrow \R$ la fonction d\'efinie par la formule
	suivante:\quad\\[-0.3cm]
	$$ f (x) = \frac{1 + x^n}{(1 + x)^n}, \ \ x \geq 0.$$
	\begin{enumerate}
		\item
		\begin{enumerate}
			\item Montrer que $f$ est d\'erivable sur $\R^{+}$ et calculer $f' (x)$
			pour $x \geq 0.$
			\item En \'etudiant le signe de $f' (x)$ sur $\R^{+},$ montrer que $f$
			atteint un minimum sur $\R^{+}$ que l'on d\'eterminera.
		\end{enumerate}
		\item
		\begin{enumerate}
			\item En d\'eduire l'in\'egalit\'e suivante:
			
			$ (1+ x)^n \leq 2^{n - 1} (1+ x^n), \ \ \forall x \in \R^{+}.$
			\item Montrer que si  $x \in \R^{+}$ et $y \in \R^{+}$ alors
			
			on a: $ (x + y)^n \leq 2^{n - 1} (x^n + y^n).$
		\end{enumerate}
	\end{enumerate}
	
	\centering
	\rule{1\linewidth}{0.6pt}
\end{exo}


\begin{exo}\textbf{(**)}\quad\\[0.2cm]
	Démontrer que les courbes d'équation $y=x^2$ et $y=1/x$ admettent une unique tangente commune.
	
	\centering
	\rule{1\linewidth}{0.6pt}
\end{exo}
	

\end{document}