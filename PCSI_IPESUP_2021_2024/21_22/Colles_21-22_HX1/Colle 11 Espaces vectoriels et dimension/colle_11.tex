\documentclass[a4paper,10pt]{article}



\usepackage{fancyhdr} % pour personnaliser les en-têtes
\usepackage[utf8]{inputenc}
\usepackage[T1]{fontenc}
\usepackage{lastpage}
\usepackage[frenchb]{babel}
\usepackage{amsfonts,amssymb}
\usepackage{amsmath,amsthm,mathtools}
\usepackage{paralist}
\usepackage{xspace}
\usepackage{xcolor,multicol}
\usepackage{variations}
\usepackage{xypic}
\usepackage{eurosym}
\usepackage{graphicx}
\usepackage[np]{numprint}
\usepackage{hyperref} 
\usepackage{listings} % pour écrire des codes avec coloration syntaxique  

\usepackage{tikz}
\usetikzlibrary{calc, arrows, plotmarks,decorations.pathreplacing}
\usepackage{colortbl}
\usepackage{multirow}
\usepackage[top=2cm,bottom=1.5cm,right=2cm,left=1.5cm]{geometry}

\newtheorem{thm}{Théorème}
\newtheorem*{pro}{Propriété}
\newtheorem*{exemple}{Exemple}

\theoremstyle{definition}
\newtheorem*{remarque}{Remarque}
\theoremstyle{definition}
\newtheorem{exo}{Exercice}
\newtheorem{definition}{Définition}


\newcommand{\vtab}{\rule[-0.4em]{0pt}{1.2em}}
\newcommand{\V}{\overrightarrow}
\renewcommand{\thesection}{\Roman{section} }
\renewcommand{\thesubsection}{\arabic{subsection} }
\renewcommand{\thesubsubsection}{\alph{subsubsection} }
\newcommand*{\transp}[2][-3mu]{\ensuremath{\mskip1mu\prescript{\smash{\mathrm t\mkern#1}}{}{\mathstrut#2}}}%

\newcommand{\C}{\mathbb{C}}
\newcommand{\R}{\mathbb{R}}
\newcommand{\Q}{\mathbb{Q}}
\newcommand{\Z}{\mathbb{Z}}
\newcommand{\N}{\mathbb{N}}



\definecolor{vert}{RGB}{11,160,78}
\definecolor{rouge}{RGB}{255,120,120}
% Set the beginning of a LaTeX document
\pagestyle{fancy}
\lhead{Optimal Sup Spé, groupe IPESUP}\chead{Année~2021-2022}\rhead{Niveau: Première année de PCSI }\lfoot{M. Botcazou}\cfoot{\thepage}\rfoot{mail: ibotca52@gmail.com }\renewcommand{\headrulewidth}{0.4pt}\renewcommand{\footrulewidth}{0.4pt}

\begin{document}
	
	
	\begin{center}
		\Large \sc colle 11 = espaces vectoriels et dimensions des espaces vectoriels
	\end{center}




\section*{Espaces vectoriels}
\begin{minipage}{1\linewidth}
	\begin{minipage}[t]{0.48\linewidth}
		\raggedright
		
		
		
		\begin{exo}\quad\\
		Parmi les ensembles suivants reconna\^\i tre ceux qui sont des
		sous-espaces vectoriels.(Justifier)\\[0.25cm]
		
		$ E_1 =\left\{ (x,y,z)\in \R^3 \mid x+y+a=0 \hbox{ et }  x +3az =0\right\}$\\[0.2cm]
		
		$ E_2 =\left\{f \in {\mathcal F}(\R,\R) \mid f(1)=0\right\}$\\[0.2cm]
		
		$ E_3 =\left\{f \in {\mathcal F}(\R,\R) \mid  f(0)=1\right\}$\\[0.2cm]
		
		$E_4 =\left\{(x,y)\in \R^2 \mid x + \alpha y +1 \geqslant 0\right\}$	
			
			\centering
			\rule{1\linewidth}{0.6pt}
		\end{exo}
	

	
	\begin{exo}\quad\\
		Soit $E$ un espace vectoriel.
	\begin{enumerate}
		\item Soient $F$ et $G$ deux sous-espaces de $E$.\\ Montrer que:\\[0.25cm]
		$F \cup G \hbox{ est un sous-espace vectoriel de } E$\\ $ \Longleftrightarrow \quad F\subset G \hbox{ ou } G \subset F.$\\[0.25cm]
		\item Soit $H$ un troisi\`eme sous-espace vectoriel de $E$. Prouver
		que
		$$G \subset F \Longrightarrow F\cap(G+H) = G + (F\cap H) .$$
	\end{enumerate}
		
		\centering
		\rule{1\linewidth}{0.6pt}
	\end{exo}
	
		\begin{exo}\quad\\
			Soient $v_1=(0,1,-2,1),
		v_2=(1,0,2,-1), v_3=(3,2,2,-1),$ $ v_4 = (0,0,1,0)$ et
		$v_5=(0,0,0,1)$ des vecteurs de $\R^4$.\\[0.25cm]  Les propositions
		suivantes sont-elles vraies ou fausses ?  Justifier votre r\'eponse.
		\begin{enumerate}
			\item$\text{Vect}\{ v_1, v_2, v_3 \}=\text{Vect}\{(1,1,0,0),(-1,1,-4,2)\}$
			\item $(1,1,0,0) \in \text{Vect}\{ v_1, v_2 \} \cap \text{Vect}\{ v_2, v_3, v_4 \}$.
			\item $\dim(\text{Vect}\{ v_1, v_2 \} \cap \text{Vect}\{ v_2, v_3, v_4 \})=1$ (c'est-à-dire c'est une droite vectorielle).
			\item $\text{Vect}\{ v_1, v_2 \} + \text{Vect}\{ v_2, v_3, v_4 \}= \R^4$.
			\item $\text{Vect}\{ v_4, v_5 \}$ est un sous-espace vectoriel 
			suppl\'ementaire de $\text{Vect}\{ v_1, v_2, v_3 \}$ dans $\R^4$.
		\end{enumerate}
		\centering
		\rule{1\linewidth}{0.6pt}
	\end{exo}

		\begin{exo}\quad\\
			Démontrer que les familles suivantes sont libres\\ dans $\mathcal{F}(\R,\R)$:\\[0.25cm]
			\begin{enumerate}
				\item $\left(x\longmapsto e^{ax}\right)_{a\in\R}$;\\[0.25cm]
				\item $\left(x\longmapsto |x-a|\right)_{a\in\R}$;\\[0.25cm]
				\item $\left(x\longmapsto \cos(ax)\right)_{a\in\R}$;\\[0.25cm]
				\item $\left(x\longmapsto (\sin x)^n\right)_{n\in\N}$;\\[0.25cm]
			\end{enumerate} 
		%Soit $\alpha \in \R$ et $f_\alpha : \R \to \R$, $x\mapsto e^{\alpha x}$.
		%Montrer que la famille $(f_\alpha)_{\alpha \in \R}$  est libre.
		
		\centering
		\rule{1\linewidth}{0.6pt}
	\end{exo}
	
		
	
	\end{minipage}	
	\hfill\vrule\hfill
	\begin{minipage}[t]{0.48\linewidth}
		\raggedright
		
	\begin{exo}\quad\\
		\begin{enumerate}
			\item Soient $v_1=(2,1,4)$, $v_2=(1,-1,2)$ et $v_3=(3,3,6)$ des vecteurs de $\R^3$, 
			trouver trois r\'eels non tous nuls $\alpha,\beta,\gamma$ tels que $\alpha v_1+ \beta v_2 + \gamma v_3=0$.
			
			\item On considère deux plans vectoriels
			$$P_1=\{(x,y,z) \in \R^3 \mid x-y+z=0\}$$
			$$P_2=\{(x,y,z) \in \R^3 \mid x-y=0\}$$
			trouver un vecteur directeur de la droite $D=P_1\cap P_2$ ainsi qu'une \'equation param\'etr\'ee.
		\end{enumerate}
		
		\centering
		\rule{1\linewidth}{0.6pt}
	\end{exo}

\begin{exo}\quad\\
	Soient $E = \mathcal{D} (\R, \R)$ l'espace des fonctions dérivables
	et $F = \left\{ f \in E \mid f (0) = f' (0) = 0\right\}$.\\[0.25cm] Montrer que $F$
	est un sous-espace vectoriel de $E$ et d\'eterminer un
	suppl\'ementaire de $F$ dans $E$.
	
	\centering
	\rule{1\linewidth}{0.6pt}
\end{exo}

\begin{exo}\quad\\
	Pour $E=\R^4$, dire si les familles de vecteurs suivantes peuvent être complétées en une base de $E$. Si oui, le faire. 
	\begin{enumerate}
		\item $(u,v,w)$ avec $u=(1,2,-1,0)$, $v=(0,1,-4,1)$ et $w=(2,5,-6,1)$;
		\item $(u,v,w)$ avec $u=(1,0,2,3)$, $v=(0,1,2,3)$ et $w=(1,2,0,3)$;
	\end{enumerate}

	
	\centering
	\rule{1\linewidth}{0.6pt}
\end{exo}


\begin{exo}\quad\\
	Soit $\mathbf{E}$ l'ensemble des fonctions continues sur $[-1,1]$ qui sont affines sur $[-1,0]$ et sur $[0,1]$. Démontrer que $\mathbf{E}$ est un espace vectoriel et en donner une base.
	
	\centering
	\rule{1\linewidth}{0.6pt}
\end{exo}
	\begin{exo}\quad\\
	Soit $$E=\big\{(u_{n})_{n\in
		\N}\in \R^{\N}\ |\ (u_{n})_{n} \text{ converge }\big\}.$$
	Montrer que
	l'ensemble des suites constantes et l'ensemble des suites convergeant
	vers $0$ sont des sous-espaces suppl\'{e}mentaires dans $E.$
		
	\centering
	\rule{1\linewidth}{0.6pt}
\end{exo}

		\begin{exo}\quad\\
	Soit $\mathbf{E}$ un espace vectoriel dans lequel tout sous-espace vectoriel admet un supplémentaire. Soit $\mathbf{F}$ un sous-espace vectoriel propre de $\mathbf{E}$ (c'est-à-dire que $\mathbf{F}\neq \{0\}$ et que $\mathbf{E}\neq \mathbf{F}$). Démontrer que $\mathbf{F}$ admet au moins deux supplémentaires distincts.
	
	\centering
	\rule{1\linewidth}{0.6pt}
\end{exo}



	
	\end{minipage}
\end{minipage}
\end{document}