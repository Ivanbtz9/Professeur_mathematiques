\documentclass[a4paper,11pt]{article}

\usepackage{inputenc}
\usepackage[T1]{fontenc}
\usepackage[frenchb]{babel}
\usepackage{fancyhdr,fancybox} % pour personnaliser les en-têtes
\usepackage{lastpage,setspace}
\usepackage{amsfonts,amssymb,amsmath,amsthm,mathrsfs}
\usepackage{mathdots}
\usepackage{relsize,exscale,bbold}
\usepackage{paralist}
\usepackage{xspace,multicol,diagbox,array}
\usepackage{xcolor}
\usepackage{variations}
\usepackage{xypic}
\usepackage{eurosym,stmaryrd}
\usepackage{graphicx}
\usepackage[np]{numprint}
\usepackage{hyperref} 
\usepackage{tikz}
\usepackage{colortbl}
\usepackage{multirow}
\usepackage{MnSymbol,wasysym}
\usepackage[top=1.5cm,bottom=1.5cm,right=1.2cm,left=1.5cm]{geometry}
\usetikzlibrary{calc, arrows, plotmarks, babel,decorations.pathreplacing}
\setstretch{1.25}
%\usepackage{lipsum} %\usepackage{enumitem} %\setlist[enumerate]{itemsep=1mm} bug avec enumerate



\newtheorem{thm}{Théorème}
\newtheorem{rmq}{Remarque}
\newtheorem{prop}{Propriété}
\newtheorem{cor}{Corollaire}
\newtheorem{lem}{Lemme}
\newtheorem{prop-def}{Propriété-définition}

\theoremstyle{definition}

\newtheorem{defi}{Définition}
\newtheorem{ex}{Exemple}
\newtheorem*{rap}{Rappel}
\newtheorem{cex}{Contre-exemple}
\newtheorem{exo}{Exercice} % \large {\fontfamily{ptm}\selectfont EXERCICE}
\newtheorem{nota}{Notation}
\newtheorem{ax}{Axiome}
\newtheorem{appl}{Application}
\newtheorem{csq}{Conséquence}
\def\di{\displaystyle}



\renewcommand{\thesection}{\Roman{section}}\renewcommand{\thesubsection}{\arabic{subsection} }\renewcommand{\thesubsubsection}{\alph{subsubsection} }


\newcommand{\bas}{~\backslash}\newcommand{\ba}{\backslash}
\newcommand{\C}{\mathbb{C}}\newcommand{\K}{\mathbb{K}}\newcommand{\R}{\mathbb{R}}\newcommand{\Q}{\mathbb{Q}}\newcommand{\Z}{\mathbb{Z}}\newcommand{\N}{\mathbb{N}}\newcommand{\V}{\overrightarrow}\newcommand{\Cs}{\mathscr{C}}\newcommand{\Ps}{\mathscr{P}}\newcommand{\Rs}{\mathscr{R}}\newcommand{\Gs}{\mathscr{G}}\newcommand{\Ds}{\mathscr{D}}\newcommand{\happy}{\huge\smiley}\newcommand{\sad}{\huge\frownie}\newcommand{\danger}{\begin{tikzpicture}[x=1.5pt,y=1.5pt,rotate=-14.2]
	\definecolor{myred}{rgb}{1,0.215686,0}
	\draw[line width=0.1pt,fill=myred] (13.074200,4.937500)--(5.085940,14.085900)..controls (5.085940,14.085900) and (4.070310,15.429700)..(3.636720,13.773400)
	..controls (3.203130,12.113300) and (0.917969,2.382810)..(0.917969,2.382810)
	..controls (0.917969,2.382810) and (0.621094,0.992188)..(2.097660,1.359380)
	..controls (3.574220,1.726560) and (12.468800,3.984380)..(12.468800,3.984380)
	..controls (12.468800,3.984380) and (13.437500,4.132810)..(13.074200,4.937500)
	--cycle;
	\draw[line width=0.1pt,fill=white] (11.078100,5.511720)--(5.406250,11.875000)..controls (5.406250,11.875000) and (4.683590,12.812500)..(4.367190,11.648400)
	..controls (4.050780,10.488300) and (2.375000,3.675780)..(2.375000,3.675780)
	..controls (2.375000,3.675780) and (2.156250,2.703130)..(3.214840,2.964840)
	..controls (4.273440,3.230470) and (10.640600,4.847660)..(10.640600,4.847660)
	..controls (10.640600,4.847660) and (11.332000,4.953130)..(11.078100,5.511720)
	--cycle;
	\fill (6.144520,8.839900)..controls (6.460940,7.558590) and (6.464840,6.457090)..(6.152340,6.378910)
	..controls (5.835930,6.300840) and (5.320300,7.277400)..(5.003900,8.554750)
	..controls (4.683590,9.835940) and (4.679690,10.941400)..(4.996090,11.019600)
	..controls (5.312490,11.097700) and (5.824210,10.121100)..(6.144520,8.839900)
	--cycle;
	\fill (7.292960,5.261780)..controls (7.382800,4.898500) and (7.128900,4.523500)..(6.730460,4.421880)
	..controls (6.328120,4.324220) and (5.929680,4.535220)..(5.835930,4.898500)
	..controls (5.746080,5.261780) and (5.999990,5.640630)..(6.402340,5.738340)
	..controls (6.804690,5.839840) and (7.203110,5.625060)..(7.292960,5.261780)
	--cycle;
	\end{tikzpicture}}\newcommand{\alors}{\Large\Rightarrow}\newcommand{\equi}{\Leftrightarrow}
\newcommand{\fonction}[5]{\begin{array}{l|rcl}
		#1: & #2 & \longrightarrow & #3 \\
		& #4 & \longmapsto & #5 \end{array}}


\definecolor{vert}{RGB}{11,160,78}
\definecolor{rouge}{RGB}{255,120,120}
\definecolor{bleu}{RGB}{15,5,107}



\pagestyle{fancy}
\lhead{Groupe IPESUP}\chead{}\rhead{Année~2022-2023}\lfoot{M. Botcazou \& M.Dupré}\cfoot{\thepage/4}\rfoot{PCSI }\renewcommand{\headrulewidth}{0.4pt}\renewcommand{\footrulewidth}{0.4pt}


\begin{document}
 %%%%BIBMATH%%%%
 
 %(1) https://www.bibmath.net/ressources/index.php?action=affiche&quoi=bde/algebrelineaire/matricesal&type=fexo
	

\noindent\shadowbox{
	\begin{minipage}{1\linewidth}
		\centering
		\huge{\textbf{ TD 15 : Matrices et applications linéaires }}
	\end{minipage}}

\smallskip
\section*{Connaître son cours:}
\begin{itemize}[$\bullet$]
	\item Soit $E$, $F$ deux espaces vectoriels de dimension finie, $\beta$ une base de E et $\zeta$ une base de $F$. Montrer que pour tout $u \in \mathcal{L} (E , F )$ et tout $x \in E$ on a : $\text{Mat}_\zeta (u (x )) = \text{Mat}_{\beta,\zeta}(u) \cdot \text{Mat}_\beta(x )$.
	\item Soit $E $, $F $, $G$ trois espaces vectoriels de dimension finie, $\beta$ une base de $E$ , $\zeta$ une base de $F$ et $\gamma$
	une base de $G $. Montrer que pour tout $u \in \mathcal{L} (E , F )$ et tout $v \in \mathcal{L} (F , G )$, on a : $\text{Mat}_{\beta, \gamma} (v\circ u ) = \text{Mat}_{\zeta, \gamma} (v ) \cdot \text{Mat}_{\beta, \zeta} ( u )$
	\item Soit $E$ un espace vectoriel de dimension finie, $\beta =(e_1,e_2,\dots,e_n)$ et $\beta'=(e_1',e_2',\dots,e_n')$ deux bases de $E $. Exprimer la matrice de passage $P_\beta^{\beta'}$ en lien avec une représentation matricielle de l'endomorphisme $Id_E$.
	
	En déduire que pour tout $x \in E $, on a : $\text{Mat}_\beta (x ) = P_\beta^{\beta'} \cdot \text{Mat}_{\beta'}(x )$
	\item Soient $E$ et $F$ deux $\K$-espaces vectoriels de dimensions respectives $p$ et $n$ et $u \in\mathcal L (E, F )$ de rang $r$.  Montrer qu'il existe une base $\beta$ de $E$ et une base $\zeta$ de $F$ tel que : $\text{Mat}_{\beta,\zeta}(u) = J_r = \begin{pmatrix}
	I_r & 0_{r,p-r} \\
	0_{n-r,r} & 0_{n-r,p-r}
	\end{pmatrix}$. 

\end{itemize}
\raggedright

\section*{Matrice d’une application linéaire dans une base:}\hfill\\%[-0.25cm]

   
\begin{minipage}{1\linewidth}\begin{minipage}[t]{0.48\linewidth}\raggedright
	
\begin{exo}\textbf{(*)}\quad\\[0.2cm]
Soient $S$ et $T$ les deux endomorphismes de $\mathbb R^2$ définis par\quad\\[-0.5cm]

$$
S(x,y)=(2x-5y,\ -3x+4y)\ \text{et}\ T(x,y)=(-8y,\ 7x+y).
$$
\begin{enumerate}
	\item Déterminer les matrices de $S$ et $T$ dans la base canonique de $\mathbb R^2$.
	\item Déterminer les applications linéaires $S+T$, $S\circ T$, $T\circ S$ et $S\circ S$ ainsi que leurs matrices dans la base canonique de $\mathbb R^2$.
\end{enumerate}

	
\centering\rule{1\linewidth}{0.6pt}\end{exo}



\begin{exo}\textbf{(*)}\quad\\[0.2cm]
On considère l'endomorphisme $f$ de $\mathbb R^3$ dont la matrice 
dans la base canonique est :

$$M=\left(
\begin{array}{ccc}
1&1&-1\\
-3&-3&3\\
-2&-2&2
\end{array}\right).$$


Donner une base de $\ker(f)$ et de $\textrm{Im}(f)$.

En déduire que $M^n=0$ pour tout $n\geq 2$.
	
	\centering\rule{1\linewidth}{0.6pt}\end{exo}




%%%%%%%%%%%%%%%%%%%%%%%%%%%%%%%%%%%%%%%%%%%%%%%%%%%%%%%%%%%%%%%%%%%%%%%%%%%%%%%%%%%%%%%%%%
\end{minipage}\hfill\vrule\hfill\begin{minipage}[t]{0.48\linewidth}\raggedright
%%%%%%%%%%%%%%%%%%%%%%%%%%%%%%%%%%%%%%%%%%%%%%%%%%%%%%%%%%%%%%%%%%%%%%%%%%%%%%%%%%%%%%%%%%

\begin{exo}\textbf{(*)}\quad\\[0.2cm]
Soient $$A=\left(\begin{array}{cc}-1&2\\1&0\end{array}\right)$$

et $f$ l'application de $M_2(\mathbb R)$ dans $M_2(\mathbb R)$
définie par $f(M)=AM$.
\begin{enumerate}
	\item Montrer que $f$ est linéaire.
	\item Déterminer sa matrice dans la base canonique de $M_2(\mathbb R)$.
\end{enumerate}

\centering\rule{1\linewidth}{0.6pt}\end{exo}


\begin{exo}\textbf{(*)}\quad\\[0.2cm]
Soit $$A = \begin{pmatrix} 
0&&\dots&0&1 \\ 
\vdots&&&1&0 \\
& & \udots && \\
0&1& & &\vdots \\ 
1&0&&\dots&0
\end{pmatrix}$$ \quad \\[0.2cm]


En utilisant l'application linéaire associée de 
$\mathcal{L} (\R^n,\R^n)$, calculer $A^p$ pour $p \in \Z$.

\centering\rule{1\linewidth}{0.6pt}\end{exo}


\end{minipage}\end{minipage} \newpage

\begin{minipage}{1\linewidth}\begin{minipage}[t]{0.48\linewidth}\raggedright
		
		\begin{exo}\textbf{(*)}\quad\\[0.2cm]
		Soit  $E$  un espace vectoriel et  $f$  une projection sur $E$.
		\begin{enumerate}
			\item Montrer que  $E= \text{Ker} f \oplus \text{Im} f$.
			
			\item Supposons que  $E$ soit de dimension finie  $n$. 
			Posons  $r= \text{rg}(f)$. 
			Montrer qu'il existe une base 
			$\mathcal{B}= ( e_1, \ldots ,e_n)$ de  $E$  telle que : 
			$f(e_i)=e_i$ si $i\le r$ et $f(e_i)=0$ si $i>r$. 
			
			Déterminer la matrice de  $f$ dans cette base $\mathcal{B}$.
		\end{enumerate}	
			
			
			\centering\rule{1\linewidth}{0.6pt}\end{exo}
		
		
		
		\begin{exo}\textbf{(*)}\quad\\[0.2cm]
		
			\begin{enumerate}
				
				\item Montrer que si $f$ est un endomorphisme d'un espace
				vectoriel $E$ de dimension $n$, $M$ sa matrice par rapport à
				une base $e$, $M'$ sa matrice par rapport à une base $e'$,
				alors $\textrm{tr}\, M = \textrm{tr}\, M'$. 
				On note $\textrm{tr}\, f$ la valeur commune de ces quantités.
				
				\item Montrer que si $g$ est un autre endomorphisme de $E$,
				$\textrm{tr}(f\circ g - g\circ f) = 0$.
			\end{enumerate}
			
			\centering\rule{1\linewidth}{0.6pt}\end{exo}
		
		\begin{exo}\textbf{(**)}\quad\\[0.2cm]
		Soit $\R^2$ muni de la base canonique $\mathcal{B}=(e_1, e_2)$.
		Soit $f : \R^2 \to \R^2$ la projection sur l'axe des abscisses $ \text{Vect}(e_1)$  
		parall\`element à $\text{Vect}(e_1+e_2)$.
		Déterminer $\textrm{Mat}_{\mathcal{B},\mathcal{B}}(f)$, la matrice de $f$ dans la base $\mathcal{B}$.
		
		Même question avec $\textrm{Mat}_{\mathcal{B}',\mathcal{B}}(f)$ où $\mathcal{B'}$ est la base 
		$(e_1 - e_2, -2e_1+3e_2)$ de $\R^2$.
		Même question avec $\textrm{Mat}_{\mathcal{B}',\mathcal{B}'}(f)$.	
			
		\centering\rule{1\linewidth}{0.6pt}\end{exo}
	
		\begin{exo}\textbf{(**)}\quad\\[0.2cm]
			Soit $f$ l'endomorphisme de $\R^3$ dont la matrice par
			rapport \`a la base canonique $(e_1, e_2, e_3)$ est
			$$A= \left( 
			\begin{array}{ccc}
			15 & -11 & 5 \\
			20 & -15 & 8 \\
			8 &  -7 & 6
			\end{array}
			\right).$$
			Montrer que les vecteurs : $ e'_1 = 2e_1+3e_2+e_3,$ $ e'_2 = 3e_1+4e_2+e_3,$\quad $ e'_3 = e_1+2e_2+2e_3$
			
			forment une base de $\R^3$ et calculer la matrice de $f$ par
			rapport \`a cette base.
	
	\centering\rule{1\linewidth}{0.6pt}\end{exo}
		
		
		%%%%%%%%%%%%%%%%%%%%%%%%%%%%%%%%%%%%%%%%%%%%%%%%%%%%%%%%%%%%%%%%%%%%%%%%%%%%%%%%%%%%%%%%%%
	\end{minipage}\hfill\vrule\hfill\begin{minipage}[t]{0.48\linewidth}\raggedright
		%%%%%%%%%%%%%%%%%%%%%%%%%%%%%%%%%%%%%%%%%%%%%%%%%%%%%%%%%%%%%%%%%%%%%%%%%%%%%%%%%%%%%%%%%%
		
					\begin{exo}\textbf{(**)}\quad\\[0.2cm]
			Soient trois vecteurs $e_1,e_2,e_3$ formant une base de $\R^3$.
			On note $\phi$ l'application linéaire définie par
			$\phi(e_1)=e_3$, $\phi(e_2)=-e_1+e_2+e_3$ et $\phi(e_3)=e_3$.
			
			\begin{enumerate}
				\item Écrire la matrice $A$ de $\phi$ dans la base $(e_1,e_2,e_3)$.
				
				Déterminer le noyau de cette application. 
				
				\item On pose $f_1=e_1-e_3$, $f_2=e_1-e_2$,  $f_3=-e_1+e_2+e_3$.
				
				Calculer $e_1,e_2,e_3$ en fonction de $f_1,f_2,f_3$.
				
				
				Les vecteurs $f_1,f_2,f_3$ forment-ils une base de $\R^3$ ?
				
				\item Calculer $\phi(f_1), \phi(f_2), \phi(f_3)$ en fonction de $f_1,f_2,f_3$.
				Écrire la matrice $B$ de $\phi$ dans la base $(f_1,f_2,f_3)$ et trouver la nature
				de l'application $\phi$.
				
				\item On pose $P=\begin{pmatrix}1&1&-1\cr 0&-1&1\cr-1&0&1\cr\end{pmatrix}$. Vérifier que $P$ est
				inversible et calculer $P^{-1}$.
				
				Quelle relation lie $A$, $B$, $P$ et $P^{-1}$ ?
			\end{enumerate}
			
			\centering\rule{1\linewidth}{0.6pt}\end{exo}
		
		
		\begin{exo}\textbf{(***)}\quad\\[0.2cm]
			Soit $f$ l'application de $\R_n[X]$  dans  $\R[X]$
			définie en posant pour tout  
			$$P\in \R_n[X],\ f(P)(X)=P(X+1)+P(X-1)-2P(X).$$
			\begin{enumerate}
				\item Montrer que  $f$  est linéaire et que son image est
				incluse dans  $\R_n[X]$.
				\item Dans le cas o\`u  $n=3$, donner la matrice de  $f$  dans
				la base  $1,X, X^2, X^3$. Déterminer ensuite, pour une valeur de  $n$
				quelconque, la matrice de  $f$  dans la base  $1,X,\ldots,X^n$.
				\item Déterminer le noyau et l'image de  $f$. 
				
				Calculer leur dimension respective.
				\item Soit  $Q$  un élément de l'image de  $f$.
				Montrer qu'il existe un unique  $P\in \R_n[X]$
				tel que : $f(P)=Q$  et  $P(0)=P'(0)=0$.
			\end{enumerate}	
			
			\centering\rule{1\linewidth}{0.6pt}\end{exo}
		
		
		
		
\end{minipage}\end{minipage}\newpage

\begin{minipage}{1\linewidth}\begin{minipage}[t]{0.48\linewidth}\raggedright
		
		\begin{exo}\textbf{(**)}\quad\\[0.2cm]
			Soit $E=\K_n[X]$. $u$ est l'endomorphisme de $E$ défini par : $\forall P\in E,\; u(P)(X)=P(X+1)-P(X)$.
			
			\begin{enumerate}
				\item Donner la représentation matricielle de cet endomorphisme dans la base canonique de $E$.
				\item  Déterminer $\text{Ker}u$  et $\text{Im}u$.
				
				\item  Déterminer explicitement une base dans laquelle la matrice de $u$ est  $$\left(
				\begin{array}{ccccc}
				0&1&0&\ldots&0\\
				\vdots&\ddots&\ddots&\ddots&\vdots\\
				& & &\ddots&0\\
				\vdots& & &\ddots&1\\
				0&\ldots& &\ldots&0
				\end{array}
				\right)$$
				\item En déduire que $u$ est nilpotent et donner son indice de nilpotence. 
			\end{enumerate}
			
			
			\centering\rule{1\linewidth}{0.6pt}\end{exo}
		
		
		
		
		
		\begin{exo}\textbf{(**)}\quad\\[0.2cm]
			Calculer l'inverse de la matrice suivante:
			 $$\left(
			\begin{array}{cccccc}
			\dbinom{0}{0}&\dbinom{1}{0}&\dbinom{2}{0}&\ldots&\dbinom{n-1}{0}&\dbinom{n}{0}\\
			\rule{0mm}{7mm}0&\dbinom{1}{1}&\dbinom{2}{1}&\ldots&\ldots&\dbinom{n}{1}\\
			\rule{0mm}{7mm}\vdots&\ddots&\dbinom{2}{2}& & &\vdots\\
			& & &\ddots& & \\
			\vdots& & &\ddots&\dbinom{n-1}{n-1}&\vdots\\
			0&\ldots& &\ldots&0&\dbinom{n}{n}
			\end{array}
			\right)$$
			
			\centering\rule{1\linewidth}{0.6pt}\end{exo}
		
		
		
		
		
		%%%%%%%%%%%%%%%%%%%%%%%%%%%%%%%%%%%%%%%%%%%%%%%%%%%%%%%%%%%%%%%%%%%%%%%%%%%%%%%%%%%%%%%%%%
	\end{minipage}\hfill\vrule\hfill\begin{minipage}[t]{0.48\linewidth}\raggedright
		%%%%%%%%%%%%%%%%%%%%%%%%%%%%%%%%%%%%%%%%%%%%%%%%%%%%%%%%%%%%%%%%%%%%%%%%%%%%%%%%%%%%%%%%%%
		
		\begin{exo}\textbf{(**)}\quad\\[0.2cm]
			Soient $E$ un espace vectoriel de dimension finie, $a$ un réel non nul et $u,v\in \mathcal{L}(E)$.
			
			Résoudre dans $\mathcal{L}(E)$ l'équation d'inconnue $w$ : $$a\cdot w+ \text{Tr}(w)\cdot u=v$$\quad\\[-0.5cm]
			
			\centering\rule{1\linewidth}{0.6pt}\end{exo}
		
		
		\begin{exo}\textbf{(**)}\quad\\[0.2cm]
			Soit $f$ une forme linéaire sur $\mathcal{M}_n(\C)$ telle que $\forall (A,B)\in(\mathcal{M}_n(\C))^2$, $f(AB) = f(BA)$. 
			
			Montrer qu'il existe un complexe $a$ tel que $f=a\text{Tr}$.
			
			\centering\rule{1\linewidth}{0.6pt}\end{exo}
		
		
		
		\begin{exo}\textbf{(***)}\quad\\[0.2cm]
			Soit $A=(a_{i,j})_{1\leqslant i,j\leqslant n}$ définie par $a_{i,j}=1$ si $i=j$, $j$ si $i=j-1$ et $0$ sinon.
			
			
			Montrer que $A$ est inversible et calculer $A^{-1}$.
			
			\centering\rule{1\linewidth}{0.6pt}\end{exo}
		
		\begin{exo}\textbf{(***)}\quad (\textit{Matrice de \textsc{Vandermonde}})\\[0.2cm]
			Soient $x_1 , \dots , x_n \in \K$.
				$$(x_i^{j-1})_{1\leq i,j\leq n} \ = \ \left(
			\begin{array}{ccccc}
			1&x_1&x_1^2&\ldots&x_1^{n-1}\\
			1&x_2&x_2^2&\ldots&x_2^{n-1}\\
			\vdots& \vdots& \vdots&\ddots&\vdots\\
			1&x_n&x_n^2&\ldots&x_n^{n-1}
			\end{array}
			\right)$$
			Montrer que cette matrice est inversible si et seulement si les scalaires $x_1 , \dots , x_n$ sont deux à deux distincts.
			
			\centering\rule{1\linewidth}{0.6pt}\end{exo}
		
		
	
		
		
\end{minipage}\end{minipage}

\section*{Rang d'une matrice et propriétés:}\hfill\\%[-0.25cm]


\begin{minipage}{1\linewidth}\begin{minipage}[t]{0.48\linewidth}\raggedright

		
		\begin{exo}\textbf{(**)}\quad\\[0.2cm]
			Soient $\alpha,\beta$ deux réels et 
			$$M_{\alpha,\beta}=\left(\begin{array}{cccc}
			1&3&\alpha&\beta\\
			2&-1&2&1\\
			-1&1&2&0
			\end{array}\right).$$
			Déterminer les valeurs de $\alpha$ et $\beta$ pour lesquelles l'application linéaire associée à $M_{\alpha,\beta}$
			est surjective.
			
			\centering\rule{1\linewidth}{0.6pt}\end{exo}
		
		
		

		
		
		%%%%%%%%%%%%%%%%%%%%%%%%%%%%%%%%%%%%%%%%%%%%%%%%%%%%%%%%%%%%%%%%%%%%%%%%%%%%%%%%%%%%%%%%%%
	\end{minipage}\hfill\vrule\hfill\begin{minipage}[t]{0.48\linewidth}\raggedright
		%%%%%%%%%%%%%%%%%%%%%%%%%%%%%%%%%%%%%%%%%%%%%%%%%%%%%%%%%%%%%%%%%%%%%%%%%%%%%%%%%%%%%%%%%%
		
		
				
		\begin{exo}\textbf{(**)}\quad\\[0.2cm]
			Soient  
			$A=\begin{pmatrix} 
			1 & 2 & 1 \cr
			3 & 4 & 1 \cr
			5 & 6 & 1 \cr
			7 & 8 & 1 \cr
			\end{pmatrix},\ 
			B=\begin{pmatrix} 
			2 & 2 & -1 & 7  \cr
			4 & 3 & -1 & 11 \cr
			0 & -1 & 2 & -4 \cr
			3 & 3 & -2 & 11 \cr 
			\end{pmatrix} $.\quad\\[0.2cm]
			Calculer $\textrm{rg}(A)$ et $\textrm{rg}(B)$. Déterminer une base du
			noyau et une base de l'image pour chacune des applications linéaires associées $f_A$ et $f_B$.
			
			\centering\rule{1\linewidth}{0.6pt}\end{exo}
		
		
\end{minipage}\end{minipage} \newpage


\begin{minipage}{1\linewidth}\begin{minipage}[t]{0.48\linewidth}\raggedright
		
		
		\begin{exo}\textbf{(*)}\quad\\[0.2cm]
			Soient $A, B$ deux matrices semblables (i.e. il existe $P$
			inversible telle que $B = P^{-1} A P$). Montrer que si l'une est
			inversible, l'autre aussi\,; que si l'une est idempotente, l'autre
			aussi\,; que si l'une est nilpotente, l'autre aussi\,; que si $A =
			\lambda I$, alors $A = B$.
			
			\centering\rule{1\linewidth}{0.6pt}\end{exo}
		
			\begin{exo}\textbf{(**)}\quad\\[0.2cm]
			Trouver toutes les matrices de $\mathcal{M}_3(\R)$ qui vérifient
			\begin{enumerate}
				\item $M^2 = 0$ ;
				\item $M^2 = M$ ; 
				\item $M^2 = I$. 
			\end{enumerate}
			
			\centering\rule{1\linewidth}{0.6pt}\end{exo}
		
		
		
		
		\begin{exo}\textbf{(**)}\quad\\[0.2cm]
			Prouver qu'une matrice $A$ de $M_{n,p}(\mathbb K)$ de rang $r$ s'écrit comme somme de $r$ matrices de rang 1.
			
			\centering\rule{1\linewidth}{0.6pt}\end{exo}
		
			\begin{exo}\textbf{(***)}\quad\\[0.2cm]
				Montrer qu'une matrice de $\mathcal M_n(\mathbb K)$ qui n'est pas inversible est équivalente à une matrice nilpotente.
				
				\centering\rule{1\linewidth}{0.6pt}\end{exo}
		
		
		%%%%%%%%%%%%%%%%%%%%%%%%%%%%%%%%%%%%%%%%%%%%%%%%%%%%%%%%%%%%%%%%%%%%%%%%%%%%%%%%%%%%%%%%%%
	\end{minipage}\hfill\vrule\hfill\begin{minipage}[t]{0.48\linewidth}\raggedright
		%%%%%%%%%%%%%%%%%%%%%%%%%%%%%%%%%%%%%%%%%%%%%%%%%%%%%%%%%%%%%%%%%%%%%%%%%%%%%%%%%%%%%%%%%%
		
		\begin{exo}\textbf{(**)}\quad\\[0.2cm]
			Soit $B$ la matrice diagonale par blocs 
			$$B=\left(
			\begin{array}{cccc}
			A_1&0&\dots&0\\
			0&A_2&\ddots&\vdots\\
			\vdots&\dots&\ddots&\vdots\\
			0&\dots&0&A_n
			\end{array}
			\right).$$
			Calculer le rang de $B$ en fonction du rang des $A_i$.	
			
			\centering\rule{1\linewidth}{0.6pt}\end{exo}
		
		
		\begin{exo}\textbf{(**)}\quad(\textit{Théorème de \textsc{Hadamard}})\\[0.2cm]
			Soit $A=(a_{i,j})_{1\leqslant i,j\leqslant n}\in\mathcal{M}_n(\C)$ telle que $\forall i\in\llbracket1,n\rrbracket$, $|a_{i,i}| >\sum_{j\neq i}^{}|a_{i,j}|$. Montrer que $A\in\mathcal{G}l_n(\C)$. 
			%(Une matrice à diagonale strictement dominante est inversible.)
			
			\centering\rule{1\linewidth}{0.6pt}\end{exo}
		
		
		\begin{exo}\textbf{(**)}\quad\\[0.2cm]
			
			Donner le rang de la matrice $(i+j+ij)_{1\leqslant i,j\leqslant n}$.
			\centering\rule{1\linewidth}{0.6pt}\end{exo}
		
		\begin{exo}\textbf{(***)}\quad\\[0.2cm]
		
		Soit $M\in\mathcal{M}_3(\R)$. Montrer que les deux propriétés suivantes sont équivalentes :
		
		\begin{center}
			(1) $M^2 = 0$ et (2) $\text{rg}M\leqslant 1$ et $\text{tr}M = 0$.
		\end{center}
	
	\centering\rule{1\linewidth}{0.6pt}\end{exo}
		
\end{minipage}\end{minipage}


\end{document}