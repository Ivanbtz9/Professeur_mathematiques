\documentclass[a4paper,11pt]{article}


\usepackage[latin1]{inputenc}
\usepackage[T1]{fontenc}
%\usepackage{tgbonum}
\usepackage{fancyhdr,fancybox} % pour personnaliser les en-t�tes
\usepackage{lastpage}
\usepackage[frenchb]{babel}
\usepackage{amsfonts,amssymb,amsmath,amsthm,mathrsfs}
\usepackage{paralist}
\usepackage{xspace,multicol}
\usepackage{xcolor}
\usepackage{variations}
\usepackage{xypic}
\usepackage{eurosym}
\usepackage{graphicx}
\usepackage[np]{numprint}
\usepackage{hyperref} 
\usepackage{lipsum}

\usepackage{enumitem}

\usepackage{tikz}
\usetikzlibrary{calc, arrows, plotmarks, babel,decorations.pathreplacing}
\usepackage{colortbl}
\usepackage{multirow}
\usepackage[top=1.5cm,bottom=1.5cm,right=1.2cm,left=1.5cm]{geometry}

\newtheorem{thm}{Th�or�me}
\newtheorem{rmq}{Remarque}
\newtheorem{prop}{Propri�t�}
\newtheorem{cor}{Corollaire}
\newtheorem{lem}{Lemme}
\newtheorem{prop-def}{Propri�t�-d�finition}

\theoremstyle{definition}

\newtheorem{defi}{D�finition}
\newtheorem{ex}{Exemple}
\newtheorem{cex}{Contre-exemple}
\newtheorem{exer}{Exercice} % \large {\fontfamily{ptm}\selectfont EXERCICE}
\newtheorem{nota}{Notation}
\newtheorem{ax}{Axiome}
\newtheorem{appl}{Application}
\newtheorem{csq}{Cons�quence}
\def\di{\displaystyle}



\renewcommand{\thesection}{\Roman{section}}\renewcommand{\thesubsection}{\arabic{subsection} }\renewcommand{\thesubsubsection}{\alph{subsubsection} }

\newcommand{\C}{\mathbb{C}}\newcommand{\R}{\mathbb{R}}\newcommand{\Q}{\mathbb{Q}}\newcommand{\Z}{\mathbb{Z}}\newcommand{\N}{\mathbb{N}}\newcommand{\V}{\overrightarrow}\newcommand{\Cs}{\mathscr{C}}\newcommand{\Ps}{\mathscr{P}}\newcommand{\Ds}{\mathscr{D}}\newcommand{\happy}{\huge\smiley}\newcommand{\sad}{\huge\frownie}\newcommand{\danger}{\begin{tikzpicture}[x=1.5pt,y=1.5pt,rotate=-14.2]
	\definecolor{myred}{rgb}{1,0.215686,0}
	\draw[line width=0.1pt,fill=myred] (13.074200,4.937500)--(5.085940,14.085900)..controls (5.085940,14.085900) and (4.070310,15.429700)..(3.636720,13.773400)
	..controls (3.203130,12.113300) and (0.917969,2.382810)..(0.917969,2.382810)
	..controls (0.917969,2.382810) and (0.621094,0.992188)..(2.097660,1.359380)
	..controls (3.574220,1.726560) and (12.468800,3.984380)..(12.468800,3.984380)
	..controls (12.468800,3.984380) and (13.437500,4.132810)..(13.074200,4.937500)
	--cycle;
	\draw[line width=0.1pt,fill=white] (11.078100,5.511720)--(5.406250,11.875000)..controls (5.406250,11.875000) and (4.683590,12.812500)..(4.367190,11.648400)
	..controls (4.050780,10.488300) and (2.375000,3.675780)..(2.375000,3.675780)
	..controls (2.375000,3.675780) and (2.156250,2.703130)..(3.214840,2.964840)
	..controls (4.273440,3.230470) and (10.640600,4.847660)..(10.640600,4.847660)
	..controls (10.640600,4.847660) and (11.332000,4.953130)..(11.078100,5.511720)
	--cycle;
	\fill (6.144520,8.839900)..controls (6.460940,7.558590) and (6.464840,6.457090)..(6.152340,6.378910)
	..controls (5.835930,6.300840) and (5.320300,7.277400)..(5.003900,8.554750)
	..controls (4.683590,9.835940) and (4.679690,10.941400)..(4.996090,11.019600)
	..controls (5.312490,11.097700) and (5.824210,10.121100)..(6.144520,8.839900)
	--cycle;
	\fill (7.292960,5.261780)..controls (7.382800,4.898500) and (7.128900,4.523500)..(6.730460,4.421880)
	..controls (6.328120,4.324220) and (5.929680,4.535220)..(5.835930,4.898500)
	..controls (5.746080,5.261780) and (5.999990,5.640630)..(6.402340,5.738340)
	..controls (6.804690,5.839840) and (7.203110,5.625060)..(7.292960,5.261780)
	--cycle;
	\end{tikzpicture}}\newcommand{\alors}{\Large\Rightarrow}\newcommand{\equi}{\Leftrightarrow}



\definecolor{vert}{RGB}{11,160,78}
\definecolor{rouge}{RGB}{255,120,120}
\definecolor{bleu}{RGB}{15,5,107}

\setlist[enumerate]{itemsep=1mm}

\pagestyle{fancy}
\lhead{Groupe IPESUP}\chead{}\rhead{Ann�e~2022-2023}\lfoot{M. Botcazou \& M.Dupr�}\cfoot{\thepage/2}\rfoot{\textbf{PCSI}}\renewcommand{\headrulewidth}{0.4pt}\renewcommand{\footrulewidth}{0.4pt}


\begin{document}

	\noindent\shadowbox{
		\begin{minipage}{1\linewidth}
		$$\huge{\textbf{ DEVOIR MAISON N�3 }}$$
		$$\left(\text{Temps : }3\text{ heures}\right) $$
	
		\end{minipage}
	}
\medskip

L'objectif de ce devoir est de vous amener � calculer trois "belles" limites:\\


\centering\rule{1\linewidth}{0.6pt}\\

\raggedright

\begin{enumerate}
	\item[$\square$] $\lim\limits_{n \rightarrow+\infty} \sum\limits_{k=1}^{n} \dfrac{1}{k^{2}}=\dfrac{\pi^{2}}{6}$ \hfill (z�ta de Riemann en 2)
	\item[$\square$] $\lim\limits_{n \rightarrow+\infty} \dfrac{\sqrt{n}}{2^{2 n}}\begin{pmatrix}2 n \\ n\end{pmatrix}=\dfrac{1}{\sqrt{\pi}}$ \hfill(formule de Wallis)
	\item[$\square$] $\lim\limits_{n \rightarrow+\infty} \dfrac{1}{n !}\left(\dfrac{n}{e}\right)^{n} \sqrt{2 n \pi}=1$ \hfill
	(formule de Stirling)
\end{enumerate}
\centering\rule{1\linewidth}{0.6pt}\\

\raggedright


\section{Pr�liminaires}


Pour tout $n \in \mathbb{N}$, on pose: $a_{n}=\int_{0}^{\frac{\pi}{2}} \cos ^{n}(t) d t \quad$\hfill(int�grales de Wallis).

\begin{enumerate}

	\item Calculer $a_{0}$ et $a_{1}$, puis montrer que pour tout $n \in \mathbb{N}: a_{n}>0$.
	\item Montrer que pour tout $n \in \mathbb{N}: a_{n+2}=\dfrac{n+1}{n+2} a_{n}$.
\end{enumerate}


\section{Calcul de la valeur en 2 de la fonction z�ta de Riemann}

Pour tout $n \in \mathbb{N}$, on pose : $b_{n}=\int_{0}^{\frac{\pi}{2}} t^{2} \cos ^{2 n}(t) d t$. 

\begin{enumerate}
	\item \begin{enumerate}
		\item Montrer que pour tout $t \in\left[0, \frac{\pi}{2}\right]: t \leqslant \frac{\pi}{2} \sin (t)$.
		\item En d�duire que pour tout $n \in \mathbb{N}: 0 \leqslant b_{n} \leqslant \frac{\pi^{2}}{4}\left(a_{2 n}-a_{2 n+2}\right)$.
		\item En d�duire enfin la limite: $\lim\limits_{n \rightarrow+\infty} \dfrac{b_{n}}{a_{2 n}}=0$.
	\end{enumerate}

\item \begin{enumerate}
	\item Montrer que pour tout $n \in \mathbb{N}: a_{2n+2}=(2n+2) \int_{0}^{\frac{\pi}{2}} t \sin (t) \cos ^{2 n+1}(t) d t$.
	\item  En d�duire que pour tout $n \in \mathbb{N}: \dfrac{a_{2 n+2}}{n+1}=(2 n+1) b_{n}-(2 n+2) b_{n+1}$. 
	\item En d�duire que pour tout $n \in \mathbb{N}: 2\left(\dfrac{b_{n}}{a_{2 n}}-\dfrac{b_{n+1}}{a_{2 n+2}}\right)=\dfrac{1}{(n+1)^{2}}$.
	\item En d�duire enfin l'existence et la valeur de $\lim _{n \rightarrow+\infty} \sum\limits_{k=1}^{n} \dfrac{1}{k^{2}}$, que l'on note �galement: $\sum\limits_{k=1}^{+\infty} \dfrac{1}{k^{2}}$.
\end{enumerate}

\end{enumerate}

\section{Formule de Wallis}

Pour tout $n \in \mathbb{N}$, on pose : $\rho_{n}=\dfrac{a_{2 n}}{a_{2 n+1}}$.

\begin{enumerate}
	\item Montrer que pour tout $n \in \mathbb{N}: \rho_{n}=\dfrac{(2 n+1) \pi}{2^{4 n+1}}\begin{pmatrix}2 n \\ n\end{pmatrix}^{2}$.
	\item\begin{enumerate}
	\item Montrer que la suite $\left(a_{n}\right)_{n \in \mathbb{N}}$ est d�croissante.
	\item En d�duire un encadrement de $\rho_{n}$ pour tout $n \in \mathbb{N}$, puis la limite: $\lim\limits_{n \rightarrow+\infty} \rho_{n}=1$.
	\item En d�duire la formule de Wallis.	
\end{enumerate}\end{enumerate}

\newpage
\section{Formule de Stirling}


On note $f$ la fonction $x \longmapsto\left(x+\dfrac{1}{2}\right) \ln \left(1+\dfrac{1}{x}\right)$ et $g$ la fonction $x \longmapsto f(x)-\dfrac{1}{12 x}+\dfrac{1}{12(x+1)}$,

toutes deux d�finies sur $\mathbb{R}_{+}^{*}$. 


On pose �galement pour tout $n \in \mathbb{N}^{*}: u_{n}=\dfrac{n^{n+\frac{1}{2}}}{n ! e^{n}}$  \ et \  $v_{n}=\ln \left(u_{n}\right)$

\begin{enumerate}
	\item\begin{enumerate}
		\item Montrer que pour tout $x>0: f^{\prime \prime}(x)=\dfrac{1}{2 x^{2}(x+1)^{2}}$, et simplifier de m�me $g^{\prime \prime}$ sur $\mathbb{R}_{+}^{*}$.
		\item En d�duire que $f$ est minor�e par 1 , et $g$ major�e par 1 sur $\mathbb{R}_{+}^{*}$.
	\end{enumerate}
	\item\begin{enumerate}
		\item Exprimer $v_{n+1}-v_{n}$ � l'aide de la fonction $f$ pour tout $n \in \mathbb{N}^{*}$.
		\item En d�duire que la suite $\left(v_{n}\right)_{n \in \mathbb{N}^{*}}$ est croissante et major�e, puis que la suite $\left(u_{n}\right)_{n \in \mathbb{N}^{*}}$ 
		
		converge vers un r�el strictement positif $\ell$.
		\item Montrer, en �tudiant pour tout $n \in \mathbb{N}^{*}$ le rapport $\dfrac{u_{n}^{2}}{u_{2 n}}$, que: $\ell=\dfrac{1}{\sqrt{2 \pi}}$. 
		
		En d�duire la formule de Stirling.
		
	\end{enumerate}
\end{enumerate}


\end{document}
