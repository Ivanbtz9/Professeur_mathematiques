\documentclass[a4paper,11pt]{article}

\usepackage{inputenc}
\usepackage[T1]{fontenc}
\usepackage[frenchb]{babel}
\usepackage{fancyhdr,fancybox} % pour personnaliser les en-têtes
\usepackage{lastpage,setspace}
\usepackage{amsfonts,amssymb,amsmath,amsthm,mathrsfs}
\usepackage{relsize,exscale,bbold}
\usepackage{paralist}
\usepackage{xspace,multicol,diagbox,array}
\usepackage{xcolor}
\usepackage{variations}
\usepackage{xypic}
\usepackage{eurosym,stmaryrd}
\usepackage{graphicx}
\usepackage[np]{numprint}
\usepackage{hyperref} 
\usepackage{tikz}
\usepackage{colortbl}
\usepackage{multirow}
\usepackage{MnSymbol,wasysym}
\usepackage[top=1.5cm,bottom=1.5cm,right=1.2cm,left=1.5cm]{geometry}
\usetikzlibrary{calc, arrows, plotmarks, babel,decorations.pathreplacing}
\setstretch{1.25}
%\usepackage{lipsum} %\usepackage{enumitem} %\setlist[enumerate]{itemsep=1mm} bug avec enumerate



\newtheorem{thm}{Théorème}
\newtheorem{rmq}{Remarque}
\newtheorem{prop}{Propriété}
\newtheorem{cor}{Corollaire}
\newtheorem{lem}{Lemme}
\newtheorem{prop-def}{Propriété-définition}

\theoremstyle{definition}

\newtheorem{defi}{Définition}
\newtheorem{ex}{Exemple}
\newtheorem*{rap}{Rappel}
\newtheorem{cex}{Contre-exemple}
\newtheorem{exo}{Exercice} % \large {\fontfamily{ptm}\selectfont EXERCICE}
\newtheorem{nota}{Notation}
\newtheorem{ax}{Axiome}
\newtheorem{appl}{Application}
\newtheorem{csq}{Conséquence}
\def\di{\displaystyle}



\renewcommand{\thesection}{\Roman{section}}\renewcommand{\thesubsection}{\arabic{subsection} }\renewcommand{\thesubsubsection}{\alph{subsubsection} }


\newcommand{\bas}{~\backslash}\newcommand{\ba}{\backslash}
\newcommand{\C}{\mathbb{C}}\newcommand{\R}{\mathbb{R}}\newcommand{\Q}{\mathbb{Q}}\newcommand{\Z}{\mathbb{Z}}\newcommand{\N}{\mathbb{N}}\newcommand{\V}{\overrightarrow}\newcommand{\Cs}{\mathscr{C}}\newcommand{\Ps}{\mathscr{P}}\newcommand{\Rs}{\mathscr{R}}\newcommand{\Gs}{\mathscr{G}}\newcommand{\Ds}{\mathscr{D}}\newcommand{\happy}{\huge\smiley}\newcommand{\sad}{\huge\frownie}\newcommand{\danger}{\begin{tikzpicture}[x=1.5pt,y=1.5pt,rotate=-14.2]
	\definecolor{myred}{rgb}{1,0.215686,0}
	\draw[line width=0.1pt,fill=myred] (13.074200,4.937500)--(5.085940,14.085900)..controls (5.085940,14.085900) and (4.070310,15.429700)..(3.636720,13.773400)
	..controls (3.203130,12.113300) and (0.917969,2.382810)..(0.917969,2.382810)
	..controls (0.917969,2.382810) and (0.621094,0.992188)..(2.097660,1.359380)
	..controls (3.574220,1.726560) and (12.468800,3.984380)..(12.468800,3.984380)
	..controls (12.468800,3.984380) and (13.437500,4.132810)..(13.074200,4.937500)
	--cycle;
	\draw[line width=0.1pt,fill=white] (11.078100,5.511720)--(5.406250,11.875000)..controls (5.406250,11.875000) and (4.683590,12.812500)..(4.367190,11.648400)
	..controls (4.050780,10.488300) and (2.375000,3.675780)..(2.375000,3.675780)
	..controls (2.375000,3.675780) and (2.156250,2.703130)..(3.214840,2.964840)
	..controls (4.273440,3.230470) and (10.640600,4.847660)..(10.640600,4.847660)
	..controls (10.640600,4.847660) and (11.332000,4.953130)..(11.078100,5.511720)
	--cycle;
	\fill (6.144520,8.839900)..controls (6.460940,7.558590) and (6.464840,6.457090)..(6.152340,6.378910)
	..controls (5.835930,6.300840) and (5.320300,7.277400)..(5.003900,8.554750)
	..controls (4.683590,9.835940) and (4.679690,10.941400)..(4.996090,11.019600)
	..controls (5.312490,11.097700) and (5.824210,10.121100)..(6.144520,8.839900)
	--cycle;
	\fill (7.292960,5.261780)..controls (7.382800,4.898500) and (7.128900,4.523500)..(6.730460,4.421880)
	..controls (6.328120,4.324220) and (5.929680,4.535220)..(5.835930,4.898500)
	..controls (5.746080,5.261780) and (5.999990,5.640630)..(6.402340,5.738340)
	..controls (6.804690,5.839840) and (7.203110,5.625060)..(7.292960,5.261780)
	--cycle;
	\end{tikzpicture}}\newcommand{\alors}{\Large\Rightarrow}\newcommand{\equi}{\Leftrightarrow}
\newcommand{\fonction}[5]{\begin{array}{l|rcl}
		#1: & #2 & \longrightarrow & #3 \\
		& #4 & \longmapsto & #5 \end{array}}


\definecolor{vert}{RGB}{11,160,78}
\definecolor{rouge}{RGB}{255,120,120}
\definecolor{bleu}{RGB}{15,5,107}


\pagestyle{fancy}
\lhead{Optimal Sup Spé, groupe IPESUP}\chead{Année~2022-2023}\rhead{Niveau: Première année de PCSI }\lfoot{M. Botcazou}\cfoot{\thepage}\rfoot{mail: i.botcazou@gmx.fr }\renewcommand{\headrulewidth}{0.4pt}\renewcommand{\footrulewidth}{0.4pt}

\begin{document}
	
	
	\begin{center}
		\Large \sc colle 9 = Fonctions dérivables
	\end{center}
\raggedright

\section*{Connaître son cours:}
\begin{enumerate}
	\item Soit $f : I \rightarrow \R$ dérivable, montrer que: 
	
	Si $f$ admet un extremum local en un point $a$ intérieur à $I$, alors $f'(a) = 0 $.
	\item Énoncer le théorème de \emph{Rolle} et donner une démonstration de celui-ci. 
	\item Soit $f , g : I \rightarrow \R$ des fonctions $n$ fois dérivables. Énoncer la formule de Leibniz pour la dérivée $n^{\text{ième}}$ de la fonction $f\times g$. En déduire la dérivée $n$-ième de la fonction suivante :
	$x\mapsto x^{n-1}\ln(1+x).$
	
\end{enumerate}

\section*{Exercices:}	


\begin{exo}\textbf{(**)}\quad\\[0.25cm]
		On considère dans tout cet exercice les deux fonctions $F$ et $G$ définies sur $\mathbb{R}_{+}^{*}$ par :
	
	$$
	F(x)=\frac{\sin (x)}{x} \quad G(x)=\frac{1-\cos (x)}{x}
	$$
	\begin{enumerate}
		\item \begin{enumerate}
			\item Montrer que $F$ et $G$ sont continues et prolongeables par continuité en 0. On notera encore $F$ et $G$ ces prolongements.
			\item Montrer que les fonctions $F$ et $G$ sont dérivables sur $\mathbb{R}^{*}$ et calculer leurs dérivées.
			\item Montrer que les fonctions $F$ et $G$ sont dérivables en 0 . Préciser les valeurs de $F^{\prime}(0)$ et $G^{\prime}(0)$.
		\end{enumerate}
		\item \begin{enumerate}
			\item  Montrer que les réels strictement positifs tels que $F(x)=0$ constituent une suite $\left(a_{k}\right)_{k \in \mathbb{N}^{*}}$ strictement croissante. On donnera explicitement la valeur de $a_{k}$
			\item Montrer que les réels strictement positifs tels que $G(x)=0$ constituent une suite $\left(b_{k}\right)_{k \in \mathbb{N}^{*}}$ strictement croissante. Y a-t-il un lien entre les suites $\left(a_{k}\right)_{k \in \mathbb{N}^{*}}$ et $\left(b_{k}\right)_{k \in \mathbb{N}^{*}}$ ?
		\end{enumerate}
	\item \begin{enumerate}
		\item Soit $k \in \mathbb{N}^{*}$. Montrer qu'il existe un réel $\left.x_{k} \in\right] a_{k}, a_{k+1}\left[\right.$ tel que $F^{\prime}\left(x_{k}\right)=0$.
		\item Montrer que la fonction $F^{\prime}$ est de même signe que $h: x \longmapsto x \cos (x)-\sin (x)$ sur $\mathbb{R}_{+}^{*}$ et que pour tout $k \in \mathbb{N}^{*}$, la fonction $h$ est strictement monotone sur $\left[a_{k}, a_{k+1}\right]$.
		\item En déduire l'unicité du réel $x_{k}$ défini dans la question 4.(a).
		\item Etablir que: $\left.\forall k \in \mathbb{N}^{*}, \ x_{k} \in\right] a_{k}, a_{k}+\frac{\pi}{2}[$.
	\end{enumerate}
	\item Calculer $\lim\limits_{k \rightarrow+\infty} x_{k}$ puis déterminer un équivalent simple de la suite $\left(x_{k}\right)_{k \in \mathbb{N}^{*}}$.
	\end{enumerate}


	\centering
\rule{1\linewidth}{0.6pt}
\end{exo}
	
	

\begin{exo}\textbf{(**)}\quad\textit{Polynômes de Laguerre}\\[0.25cm]
On pose, pour tout entier naturel $n$ et pour tout réel $x$, 
$$h_n(x)=x^ne^{-x}\textrm{ et }L_n(x)=\frac{e^x}{n!}h_n^{(n)}(x).$$
\begin{enumerate}
	\item Montrer que, pour tout entier $n$, $L_n$ est une fonction polynômiale. 
	
	Préciser son degré et son coefficient dominant.
	\item Montrer que, pour tout $k\in\{0,\dots,n\}$, il existe $Q_k\in\mathbb R[X]$ tel que, pour tout $x\in\mathbb R$, on a
	$$h_n^{(k)}(x)=x^{n-k}e^{-x}Q_k(x).$$
\end{enumerate}

	
	%\centering
	%\rule{1\linewidth}{0.6pt}
\end{exo}


\begin{exo}\textbf{(***)}\quad\\[0.25cm]
	Soit $f$ définie sur un intervalle ouvert $I$ contenant $0$, continue sur $I$.
	On suppose en outre que $\lim_{x\to 0}\frac{f(2x)-f(x)}{x}=0$. Montrer que $f$ est dérivable en 0.
	
	\centering
	\rule{1\linewidth}{0.6pt}
\end{exo}

\begin{exo}\textbf{(**)}\quad\\[0.25cm]
	Soit $f:]0,+\infty[\to\mathbb R$ une fonction dérivable et $\ell\in\mathbb R$ tel que $\lim\limits_{x\to+\infty}f'(x)=\ell$.
	
	 L'objectif de cet exercice est de démontrer que $\lim\limits_{x\to+\infty}\frac{f(x)}x=\ell$. 
	\begin{enumerate}
		\item On suppose dans cette question que $\ell=0$. Soit $\varepsilon>0$.
		\begin{enumerate}
			\item Montrer qu'il existe $A>0$ tel que, pour tout $x\geq A$, on a 
			$$\left|\frac{f(x)}{x}\right|\leq \left|\frac{f(A)}{x}\right|+\varepsilon.$$
			\item En déduire le résultat dans ce cas.
		\end{enumerate}
		\item Démontrer le résultat dans le cas général.
		\item Réciproquement, est-il vrai que pour toute fonction dérivable $f:]0,+\infty[\to \mathbb R$ telle que $\lim\limits_{x\to+\infty}\frac{f(x)}x=\ell$, alors on a $\lim\limits_{x\to+\infty}f'(x)=\ell$?
	\end{enumerate}
	
	\centering
	\rule{1\linewidth}{0.6pt}
\end{exo}

\begin{exo}\textbf{(**)}\quad\\[0.25cm]
Déterminer la dérivée d'ordre $n$ de la fonction $f$ définie par $f(x)=(x-a)^n (x-b)^n$
($a,b$ sont des réels). En étudiant le cas $a=b$, trouver la valeur de $\sum_{k=0}^n \binom{n}{k}^2$.
	
	\centering
	\rule{1\linewidth}{0.6pt}
\end{exo}


\begin{exo}\textbf{(**)}\quad\\[0.25cm]
Soit $n\in\mathbb N$. Montrer que la dérivée d'ordre $n+1$ de $x^ne^{1/x}$ est
$$\frac{(-1)^{n+1}}{x^{n+2}}e^{1/x}.$$
	
	\centering
	\rule{1\linewidth}{0.6pt}
\end{exo}


\end{document}
