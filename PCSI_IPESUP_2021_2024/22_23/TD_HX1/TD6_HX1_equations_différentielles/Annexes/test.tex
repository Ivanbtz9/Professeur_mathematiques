\documentclass[a4paper,11pt]{article}

\usepackage{inputenc}
\usepackage[T1]{fontenc}
\usepackage[frenchb]{babel}
\usepackage{fancyhdr,fancybox} % pour personnaliser les en-têtes
\usepackage{lastpage,setspace}
\usepackage{amsfonts,amssymb,amsmath,amsthm,mathrsfs}
\usepackage{relsize,exscale,bbold}
\usepackage{paralist}
\usepackage{xspace,multicol,diagbox,array}
\usepackage{xcolor}
\usepackage{variations}
\usepackage{xypic}
\usepackage{eurosym,stmaryrd}
\usepackage{graphicx}
\usepackage[np]{numprint}
\usepackage{hyperref} 
\usepackage{tikz}
\usepackage{colortbl}
\usepackage{multirow}
\usepackage{MnSymbol,wasysym}
\usepackage[top=1.5cm,bottom=1.5cm,right=1.2cm,left=1.5cm]{geometry}
\usetikzlibrary{calc, arrows, plotmarks, babel,decorations.pathreplacing}
\setstretch{1.25}
%\usepackage{lipsum} %\usepackage{enumitem} %\setlist[enumerate]{itemsep=1mm} bug avec enumerate



\newtheorem{thm}{Théorème}
\newtheorem{rmq}{Remarque}
\newtheorem{prop}{Propriété}
\newtheorem{cor}{Corollaire}
\newtheorem{lem}{Lemme}
\newtheorem{prop-def}{Propriété-définition}

\theoremstyle{definition}

\newtheorem{defi}{Définition}
\newtheorem{ex}{Exemple}
\newtheorem*{rap}{Rappel}
\newtheorem{cex}{Contre-exemple}
\newtheorem{exo}{Exercice} % \large {\fontfamily{ptm}\selectfont EXERCICE}
\newtheorem{nota}{Notation}
\newtheorem{ax}{Axiome}
\newtheorem{appl}{Application}
\newtheorem{csq}{Conséquence}
\def\di{\displaystyle}



\renewcommand{\thesection}{\Roman{section}}\renewcommand{\thesubsection}{\arabic{subsection} }\renewcommand{\thesubsubsection}{\alph{subsubsection} }


\newcommand{\bas}{~\backslash}\newcommand{\ba}{\backslash}
\newcommand{\C}{\mathbb{C}}\newcommand{\R}{\mathbb{R}}\newcommand{\Q}{\mathbb{Q}}\newcommand{\Z}{\mathbb{Z}}\newcommand{\N}{\mathbb{N}}\newcommand{\V}{\overrightarrow}\newcommand{\Cs}{\mathscr{C}}\newcommand{\Ps}{\mathscr{P}}\newcommand{\Rs}{\mathscr{R}}\newcommand{\Gs}{\mathscr{G}}\newcommand{\Ds}{\mathscr{D}}\newcommand{\happy}{\huge\smiley}\newcommand{\sad}{\huge\frownie}\newcommand{\danger}{\begin{tikzpicture}[x=1.5pt,y=1.5pt,rotate=-14.2]
	\definecolor{myred}{rgb}{1,0.215686,0}
	\draw[line width=0.1pt,fill=myred] (13.074200,4.937500)--(5.085940,14.085900)..controls (5.085940,14.085900) and (4.070310,15.429700)..(3.636720,13.773400)
	..controls (3.203130,12.113300) and (0.917969,2.382810)..(0.917969,2.382810)
	..controls (0.917969,2.382810) and (0.621094,0.992188)..(2.097660,1.359380)
	..controls (3.574220,1.726560) and (12.468800,3.984380)..(12.468800,3.984380)
	..controls (12.468800,3.984380) and (13.437500,4.132810)..(13.074200,4.937500)
	--cycle;
	\draw[line width=0.1pt,fill=white] (11.078100,5.511720)--(5.406250,11.875000)..controls (5.406250,11.875000) and (4.683590,12.812500)..(4.367190,11.648400)
	..controls (4.050780,10.488300) and (2.375000,3.675780)..(2.375000,3.675780)
	..controls (2.375000,3.675780) and (2.156250,2.703130)..(3.214840,2.964840)
	..controls (4.273440,3.230470) and (10.640600,4.847660)..(10.640600,4.847660)
	..controls (10.640600,4.847660) and (11.332000,4.953130)..(11.078100,5.511720)
	--cycle;
	\fill (6.144520,8.839900)..controls (6.460940,7.558590) and (6.464840,6.457090)..(6.152340,6.378910)
	..controls (5.835930,6.300840) and (5.320300,7.277400)..(5.003900,8.554750)
	..controls (4.683590,9.835940) and (4.679690,10.941400)..(4.996090,11.019600)
	..controls (5.312490,11.097700) and (5.824210,10.121100)..(6.144520,8.839900)
	--cycle;
	\fill (7.292960,5.261780)..controls (7.382800,4.898500) and (7.128900,4.523500)..(6.730460,4.421880)
	..controls (6.328120,4.324220) and (5.929680,4.535220)..(5.835930,4.898500)
	..controls (5.746080,5.261780) and (5.999990,5.640630)..(6.402340,5.738340)
	..controls (6.804690,5.839840) and (7.203110,5.625060)..(7.292960,5.261780)
	--cycle;
	\end{tikzpicture}}\newcommand{\alors}{\Large\Rightarrow}\newcommand{\equi}{\Leftrightarrow}
\newcommand{\fonction}[5]{\begin{array}{l|rcl}
		#1: & #2 & \longrightarrow & #3 \\
		& #4 & \longmapsto & #5 \end{array}}


\definecolor{vert}{RGB}{11,160,78}
\definecolor{rouge}{RGB}{255,120,120}
\definecolor{bleu}{RGB}{15,5,107}



\begin{document}

TD Suites
Exercice 1 Pour chacune des assertions suivantes, dire si elle est vraie ou fausse, $5 i$ elle est vraie donner une preuve, sinon, donner un contre-exemple.
Fxercice 5 Soit $\left(u_N\right)$ une suite reclle.
1. On suppose que les suites extraites $\left(t_{2 n}\right)_{n \in N}$ et $\left(u_{2 n+1}\right)_{n \in N}$ convergent vers une meme limite $\ell$. R. Montrer que ( $\left.u_n\right)$ converge vers $f$.
2. On suppose que $\left(u_{2 n}\right) \cdot\left(u_{2 n+1}\right)$ et $\left(u_{m n}\right)$ convergent. Montrer que $\left(t_n\right)$ eonverge. xercice 6 Soit $\left(u_s\right)$ une suite de termes strictement positifs. On suppose que $\frac{u_{n-1}}{u_n} \frac{1}{\pi+\infty} \ell_4$ avec $\ell>1$.
1. Mantrer que la suite $\left(u_n\right)$ est croisfante A partir diun certain rang.
2. Montrer que $t_n \rightarrow+\infty$
Exercice 7 Soit $\left(x_N\right)$ une suite de recels
1. Montrer que si $x_{n+1}-x_{n i} \rightarrow \ell, \frac{\pi}{n} \rightarrow \ell$
Indication : Lemme de Cesaro.
2. On suppose que $\forall n \in N_1$, Fn $>0$ et $\frac{x_{n-1}}{x_n} \rightarrow \ell$. Montrer que $\sqrt[3]{x_n} \rightarrow \ell$
3. Déterminer les limites eventuelles des suites $\sqrt[n]{\left(\begin{array}{c}2 n \\ n\end{array}\right)}$ et $\frac{\sqrt[5]{\pi}}{n}$.
Exercice 8 Soit $\left(u_n\right),\left(v_n\right)$ deux suites a valeurs dans $[0,1]$ telles que $u_n v_n \rightarrow 1$. Montrer que $u_n \rightarrow 1$ et $v_n \rightarrow 1$.
Exercice 9 On suppose $u_n \rightarrow \ell, v_n \rightarrow \ell^{\prime}$. Montrer que max $\left(u_n, v_a\right) \rightarrow \operatorname{max}\left(\ell, \ell^{\prime}\right)$.
Exercice 10. Déterminer la limite; quand $n \rightarrow+\infty$; de $\left(\sin \frac{1}{n}+y^3 \cos n\right)^n$.
Exercice 11 Etudier la limite de sin $\left(\pi \sqrt{\pi^2+b}\right)$
Exercice 12 Déterminer la limite, quand $n \rightarrow+\infty$, de $E\left(a^n\right)^{2 / a}$, en fonction de $a \in R_{+}$.
Exercice 13. Soit $a_i, \ldots, a_m>0$. Montrer que $\left(\sum_{i=1}^m a_i^n\right)^{\frac{1}{k}} \frac{n_{n \rightarrow+\infty}}{\operatorname{minx}} a_{i \leq 1} a_i$.
Exercice 14 Soit $u_r=\sum_e^n(-1)\left(\begin{array}{l}1 \\ s\end{array}\right)$
1. Etudier la limite de une-
2. Montrer que $\sum_{k-0}^i \frac{1}{(\text{ iv) }}=\frac{m_a}{2}$.
Exercice 16 * Soit $f: N-+N$ une bijection.
1. Montrer que $f(m) \rightarrow+\infty$.
2. On suppose que $\frac{f(b)}{\pi}+L$ Montrer que $L=1$
3. La limite precedente existe:t-elle toujours?


Exercice 17 Soit $\left(u_n\right)$ une suite minorée telle que $u_n+\left(\tan u_n\right)^2$ converge.
1. Montrer que $\left(u_n\right)$ est majorée.
2. Montrer que $\left(\tan ^2 u_n\right)$ est bornée.
3. On suppose que $\left(\tan u_n\right)$ est croissante. Montrer que $\left(u_n\right)$ converge.
Exercice 18 On considère la suite $\left(u_n\right)_{n \in \mathbb{N}}$ définie par $u_0>0$ et $\forall n \in \mathbb{N}, u_{n+1}=u_n+\frac{1}{u_n}$
1. Montrer que $\forall n \in \mathbb{N}, u_n \geq 0$.
2. Montrer que $u_n \rightarrow+\infty$.
3. Déterminer la limite de $u_{n+1}^2-u_n^2$. En utilisant le théorème de Cesàro, montrer que $u_n \sim \sqrt{2 n}$.
Exercice 19 On considère les suites définies par $u_n=\sum_{p=0}^n \frac{1}{p !}$ et $v_n=u_n+\frac{1}{n !}$.
1. Montrer que ces deux suites sont adjacentes.
On admet que leur limite est $e=e^1$.
2. Montrer que leur limite est irrationnelle.
Indication : Procéder par l'absurde et écrire les inégalités pour un n bien choisi.
Exercice $20 \star \operatorname{Soit}\left(u_n\right)$ une suite réelle positive vérifiant $\forall n \in \mathcal{N}, u_{n+2} \leq \frac{u_{n+1}+u_n}{3}$.
1. Soit $v_n=\max \left(u_n, u_{n+1}\right)$. Montrer que $v_n$ est décroissante.
2. Montrer que $v_n \rightarrow 0$, puis que $u_n \rightarrow 0$.
Exercice 21 Distance À UNE PARTIE Soit $A \subset \mathbb{R}$ non vide. Pour $x \in \mathbb{R}$, on pose $d(x, A)=\inf _{y \in A}|x-y|$.
1. Justifier la définition de $d(x, A)$.
2. Montrer que $x \mapsto d(x, A)$ est 1 -lipschitzienne, c'est-à-dire
$$
\forall x, y \in \mathbb{R}, \quad|d(x, A)-d(y, A)| \leq|x-y|
$$
En particulier, la fonction $x \mapsto d(x, A)$ est continue.
Exercice 22 Soit $\left(u_n\right)$ une suite réelle et $\ell \in \bar{R}$. Montrer que $\left(u_n\right)$ ne tend pas vers $\ell$ si et seulement s'il existe $\ell^{\prime} \in \bar{R}, \ell^{\prime} \neq \ell$ et une suite $\left(v_n\right)$ extraite de $\left(u_n\right)$ tels que $v_n \rightarrow \ell^{\prime}$.
Exercice 23 LIMITES INFÉRIEURE ET SUPÉRIEURE Soit $\left(u_n\right)_{n \in \mathbb{N}}$ une suite bornée. Pour $n \in \mathbb{N}$ On pose
$$
\alpha_n=\inf \left\{u_k, k \geq n\right\} \quad \text { et } \quad \beta_n=\sup \left\{u_k, k \geq n\right\} .
$$
1. Montrer que $\left(\alpha_n\right)_{n \in \mathbb{N}}$ et $\left(\beta_n\right)_{n \in \mathbb{N}}$ convergent. On note $\alpha=\lim _{n \rightarrow+\infty} \alpha_n$, appelé limite inférieure de $\left(u_n\right)_{n \in \mathbb{N}}$, et $\beta=\lim _{n \rightarrow+\infty} \beta_n$ sa limite supérieure.
2. Montrer que $\left(u_n\right)$ converge si et seulement si $\alpha=\beta$.
3. Montrer que $\beta$ est la plus grande valeur d'adhérence de $\left(u_n\right)$.
En particulier, on a démontré le théorème de Bolzano-Weierstrass.
Exercice 24 On considère une suite $\left(u_n\right)$ vérifiant $u_{n+1}-u_n \rightarrow 0$ et $u_n \rightarrow+\infty$.
1. Justifier que si $\left|u_{n+1}-u_n\right| \leq \varepsilon$ à partir du rang $n_0$, et que $x \geq x_{n_0}$, il existe $p \in \mathbb{N}$ tel que $\left|u_p-x\right| \leq \varepsilon$.
2. Soit $\left(v_n\right)_{n \in \mathbb{N}}$ une suite telle que $v_n \rightarrow+\infty$. Montrer que $\left\{u_n-v_m, n, m \in \mathbb{N}\right\}$ est dense dans $\mathbb{R}$.
3. En déduire que $\left\{u_n-\left\lfloor u_n\right\rfloor, n \in \mathbb{N}\right\}$ est dense dans $[0,1]$.
4. Montrer que $\{\sin \ln (n), n \in \mathbb{N}\}$ est dense dans $[-1,1]$.
Exercice 25 1. Soit $\left(a_n\right),\left(b_n\right)$ deux suites réelles telles que $a_n+b_n \rightarrow 0$ et $e^{a_n}+e^{b_n} \rightarrow 2$. Montrer que $\left(a_n\right)$ et $\left(b_n\right)$ convergent.
2. Soit $\left(a_n\right),\left(b_n\right),\left(c_n\right)$ trois suites réelles telles que $a_n+b_n+c_n \rightarrow 0$ et $e^{a_n}+e^{b_n}+e^{c_n} \rightarrow 3$. Montrer que $\left(a_n\right)$, ( $\left.b_n\right)$, ( $\left.c_n\right)$ convergent.
Exercice $26 \star \$$ Suites sous-AdDitives Soit $\left(u_n\right)$ une suite réelle vérifiant $\forall p, q \in \mathbb{N}, u_{p+q} \leq u_p+u_q$. Montrer que la suite $\left(\frac{u_n}{n}\right)$ converge vers sa borne inférieure $\inf _{n \in \mathbb{N}^*} \frac{u_n}{n}$.
2

\end{document}