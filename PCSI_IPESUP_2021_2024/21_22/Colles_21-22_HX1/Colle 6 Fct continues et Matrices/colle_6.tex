\documentclass[a4paper,10pt]{article}



\usepackage{fancyhdr} % pour personnaliser les en-têtes
\usepackage[utf8]{inputenc}
\usepackage[T1]{fontenc}
\usepackage{lastpage}
\usepackage[frenchb]{babel}
\usepackage{amsfonts,amssymb}
\usepackage{amsmath,amsthm,mathtools}
\usepackage{paralist}
\usepackage{xspace}
\usepackage{xcolor,multicol}
\usepackage{variations}
\usepackage{xypic}
\usepackage{eurosym}
\usepackage{graphicx}
\usepackage[np]{numprint}
\usepackage{hyperref} 
\usepackage{listings} % pour écrire des codes avec coloration syntaxique  

\usepackage{tikz}
\usetikzlibrary{calc, arrows, plotmarks,decorations.pathreplacing}
\usepackage{colortbl}
\usepackage{multirow}
\usepackage[top=2cm,bottom=1.5cm,right=2cm,left=1.5cm]{geometry}

\newtheorem{thm}{Théorème}
\newtheorem*{pro}{Propriété}
\newtheorem*{exemple}{Exemple}

\theoremstyle{definition}
\newtheorem*{remarque}{Remarque}
\theoremstyle{definition}
\newtheorem{exo}{Exercice}
\newtheorem{definition}{Définition}


\newcommand{\vtab}{\rule[-0.4em]{0pt}{1.2em}}
\newcommand{\V}{\overrightarrow}
\renewcommand{\thesection}{\Roman{section} }
\renewcommand{\thesubsection}{\arabic{subsection} }
\renewcommand{\thesubsubsection}{\alph{subsubsection} }
\newcommand*{\transp}[2][-3mu]{\ensuremath{\mskip1mu\prescript{\smash{\mathrm t\mkern#1}}{}{\mathstrut#2}}}%

\newcommand{\C}{\mathbb{C}}
\newcommand{\R}{\mathbb{R}}
\newcommand{\Q}{\mathbb{Q}}
\newcommand{\Z}{\mathbb{Z}}
\newcommand{\N}{\mathbb{N}}



\definecolor{vert}{RGB}{11,160,78}
\definecolor{rouge}{RGB}{255,120,120}
% Set the beginning of a LaTeX document
\pagestyle{fancy}
\lhead{Optimal Sup Spé, groupe IPESUP}\chead{Année~2021-2022}\rhead{Niveau: Première année de PCSI }\lfoot{M. Botcazou}\cfoot{\thepage/2}\rfoot{mail: ibotca52@gmail.com }\renewcommand{\headrulewidth}{0.4pt}\renewcommand{\footrulewidth}{0.4pt}

\begin{document}
	
	
	\begin{center}
		\Large \sc colle 6 = fonctions continues et Matrices
	\end{center}
	
\section *{Questions de cours:}
%\hyperref{http://www.normalesup.org/~sage/Enseignement/TSI/index.html}
\noindent Soient $I$ un intervalle de $\R$, $f:I\rightarrow\R$ et $g:I\rightarrow\R$ deux fonctions, $\lambda,\beta\in\R$,  $a\in I$ et $l\in\R$.
\begin{enumerate}
\item Démontrer la propriété suivante:
\begin{pro}\hfil\\
$f$ tend vers $l$  en $a$ si et seulement si  pour toute suite $(u_n)_{n\in\N}$ convergeant vers $a$, $(f(u_n))_{n\in\N}$ converge vers $l$
\end{pro} 
\item  Démontrer la propriété suivante:
\begin{pro}\hfil\\
Si $f$ tend vers $l$  en $a$ et $g$ tend vers $l'$  en $a$   alors $\lambda f+\beta g$ tend vers $\lambda l+ \beta l'$ en $a$.
\end{pro} 
\item  Démontrer la propriété suivante:
\begin{pro}\hfil\\
Si $f$ tend vers $l$  en $a$ et $g$ tend vers $l'$  en $a$   alors $fg$ tend vers $ll'$ en $a$.
\end{pro}
\item  Démontrer la propriété suivante:
\begin{pro}\hfil\\
Si $f$ tend vers $l$  en $a$ avec $l\neq 0$   alors $\dfrac{1}{f}$ tend vers $\dfrac{1}{l}$  en $a$.
\end{pro} 
\end{enumerate}
\noindent Soient $n\in\N$ et  $A,B\in\mathcal{M}_{n}\left(\R\right)$.
\begin{enumerate} 
\item[6.]Que signifie que la matrice $A$ et est une matrice symétrique? antisymétrique? Montrer que l'on peut toujours écrire la matrice $A$ comme la somme d'une matrice symétrique avec une matrice antisymétrique. 
\item[7.] Rappeler la définition de $Tr(A)$, et calculer $\sum_{k=1}^{n}Tr(I_k)$.
\item[8.] Pour $(i,j)\in\{1,..,n\}^2$ donner le coefficient $AB_{ij}$ en fonction des coefficients des matrices A et B. Montrer que $Tr(AB) = Tr(BA)$.
\end{enumerate}

\section*{Fonctions continues:}
\begin{minipage}{1\linewidth}
\begin{minipage}[t]{0.48\linewidth}
\raggedright

\begin{exo}\quad\\
Donner si elles existe les limites suivantes:
\begin{multicols}{2}
\begin{enumerate}
\item $\lim\limits_{x\rightarrow+\infty} \dfrac{\lfloor 2x\rfloor}{\lfloor x\rfloor}$
\item $\lim\limits_{x\rightarrow 0}~ \left\lfloor\dfrac{ 1}{ x} \right\rfloor$
\item $\lim\limits_{x\rightarrow 0} ~x\left\lfloor\dfrac{ 1}{ x} \right\rfloor$
\item$\lim\limits_{x\rightarrow 0} ~x^2\left\lfloor\dfrac{ 1}{ x} \right\rfloor$
\end{enumerate}
\end{multicols}
\centering\rule{1\linewidth}{0.6pt}
\end{exo}

\begin{exo}\quad\\
Soit $f: \R^*\rightarrow\R$ la fonction définie par $$f(x) = x\sqrt{1+\dfrac{1}{x^2}}$$
La fonction $f$ admet-elle un prolongement par continuité en $0$ ? 

\centering
\rule{1\linewidth}{0.6pt}
\end{exo}

\begin{exo}\quad\\
Soit $f: \R\rightarrow\R$ la fonction définie par $$f(x) =\lfloor x\rfloor + \sqrt{x - \lfloor x\rfloor }$$
Montrer que la fonction $f$ est continue sur $\R$.\\
(\textit{Indication: Étudier }$f(x+1)$)

\centering
\rule{1\linewidth}{0.6pt}
\end{exo}

\end{minipage}	
\hfill\vrule\hfill
\begin{minipage}[t]{0.48\linewidth}
\raggedright


\begin{exo}\quad\\
Soit $f: \R\rightarrow\R$ périodique et admettant une limite finie $l$ en $+\infty$. Montrer que $f$ est constante.

\centering
\rule{1\linewidth}{0.6pt}
\end{exo}

\begin{exo}\quad\\
Étudier les limites suivantes:

\begin{tabular}{cc}
 1. \   $\lim\limits_{x\rightarrow+\infty} \dfrac{e^{3x}+2x+7}{e^x + e^{-x}}$
&2. \ $\lim\limits_{x\rightarrow 0}~\dfrac{\sqrt{1+x} - \left(1+ \dfrac{x}{2}\right) }{x^2 } $
\end{tabular}
\hfill\\ \hfill\\
\centering\rule{1\linewidth}{0.6pt}
\end{exo}
\begin{exo}\quad\\
Soit $f:\R_+\rightarrow\R_+^*$ une fonction continue telle que $\lim\limits_{x\rightarrow+\infty}\dfrac{f(x)}{x} = 0$. Montrer que la fonction $f$ admet un point fixe sur $\R^+$.

\centering
\rule{1\linewidth}{0.6pt}
\end{exo}
\begin{exo}\quad\\
Soit $f: \R\rightarrow\R$ la fonction définie par 
$$f(x) =\left\{\begin{array}{cl}
0  & \text{Si} \ x \   \text{est irrationnel ou} \  x=0. \\
\dfrac{1}{q} & \text{Si} \  x=\dfrac{p}{q},\ \text{avec} \  p\in\Z, q\geq 1 \ \text{et} \ pgcd(p,q) =1  
\end{array}\right.$$
Montrer que la fonction $f$ est continue sur $\R \backslash \Q\cup\{0\}$, discontinue sur $\Q^*$ \\

\centering
\rule{1\linewidth}{0.6pt}
\end{exo}

\end{minipage}
\end{minipage}
\newpage
\section*{Matrices:}
\begin{minipage}{1\linewidth}
\begin{minipage}[t]{0.48\linewidth}
\raggedright

\begin{exo}\quad\\
Soient  $A,B\in\mathcal{M}_{n}\left(\R\right)$ et $\lambda,\beta\in\R$.
\begin{enumerate}
\item Montrer que  $Tr(\transp{A}A) \geq 0$. Que peut-on en déduire sur la matrice $A$ si $Tr(\transp{A}A) = 0$?
\item Montrer que  $Tr(\lambda A + \beta B) = \lambda Tr(A) + \beta Tr(B)$.
\item En déduire que si pour tout $M\in \mathcal{M}_{n}\left(\R\right)$  on a: $Tr(XA) = Tr(XB) $ alors $A=B$.
\end{enumerate}

\centering\rule{1\linewidth}{0.6pt}
\end{exo}

\begin{exo}\quad\\
Soit $$A \ = \ \begin{pmatrix}
0 & 1 & -1\\
-1 &  2& -1\\
1 & -1 & 2
\end{pmatrix}$$
\begin{enumerate}
\item Montrer que le polynôme $P(X) = X^2-3X+2$ est anulateur de la matrice $A$.
\item Donner le reste de la division Euclidienne de $X^n$ par  $X^2-3X+2$ pour $n\geq 2$.
\item En déduire la valeur de $A^n$. 
\end{enumerate}

\centering
\rule{1\linewidth}{0.6pt}
\end{exo}

\end{minipage}	
\hfill\vrule\hfill
\begin{minipage}[t]{0.48\linewidth}
\raggedright

\begin{exo}\quad\\
Soient $$A  =  \begin{pmatrix}
1 & 1 & 0\\
0 &  1& 1\\
0 & 0& 1
\end{pmatrix} \ , \ I_3  =  \begin{pmatrix}
1 & 0 & 0\\
0 &  1& 0\\
0 & 0& 1
\end{pmatrix} \ \text{et}  \ \ B  =  A-I_3
$$
Montrer que la matrice $B$ est nilpotente et en déduire pour tout $n\in\N$ l'expression de $A^n$.

\centering\rule{1\linewidth}{0.6pt}
\end{exo}

\begin{exo}\quad\\
Soit  $A\in\mathcal{M}_{3}\left(\R\right)$ 
\begin{enumerate}
\item Pour tout $(i,j)\in\{1,2,3\}$ on note $E_{ij}$ une matrice élémentaire de $\mathcal{M}_{3}\left(\R\right)$. Expliquer ce que donne les produits matriciels $ AE_{ij}$ et $E_{ij}A$. 
\item Considérons le centre de $\mathcal{M}_{3}\left(\R\right)$ :
 $$ \mathcal{Z}\left(\mathcal{M}_{3}\left(\R\right)\right) : \left\{A\in\mathcal{M}_{3}\left(\R\right); \forall M\in\mathcal{M}_{3}\left(\R\right)  : MA  = AM \right \}$$
 Montrer que: 
 $$\mathcal{Z}\left(\mathcal{M}_{3}\left(\R\right)\right) = \left\{\lambda I_3 : \lambda\in\R \right \}$$
 \item Que peut-on dire de $\mathcal{Z}\left(\mathcal{M}_{n}\left(\R\right)\right) $ ?
 
\end{enumerate}
\centering
\rule{1\linewidth}{0.6pt}
\end{exo}




\end{minipage}
\end{minipage}

\end{document}