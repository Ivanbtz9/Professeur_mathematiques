
%%%%%%%%%%%%%%%%%% PREAMBULE %%%%%%%%%%%%%%%%%%

\documentclass[11pt,a4paper]{article}

\usepackage{amsfonts,amsmath,amssymb,amsthm}
\usepackage[utf8]{inputenc}
\usepackage[T1]{fontenc}
\usepackage[francais]{babel}
\usepackage{mathptmx}
\usepackage{fancybox}
\usepackage{graphicx}
\usepackage{ifthen}

\usepackage{tikz}   

\usepackage{hyperref}
\hypersetup{colorlinks=true, linkcolor=blue, urlcolor=blue,
pdftitle={Exo7 - Exercices de mathématiques}, pdfauthor={Exo7}}

\usepackage{geometry}
\geometry{top=2cm, bottom=2cm, left=2cm, right=2cm}

%----- Ensembles : entiers, reels, complexes -----
\newcommand{\Nn}{\mathbb{N}} \newcommand{\N}{\mathbb{N}}
\newcommand{\Zz}{\mathbb{Z}} \newcommand{\Z}{\mathbb{Z}}
\newcommand{\Qq}{\mathbb{Q}} \newcommand{\Q}{\mathbb{Q}}
\newcommand{\Rr}{\mathbb{R}} \newcommand{\R}{\mathbb{R}}
\newcommand{\Cc}{\mathbb{C}} \newcommand{\C}{\mathbb{C}}
\newcommand{\Kk}{\mathbb{K}} \newcommand{\K}{\mathbb{K}}

%----- Modifications de symboles -----
\renewcommand{\epsilon}{\varepsilon}
\renewcommand{\Re}{\mathop{\mathrm{Re}}\nolimits}
\renewcommand{\Im}{\mathop{\mathrm{Im}}\nolimits}
\newcommand{\llbracket}{\left[\kern-0.15em\left[}
\newcommand{\rrbracket}{\right]\kern-0.15em\right]}
\renewcommand{\ge}{\geqslant} \renewcommand{\geq}{\geqslant}
\renewcommand{\le}{\leqslant} \renewcommand{\leq}{\leqslant}

%----- Fonctions usuelles -----
\newcommand{\ch}{\mathop{\mathrm{ch}}\nolimits}
\newcommand{\sh}{\mathop{\mathrm{sh}}\nolimits}
\renewcommand{\tanh}{\mathop{\mathrm{th}}\nolimits}
\newcommand{\cotan}{\mathop{\mathrm{cotan}}\nolimits}
\newcommand{\Arcsin}{\mathop{\mathrm{arcsin}}\nolimits}
\newcommand{\Arccos}{\mathop{\mathrm{arccos}}\nolimits}
\newcommand{\Arctan}{\mathop{\mathrm{arctan}}\nolimits}
\newcommand{\Argsh}{\mathop{\mathrm{argsh}}\nolimits}
\newcommand{\Argch}{\mathop{\mathrm{argch}}\nolimits}
\newcommand{\Argth}{\mathop{\mathrm{argth}}\nolimits}
\newcommand{\pgcd}{\mathop{\mathrm{pgcd}}\nolimits} 

%----- Structure des exercices ------

\newcommand{\exercice}[1]{\video{0}}
\newcommand{\finexercice}{}
\newcommand{\noindication}{}
\newcommand{\nocorrection}{}

\newcounter{exo}
\newcommand{\enonce}[2]{\refstepcounter{exo}\hypertarget{exo7:#1}{}\label{exo7:#1}{\bf Exercice \arabic{exo}}\ \  #2\vspace{1mm}\hrule\vspace{1mm}}

\newcommand{\finenonce}[1]{
\ifthenelse{\equal{\ref{ind7:#1}}{\ref{bidon}}\and\equal{\ref{cor7:#1}}{\ref{bidon}}}{}{\par{\footnotesize
\ifthenelse{\equal{\ref{ind7:#1}}{\ref{bidon}}}{}{\hyperlink{ind7:#1}{\texttt{Indication} $\blacktriangledown$}\qquad}
\ifthenelse{\equal{\ref{cor7:#1}}{\ref{bidon}}}{}{\hyperlink{cor7:#1}{\texttt{Correction} $\blacktriangledown$}}}}
\ifthenelse{\equal{\myvideo}{0}}{}{{\footnotesize\qquad\texttt{\href{http://www.youtube.com/watch?v=\myvideo}{Vidéo $\blacksquare$}}}}
\hfill{\scriptsize\texttt{[#1]}}\vspace{1mm}\hrule\vspace*{7mm}}

\newcommand{\indication}[1]{\hypertarget{ind7:#1}{}\label{ind7:#1}{\bf Indication pour \hyperlink{exo7:#1}{l'exercice \ref{exo7:#1} $\blacktriangle$}}\vspace{1mm}\hrule\vspace{1mm}}
\newcommand{\finindication}{\vspace{1mm}\hrule\vspace*{7mm}}
\newcommand{\correction}[1]{\hypertarget{cor7:#1}{}\label{cor7:#1}{\bf Correction de \hyperlink{exo7:#1}{l'exercice \ref{exo7:#1} $\blacktriangle$}}\vspace{1mm}\hrule\vspace{1mm}}
\newcommand{\fincorrection}{\vspace{1mm}\hrule\vspace*{7mm}}

\newcommand{\finenonces}{\newpage}
\newcommand{\finindications}{\newpage}


\newcommand{\fiche}[1]{} \newcommand{\finfiche}{}
%\newcommand{\titre}[1]{\centerline{\large \bf #1}}
\newcommand{\addcommand}[1]{}

% variable myvideo : 0 no video, otherwise youtube reference
\newcommand{\video}[1]{\def\myvideo{#1}}

%----- Presentation ------

\setlength{\parindent}{0cm}

\definecolor{myred}{rgb}{0.93,0.26,0}
\definecolor{myorange}{rgb}{0.97,0.58,0}
\definecolor{myyellow}{rgb}{1,0.86,0}

\newcommand{\LogoExoSept}[1]{  % input : echelle       %% NEW
{\usefont{U}{cmss}{bx}{n}
\begin{tikzpicture}[scale=0.1*#1,transform shape]
  \fill[color=myorange] (0,0)--(4,0)--(4,-4)--(0,-4)--cycle;
  \fill[color=myred] (0,0)--(0,3)--(-3,3)--(-3,0)--cycle;
  \fill[color=myyellow] (4,0)--(7,4)--(3,7)--(0,3)--cycle;
  \node[scale=5] at (3.5,3.5) {Exo7};
\end{tikzpicture}}
}


% titre
\newcommand{\titre}[1]{%
\vspace*{-4ex} \hfill \hspace*{1.5cm} \hypersetup{linkcolor=black, urlcolor=black} 
\href{http://exo7.emath.fr}{\LogoExoSept{3}} 
 \vspace*{-5.7ex}\newline 
\hypersetup{linkcolor=blue, urlcolor=blue}  {\Large \bf #1} \newline 
 \rule{12cm}{1mm} \vspace*{3ex}}

%----- Commandes supplementaires ------



\begin{document}

%%%%%%%%%%%%%%%%%% EXERCICES %%%%%%%%%%%%%%%%%%
\fiche{f00150, quinio, 2011/05/18}

Exercices : Martine Quinio

\titre{Probabilité et dénombrement ; indépendance}

\exercice{5983, quinio, 2011/05/18}

\enonce{005983}{}
Une entreprise décide de classer $20$ personnes
susceptibles d'être embauchées; leurs CV étant très proches,
le patron décide de recourir au hasard : combien y-a-il de classements
possibles : sans ex-aequo; avec exactement $2$ ex-aequo ?
\finenonce{005983}



\finexercice
\exercice{5984, quinio, 2011/05/18}

\enonce{005984}{}
Un étudiant s'habille très vite le matin et prend, au hasard
dans la pile d'habits, un pantalon, un tee-shirt, une paire de chaussettes;
il y a ce jour-là dans l'armoire $5$ pantalons dont $2$ noirs, 
$6$ tee-shirt dont $4$ noirs, $8$ paires de chaussettes, dont $5$ paires
noires. Combien y-a-t-il de façons de s'habiller? Quelles sont les
probabilités des événements suivants : il est tout en noir; une
seule pièce est noire sur les trois.
\finenonce{005984}


\finexercice
\exercice{5985, quinio, 2011/05/18}

\enonce{005985}{}
Si $30$ personnes sont présentes à un réveillon et si, à minuit, chaque personne fait $2$ bises à toutes les autres,
combien de bises se sont-elles échangées en tout ? (On appelle bise un contact entre deux joues...)
\finenonce{005985}


\finexercice
\exercice{5986, quinio, 2011/05/18}

\enonce{005986}{}
Un QCM comporte $10$ questions, pour chacune desquelles $4$ réponses sont proposées, une seule est exacte. Combien y-a-t-il de
grilles-réponses possibles? Quelle est la probabilité de répondre au hasard au moins $6$ fois correctement?
\finenonce{005986}


\finexercice
\exercice{5987, quinio, 2011/05/18}

\enonce{005987}{}
 Amédée, Barnabé, Charles tirent sur un oiseau; si les
probabilités de succès sont pour Amédée : $70$\%, Barnabé : $50$\%, Charles : $90$\%, quelle est la probabilité que l'oiseau soit
touché?
\finenonce{005987}


\finexercice
\exercice{5988, quinio, 2011/05/18}

\enonce{005988}{}
Lors d'une loterie de Noël, $300$ billets sont
vendus aux enfants de l'école ; $4$ billets sont gagnants.
J'achète $10$ billets, quelle est la probabilité pour que je gagne au moins un lot?
\finenonce{005988}


\finexercice
\exercice{5989, quinio, 2011/05/18}

\enonce{005989}{}
La probabilité pour une population d'être atteinte
d'une maladie $A$ est $p$ donné; dans cette même population, un individu
peut être atteint par une maladie $B$ avec une probabilité $q$ donnée aussi; 
on suppose que les maladies sont indépendantes : quelle est la
probabilité d'être atteint par l'une et l'autre de ces maladies?
Quelle est la probabilité d'être atteint par l'une ou l'autre de ces
maladies?
\finenonce{005989}


\finexercice
\exercice{5990, quinio, 2011/05/18}
\enonce{005990}{}
Dans un jeu de $52$ cartes, on prend une carte au hasard : les événements <<tirer un roi>> et 
<<tirer un pique>> sont-ils indépendants? quelle est la probabilité de <<tirer un roi ou
un pique>> ? 
\finenonce{005990}


\finexercice
\exercice{5991, quinio, 2011/05/18}

\enonce{005991}{}
La famille Potter comporte $2$ enfants; les événements $A$ : 
<<il y a deux enfants de sexes différents chez les Potter>> 
et $B$ : <<la famille Potter a au
plus une fille>> sont-ils indépendants? 
Même question si la famille Potter comporte $3$ enfants. Généraliser.
\finenonce{005991}


\finexercice

\finfiche 

 \finenonces 



 \finindications 

\noindication
\noindication
\noindication
\noindication
\noindication
\noindication
\noindication
\noindication
\noindication
\noindication


\newpage

\correction{005983}
Classements possibles : sans ex-aequo, il y en a 20!.

Avec exactement $2$ ex-aequo, il y en a :
\begin{enumerate}
\item Choix des deux ex-aequo : $\binom{20}{2}=$ $190$ choix;
\item Place des ex-aequo : il y a $19$ possibilités;
\item Classements des $18$ autres personnes, une fois les ex-aequo placés : il y a $18!$ choix.
\end{enumerate}
Il y a au total : $19\binom{20}{2}(18!)$ choix possibles.
\fincorrection
\correction{005984}
- Une tenue est un triplet $(P, T, C)$ : il y a $5\times 6\times 8=240$ tenues
différentes;

- <<Il est tout en noir>> : de combien de façons différentes ? 
Réponse : de $2\times 4\times 5=40$ façons.

La probabilité de l'événement <<Il est tout en
noir>> est donc : $\frac{40}{240}=\frac{1}{6}$.

- <<Une seule pièce est noire sur les trois >> : notons les événements :
$N_{1}$ la première pièce (pantalon) est noire, $N_{2}$ la deuxième pièce (tee-shirt)
 est noire, $N_{3}$ la troisième pièce
(chaussette) est noire: l'événement est représenté par :
$(N_{1}\cap \overline{N_{2}}\cap \overline{N_{3}})\cup (\overline{N_{1}}\cap
N_{2}\cap \overline{N_{3}})\cup (\overline{N_{1}}\cap \overline{N_{2}}\cap
N_{3})$.
Ces trois événements sont disjoints, leurs probabilités
s'ajoutent. La probabilité de l'événement <<une seule pièce est noire sur les trois>> est donc : 
$0.325$.
\fincorrection
\correction{005985}
Il y a $\binom{30}{2}$ façons de choisir $2$ personnes parmi $30$ et donc $2\cdot\binom{30}{2} = 870$ bises.
\fincorrection
\correction{005986}
\begin{enumerate}
\item Une grille-réponses est une suite ordonnée de $10$ réponses,
il y a $4$ choix possibles pour chacune. Il y a donc $4^{10}$ grilles-réponses possibles.
\item L'événement $E$ <<répondre au hasard au moins $6$ fois correctement>> est 
réalisé si le candidat répond bien à $6$ ou $7$ ou $8$ ou $9$ ou $10$ questions. Notons $A_{n}$ l'événement : <<répondre au hasard
exactement $n$ fois correctement>>. Alors, $A_{n}$ est réalisé si $n$ réponses sont correctes et $10-n$
sont incorrectes : $3$ choix sont possibles pour chacune de ces dernières.
Comme il y a $\binom{10}{n}$ choix de $n$ objets parmi $10$, et donc il y a :
$\binom{10}{n} \times 3^{10-n}$ façons de réaliser $A_{n}$ et :
\begin{equation*}
P(A_n)=\frac{\binom{10}{n}\cdot 3^{10-n}}{4^{10}}
\end{equation*}
pour $n= 6, 7, 8, 9, 10$.
$P(E)=\sum_{n=6}^{10}\frac{\binom{10}{n}\cdot 3^{10-n}}{4^{10}} \simeq 1.9728\times 10^{-2}$, soit environ $2$\%.
\end{enumerate}
\fincorrection
\correction{005987}

Considérons plutôt l'événement complémentaire :
l'oiseau n'est pas touché s'il n'est touché ni par Amédée,
ni par Barnabé, ni par Charles.
Cet événement a pour probabilité : $(1-0.7)\cdot (1-0.5)\cdot (1-0.9)=0.015$.
La probabilité que l'oiseau soit touché est donc : $1-0.015=0.985$.
\fincorrection
\correction{005988}

L'univers des possibles est ici l'ensemble des combinaisons de $10$ billets
parmi les $300$ ; il y en a $\binom{300}{10}$.
Je ne gagne rien si les $10$ billets achetés se trouvent parmi les $296$
billets perdants, ceci avec la probabilité : 
\begin{equation*}
\frac{\binom{296}{10}}{\binom{300}{10}}.
\end{equation*}

La probabilité cherchée est celle de l'événement complémentaire : 
\begin{equation*}
1-\frac{\binom{296}{10}}{\binom{300}{10}}\simeq 0.127.
\end{equation*}

La probabilité est environ $12.7$\% de gagner au moins un lot.
\fincorrection
\correction{005989}
$P(A\cap B)=pq$ car les maladies sont indépendantes.
$P(A\cup B)=P(A)+P(B)-P(A\cap B)=p+q-pq$
\fincorrection
\correction{005990}
Soit $A$ : l'événement <<tirer un roi>>
et $B$ : <<tirer un pique>>.

$P(A\cap B)=\frac{1}{52};P(A)=\frac{4}{52}=\frac{1}{13};P(B)=\frac{13}{52}=\frac{1}{4}.$

Donc $P(A\cap B)=P(A)P(B)$ et donc les événements $A$ et $B$ sont indépendants.

$P(A\cup B)=P(A)+P(B)-P(A\cap B)=\frac{1}{13}+\frac{1}{4}-\frac{1}{52}=\frac{4}{13}$.
\fincorrection
\correction{005991}
Notons, pour le cas où la famille Potter comporte $2$
enfants, l'univers des possibles pour les enfants :
$\Omega =\{(G,G),(G,F),(F,G),(F,F)\}$, représente les cas possibles, 
équiprobables, d'avoir garçon-garçon, garçon-fille etc... :
Alors $P(A)=\frac{2}{4},$ $P(B)=\frac{3}{4}, P(A\cap B) = \frac{2}{4}$.
On en conclut que : $P(A\cap B)\neq P(A)P(B)$ et donc que les événements $A$ et $B$ ne sont pas indépendants.

Si maintenant la famille Potter comporte $3$ enfants :
Alors $\Omega' = \{ (a,b,c) \mid a \in \{ G,F \}, b\in \{G,F\}, c\in \{G,F\}\}$ 
représente les $2^{3}=8$ cas possibles, équiprobables.
Cette fois, $P(A)=1-P(\{(G,G,G),(F,F,F)\})=\frac{6}{8}$ ;
$P(B)=\frac{4}{8}, P(A\cap B)=P\{(F,G,G),(G,F,G),\{(G,G,F)\}=\frac{3}{8}$.
On a $P(A)P(B)=\frac{3}{8}=P(A\cap B),$ et les événements $A$ et $B$
sont indépendants

Avec $n$ enfants, on peut généraliser sans difficulté : 
$P(A)=1-\frac{2}{2^{n}},$ $P(B)=\frac{1+n}{2^{n}}$
$P(A\cap B)=\frac{n}{2^{n}}$
Un petit calcul montre que
$P(A)P(B)=P(A\cap B)$ si et seulement si $n=3$.
\fincorrection


\end{document}

