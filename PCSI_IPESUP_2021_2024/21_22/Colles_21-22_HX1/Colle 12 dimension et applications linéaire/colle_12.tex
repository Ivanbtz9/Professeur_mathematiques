\documentclass[a4paper,10pt]{article}



\usepackage{fancyhdr} % pour personnaliser les en-têtes
\usepackage[utf8]{inputenc}
\usepackage[T1]{fontenc}
\usepackage{lastpage}
\usepackage[frenchb]{babel}
\usepackage{amsfonts,amssymb}
\usepackage{amsmath,amsthm,mathtools}
\usepackage{paralist}
\usepackage{xspace}
\usepackage{xcolor,multicol}
\usepackage{variations}
\usepackage{xypic}
\usepackage{eurosym}
\usepackage{graphicx}
\usepackage[np]{numprint}
\usepackage{hyperref} 
\usepackage{listings} % pour écrire des codes avec coloration syntaxique  

\usepackage{tikz}
\usetikzlibrary{calc, arrows, plotmarks,decorations.pathreplacing}
\usepackage{colortbl}
\usepackage{multirow}
\usepackage[top=2cm,bottom=1.5cm,right=2cm,left=1.5cm]{geometry}

\newtheorem{thm}{Théorème}
\newtheorem*{pro}{Propriété}
\newtheorem*{exemple}{Exemple}

\theoremstyle{definition}
\newtheorem*{remarque}{Remarque}
\theoremstyle{definition}
\newtheorem{exo}{Exercice}
\newtheorem{definition}{Définition}


\newcommand{\vtab}{\rule[-0.4em]{0pt}{1.2em}}
\newcommand{\V}{\overrightarrow}
\renewcommand{\thesection}{\Roman{section} }
\renewcommand{\thesubsection}{\arabic{subsection} }
\renewcommand{\thesubsubsection}{\alph{subsubsection} }
\newcommand*{\transp}[2][-3mu]{\ensuremath{\mskip1mu\prescript{\smash{\mathrm t\mkern#1}}{}{\mathstrut#2}}}%

\newcommand{\C}{\mathbb{C}}
\newcommand{\R}{\mathbb{R}}
\newcommand{\Q}{\mathbb{Q}}
\newcommand{\Z}{\mathbb{Z}}
\newcommand{\N}{\mathbb{N}}



\definecolor{vert}{RGB}{11,160,78}
\definecolor{rouge}{RGB}{255,120,120}
% Set the beginning of a LaTeX document
\pagestyle{fancy}
\lhead{Optimal Sup Spé, groupe IPESUP}\chead{Année~2021-2022}\rhead{Niveau: Première année de PCSI }\lfoot{M. Botcazou}\cfoot{\thepage}\rfoot{mail: ibotca52@gmail.com }\renewcommand{\headrulewidth}{0.4pt}\renewcommand{\footrulewidth}{0.4pt}

\begin{document}
	
	
	\begin{center}
		\Large \sc colle 12 = dimensions des espaces vectoriels 
	\end{center}




\section*{Dimensions des espaces vectoriels:}
\begin{minipage}{1\linewidth}
	\begin{minipage}[t]{0.48\linewidth}
		\raggedright
		
		
		
		\begin{exo}\quad\\
		Soit $E = \C_{n-1}[X]$ 
			et soit $\alpha_1,...,\alpha_n$ des nombres complexes deux à deux distincts. On pose, pour $k= 1,...n$, 
			$$L_k = \dfrac{\prod\limits^{n}\mathop{}_{\mkern-5mu \substack{i=1\\i\neq k}}(X-\alpha_i)}{\prod\limits^{n}\mathop{}_{\mkern-5mu \substack{i=1\\i\neq k}}(\alpha_k-\alpha_i)}$$
			Démontrer que $(L_k)_{k=1,...,n}$ est une base de E. Déterminer les coordonnées d'un élément $P\in E$ dans cette base.
			
			\centering
			\rule{1\linewidth}{0.6pt}
		\end{exo}
	
		\begin{exo}\quad\\
			Soit $E=\R_3[X]$ l'espace vectoriel des polynômes à coefficients réels de degré inférieur ou égal à 3. On définit $u$ l'application de $E$ dans lui-même par $$u(P)=P+(1-X)P'$$
			\begin{enumerate}
				\item Montrer que $u$ est un endomorphisme de $E$. 
				\item Déterminer une base de $Im(u)$.
				\item Déterminer une base de $ker(u)$.
				\item Montrer que $ker(u)$
				et $Im(u)$ sont deux sous-espaces vectoriels supplémentaires de $E$. 
				
				
			\end{enumerate}
			
		\centering
		\rule{1\linewidth}{0.6pt}
	\end{exo}

	\begin{exo}\quad\\
	Soient $\alpha\in\R$ et $F=\{P\in \R_n[X] ; P(\alpha)=0\}$. \\[0.25cm] Démontrer que $B=\{(X-\alpha)X^k; 0\leq k \leq n-1\}$ est une base de $F$. Quelle est la dimension de $F$? Donner les coordonnées de $(X-\alpha)^n$ dans cette base.
	
	\centering
	\rule{1\linewidth}{0.6pt}
	\end{exo}

		\begin{exo}\quad\\
			Démontrer que les familles suivantes sont libres\\ dans $\mathcal{F}(\R,\R)$:\\[0.25cm]
			\begin{enumerate}
				\item $\left(x\longmapsto e^{ax}\right)_{a\in\R}$;\\[0.25cm]
				\item $\left(x\longmapsto |x-a|\right)_{a\in\R}$;\\[0.25cm]
				\item $\left(x\longmapsto \cos(ax)\right)_{a\in\R}$;\\[0.25cm]
				\item $\left(x\longmapsto (\sin x)^n\right)_{n\in\N}$;\\[0.25cm]
			\end{enumerate} 
		%Soit $\alpha \in \R$ et $f_\alpha : \R \to \R$, $x\mapsto e^{\alpha x}$.
		%Montrer que la famille $(f_\alpha)_{\alpha \in \R}$  est libre.
		
		\centering
		\rule{1\linewidth}{0.6pt}
	\end{exo}
	
		
	
	\end{minipage}	
	\hfill\vrule\hfill
	\begin{minipage}[t]{0.48\linewidth}
		\raggedright
		
	\begin{exo}\quad\\
		Soit $E$ l'ensemble des fonctions continues sur $[-1,1]$ qui sont affines sur $[-1,0]$ et sur $[0,1]$. Démontrer que $E$ est un espace vectoriel et en donner une base.
		
		\centering
		\rule{1\linewidth}{0.6pt}
	\end{exo}
		
	\begin{exo}\quad\\
		Soit $E$ un espace vectoriel et $f\in\mathcal{L}(E)$.
		\begin{enumerate}
		\item Montrer que les conditions suivantes sont équivalentes:
		\begin{itemize}[$\square$]
		\item $ker(f) = ker(f^2)$.
		\item $ Im(f) \cap ker(f) = \{0\}$.
		\end{itemize}
		\item On suppose maintenant que $E$ est de dimension finie. Montrer que les conditions suivantes sont équivalentes:
		\begin{itemize}[$\square$]
			\item $ker(f) = ker(f^2)$.
			\item $ker(f) \oplus Im(f) = E$.
			\item $Im(f) = Im(f^2)$.
		\end{itemize}
		\end{enumerate}
		
		\centering
		\rule{1\linewidth}{0.6pt}
	\end{exo}

\begin{exo}\quad\\
	Démontrer que l'ensemble des suites arithmétiques complexes est un espace vectoriel. Quelle est sa dimension ? 
	
	\centering
	\rule{1\linewidth}{0.6pt}
\end{exo}

\begin{exo}\quad\\
	Soit $n\geq 1$, $E=\R_n[X]$ et $\phi\in\mathcal{L}(E)$ défini par $\phi(P) = P(X+1)-P(X)$. Déterminer le noyau et l'image de $\phi$. 

	
	\centering
	\rule{1\linewidth}{0.6pt}
\end{exo}

		\begin{exo}\quad\\
	Soit $E$ un espace vectoriel dans lequel tout sous-espace vectoriel admet un supplémentaire. Soit $F$ un sous-espace vectoriel propre de $E$ (c'est-à-dire que $F\neq \{0\}$ et que $E\neq F$). Démontrer que $F$ admet au moins deux supplémentaires distincts.
	
	\centering
	\rule{1\linewidth}{0.6pt}
\end{exo}



	
	\end{minipage}
\end{minipage}
\end{document}