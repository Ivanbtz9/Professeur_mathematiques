\documentclass[a4paper,11pt]{article}

\usepackage{inputenc}
\usepackage[T1]{fontenc}
\usepackage[frenchb]{babel}
\usepackage{fancyhdr,fancybox} % pour personnaliser les en-têtes
\usepackage{lastpage,setspace}
\usepackage{amsfonts,amssymb,amsmath,amsthm,mathrsfs}
\usepackage{relsize,exscale,bbold}
\usepackage{paralist}
\usepackage{xspace,multicol,diagbox,array}
\usepackage{xcolor}
\usepackage{variations}
\usepackage{xypic}
\usepackage{eurosym,stmaryrd}
\usepackage{graphicx}
\usepackage[np]{numprint}
\usepackage{hyperref} 
\usepackage{tikz}
\usepackage{colortbl}
\usepackage{multirow}
\usepackage{MnSymbol,wasysym}
\usepackage[top=1.5cm,bottom=1.5cm,right=1.2cm,left=1.5cm]{geometry}
\usetikzlibrary{calc, arrows, plotmarks, babel,decorations.pathreplacing}
\setstretch{1.25}
%\usepackage{lipsum} %\usepackage{enumitem} %\setlist[enumerate]{itemsep=1mm} bug avec enumerate



\newtheorem{thm}{Théorème}
\newtheorem{rmq}{Remarque}
\newtheorem{prop}{Propriété}
\newtheorem{cor}{Corollaire}
\newtheorem{lem}{Lemme}
\newtheorem{prop-def}{Propriété-définition}

\theoremstyle{definition}

\newtheorem{defi}{Définition}
\newtheorem{ex}{Exemple}
\newtheorem*{rap}{Rappel}
\newtheorem{cex}{Contre-exemple}
\newtheorem{exo}{Exercice} % \large {\fontfamily{ptm}\selectfont EXERCICE}
\newtheorem{nota}{Notation}
\newtheorem{ax}{Axiome}
\newtheorem{appl}{Application}
\newtheorem{csq}{Conséquence}
\def\di{\displaystyle}



\renewcommand{\thesection}{\Roman{section}}\renewcommand{\thesubsection}{\arabic{subsection} }\renewcommand{\thesubsubsection}{\alph{subsubsection} }


\newcommand{\bas}{~\backslash}\newcommand{\ba}{\backslash}
\newcommand{\C}{\mathbb{C}}\newcommand{\R}{\mathbb{R}}\newcommand{\K}{\mathbb{K}}\newcommand{\Q}{\mathbb{Q}}\newcommand{\Z}{\mathbb{Z}}\newcommand{\N}{\mathbb{N}}\newcommand{\V}{\overrightarrow}\newcommand{\Cs}{\mathscr{C}}\newcommand{\Ps}{\mathscr{P}}\newcommand{\Rs}{\mathscr{R}}\newcommand{\Gs}{\mathscr{G}}\newcommand{\Ds}{\mathscr{D}}\newcommand{\happy}{\huge\smiley}\newcommand{\sad}{\huge\frownie}\newcommand{\danger}{\begin{tikzpicture}[x=1.5pt,y=1.5pt,rotate=-14.2]
	\definecolor{myred}{rgb}{1,0.215686,0}
	\draw[line width=0.1pt,fill=myred] (13.074200,4.937500)--(5.085940,14.085900)..controls (5.085940,14.085900) and (4.070310,15.429700)..(3.636720,13.773400)
	..controls (3.203130,12.113300) and (0.917969,2.382810)..(0.917969,2.382810)
	..controls (0.917969,2.382810) and (0.621094,0.992188)..(2.097660,1.359380)
	..controls (3.574220,1.726560) and (12.468800,3.984380)..(12.468800,3.984380)
	..controls (12.468800,3.984380) and (13.437500,4.132810)..(13.074200,4.937500)
	--cycle;
	\draw[line width=0.1pt,fill=white] (11.078100,5.511720)--(5.406250,11.875000)..controls (5.406250,11.875000) and (4.683590,12.812500)..(4.367190,11.648400)
	..controls (4.050780,10.488300) and (2.375000,3.675780)..(2.375000,3.675780)
	..controls (2.375000,3.675780) and (2.156250,2.703130)..(3.214840,2.964840)
	..controls (4.273440,3.230470) and (10.640600,4.847660)..(10.640600,4.847660)
	..controls (10.640600,4.847660) and (11.332000,4.953130)..(11.078100,5.511720)
	--cycle;
	\fill (6.144520,8.839900)..controls (6.460940,7.558590) and (6.464840,6.457090)..(6.152340,6.378910)
	..controls (5.835930,6.300840) and (5.320300,7.277400)..(5.003900,8.554750)
	..controls (4.683590,9.835940) and (4.679690,10.941400)..(4.996090,11.019600)
	..controls (5.312490,11.097700) and (5.824210,10.121100)..(6.144520,8.839900)
	--cycle;
	\fill (7.292960,5.261780)..controls (7.382800,4.898500) and (7.128900,4.523500)..(6.730460,4.421880)
	..controls (6.328120,4.324220) and (5.929680,4.535220)..(5.835930,4.898500)
	..controls (5.746080,5.261780) and (5.999990,5.640630)..(6.402340,5.738340)
	..controls (6.804690,5.839840) and (7.203110,5.625060)..(7.292960,5.261780)
	--cycle;
	\end{tikzpicture}}\newcommand{\alors}{\Large\Rightarrow}\newcommand{\equi}{\Leftrightarrow}
\newcommand{\fonction}[5]{\begin{array}{l|rcl}
		#1: & #2 & \longrightarrow & #3 \\
		& #4 & \longmapsto & #5 \end{array}}
\newcommand{\Pb}{\mathbf{P}}


\definecolor{vert}{RGB}{11,160,78}
\definecolor{rouge}{RGB}{255,120,120}
\definecolor{bleu}{RGB}{15,5,107}


\pagestyle{fancy}
\lhead{Optimal Sup Spé, groupe IPESUP}\chead{Année~2022-2023}\rhead{Niveau: Première année de PCSI }\lfoot{M. Botcazou}\cfoot{\thepage}\rfoot{mail: i.botcazou@gmx.fr }\renewcommand{\headrulewidth}{0.4pt}\renewcommand{\footrulewidth}{0.4pt}

\begin{document}
	
	
	\begin{center}
		\Large \sc colle 21 = Probabilité et variables aléatoires 
	\end{center}
\raggedright


\section*{Connaître son cours:}
\begin{enumerate}
\item Soit $(\Omega, \mathcal P(\Omega),\Pb)$ un espace probabilisé fini, $A$ et $B$ deux éléments de $\mathcal P(\Omega)$ de probabilité non nulle. Donner et démontrer la formule de Bayes.
\item Soit $(\Omega, \mathcal P(\Omega),\Pb)$ un espace probabilisé fini et $(A_k )_{k\leq n}$ une famille d’événements vérifiant $\displaystyle\Pb\left(\underset{k\leq n}{\bigcap} A_k \right) >   0$. Donner et démontrer la formule des probabilités composées. 
\item Soit $(\Omega, \mathcal P(\Omega),\Pb)$ un espace probabilisé fini et $(A_1,\dots,A_n)_{k\leq n}$ une partition de $\Omega$ d’événements de probabilités non nulles. Soit $B \subset \Omega$, donner et démontrer la formule des probabilités totales.
\hfill \\[-1.5cm]
\end{enumerate}


\section*{Exercices:} 	

\begin{exo}\textbf{(*)}\quad\\[0.15cm]
	On prend au hasard, en même temps, trois ampoules dans un lot de 15 dont
	5 sont défectueuses. Calculer la probabilité des événements:
	
	$A$ : au moins une ampoule est défectueuse;
	
	$B$ : les 3 ampoules sont défectueuses;
	
	$C$ : exactement une ampoule est défectueuse.
	
	\centering
	\rule{1\linewidth}{0.6pt}
\end{exo}


\begin{exo}\textbf{(**)}\quad\\[0.15cm]%
	Une urne contient des boules blanches et noires en proportion $p$ et $q$ (avec
	$p + q = 1$). On opère à des tirages successifs avec remise.
	\begin{enumerate}
		\item Quelle est la probabilité que la première boule blanche tirée apparaisse lors
		du n-ième tirage ?
		\item Quelle est la probabilité que la k-ième boule blanche tirée apparaisse lors du
		n-ième tirage ?
	\end{enumerate}


	\centering
\rule{1\linewidth}{0.6pt}
\end{exo}

\begin{exo}\textbf{(**)}\quad\\[0.15cm]%
	Dans une population, une personne sur 10 000 souffre d'une pathologie. Un
	laboratoire pharmaceutique met sur le marché un test sanguin. Celui-ci est positif
	chez 99 \% des malades mais aussi faussement positif chez 0,1 \% des personnes non
	atteintes. Un individu passe ce test et obtient un résultat positif.
	Quelle est sa probabilité d'être malade ? Qu'en conclure ?
	
	
	\centering
	\rule{1\linewidth}{0.6pt}
\end{exo}


\begin{exo}\textbf{(*)}\quad\\[0.15cm]%
	L'oral d'un concours comporte au total 100 sujets; les candidats
	tirent au sort trois sujets et choisissent alors le sujet traité parmi
	ces trois sujets. Un candidat se présente en ayant révisé 60
	sujets sur les 100.
	\begin{enumerate}
		\item Quelle est la probabilité pour que le candidat ait révisé:
		\begin{enumerate}
			\item les trois sujets tirés;
			\item exactement deux sujets sur les trois sujets;
			\item aucun des trois sujets.
		\end{enumerate}
		\item Définir une variable aléatoire associée à ce problème 
		et donner sa loi de probabilité, son espérance.
	\end{enumerate}
	
%	\centering
%	\rule{1\linewidth}{0.6pt}
\end{exo}


\begin{exo}\textbf{(***)}\quad\\[0.25cm]%
	Une urne contient $n \in \N^*$ boules numérotées de $1$ à $n $. On tire avec remise des
	boules dans cette urne jusqu'à ce qu'une boule ait été tirée deux fois. On note $T$
	la variable aléatoire à valeurs dans $\llbracket2 ; n + 1\rrbracket$ précisant le nombre de tirages alors	effectués.
	\begin{enumerate}
		\item Proposer un espace probabilisé $(\Omega, \mathcal P(\Omega),\Pb)$ modélisant cette expérience.
		\item Calculer $\Pb(T = 2) $.
		\item Soit $k \in \llbracket1 ; n + 1\rrbracket$. Exprimer $\Pb(T > k \ | \ T > k - 1)$.
		\item Donner un expression de $\Pb(T = k)$ pour tout $k \in \llbracket1 ; n + 1\rrbracket$
	\end{enumerate}

	
	\centering
	\rule{1\linewidth}{0.6pt}
\end{exo}


\begin{exo}\textbf{(**)}\quad\\[0.25cm]%
	Un industriel doit vérifier l'état de marche de ses machines et en
	remplacer certaines le cas échéant. D'après des statistiques précédentes, 
	il évalue à 30\% la probabilité pour une
	machine de tomber en panne en 5 ans; parmi ces dernières, 
	la probabilité de devenir hors d'usage suite à une panne plus grave est 
	évaluée à 75\%; cette probabilité est de 40\% pour une machine
	n'ayant jamais eu de panne.
	\begin{enumerate}
		\item Quelle est la probabilité pour une machine donnée de plus de cinq
		ans d'être hors d'usage ?
		
		\item Quelle est la probabilité pour une machine hors d'usage de n'avoir
		jamais eu de panne auparavant ?
		
		\item Soit $X$ la variable aléatoire <<nombre de machines
		qui tombent en panne au bout de 5 ans, parmi 10 machines choisies au
		hasard>>. Quelle est la loi de probabilité de $X$, (on
		donnera le type de loi et les formules de calcul), son espérance, sa
		variance et son écart-type ?
		
		\item Calculer $P[X=5]$.
	\end{enumerate}
	
	\centering
	\rule{1\linewidth}{0.6pt}
\end{exo}



\begin{exo}\textbf{(****)}\quad\textit{(Urne de \sc Polya)}\\[0.2cm]
	
	On considère une urne contenant $a$ boules colorées et $b$ boules blanches. Après chaque tirage, la boule extraite est remise dans l'urne avec $c$ boules de la même couleur.
	\begin{enumerate}
		\item Pour $a=2, b=2, c=5$, faire un schéma pour le premier tirage
		\item Déterminer la probabilité que la n-ième boule tirée soit blanche.
	\end{enumerate}
	
	
	
	\centering\rule{1\linewidth}{0.6pt}\end{exo}









\end{document}
