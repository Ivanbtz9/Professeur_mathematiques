\documentclass[a4paper,11pt]{article}

\usepackage{inputenc}
\usepackage[T1]{fontenc}
\usepackage[french]{babel}
\usepackage{fancyhdr,fancybox} % pour personnaliser les en-têtes
\usepackage{lastpage,setspace}
\usepackage{amsfonts,amssymb,amsmath,amsthm,mathrsfs}
\usepackage{mathdots}
\usepackage{relsize,exscale,bbold}
\usepackage{paralist}
\usepackage{xspace,multicol,diagbox,array}
\usepackage{xcolor}
\usepackage{variations}
\usepackage{xypic}
\usepackage{eurosym,stmaryrd}
\usepackage{graphicx}
\usepackage[np]{numprint}
\usepackage{hyperref} 
\usepackage{tikz}
\usepackage{colortbl}
\usepackage{multirow}
\usepackage{MnSymbol,wasysym}
\usepackage[top=1.5cm,bottom=1.5cm,right=1.2cm,left=1.5cm]{geometry}
\usetikzlibrary{calc, arrows, plotmarks, babel,decorations.pathreplacing}
\setstretch{1.25}
%\usepackage{lipsum} %\usepackage{enumitem} %\setlist[enumerate]{itemsep=1mm} bug avec enumerate



\newtheorem{thm}{Théorème}
\newtheorem{rmq}{Remarque}
\newtheorem{prop}{Propriété}
\newtheorem{cor}{Corollaire}
\newtheorem{lem}{Lemme}
\newtheorem{prop-def}{Propriété-définition}

\theoremstyle{definition}

\newtheorem{defi}{Définition}
\newtheorem{ex}{Exemple}
\newtheorem*{rap}{Rappel}
\newtheorem{cex}{Contre-exemple}
\newtheorem{exo}{Exercice} % \large {\fontfamily{ptm}\selectfont EXERCICE}
\newtheorem{nota}{Notation}
\newtheorem{ax}{Axiome}
\newtheorem{appl}{Application}
\newtheorem{csq}{Conséquence}
\def\di{\displaystyle}



\renewcommand{\thesection}{\Roman{section}}\renewcommand{\thesubsection}{\arabic{subsection} }\renewcommand{\thesubsubsection}{\alph{subsubsection} }


\newcommand{\bas}{~\backslash}\newcommand{\ba}{\backslash}
\newcommand{\C}{\mathbb{C}}\newcommand{\K}{\mathbb{K}}\newcommand{\R}{\mathbb{R}}\newcommand{\Q}{\mathbb{Q}}\newcommand{\Z}{\mathbb{Z}}\newcommand{\N}{\mathbb{N}}\newcommand{\V}{\overrightarrow}\newcommand{\Cs}{\mathscr{C}}\newcommand{\Ps}{\mathscr{P}}\newcommand{\Rs}{\mathscr{R}}\newcommand{\Gs}{\mathscr{G}}\newcommand{\Ds}{\mathscr{D}}\newcommand{\happy}{\huge\smiley}\newcommand{\sad}{\huge\frownie}\newcommand{\danger}{\begin{tikzpicture}[x=1.5pt,y=1.5pt,rotate=-14.2]
	\definecolor{myred}{rgb}{1,0.215686,0}
	\draw[line width=0.1pt,fill=myred] (13.074200,4.937500)--(5.085940,14.085900)..controls (5.085940,14.085900) and (4.070310,15.429700)..(3.636720,13.773400)
	..controls (3.203130,12.113300) and (0.917969,2.382810)..(0.917969,2.382810)
	..controls (0.917969,2.382810) and (0.621094,0.992188)..(2.097660,1.359380)
	..controls (3.574220,1.726560) and (12.468800,3.984380)..(12.468800,3.984380)
	..controls (12.468800,3.984380) and (13.437500,4.132810)..(13.074200,4.937500)
	--cycle;
	\draw[line width=0.1pt,fill=white] (11.078100,5.511720)--(5.406250,11.875000)..controls (5.406250,11.875000) and (4.683590,12.812500)..(4.367190,11.648400)
	..controls (4.050780,10.488300) and (2.375000,3.675780)..(2.375000,3.675780)
	..controls (2.375000,3.675780) and (2.156250,2.703130)..(3.214840,2.964840)
	..controls (4.273440,3.230470) and (10.640600,4.847660)..(10.640600,4.847660)
	..controls (10.640600,4.847660) and (11.332000,4.953130)..(11.078100,5.511720)
	--cycle;
	\fill (6.144520,8.839900)..controls (6.460940,7.558590) and (6.464840,6.457090)..(6.152340,6.378910)
	..controls (5.835930,6.300840) and (5.320300,7.277400)..(5.003900,8.554750)
	..controls (4.683590,9.835940) and (4.679690,10.941400)..(4.996090,11.019600)
	..controls (5.312490,11.097700) and (5.824210,10.121100)..(6.144520,8.839900)
	--cycle;
	\fill (7.292960,5.261780)..controls (7.382800,4.898500) and (7.128900,4.523500)..(6.730460,4.421880)
	..controls (6.328120,4.324220) and (5.929680,4.535220)..(5.835930,4.898500)
	..controls (5.746080,5.261780) and (5.999990,5.640630)..(6.402340,5.738340)
	..controls (6.804690,5.839840) and (7.203110,5.625060)..(7.292960,5.261780)
	--cycle;
	\end{tikzpicture}}\newcommand{\alors}{\Large\Rightarrow}\newcommand{\equi}{\Leftrightarrow}
\newcommand{\fonction}[5]{\begin{array}{l|rcl}
		#1: & #2 & \longrightarrow & #3 \\
		& #4 & \longmapsto & #5 \end{array}}


\definecolor{vert}{RGB}{11,160,78}
\definecolor{rouge}{RGB}{255,120,120}
\definecolor{bleu}{RGB}{15,5,107}



\pagestyle{fancy}
\lhead{Groupe IPESUP}\chead{}\rhead{Année~2022-2023}\lfoot{M. Botcazou \& M.Dupré}\cfoot{\thepage/3}\rfoot{PCSI }\renewcommand{\headrulewidth}{0.4pt}\renewcommand{\footrulewidth}{0.4pt}


\begin{document}
 %%%%BIBMATH%%%%
 
 %(1) https://www.bibmath.net/ressources/index.php?action=affiche&quoi=mathspe/feuillesexo/calculdiff&type=fexo

\noindent\shadowbox{
	\begin{minipage}{1\linewidth}
		\centering
		\huge{\textbf{ TD 23 : Fonctions à plusieurs variables }}
	\end{minipage}}

\smallskip
\section*{Connaître son cours:}
\begin{itemize}[$\bullet$]
	\item Dessinez le domaine de définition de $f := (x, y ) \mapsto x \ln(x + y ) - y\sqrt{y - x}.$.
	\item Soit $f:\R^2\to\R$, donner la signification de : $f$ admet des dérivées partielles en un point $(x_0,y_0)\in\R^2$.
	
	Donner les dérivées partielles de la fonction $f := (x, y ) \mapsto xy + y^2 + \cos(xy)$.
	\item Calculez le gradient de $f := (x, y ) \mapsto xe^y -3yx^2$ en $(1, 1)$.
	\item Soit $f$ une fonction à deux variables sur $U$ un ouvert de $\R^2$, admettant des dérivées partielles en tout point de $U$. Rappeler la définition d'un point critique pour la fonction $f$.
	
	Donner les points critiques de la fonction $f := (x, y ) \mapsto x^3 - 3x + y^2$. 
	

\end{itemize}
\raggedright
\smallskip
\section*{Continuité et dérivées partielles :}\hfill\\%[-0.25cm]
\smallskip
   
\begin{minipage}{1\linewidth}\begin{minipage}[t]{0.48\linewidth}\raggedright
	
\begin{exo}\textbf{(*)}\quad\\[0.2cm]
	Étudier les limites en $(0,0)$ des fonctions suivantes :
	\begin{multicols}{2}
		\begin{enumerate}
			\item $f(x, y)= \frac{x^{3}}{y}$
			\item $f(x, y)= \frac{x+2 y}{x^{2}-y^{2}}$
			\item $f(x, y)= \frac{1-\cos (x y)}{x y^{2}}$
			\item $f(x, y)=\frac{\sin x y}{\sqrt{x^{2}+y^{2}}}$
		\end{enumerate}
	\end{multicols}

	
\centering\rule{1\linewidth}{0.6pt}\end{exo}


\begin{exo}\textbf{(*)}\quad\\[0.2cm]
	
	Soit $f: \mathbb{R} \rightarrow \mathbb{R}$ une fonction de classe $\mathcal{C}^{1}$ et $F: \mathbb{R}^{2} \backslash\{(0,0)\} \rightarrow \mathbb{R}$ définie par
	
	$$
	F(x, y)=\frac{f\left(x^{2}+y^{2}\right)-f(0)}{x^{2}+y^{2}}
	$$
	
	Déterminer $\lim\limits_{(x, y) \rightarrow(0,0)} F(x, y)$.

\centering\rule{1\linewidth}{0.6pt}\end{exo}


\begin{exo}\textbf{(*)}\quad\\[0.2cm]
	
	Soit $A$ une partie convexe non vide de $\mathbb{R}^{2}$ et $f: A \rightarrow \mathbb{R}$ une fonction continue.
	
	Soit $a$ et $b$ deux points de $A$ et $y$ un réel tels que \begin{center}
		$f(a) \leq y \leq f(b)$.
	\end{center}
	
	Montrer qu'il existe $x \in A$ tel que $f(x)=y$.
	
\centering\rule{1\linewidth}{0.6pt}\end{exo}



%%%%%%%%%%%%%%%%%%%%%%%%%%%%%%%%%%%%%%%%%%%%%%%%%%%%%%%%%%%%%%%%%%%%%%%%%%%%%%%%%%%%%%%%%%
\end{minipage}\hfill\vrule\hfill\begin{minipage}[t]{0.48\linewidth}\raggedright
%%%%%%%%%%%%%%%%%%%%%%%%%%%%%%%%%%%%%%%%%%%%%%%%%%%%%%%%%%%%%%%%%%%%%%%%%%%%%%%%%%%%%%%%%%

\begin{exo}\textbf{(*)}\quad\\[0.2cm]
	Soit $f: \mathbb{R}^{2} \rightarrow \mathbb{R}$ définie par
	
	$$
	f(x, y)= \begin{cases}\frac{1}{2} x^{2}+y^{2}-1 & \text { si } x^{2}+y^{2}>1 \\ -\frac{1}{2} x^{2} & \text { sinon }\end{cases}
	$$
	
	Montrer que $f$ est continue.
	
	\centering\rule{1\linewidth}{0.6pt}\end{exo}

\begin{exo}\textbf{(**)}\quad\\[0.2cm]%\textit{Contre exemple de Peano} P.464
Pour $(x,y)\in\R^2$ avec $(x,y)\neq (0,0)$ on pose 
$$f(x,y) = \dfrac{xy(x^2-y^2)}{x^2+y^2}$$
\begin{enumerate}
	\item Par quelle valeur peut-on prolonger $f$ par continuité en $(0,0)$ ?

On note encore $f$ cette fonction définie par prolongement.
	\item Calculer les dérivées partielles de $f$ en $(x,y)\neq(0,0)$.
	\item Calculer les dérivées partielles de $f$ en $(0,0)$.
	\item Montrer que la fonction $f$ est de classe $\mathcal{C}^1$ sur $\R^2$.
\end{enumerate}

\centering\rule{1\linewidth}{0.6pt}\end{exo}



\end{minipage}\end{minipage} \newpage
\begin{minipage}{1\linewidth}\begin{minipage}[t]{0.48\linewidth}\raggedright
		
		\begin{exo}\textbf{(*)}\quad\\[0.1cm]
			Justifier l'existence des dérivées partielles des fonctions suivantes et les calculer. En déduire respectivement un développement limité à l'ordre $1$ en $(0;\frac{\pi}{3}) , \ (0,0) \text{ et } (\sqrt{3},\sqrt{2})$. 
			\begin{enumerate}
				\item $f(x,y)=e^x\cos y.$
				\item $f(x,y)=(x^2+y^2)\cos(xy).$
				\item $f(x,y)=\sqrt{1+x^2y^2}.$
			\end{enumerate}
			
			
			\centering\rule{1\linewidth}{0.6pt}\end{exo}
		
		
		
		\begin{exo}\textbf{(**)}\quad\\\textit{(\textbf{Continue sans une des dérivées partielles})}\\[0.1cm]
			On définit $f:\mathbb R^2\backslash\{(0,0)\}\to\mathbb R$ par 
			$$f(x,y)=\frac{x^2}{(x^2+y^2)^{3/4}}.$$
			Justifier que l'on peut prolonger $f$ en une fonction continue sur $\mathbb R^2$. \'Etudier l'existence de dérivées partielles en $(0,0)$ pour ce prolongement.
			
			\centering\rule{1\linewidth}{0.6pt}\end{exo}
		
		\begin{exo}\textbf{(**)}\quad\\\textit{(\textbf{Des dérivées partielles sans la continuité})}\\[0.1cm]
			Pour les fonctions suivantes, démontrer qu'elles admettent une dérivée suivant tout vecteur en $(0,0)$ sans pour autant y être continue.
			\begin{enumerate}
				\item $\displaystyle f(x,y)=\left\{
				\begin{array}{ll}
				y^2\ln |x|&\textrm{ si }x\neq 0\\
				0&\textrm{ sinon.}
				\end{array}
				\right.
				$
				\item $\displaystyle g(x,y)=\left\{
				\begin{array}{ll}
				\frac{x^2y}{x^4+y^2}&\textrm{ si }(x,y)\neq (0,0)\\
				0&\textrm{ sinon.}
				\end{array}
				\right.
				$
			\end{enumerate}
			
			\centering\rule{1\linewidth}{0.6pt}\end{exo}
		
		\begin{exo}\textbf{(**)}\quad\\[0.1cm]
			Les fonctions suivantes, définies sur $\mathbb R^2$, sont-elles de classe $C^1$? \\
			\begin{enumerate}
				\item $\displaystyle f(x,y)=\frac{x^2y^3}{x^2+y^2}\textrm{ si }(x,y)\neq (0,0)$ \\ $\textrm{ et }f(0,0)=0$;\\
				\item $\displaystyle f(x,y)=x^2y^2\ln(x^2+y^2)$ \\ $\textrm{ si }(x,y)\neq (0,0)\textrm{ et }f(0,0)=0$.
				\item $\displaystyle f(x,y)=x\frac{x^2-y^2}{x^2+y^2}\textrm{ si }(x,y)\neq (0,0)$ \\ $\textrm{ et }f(0,0)=0$;
				\item $\displaystyle f(x,y)=e^{-\frac 1{x^2+y^2}}\textrm{ si }(x,y)\neq (0,0)$ \\ $\textrm{ et }f(0,0)=0$.
			\end{enumerate}\quad\\[-0.25cm]
			
			\centering\rule{1\linewidth}{0.6pt}\end{exo}
		
		
		%%%%%%%%%%%%%%%%%%%%%%%%%%%%%%%%%%%%%%%%%%%%%%%%%%%%%%%%%%%%%%%%%%%%%%%%%%%%%%%%%%%%%%%%%%
	\end{minipage}\hfill\vrule\hfill\begin{minipage}[t]{0.48\linewidth}\raggedright
		%%%%%%%%%%%%%%%%%%%%%%%%%%%%%%%%%%%%%%%%%%%%%%%%%%%%%%%%%%%%%%%%%%%%%%%%%%%%%%%%%%%%%%%%%%
				\begin{exo}\textbf{(***)}\quad\\[0.1cm]
			Soit  $\R^2$ munit de la base canonique ainsi que du produit scalaire usuel $\langle\cdot,\cdot\rangle$. Soit $u$ un endomorphisme symétrique de $\R^2$. 
			
			\begin{enumerate}
				\item Montrer que l'application $f: x \mapsto \langle u(x),x\rangle$ est de classe $\mathcal C^1$ sur $\R^2$. Donner un développement limiter à l'ordre $1$ en $a\in \R^2$.
				\item Soit $a\neq (0,0)$ donner $\nabla F(a)$ pour la fonction $F: x\in \R^2\backslash(0,0) \mapsto \dfrac{\langle u(x),x\rangle}{\langle x,x\rangle} $.
				\item Montrer que $\nabla F(a) = 0$ si et seulement si il existe $\lambda\in\R$ telle que $u(a) = \lambda a$. 
			\end{enumerate} 
			
			
			\centering\rule{1\linewidth}{0.6pt}\end{exo}
		\subsubsection*{Formule de la chaîne}
		\begin{exo}\textbf{(*)}\quad\\[0.1cm]
			Soit $f$ une application de classe $C^1$ sur $\R^2$ à valeur da $\R$. Calculer les dérivées (éventuellement partielles) des fonctions suivantes :
			
			\begin{enumerate}
				\item $g(x,y)=f(y,x)$.
				\item $g(x)=f(x,x)$.
				\item $g(x,y)=f(y,f(x,x))$.
				\item $g(x)=f(x,f(x,x))$.
			\end{enumerate}

			
			\centering\rule{1\linewidth}{0.6pt}\end{exo}
		
		
		\begin{exo}\textbf{(**)}\quad\\[0.1cm]
			Soit $f$ une application de classe $C^2$ sur $\R^2$ à valeur da $\R$ et $g:\R^2\to\R$ définie par $g(u,v) = f(uv, u^2 + v^2)$
			
			\begin{enumerate}
				\item Exprimer les dérivées partielles de $g$ en fonction de celles de $f$.
				\item Exprimer les dérivées partielles d'ordre 2 de $g$ en fonction de celles de $f$.
			\end{enumerate} 
			
			\centering\rule{1\linewidth}{0.6pt}\end{exo}
		
		\begin{exo}\textbf{(**)}\quad\\[0.1cm]
			Soit $f:\R^2\to\R$ de classe $C^1$ et $g:\R^2\to\R$ définie par 
			
			$g(r,\theta) = f(r\cos(\theta), r\sin(\theta)).$\\[0.15cm]
			
			 Soit $(x,y)\in\R^2\backslash\{(0,0)\}$ et $(r,\theta)\in \R^*_+\times\R$ tels que $x= r\cos(\theta)$ et $y= r\sin(\theta)$.\\Exprimer\\[0.25cm] 
			 
			  $\dfrac{\partial f}{\partial x}(x,y)$ et $\dfrac{\partial f}{\partial y}(x,y)$  avec  $\dfrac{\partial g}{\partial r}(r,\theta)$ et $\dfrac{\partial g}{\partial \theta}(r,\theta)$. 
			
			\centering\rule{1\linewidth}{0.6pt}\end{exo}
		

		
		
\end{minipage}\end{minipage}\newpage

\section*{Extrema :}\hfill\\%[-0.25cm]


\begin{minipage}{1\linewidth}\begin{minipage}[t]{0.48\linewidth}\raggedright
		
		\begin{exo}\textbf{(*)}\quad\\[0.2cm]
			On pose $$f(x,y)=x^2+y^2+xy+1$$ et $$g(x,y)=x^2+y^2+4xy-2$$
			\begin{enumerate}
				\item Déterminer les points critiques de $f$, de $g$.
				\item En reconnaissant le début du développement d'un carré, étudier les extrema locaux de $f$.
				\item En étudiant les valeurs de $g$ sur deux droites vectorielles bien choisies, étudier les extrema locaux de $g$.
			\end{enumerate}
			
			
			\centering\rule{1\linewidth}{0.6pt}\end{exo}
		
		
		
		
		%%%%%%%%%%%%%%%%%%%%%%%%%%%%%%%%%%%%%%%%%%%%%%%%%%%%%%%%%%%%%%%%%%%%%%%%%%%%%%%%%%%%%%%%%%
	\end{minipage}\hfill\vrule\hfill\begin{minipage}[t]{0.48\linewidth}\raggedright
		%%%%%%%%%%%%%%%%%%%%%%%%%%%%%%%%%%%%%%%%%%%%%%%%%%%%%%%%%%%%%%%%%%%%%%%%%%%%%%%%%%%%%%%%%%
		
	\begin{exo}\textbf{(**)}\quad\\[0.2cm]
	Déterminer les extrema locaux des fonctions $f:\mathbb{R}^2  \to \mathbb{R}$ suivantes :
	\begin{enumerate}
		\item $f(x,y) = x^2  + 2y^2  - 2xy - 2y + 1$
		\item $f(x,y) = x^3  + y^3 $
		\item $f(x,y) = (x - y)^2  + (x + y)^3 $
	\end{enumerate}
	
	\centering\rule{1\linewidth}{0.6pt}\end{exo}

	\begin{exo}\textbf{(**)}\quad\\[0.2cm]
		
		Calculer $$\displaystyle \underset{x,y>0}{\inf}\left(\dfrac{1}{x} + \dfrac{1}{y} +xy\right).$$
		
		\centering\rule{1\linewidth}{0.6pt}\end{exo}
		
		
\end{minipage}\end{minipage} \section*{Équations de dérivées partielles:}\hfill\\%[-0.25cm]


\begin{minipage}{1\linewidth}\begin{minipage}[t]{0.48\linewidth}\raggedright

\begin{exo}\textbf{(*)}\quad\\[0.2cm]
	Déterminer toutes les fonctions $f:\mathbb R^2\to\mathbb R$ de classe $C^1$ solutions des systèmes suivants :
	$$
	\mathbf 1.\left\{
	\begin{array}{rcl}
	\displaystyle \frac{\partial f}{\partial x}&=&xy^2\\[3mm]
	\displaystyle \frac{\partial f}{\partial y}&=&yx^2.
	\end{array}\right.
	\quad\quad
	\mathbf 2.\left\{
	\begin{array}{rcl}
	\displaystyle \frac{\partial f}{\partial x}&=&e^xy\\[3mm]
	\displaystyle \frac{\partial f}{\partial y}&=&e^x+2y.
	\end{array}\right.
	$$
	
	
	
	\centering\rule{1\linewidth}{0.6pt}\end{exo}


\begin{exo}\textbf{(**)}\quad\\[0.2cm]
	Déterminer les fonctions $f:\mathbb R^2\to\mathbb R$ de classe $C^1$ vérifiant l'équation aux dérivées partielles
	
	$$(E):\quad \frac{\partial f}{\partial x}(x,y)+\frac{\partial f}{\partial y}(x,y) =  f(x,y).$$
	
	 On opérera le changement de variables de la forme 
	 
	 $\left\{\begin{array}{l}
	 u = x+y\\v=x-y
	 \end{array}\right.$
	\centering\rule{1\linewidth}{0.6pt}\end{exo}






%%%%%%%%%%%%%%%%%%%%%%%%%%%%%%%%%%%%%%%%%%%%%%%%%%%%%%%%%%%%%%%%%%%%%%%%%%%%%%%%%%%%%%%%%%
\end{minipage}\hfill\vrule\hfill\begin{minipage}[t]{0.48\linewidth}\raggedright
%%%%%%%%%%%%%%%%%%%%%%%%%%%%%%%%%%%%%%%%%%%%%%%%%%%%%%%%%%%%%%%%%%%%%%%%%%%%%%%%%%%%%%%%%%

\begin{exo}\textbf{(**)}\quad\textit{(Équation des ondes)}\\[0.2cm]
	Soit $c\neq 0$. Chercher les solutions de classe $C^2$ de l'équation aux dérivées partielles suivantes 
	$$(E):\quad c^2\frac{\partial^2 f}{\partial x^2}(x,t)=\frac{\partial^2 f}{\partial t^2}(x,t),$$
	à l'aide d'un changement de variables de la forme 
	
	$u=x+at$, $v=x+bt$.
	\centering\rule{1\linewidth}{0.6pt}\end{exo}




\begin{exo}\textbf{(***)}\quad\\[0.2cm]
		Cherche toutes les fonctions $f$ de classe $C^1$ sur $\mathbb R^2$ vérifiant 
		$$\frac{\partial f}{\partial x}(x,y)-3\frac{\partial f}{\partial y}(x,y)=0.$$
		
		\textit{(penser à un changement de variables linéaire à l'aide d'une matrice inversible)}
	
	\centering\rule{1\linewidth}{0.6pt}\end{exo}


\end{minipage}\end{minipage} \newpage

\end{document}