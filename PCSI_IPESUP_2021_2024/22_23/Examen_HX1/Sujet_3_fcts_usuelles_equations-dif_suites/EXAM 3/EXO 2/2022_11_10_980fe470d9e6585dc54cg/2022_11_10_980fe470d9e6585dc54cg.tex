\documentclass[10pt]{article}
\usepackage[utf8]{inputenc}
\usepackage[T1]{fontenc}
\usepackage{amsmath}
\usepackage{amsfonts}
\usepackage{amssymb}
\usepackage[version=4]{mhchem}
\usepackage{stmaryrd}

\title{EXERCICE 2. }

\author{}
\date{}


\begin{document}
\maketitle
\section{Partie I}
\begin{enumerate}
  \item Soit $h$ la fonction qui, à tout réel strictement positif $x$, associe : $\operatorname{Arctan}(x)+\operatorname{Arctan}\left(\frac{1}{x}\right)$.
\end{enumerate}

Montrer que la fonction $h$ est constante sur $] 0,+\infty$ [ (on précisera la valeur prise par $h$ sur $] 0,+\infty[$ ).

\begin{enumerate}
  \setcounter{enumi}{1}
  \item (a) Pour tout réel $t$ de $\left[0, \frac{\pi}{2}\right]$, exprimer $\cos (t)$ en fonction de $\cos \left(\frac{t}{2}\right)$.
\end{enumerate}

(b) Pour tout réel $t$ de $\left[0, \frac{\pi}{2}\right]$, comparer $\frac{1}{1+\tan ^{2}\left(\frac{t}{2}\right)}$ et $\cos ^{2}\left(\frac{t}{2}\right)$.

(c) Pour tout réel $t$ de $\left[0, \frac{\pi}{2}\right]$, on pose:

$$
u=\tan \left(\frac{t}{2}\right)
$$

Exprimer $\cos (t)$ en fonction de $u$.

\section{Partie II}
Pour tout réel $x$ de $]-1,1\left[\right.$, on pose $: F(x)=\int_{0}^{\frac{\pi}{2}} \frac{d t}{1-x \cos (t)}$.

\begin{enumerate}
  \setcounter{enumi}{2}
  \item Que vaut $F(0)$ ?

  \item A l'aide du changement de variable $u=\tan \left(\frac{t}{2}\right)$, montrer que, pour tout réel $x$ de $]-1,1[:$

\end{enumerate}

$$
F(x)=\frac{2}{\sqrt{1-x^{2}}} \operatorname{Arctan} \sqrt{\frac{1+x}{1-x}}
$$

Indication: on pensera à utiliser la première partie.

\begin{enumerate}
  \setcounter{enumi}{4}
  \item En déduire, pour tout réel $x$ de $]-1,1[$, une relation entre $F(x)$ et $F(-x)$.

  \item En déduire que $F$ est dérivable sur] - 1,1 [ et démontrer que pour tout réel $x$ de] $-1,1[:$

\end{enumerate}

$$
\left(1-x^{2}\right) F^{\prime}(x)=x F(x)+1
$$

\begin{enumerate}
  \setcounter{enumi}{6}
  \item (a) Donner la solution générale sur ] - 1,1[ de l'équation différentielle homogène :
\end{enumerate}

$$
\left(E_{0}\right) \quad\left(1-x^{2}\right) y^{\prime}-x y=0
$$

(b) A l'aide de la méthode de variation de la constante, donner la solution générale sur $]-1,1[$ de l'équation différentielle :

$$
\text { (E) } \quad\left(1-x^{2}\right) y^{\prime}-x y=1
$$

(c) Donner les solutions respectives des problèmes de Cauchy :

$$
\left(P_{0}\right) \quad\left\{\begin{array}{l}
\left(1-x^{2}\right) y^{\prime}-x y=1 \\
y(0)=0
\end{array}\right.
$$

et

$$
\left(P_{1}\right) \quad\left\{\begin{array}{l}
\left(1-x^{2}\right) y^{\prime}-x y=1 \\
y(0)=\frac{\pi}{2}
\end{array}\right.
$$

(d) Pour tout réel $x$ de $]-1,1\left[\right.$, déduire de la résolution de $\left(P_{1}\right)$ une expression simplifiée de $F(x)$ avec la fonction Arcsin.

\begin{enumerate}
  \setcounter{enumi}{7}
  \item On admet que $F$ est dérivable sur $]-1,1[$ avec pour tout $x \in]-1,1\left[: F^{\prime}(x)=\int_{0}^{\frac{\pi}{2}} \frac{\cos (t)}{(1-x \cos (t))^{2}} d t\right.$.
\end{enumerate}

En déduire la valeur de $\int_{0}^{\frac{\pi}{2}} \frac{\cos (t)-x}{(1-x \cos (t))^{2}} d t$ pour tout réel $x$ de $]-1,1[$.


\end{document}