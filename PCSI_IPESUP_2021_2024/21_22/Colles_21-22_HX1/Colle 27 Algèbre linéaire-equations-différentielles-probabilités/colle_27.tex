\documentclass[a4paper,10pt]{article}



\usepackage{fancyhdr} % pour personnaliser les en-têtes
\usepackage[utf8]{inputenc}
\usepackage[T1]{fontenc}
\usepackage{lastpage}
\usepackage[frenchb]{babel}
\usepackage{amsfonts,amssymb,stmaryrd}
\usepackage{amsmath,amsthm,mathtools}
\usepackage{paralist}
\usepackage{xspace,xypic}
\usepackage{xcolor,multicol,tabularx}
\usepackage{variations}
\usepackage{xypic}
\usepackage{eurosym,multicol}
\usepackage{graphicx}
\usepackage{mathdots}%faire des points suspendus en diagonale
\usepackage[np]{numprint}
\usepackage{hyperref} 
\usepackage{relsize,exscale}
\usepackage{listings} % pour écrire des codes avec coloration syntaxique  

\usepackage{tikz}
\usetikzlibrary{calc, arrows, plotmarks,decorations.pathreplacing}
\usepackage{colortbl}
\usepackage{multirow}
\usepackage[top=2cm,bottom=1.5cm,right=2cm,left=1.5cm]{geometry}

\newtheorem{thm}{Théorème}
\newtheorem*{pro}{Propriété}
\newtheorem*{exemple}{Exemple}

\theoremstyle{definition}
\newtheorem*{remarque}{Remarque}
\theoremstyle{definition}
\newtheorem{exo}{Exercice}
\newtheorem{definition}{Définition}


\newcommand{\vtab}{\rule[-0.4em]{0pt}{1.2em}}
\newcommand{\V}{\overrightarrow}
\renewcommand{\thesection}{\Roman{section} }
\renewcommand{\thesubsection}{\arabic{subsection} }
\renewcommand{\thesubsubsection}{\alph{subsubsection} }
\newcommand*{\transp}[2][-3mu]{\ensuremath{\mskip1mu\prescript{\smash{\mathrm t\mkern#1}}{}{\mathstrut#2}}}%

\newcommand{\K}{\mathbb{K}}
\newcommand{\C}{\mathbb{C}}
\newcommand{\R}{\mathbb{R}}
\newcommand{\Q}{\mathbb{Q}}
\newcommand{\Z}{\mathbb{Z}}
\newcommand{\N}{\mathbb{N}}
\newcommand{\p}{\mathbb{P}}
\newcommand{\M}{\mathcal{M}}

\renewcommand{\Im}{\mathop{\mathrm{Im}}\nolimits}



\definecolor{vert}{RGB}{11,160,78}
\definecolor{rouge}{RGB}{255,120,120}
% Set the beginning of a LaTeX document
\pagestyle{fancy}
\lhead{Optimal Sup Spé, groupe IPESUP}\chead{Année~2021-2022}\rhead{Niveau: Première année de PCSI }\lfoot{M. Botcazou}\cfoot{\thepage}\rfoot{mail: ibotca52@gmail.com }\renewcommand{\headrulewidth}{0.4pt}\renewcommand{\footrulewidth}{0.4pt}

\begin{document}
 	

\begin{center}
\Large \sc colle 27 = Algèbre linéaire, équations différentielles et probabilités
\end{center}

\section*{Exercices mixtes:}%[-0.25cm]

\raggedright

\begin{exo}\textbf{}\quad\\[0.25cm]
Soit $P$ le plan d’équation $x + y + z = 0$ et $D$ la droite d’équation $x =\dfrac{y}{2} = \dfrac{z}{3} $.
\begin{enumerate}
	\item Vérifier que $\R^3 = P \oplus D$.
	\item Soit $p$ la projection vectorielle de $\R^3$ sur $P$ parallèlement à $D$.
	Soit $u = (x, y, z) \in \R^3$ .\\
	Déterminer $p(u)$ et donner la matrice de $p$ dans la base canonique de $\R^3$.
	\item Déterminer une base de $\R^3$ dans laquelle la matrice de $p$ est diagonale, expliciter cette matrice.
\end{enumerate}

\centering
\rule{1\linewidth}{0.6pt}
\end{exo}

\begin{exo}\textbf{}\quad\\[0.25cm]
Une urne contient deux boules blanches et huit boules noires.\\
\begin{enumerate}
	\item Un joueur tire successivement, avec remise, cinq boules dans cette urne.
	Pour chaque boule blanche tirée, il gagne $2$ points et pour chaque boule noire tirée, il perd $3$ points.\\
	On note $X$ la variable aléatoire représentant le nombre de boules blanches tirées.\\
	On note $Y$ le nombre de points obtenus par le joueur sur une partie.
	\begin{enumerate}
		\item Déterminer la loi de $X$, son espérance et sa variance.
		\item Déterminer la loi de $Y$ , son espérance et sa variance.
	\end{enumerate}
	\item Dans cette question, on suppose que les cinq tirages successifs se font sans remise.
	\begin{enumerate}
		\item Déterminer la loi de $X$, son espérance et sa variance.
		\item Déterminer la loi de $Y$ , son espérance et sa variance.
	\end{enumerate}
\end{enumerate}
\centering
\rule{1\linewidth}{0.6pt}
\end{exo}

\begin{exo}\quad\\[0.25cm]
	\begin{enumerate}
		\item Déterminer une primitive de $x\longmapsto\cos^4(x)$
		\item Résoudre sur $\R$ l’équation différentielle : $y'' + y = \cos^3(x)$  en utilisant la méthode de variation des
		constantes.
	\end{enumerate}
	
	\centering
	\rule{1\linewidth}{0.6pt}
\end{exo}

\begin{exo}\textbf{}\quad\\[0.25cm]
Une secrétaire effectue, une première fois, un appel téléphonique vers $n$ correspondants distincts.
On admet que les $n$ appels constituent $n$ expériences indépendantes et que, pour chaque appel, la probabilité
d’obtenir le correspondant demandé est $p\in \ ]0,1[$.\\
Soit $X$ la variable aléatoire représentant le nombre de correspondants obtenus.
\begin{enumerate}
	\item Donner la loi de $X$. Justifier.
	\item La secrétaire rappelle une seconde fois, dans les mêmes conditions, chacun des $n - X$ correspondants qu’elle n’a pas pu joindre au cours de la première série d’appels. On note $Y$ la variable aléatoire représentant le nombre de personnes jointes au cours de la seconde série d’appels.
	\begin{enumerate}
		\item Soit $i \in\llbracket 0~;~ n \rrbracket $. Déterminer, pour $k\in\N$, $P (Y = k|X = i)$.
		\item Montrer l’égalité suivante : $$\begin{pmatrix}n-i\\k-i\end{pmatrix} \begin{pmatrix}n\\i\end{pmatrix}  \ = \  \begin{pmatrix}k\\i\end{pmatrix} \begin{pmatrix}n\\k\end{pmatrix}$$
		\item Prouver que $Z = X + Y$ suit une loi binomiale dont on déterminera les paramètres.
		\item Déterminer l’espérance et la variance de $Z$.
	\end{enumerate}
\end{enumerate}

\centering
\rule{1\linewidth}{0.6pt}
\end{exo}

\newpage
\begin{exo}\quad\\[0.25cm]%https://www.bibmath.net/ressources/index.php?action=affiche&quoi=mathsup/feuillesexo/equadiffs&type=fexo
On cherche à résoudre sur $\mathbb R_+^*$ l’équation différentielle :
$$x^2y''-3xy'+4y = 0 .\ (E)$$
\begin{enumerate}
	\item Cette équation est-elle linéaire ? Qu’est-ce qui change par rapport au cours ?
	\item  Analyse. Soit $y$ une solution de $(E)$ sur $\mathbb R_+^*$. Pour $t\in\mathbb R$, on pose $z(t)=y(e^t)$. En déduire que $z$ vérifie une équation différentielle linéaire d’ordre 2 à coefficients  constants que l’on précisera. En déduire l'expression $y$.
	\item Synthèse. Vérifier que la fonction $y$ trouvée précédemment est solution de $(E)$.
\end{enumerate}

\centering
\rule{1\linewidth}{0.6pt}
\end{exo}			


\begin{exo}\quad\\[0.25cm]
	Soient $E$ un espace vectoriel sur $\R$ ou $\C$ et $f \in \mathcal{L}(E)$ tel que $f^2 - f - 2Id = 0$.
	
	\begin{enumerate}
		\item Prouver que $f$ est bijectif et exprimer $f^{-1}$ en fonction de $f$ .
		\item Prouver que $E = \ker(f + Id) \oplus \ker(f - 2Id)$.
		\item Dans cette question, on suppose que $E$ est de dimension finie.\\
		Prouver que $Im(f + Id) = \ker(f - 2Id)$.
		
	\end{enumerate}
	\centering
	\rule{1\linewidth}{0.6pt}
\end{exo}








\end{document}