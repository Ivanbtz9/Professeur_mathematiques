
%%%%%%%%%%%%%%%%%% PREAMBULE %%%%%%%%%%%%%%%%%%

\documentclass[11pt,a4paper]{article}

\usepackage{amsfonts,amsmath,amssymb,amsthm}
\usepackage[utf8]{inputenc}
\usepackage[T1]{fontenc}
\usepackage[francais]{babel}
\usepackage{mathptmx}
\usepackage{fancybox}
\usepackage{graphicx}
\usepackage{ifthen}

\usepackage{tikz}   

\usepackage{hyperref}
\hypersetup{colorlinks=true, linkcolor=blue, urlcolor=blue,
pdftitle={Exo7 - Exercices de mathématiques}, pdfauthor={Exo7}}

\usepackage{geometry}
\geometry{top=2cm, bottom=2cm, left=2cm, right=2cm}

%----- Ensembles : entiers, reels, complexes -----
\newcommand{\Nn}{\mathbb{N}} \newcommand{\N}{\mathbb{N}}
\newcommand{\Zz}{\mathbb{Z}} \newcommand{\Z}{\mathbb{Z}}
\newcommand{\Qq}{\mathbb{Q}} \newcommand{\Q}{\mathbb{Q}}
\newcommand{\Rr}{\mathbb{R}} \newcommand{\R}{\mathbb{R}}
\newcommand{\Cc}{\mathbb{C}} \newcommand{\C}{\mathbb{C}}
\newcommand{\Kk}{\mathbb{K}} \newcommand{\K}{\mathbb{K}}

%----- Modifications de symboles -----
\renewcommand{\epsilon}{\varepsilon}
\renewcommand{\Re}{\mathop{\mathrm{Re}}\nolimits}
\renewcommand{\Im}{\mathop{\mathrm{Im}}\nolimits}
\newcommand{\llbracket}{\left[\kern-0.15em\left[}
\newcommand{\rrbracket}{\right]\kern-0.15em\right]}
\renewcommand{\ge}{\geqslant} \renewcommand{\geq}{\geqslant}
\renewcommand{\le}{\leqslant} \renewcommand{\leq}{\leqslant}

%----- Fonctions usuelles -----
\newcommand{\ch}{\mathop{\mathrm{ch}}\nolimits}
\newcommand{\sh}{\mathop{\mathrm{sh}}\nolimits}
\renewcommand{\tanh}{\mathop{\mathrm{th}}\nolimits}
\newcommand{\cotan}{\mathop{\mathrm{cotan}}\nolimits}
\newcommand{\Arcsin}{\mathop{\mathrm{arcsin}}\nolimits}
\newcommand{\Arccos}{\mathop{\mathrm{arccos}}\nolimits}
\newcommand{\Arctan}{\mathop{\mathrm{arctan}}\nolimits}
\newcommand{\Argsh}{\mathop{\mathrm{argsh}}\nolimits}
\newcommand{\Argch}{\mathop{\mathrm{argch}}\nolimits}
\newcommand{\Argth}{\mathop{\mathrm{argth}}\nolimits}
\newcommand{\pgcd}{\mathop{\mathrm{pgcd}}\nolimits} 

%----- Structure des exercices ------

\newcommand{\exercice}[1]{\video{0}}
\newcommand{\finexercice}{}
\newcommand{\noindication}{}
\newcommand{\nocorrection}{}

\newcounter{exo}
\newcommand{\enonce}[2]{\refstepcounter{exo}\hypertarget{exo7:#1}{}\label{exo7:#1}{\bf Exercice \arabic{exo}}\ \  #2\vspace{1mm}\hrule\vspace{1mm}}

\newcommand{\finenonce}[1]{
\ifthenelse{\equal{\ref{ind7:#1}}{\ref{bidon}}\and\equal{\ref{cor7:#1}}{\ref{bidon}}}{}{\par{\footnotesize
\ifthenelse{\equal{\ref{ind7:#1}}{\ref{bidon}}}{}{\hyperlink{ind7:#1}{\texttt{Indication} $\blacktriangledown$}\qquad}
\ifthenelse{\equal{\ref{cor7:#1}}{\ref{bidon}}}{}{\hyperlink{cor7:#1}{\texttt{Correction} $\blacktriangledown$}}}}
\ifthenelse{\equal{\myvideo}{0}}{}{{\footnotesize\qquad\texttt{\href{http://www.youtube.com/watch?v=\myvideo}{Vidéo $\blacksquare$}}}}
\hfill{\scriptsize\texttt{[#1]}}\vspace{1mm}\hrule\vspace*{7mm}}

\newcommand{\indication}[1]{\hypertarget{ind7:#1}{}\label{ind7:#1}{\bf Indication pour \hyperlink{exo7:#1}{l'exercice \ref{exo7:#1} $\blacktriangle$}}\vspace{1mm}\hrule\vspace{1mm}}
\newcommand{\finindication}{\vspace{1mm}\hrule\vspace*{7mm}}
\newcommand{\correction}[1]{\hypertarget{cor7:#1}{}\label{cor7:#1}{\bf Correction de \hyperlink{exo7:#1}{l'exercice \ref{exo7:#1} $\blacktriangle$}}\vspace{1mm}\hrule\vspace{1mm}}
\newcommand{\fincorrection}{\vspace{1mm}\hrule\vspace*{7mm}}

\newcommand{\finenonces}{\newpage}
\newcommand{\finindications}{\newpage}


\newcommand{\fiche}[1]{} \newcommand{\finfiche}{}
%\newcommand{\titre}[1]{\centerline{\large \bf #1}}
\newcommand{\addcommand}[1]{}

% variable myvideo : 0 no video, otherwise youtube reference
\newcommand{\video}[1]{\def\myvideo{#1}}

%----- Presentation ------

\setlength{\parindent}{0cm}

\definecolor{myred}{rgb}{0.93,0.26,0}
\definecolor{myorange}{rgb}{0.97,0.58,0}
\definecolor{myyellow}{rgb}{1,0.86,0}

\newcommand{\LogoExoSept}[1]{  % input : echelle       %% NEW
{\usefont{U}{cmss}{bx}{n}
\begin{tikzpicture}[scale=0.1*#1,transform shape]
  \fill[color=myorange] (0,0)--(4,0)--(4,-4)--(0,-4)--cycle;
  \fill[color=myred] (0,0)--(0,3)--(-3,3)--(-3,0)--cycle;
  \fill[color=myyellow] (4,0)--(7,4)--(3,7)--(0,3)--cycle;
  \node[scale=5] at (3.5,3.5) {Exo7};
\end{tikzpicture}}
}


% titre
\newcommand{\titre}[1]{%
\vspace*{-4ex} \hfill \hspace*{1.5cm} \hypersetup{linkcolor=black, urlcolor=black} 
\href{http://exo7.emath.fr}{\LogoExoSept{3}} 
 \vspace*{-5.7ex}\newline 
\hypersetup{linkcolor=blue, urlcolor=blue}  {\Large \bf #1} \newline 
 \rule{12cm}{1mm} \vspace*{3ex}}

%----- Commandes supplementaires ------



\begin{document}

%%%%%%%%%%%%%%%%%% EXERCICES %%%%%%%%%%%%%%%%%%

\fiche{f00072, tumpach, 2009/10/16}

Enoncés : Barbara Tumpach

\titre{Vecteurs propres et valeurs propres}

\textit{Le but de cette feuille d'exercices est d'apprendre \`a calculer les valeurs propres et vecteurs propres d'un endomorphisme de $\mathbb{R}^n$, et \`a appliquer un changement de base \`a la matrice d'un endomorphisme.}

\exercice{2761, tumpach, 2009/10/25}
\enonce{002761}{}
\begin{enumerate}
\item
\begin{enumerate}
\item Soit $f~:\mathbb{R}^2\rightarrow \mathbb{R}^2$ l'application lin\'eaire d\'efinie par 
$$
f\left(\begin{array}{c}x\\y\end{array}\right) =\frac{1}{5} \left(\begin{array}{c}3x + 4y\\ 4x - 3y\end{array}\right).
$$
\item \'Ecrire la matrice de $f$ dans la base canonique de $\mathbb{R}^2$. On la notera $A$.
\item Montrer que le vecteur $\vec{v}_1 = \left(\begin{array}{c}2 \\1\end{array}\right)$ est vecteur propre de $f$. Quelle est la valeur propre associ\'ee~?
\item Montrer que le vecteur $\vec{v}_2 = \left(\begin{array}{c}-1 \\2\end{array}\right)$ est \'egalement vecteur propre de $f$. Quelle est la valeur propre associ\'ee~?
\item Calculer graphiquement l'image du vecteur $\vec{v}_3 = \left(\begin{array}{c}1\\ 3\end{array}\right).$ Retrouver ce r\'esultat par le calcul.
\item Montrer que la famille $\{\vec{v}_1, \vec{v}_2\}$ forme une base de $\mathbb{R}^2$.
\item Quelle est la matrice de $f$ dans la base $\{\vec{v}_1, \vec{v}_2\}$~? On la notera $D$.
\item Soit $P$ la matrice  dont la premi\`ere colonne est le vecteur $\vec{v}_1$ et dont la deuxi\`eme colonne est le vecteur $\vec{v}_2$. Calculer $P^{-1}$.
\item Quelle relation y-a-t-il entre $A$, $P$, $P^{-1}$ et $D$~?
\item Calculer $A^n$, pour $n\in\mathbb{N}$.
\end{enumerate}
\item M\^eme exercice avec la matrice $A = \left(\begin{array}{cc}2 & -3\\ -1 & 0\end{array}\right)$ et les vecteurs $\vec{v}_1 = \left(\begin{array}{c}3 \\-1\end{array}\right)$, $\vec{v}_2 = \left(\begin{array}{c}1 \\1\end{array}\right)$ et $\vec{v}_3 = \left(\begin{array}{c}0 \\4\end{array}\right)$.
\end{enumerate}
\finenonce{002761}



\finexercice
\exercice{2762, tumpach, 2009/10/25}
\enonce{002762}{}
D\'eterminer le polyn\^ome caract\'eristique des matrices suivantes
$$
\left(\begin{array}{cc}0 & 1 \\1 & 0\end{array}\right),\quad
\left(\begin{array}{ccc}0 & 1 & 1\\1 & 0 & 1\\1 & 1 & 0\end{array}\right),\quad
\left(\begin{array}{cccc}0 & 1 & 1 & 1\\ 1 & 0 & 1 & 1\\ 1 & 1 & 0 & 1\\1 & 1 & 1 & 0\end{array}\right).
$$
\finenonce{002762}



\finexercice
\exercice{2763, tumpach, 2009/10/25}
\enonce{002763}{}
Rechercher les valeurs propres et vecteurs propres des matrices suivantes~:
$$
\left(\begin{array}{ccc} 1 & 0 & 0\\ 0 & 1 & 1\\ 0 & 1 & -1\end{array}\right), \quad
\left(\begin{array}{ccc} 1 & 0 & 4\\0 & 7 & -2\\ 4 & -2 & 0\end{array}\right),\quad
\left(\begin{array}{ccc} 1 & -1 & -1 \\ -1 & a^2 & 0 \\ -1 & 0 & a^2 \end{array} \right)
\quad(a\neq 0).
$$
\finenonce{002763}



\finexercice
\exercice{2764, tumpach, 2009/10/25}
\enonce{002764}{}
Trouver une matrice carr\'ee inversible $P$ telle que $B = PAP^{-1}$ soit diagonale, et \'ecrire la matrice $B$ obtenue, pour les matrices $A$ suivantes~:
$$
\left(\begin{array}{ccc}2 & 0 & 0\\0 & 1 & -4\\0 &-4 & 1\end{array}\right),\quad
\left(\begin{array}{ccc}1 & 0 & 1\\0 & 1 & 0\\1 & 0 & 1\end{array}\right),\quad
\left(\begin{array}{ccc}1 & 0 & 4\\0 & 7&-2\\4 & -2 & 0\end{array}\right).
$$
\finenonce{002764}



\finexercice
\exercice{2765, tumpach, 2009/10/25}
\enonce{002765}{DS mai 2008}
Soit la matrice 
$$
A = \left(\begin{array}{ccc}7 & 3 & -9\\-2 & -1 & 2\\2 & -1 & -4\end{array}\right)
$$
qui repr\'esente $f$, un endomorphisme de $\mathbb{R}^3$ dans la base canonique $\mathcal{B} = \{\vec{i}, \vec{j}, \vec{k}\}$.
\begin{enumerate}
\item
\begin{enumerate}
\item Montrer que les valeurs propres de $A$ sont $\lambda_1 = -2$, $\lambda_2 = 1$ et $\lambda_3 = 3$.
\item En d\'eduire que l'on peut diagonaliser $A$.
\end{enumerate}
\item 
\begin{enumerate}
\item
D\'eterminer une base $\mathcal{B}' = \{\vec{v}_1, \vec{v}_2, \vec{v}_3\}$ de vecteurs propres tels que la matrice de $f$ dans la base $\mathcal{B}'$ soit
$$
D = \left(\begin{array}{ccc}\lambda_1 & 0 & 0\\ 0 & \lambda_2 & 0\\ 0 & 0 &\lambda_3\end{array}\right).
$$
\item Pr\'eciser la matrice de passage $P$ de la base $\mathcal{B}$ \`a la base $\mathcal{B}'$ ; quelle relation lie les matrices $A$, $P$, $P^{-1}$ et $D$~?
\end{enumerate}
\item Montrer que pour tout entier $n\in\mathbb{N}$, on a $A^n = P D^{n} P^{-1}$.
\item Apr\`es avoir donn\'e $D^n$, calculer $A^n$ pour tout $n\in\mathbb{N}$.
\end{enumerate}
\finenonce{002765}



\finexercice
\exercice{2766, tumpach, 2009/10/25}
\enonce{002766}{DS mai 2008}
Soit la matrice $A = \left(\begin{array}{ccc}-3 & -2 & -2\\2 & 1 & 2\\ 2 & 2 & 1\end{array}\right).$
\begin{enumerate}
\item Calculer les valeurs propres de $A$.
\item \begin{enumerate}
\item Donner une base et la dimension de chaque sous-espace propre de $A$.
\item $A$ est diagonalisable ; justifier cette affirmation et diagonaliser $A$.
\end{enumerate}
\end{enumerate}
\finenonce{002766}



\finexercice
\exercice{2767, tumpach, 2009/10/25}
\enonce{002767}{}
On consid\`ere la matrice 
$$
A = \left(\begin{array}{cccc}1 & 0 & 0 & 0\\a & 1 & 0 & 0 \\a' & b & 2 & 0\\ a'' & b' & c & 2\end{array}\right).
$$
\`A quelles conditions les inconnue doivent-elles satisfaire pour que cette matrice soit diagonalisable~? Ces conditions \'etant remplies, fournir une base de vecteurs propres pour $A$.
\finenonce{002767}



\finexercice

\finfiche 



 \finenonces 



 \finindications 

\noindication
\noindication
\noindication
\noindication
\noindication
\noindication
\noindication


\newpage

\nocorrection
\nocorrection
\nocorrection
\nocorrection
\nocorrection
\nocorrection
\nocorrection


\end{document}

