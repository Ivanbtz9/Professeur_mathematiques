\documentclass[a4paper,11pt]{article}

\usepackage{inputenc}
\usepackage[T1]{fontenc}
\usepackage[frenchb]{babel}
\usepackage{fancyhdr,fancybox} % pour personnaliser les en-têtes
\usepackage{lastpage,setspace}
\usepackage{amsfonts,amssymb,amsmath,amsthm,mathrsfs}
\usepackage{relsize,exscale,bbold}
\usepackage{paralist}
\usepackage{xspace,multicol,diagbox,array}
\usepackage{xcolor}
\usepackage{variations}
\usepackage{xypic}
\usepackage{eurosym,stmaryrd}
\usepackage{graphicx}
\usepackage[np]{numprint}
\usepackage{hyperref} 
\usepackage{tikz}
\usepackage{colortbl}
\usepackage{multirow}
\usepackage{MnSymbol,wasysym}
\usepackage[top=1.5cm,bottom=1.5cm,right=1.2cm,left=1.5cm]{geometry}
\usetikzlibrary{calc, arrows, plotmarks, babel,decorations.pathreplacing}
\setstretch{1.25}
%\usepackage{lipsum} %\usepackage{enumitem} %\setlist[enumerate]{itemsep=1mm} bug avec enumerate



\newtheorem{thm}{Théorème}
\newtheorem{rmq}{Remarque}
\newtheorem{prop}{Propriété}
\newtheorem{cor}{Corollaire}
\newtheorem{lem}{Lemme}
\newtheorem{prop-def}{Propriété-définition}

\theoremstyle{definition}

\newtheorem{defi}{Définition}
\newtheorem{ex}{Exemple}
\newtheorem*{rap}{Rappel}
\newtheorem{cex}{Contre-exemple}
\newtheorem{exo}{Exercice} % \large {\fontfamily{ptm}\selectfont EXERCICE}
\newtheorem{nota}{Notation}
\newtheorem{ax}{Axiome}
\newtheorem{appl}{Application}
\newtheorem{csq}{Conséquence}
\def\di{\displaystyle}



\renewcommand{\thesection}{\Roman{section}}\renewcommand{\thesubsection}{\arabic{subsection} }\renewcommand{\thesubsubsection}{\alph{subsubsection} }


\newcommand{\bas}{~\backslash}\newcommand{\ba}{\backslash}
\newcommand{\C}{\mathbb{C}}\newcommand{\R}{\mathbb{R}}\newcommand{\Q}{\mathbb{Q}}\newcommand{\Z}{\mathbb{Z}}\newcommand{\N}{\mathbb{N}}\newcommand{\V}{\overrightarrow}\newcommand{\Cs}{\mathscr{C}}\newcommand{\Ps}{\mathscr{P}}\newcommand{\Rs}{\mathscr{R}}\newcommand{\Gs}{\mathscr{G}}\newcommand{\Ds}{\mathscr{D}}\newcommand{\happy}{\huge\smiley}\newcommand{\sad}{\huge\frownie}\newcommand{\danger}{\begin{tikzpicture}[x=1.5pt,y=1.5pt,rotate=-14.2]
	\definecolor{myred}{rgb}{1,0.215686,0}
	\draw[line width=0.1pt,fill=myred] (13.074200,4.937500)--(5.085940,14.085900)..controls (5.085940,14.085900) and (4.070310,15.429700)..(3.636720,13.773400)
	..controls (3.203130,12.113300) and (0.917969,2.382810)..(0.917969,2.382810)
	..controls (0.917969,2.382810) and (0.621094,0.992188)..(2.097660,1.359380)
	..controls (3.574220,1.726560) and (12.468800,3.984380)..(12.468800,3.984380)
	..controls (12.468800,3.984380) and (13.437500,4.132810)..(13.074200,4.937500)
	--cycle;
	\draw[line width=0.1pt,fill=white] (11.078100,5.511720)--(5.406250,11.875000)..controls (5.406250,11.875000) and (4.683590,12.812500)..(4.367190,11.648400)
	..controls (4.050780,10.488300) and (2.375000,3.675780)..(2.375000,3.675780)
	..controls (2.375000,3.675780) and (2.156250,2.703130)..(3.214840,2.964840)
	..controls (4.273440,3.230470) and (10.640600,4.847660)..(10.640600,4.847660)
	..controls (10.640600,4.847660) and (11.332000,4.953130)..(11.078100,5.511720)
	--cycle;
	\fill (6.144520,8.839900)..controls (6.460940,7.558590) and (6.464840,6.457090)..(6.152340,6.378910)
	..controls (5.835930,6.300840) and (5.320300,7.277400)..(5.003900,8.554750)
	..controls (4.683590,9.835940) and (4.679690,10.941400)..(4.996090,11.019600)
	..controls (5.312490,11.097700) and (5.824210,10.121100)..(6.144520,8.839900)
	--cycle;
	\fill (7.292960,5.261780)..controls (7.382800,4.898500) and (7.128900,4.523500)..(6.730460,4.421880)
	..controls (6.328120,4.324220) and (5.929680,4.535220)..(5.835930,4.898500)
	..controls (5.746080,5.261780) and (5.999990,5.640630)..(6.402340,5.738340)
	..controls (6.804690,5.839840) and (7.203110,5.625060)..(7.292960,5.261780)
	--cycle;
	\end{tikzpicture}}\newcommand{\alors}{\Large\Rightarrow}\newcommand{\equi}{\Leftrightarrow}
\newcommand{\fonction}[5]{\begin{array}{l|rcl}
		#1: & #2 & \longrightarrow & #3 \\
		& #4 & \longmapsto & #5 \end{array}}


\definecolor{vert}{RGB}{11,160,78}
\definecolor{rouge}{RGB}{255,120,120}
\definecolor{bleu}{RGB}{15,5,107}



\pagestyle{fancy}
\lhead{Groupe IPESUP}\chead{}\rhead{Année~2022-2023}\lfoot{M. Botcazou}\cfoot{\thepage/5}\rfoot{PCSI }\renewcommand{\headrulewidth}{0.4pt}\renewcommand{\footrulewidth}{0.4pt}


\begin{document}
 	
	

\noindent\shadowbox{
	\begin{minipage}{1\linewidth}
		\huge{\textbf{ TD 1 : Raisonnement et vocabulaire ensembliste  }}
	\end{minipage}
}
\bigskip




\section*{Rédaction, types de raisonnements et vocabulaire ensembliste:}\hfill\\%[-0.25cm]
\begin{minipage}{1\linewidth}
	\begin{minipage}[t]{0.48\linewidth}
		\raggedright
	
		
\subsection*{La syntaxe mathématique}
\begin{exo}\textbf{(*)}\quad\\[0.2cm]
	Vrai ou faux? Justifier. 
	\begin{enumerate}
		\item $\forall (x,y)\in (\R^*)^2$,  $x<y\alors \dfrac{1}{x}> \dfrac{1}{y}$.
		\item $\forall x\in[0,1]$,  $x-x^2 \in\N \alors x\in\{0,1\}$.
		
		\item $\forall x,y\in\R$,  $ x<y \alors \sin(x)<\sin(y)$.
		\item $\not\exists x\in\R^+$, $x<\sqrt{x}$.
		\item $\forall x \in\R$,  $ x^2+x>0 \alors x>0$.
		\item $\forall N\in\N^*, \ \exists n\in\N^*, \ \sum\limits_{k=0}^{n}k \geq N$.
	\end{enumerate}
	\centering
	\rule{1\linewidth}{0.6pt}
\end{exo}



\begin{exo}\textbf{(**)}\quad\\[0.2cm]
	Soit $f : \R \longrightarrow \R$ une fonction. Écrire avec des quantificateurs les propositions suivantes :
	\begin{multicols}{2}
		\begin{enumerate}
			\item $f$ est périodique. \item $f$ est majorée.\item $f$ est constante.\item $f$ atteint tous les réels.\item $f$ est croissante sur $\R$.\item $f$ prend des valeurs aussi grandes que l'on veut.\item $f$ s'annule au plus une fois \item $f$ admet un point fixe.
		\end{enumerate}
	\end{multicols}
 
	\centering
	\rule{1\linewidth}{0.6pt}
\end{exo}

\begin{exo}\textbf{(*)}\quad\\[0.2cm]
Donner la négation des assertions suivantes après avoir justifié si elles sont vraies ou fausses.  
\begin{enumerate}
	\item $\exists x\in \R \quad \forall y\in \R \quad x+y > 0. $
	\item $\forall x\in \R \quad \exists y\in \R \quad x+y > 0. $ 
	\item $ \forall x\in \R \quad \forall y\in \R \quad x+y > 0. $
	\item $ \exists x\in \R\quad \forall y\in \R \quad y^2 > x .$
\end{enumerate}


\centering
\rule{1\linewidth}{0.6pt}
\end{exo}



\end{minipage}	
\hfill\vrule\hfill
\begin{minipage}[t]{0.48\linewidth}
\raggedright



\begin{exo}\textbf{(**)}\quad\\[0.2cm]
	
	Montrer que 
	$$\forall \epsilon >0 \quad \exists N \in \N \text{ tel que }
	\left(n \geq N \Rightarrow 2-\epsilon < \frac{2n + 1}{n + 2} < 2 + \epsilon\right).$$
	
	\centering
	\rule{1\linewidth}{0.6pt}
\end{exo}


\begin{exo}\textbf{(*)}\quad\\[0.2cm]
	Écrire en langage mathématique l'ensemble :
	\begin{enumerate}
		\item des entiers naturels divisibles par $7$.
		\item des réels qui sont la somme de deux carrés d'entiers.
		\item des entiers relatifs qui possèdent un antécédent
		par la fonction $x\longmapsto e^x + x$.
	\end{enumerate}
	\centering
	\rule{1\linewidth}{0.6pt}
\end{exo}
\subsection*{Relations ensemblistes}
\begin{exo}\textbf{(**)}\quad\\[0.2cm]
	Soient $E$ et $F$ deux ensembles, $f:E\rightarrow F$. D\'emontrer que :\\
	$\forall A,B \in \mathcal{P}(E) \quad (A\subset B)\Rightarrow (f(A)\subset f(B))$,\\
	$\forall A,B \in \mathcal{P}(E) \quad f(A\cap B)\subset f(A)\cap f(B)$,\\
	$\forall A,B \in \mathcal{P}(E) \quad f(A\cup B) = f(A)\cup f(B)$,\\
	$\forall A,B \in \mathcal{P}(F) \quad f^{-1}(A\cup B) = f^{-1}(A)\cup f^{-1}(B)$,\\
	$\forall A \in \mathcal{P}(F) \quad \quad f(f^{-1}(A)) \subset A $,\\
	$\forall A \in \mathcal{P}(E) \quad \quad A \subset f^{-1}(f(A))$,\\
	$\forall A \in \mathcal{P}(F) \quad \quad f^{-1}(F\backslash A)=E\backslash f^{-1}(A)$.
	
	\centering
	\rule{1\linewidth}{0.6pt}
\end{exo}



\begin{exo}\textbf{(**)}\quad\\[0.2cm]
Soit $E$ un ensemble, $A$, $B$ et $C$ des parties de $E$ telles que $A \cup B = A \cup C$ et $A \cap B = A \cap C$. 

Montrer que $B = C$.

\centering
\rule{1\linewidth}{0.6pt}
\end{exo}	




\end{minipage}
\end{minipage}


\begin{minipage}{1\linewidth}
	\begin{minipage}[t]{0.48\linewidth}
		\raggedright
		
		\begin{exo}\textbf{(**)}\quad\\[0.2cm]
		Étudier les inclusions $A \subset B$ et $B \subset A$ pour :\\[0.2cm]
		$A = \left\{\dfrac{\epsilon}{k(k+1)} \ | \ k\in\N^*,\ \epsilon\in\{\pm1\}\right\}$\\[0.2cm]
		$B = \left\{\dfrac{1}{p} - \dfrac{1}{q}\ | \ p,q\in\N^*\right\}$

			\centering
			\rule{1\linewidth}{0.6pt}
		\end{exo}
	\begin{exo}\textbf{(**)}\quad\\[0.2cm]
		
		Soient $E$ un ensemble et $A, B, C \in \Ps (E)$. Montrer que 
		\begin{enumerate}
			\item $ \overline{A\cap B}~\backslash C = \left(\overline{C}~\backslash B\right) \cup \left(\overline{A}~\backslash C\right).$
			\item $A \ba(B \cap C) = (A \ba B) \cup( A \ba C ).$
			\item $A \cup B = B \cap C  \ \equi A\subset B\subset C.$
		\end{enumerate}
	
		
		\centering
		\rule{1\linewidth}{0.6pt}
	\end{exo}
	\subsection*{Raisonnement par récurrence}
	
\begin{exo}\textbf{(*)}\quad\\[0.2cm]	
	
	\begin{enumerate}
		\item Montrer que: pour $n$ dans $\N^*$, la somme des $n$ premiers entiers est donnée par la formule:
		$$1 + 2 + ... + n =
		\dfrac{n(n + 1)
		}{2}$$
		\item Montrer que: pour $n$ dans $\N^*$, la somme des $n$ premiers entiers au carré est donnée par la formule:
		$$1^2 + 2^2 + 3^2 +... + n^2 =
		\dfrac{n(n + 1)(2n+1)
		}{6}$$ 
		\item Montrons par récurrence :
		$$\forall n \in \N^*, \ 1+ \dfrac{1}{2^2}+\dfrac{1}{3^2} + ...+ \dfrac{1}{n^2} \leq 2 - \dfrac{1}{n}$$\hfil\\
	\end{enumerate}

		\centering
\rule{1\linewidth}{0.6pt}
\end{exo}

		\begin{exo}\textbf{(***)}\quad\\[0.2cm]
	
	Soit $A$ une partie de $\N^*$ possédant les trois propriétés suivantes :
	\begin{enumerate}[$\bullet$]
		\item $1\in A$.
		\item $\forall n \in \N^*, \ n\in A \alors 2n\in A$.
		\item $\forall n \in \N^*, \ (n+1)\in A \alors n\in A$.
	\end{enumerate}
	
	Démontrer que $A=\N^*$.
	
	\centering
	\rule{1\linewidth}{0.6pt}
\end{exo}
	

	
		
		
		
	\end{minipage}	
	\hfill\vrule\hfill
	\begin{minipage}[t]{0.48\linewidth}
		\raggedright
		
				\begin{exo}\textbf{(*)}\quad\\[0.2cm]
			
			\begin{enumerate}	
				\item On note $(u_n )_{n\in\N}$ la suite définie par : $u_0 = 0$,
				$u_1 = 1$ et pour tout $n \in \N : \ u_{n+2} = 5u_{ n+1} - 6u_n$.
				Montrer que pour tout $n \in \N : u_n = 3^n - 2^n $.
				
				\item On note $(u_n )_{n\in\N}$ la suite définie par : \\
				$u_0 = 0$, $u_1 = 0$, $u_2 = 2$ et pour tout $n \in \N $ \\
				$u_{ n+3} = 3u_{ n+2} - 3u_{ n+1} + u_n $.
				
				Montrer que pour tout $n \in \N : \ u_n = n(n - 1)$.			
			\end{enumerate}
			
			\centering
			\rule{1\linewidth}{0.6pt}
		\end{exo}
	
			\begin{exo}\textbf{(*)}\quad\\[0.2cm]
		On note $(u_n){ n\in\N }$ la suite définie par $u_0 = 3$ et
		pour tout $n \in \N : u_{ n+1} = u^2_n -u_n $.
		
		Montrer que pour
		tout $n \in \N : u_n\geq  3 \times 2^n $.
		
		\centering
		\rule{1\linewidth}{0.6pt}
	\end{exo}
		
		\begin{exo}\textbf{(**)}\quad\\[0.2cm]
			Soit la suite $(x_n)_{n\in \N}$ d\'efinie par
			$x_0=4$ et $\displaystyle{x_{n+1}=\frac{2x_n^2-3}{x_n+2}}$.
			\begin{enumerate}
				\item Montrer que : $\forall n\in\N\quad x_n>3$.
				\item Montrer que : $\forall n\in\N \quad x_{n+1}-3>\frac{3}{2}(x_n-3)$.
				\item Montrer que : $\forall n\in\N \quad x_n \geqslant \left(\frac{3}{2}\right)^n+3$.
				\item La suite $(x_n)_{n\in\N}$ est-elle convergente ?
				
			\end{enumerate}
			
			\centering
			\rule{1\linewidth}{0.6pt}
		\end{exo}
	
	

		\begin{exo}\textbf{(**)}\quad\\[0.2cm]
	Soit $(u_n)_{n\in\N^*} $la suite définie par $u_1=3$ et \\pour tout \ $n\geq1$,\quad $u_{n+1} = \frac{2}{n}\sum_{k=1}^{n}u_k$. 
	
	Démontrer que, pour tout $n\in\N^*$, on a $u_n=3n$.
	
	\centering
	\rule{1\linewidth}{0.6pt}
\end{exo}

		\begin{exo}\textbf{(**)}\quad\\[0.2cm]
			
	Démontrer que tout entier $n\geq1$ peut s'écrire comme somme de puissances de $2$ toutes distinctes.
	
	\centering
	\rule{1\linewidth}{0.6pt}
\end{exo}

\begin{exo}\textbf{(**)}\quad\\[0.2cm]
	Soit $X$ un ensemble. Pour $f \in \mathcal{F} (X, X)$, on d\'efinit $f^0 = id$ et
	par r\'ecurrence pour $n \in \N$ $f^{n + 1} = f^n \circ f$.
	\begin{enumerate}
		\item Montrer que $\left(\forall n \in \N, \ \  f^{n + 1} = f \circ f^n \right).$
		\item Montrer que si $f$ est bijective alors $\left(\forall n \in \N,\ \  (f^{-1})^n
		= (f^n)^{-1}\right)$.
	\end{enumerate}
	
	\centering
	\rule{1\linewidth}{0.6pt}
\end{exo}


		
	\end{minipage}
\end{minipage}


\begin{minipage}{1\linewidth}
	\begin{minipage}[t]{0.48\linewidth}
		\raggedright	
\subsection*{Raisonnement par l'absurde}

\begin{exo}\textbf{(*)}\quad\\[0.2cm]
	Montrer que $\dfrac{\ln(7)}{\ln(2)}$ est un irrationnel. 
	
	\centering
	\rule{1\linewidth}{0.6pt}
\end{exo}

\begin{exo}\textbf{(*)}\quad\\[0.2cm]
Démontrer que si vous rangez $(n+1)$ paires de chaussettes dans $n$ tiroirs distincts, alors il y a au moins un tiroir contenant au moins $2$ paires de chaussettes.

\centering
\rule{1\linewidth}{0.6pt}
\end{exo}

\begin{exo}\textbf{(*)}\quad\\[0.2cm]
	Soit $(f_n)_{n\in\N}$ une suite d'applications de
	l'ensemble $\N$ dans lui-m\^eme. On d\'efinit une application $f$
	de $\N$ dans $\N$ en posant $f(n)=f_n(n)+1$. 
	
	Démontrer qu'il
	n'existe aucun $p\in \N$ tel que $f=f_p$.
	
	\centering
	\rule{1\linewidth}{0.6pt}
\end{exo}

\begin{exo}\textbf{(**)}\quad\\[0.2cm]
	Démontrer que l'équation $9x^5-12x^4+6x-5=0$ n'admet pas de solution entière.
	
	\centering
	\rule{1\linewidth}{0.6pt}
\end{exo}
\begin{exo}\textbf{(**)}\quad\\[0.2cm]
\begin{enumerate}
	\item Montrer qu'il existe une infinité de nombres premiers.
	\item Montrer qu'il existe une infinité de nombres premiers de la forme $4k+3$.
	\item Montrer qu'il existe une infinité de nombres premiers de la forme $6k+5$.
\end{enumerate}
	\centering
\rule{1\linewidth}{0.6pt}
\end{exo}

\begin{exo}\textbf{(***)}\quad\\[0.2cm]%https://www.bibmath.net/ressources/index.php?action=affiche&quoi=bde/logique/raisonnement&type=fexo
	\begin{enumerate}
		\item Exprimer $\cos((n+1)^\circ)$ en fonction de $\cos(n^\circ)$, $\cos(1^\circ)$ et $\cos((n-1)^\circ)$.
		\item Démontrer que $\cos(1^\circ)$	est irrationnel. 
	\end{enumerate}
	
	\centering
	\rule{1\linewidth}{0.6pt}
\end{exo}

\subsection*{injectivité, surjectivité et bijectivité}
\begin{exo}\textbf{(*)}\quad\\[0.2cm]
	Soit $f  : [1,+\infty[\rightarrow[0,+\infty[$ telle que
	$f(x)=x^2-1$. 
	
	$f$ est-elle bijective ?

\end{exo}



	\end{minipage}	
\hfill\vrule\hfill
\begin{minipage}[t]{0.48\linewidth}
\raggedright


\begin{exo}\textbf{(**)}\quad\\[0.2cm]
	Les applications suivantes sont-elles injectives, surjectives, bijectives ?
	\begin{enumerate}
		\item $f : {\N} \to {\N}, {n} \mapsto {n + 1}$
		\item $g : {\Z} \to {\Z}, {n}\mapsto{n + 1}$
		\item $h : {\R^2} \to {\R^2}, {(x, y)}\mapsto{ (x + y, x-y)}$
		\item $k : {\R \setminus \left\{ 1\right\}} \to {\R}, {x}\mapsto{\frac{x + 1}{x - 1}}$
	\end{enumerate}
	\centering
	\rule{1\linewidth}{0.6pt}
\end{exo}

\begin{exo}\textbf{(***)}\quad\\[0.2cm]
	Soit $E$ un ensemble et $f : E \rightarrow E $. Montrer que $f$ est injective si, et seulement si, pour toutes parties $A$ et
	$B$ de $E $, $f (A \cap B ) = f (A) \cap f (B )$.
	
	\centering
	\rule{1\linewidth}{0.6pt}
\end{exo}

\begin{exo}\textbf{(**)}\quad\\[0.2cm]

Montrer que $$f : \mathbb{N}^{2} \longrightarrow \mathbb{N}^{*}, (m,n) \longmapsto 2^{m}(2n+1)$$ est bijective

\centering
\rule{1\linewidth}{0.6pt}
\end{exo}



\begin{exo}\textbf{(***)}\quad\\[0.2cm]
Soit $f : E \longrightarrow F$, $g : E \longrightarrow G$. On définit 
$$ \forall x \in E,\  h(x) = (f(x),g(x))$$
\begin{enumerate}
	\item Montrer que si $f$ ou $g$ est injective, alors $h$ est injective
	\item On suppose $f$ et $g$ surjective. $h$ est-elle surjective?
\end{enumerate}

	\centering
\rule{1\linewidth}{0.6pt}
\end{exo}


\subsection*{Raisonnement par contraposée}

\begin{exo}\textbf{(*)}\quad\\[0.2cm]
	Montrer que: si $n^2$ est impair, alors $n$ est impair. 
	
	\centering
	\rule{1\linewidth}{0.6pt}
\end{exo}


\begin{exo}\textbf{(**)}\quad\\[0.2cm]
Soit $a\in\R$. Montrer que 
 $$\forall \epsilon>0, \ |a|\leq\epsilon \ \alors \ a=0.$$


\end{exo}
	\end{minipage}
\end{minipage}




\begin{minipage}{1\linewidth}
	\begin{minipage}[t]{0.48\linewidth}
		\raggedright
		
\begin{exo}\textbf{(*)}\quad\\[0.2cm]
	Écrire les contraposées des implications suivantes. Sont-elles vraies ou
	fausses ?
	\begin{enumerate}
		\item Pour tous réels $x$ et $y$, si $xy = 0$
		
		alors $(x = 0$ ou $y = 0)$.
		\item Si $ABC$ est un triangle rectangle en $A$, 
		
		alors $BC^2 = AB^2 + AC^2$.
		\item $\forall n \in \N$ $n$ pair $ \alors$ $n$ non premier.
	\end{enumerate}
	
	\centering
	\rule{1\linewidth}{0.6pt}
\end{exo}

		

\begin{exo}\textbf{(**)}\quad\\[0.2cm]
	Soit $n\in\N^*$. Montrer que:
	
 Si l'entier $(n^2-1)$ n'est pas divisible par $8$, alors l'entier $n$ est pair. 
	
	\centering
	\rule{1\linewidth}{0.6pt}
\end{exo}

	\subsection*{Raisonnement par analyse-synthèse}
		
		
		
		\begin{exo}\textbf{(**)}\quad\\[0.2cm]
			Soit $f : \mathbb{R} \longrightarrow \mathbb{R}$ une application vérifiant :
			$$ \forall (x,y,z) \in \mathbb{R}^{3},\  \dfrac{f(x)-f(y)}{x-y} = \dfrac{f(x)-f(z)}{x-z}.$$
			
			\noindent Montrer qu'il existe un unique couple $(a,b) \in \mathbb{R}^{2}$ tel que pour tout $x\in\R$, \ $f(x) = ax+b$.
			
			
			\centering
			\rule{1\linewidth}{0.6pt}
		\end{exo}
		
		\begin{exo} \textbf{(**)}\quad\\[0.2cm]
			Soit $f$ une fonction de $\R$ dans $\R$. Montrer qu'existe un unique couple $(p, i)$ de fonctions de $\R$ dans $\R$ vérifiant les conditions suivantes:
			\begin{itemize}
				\item $p$ est paire, $i$ est impaire.
				\item $f = p + i$.
			\end{itemize}
			
			\centering
			\rule{1\linewidth}{0.6pt}
		\end{exo}
		
		
			\begin{exo} \textbf{(**)}\quad\\[0.2cm]

			Trouver les fonctions $f$ de $\R$ dans $\R$ dérivables de dérivée continue telles que:
			
			
			$$\forall (x, y) \in \R^{2}, \ f(x+y) = f(x)+f(y).$$
			
			\centering
			\rule{1\linewidth}{0.6pt}
		\end{exo}
		
		
	\end{minipage}	
	\hfill\vrule\hfill
	\begin{minipage}[t]{0.48\linewidth}
		\raggedright
		
	
				\begin{exo}\textbf{(*)}\quad\\[0.2cm]
			
			Déterminer l'ensemble des fonctions $f : \R \longrightarrow \R$ pour lesquelles :
			
			$\forall (x, y) \in \R^2$ , $f(y-f(x)) = 2 - x - y$.
			\centering
			\rule{1\linewidth}{0.6pt}
		\end{exo}
	
		\begin{exo}\textbf{(***)}\quad\\[0.2cm]
			On cherche toutes les isométries de $\R$, i.e. toutes les fonctions $f : \R \longrightarrow \R$ pour lesquelles:
			$$\forall( x, y) \in \R^2 : \ |f(x) - f(y)| = |x - y|$$
			
			\textit{Indication:} Soit $f$ une isométrie. 
			
			On note $\delta$ la fonction $x \longmapsto f(x) - f(0)$ sur $\R$.
			
			
			Montrer, en étudiant la quantité $\left(f(x)- f(y)\right)^2$ , que pour tous $x, y \in \R, \ \delta(x) \delta(y) = x y$.
			
			\centering
			\rule{1\linewidth}{0.6pt}
		\end{exo}
		
		
		\begin{exo}\textbf{(**)}\quad\\[0.2cm]
			On note $\mathscr{A}$ l'ensemble des fonctions affines et
			$\mathscr{B}$ l'ensemble des fonctions $f : \R \longrightarrow \R$ dérivables
			pour lesquelles $f (0) = f' (0) = 0$. Montrer par analyse-synthèse que toute fonction dérivable de $\R$ dans $\R$ est la somme, d'une et une seule manière, d'une fonction de $\mathscr{A}$ et d'une fonction de $\mathscr{B}$.
			
			\centering
			\rule{1\linewidth}{0.6pt}
		\end{exo}
	
		\begin{exo} \textbf{(***)}\quad\\[0.2cm]
			
			Soit une fonctions $f$ de $\R$ dans $\R$ continue telle que:
			
			
			$$\forall (x, y) \in \R^{2}, \ f(x+y) = f(x)+f(y).$$
			
			\begin{enumerate}
				\item Calculer $f (0)$ et montrer que $f$ est impaire.
				\item Pour $n \in \N$ et $x \in \R$, calculer $f (nx)$ en fonction de $n$ et $f (x)$.
				\item Soit $a = f (1)$. Montrer que $$ \forall x \in \Q,\ 
				f (x) = ax. $$
				
				\item Expliquer pourquoi tout nombre réel est limite d'une suite de nombres rationnels.
				\item Conclure.
			\end{enumerate}
			\centering
			\rule{1\linewidth}{0.6pt}
		\end{exo}
		
	
		
		
	\end{minipage}
\end{minipage}


\section*{Compléments sur les nombres réels}\hfill\\%[-0.25cm]
\begin{minipage}{1\linewidth}
	\begin{minipage}[c]{0.48\linewidth}
		\raggedright
		
		

		
		
		\begin{exo}\textbf{(*)}\quad\\[0.2cm]
			Le maximum de deux nombres $x,y$ (c'est-\`a-dire le plus grand des
			deux) est not\'e $\max(x,y)$. De m\^eme on notera $\min(x,y)$ le plus petit des deux nombres
			$x,y$. D\'emontrer que :
			$$ \max(x,y)= \frac{x+y+ \vert x-y\vert}{ 2}$$ \quad \hbox{et}\quad $$ \min(x,y)= \frac{x+y- \vert
				x-y\vert}{ 2}. $$
			Trouver une formule pour $\max(x,y,z)$.
			
			\centering
			\rule{1\linewidth}{0.6pt}
		\end{exo}
		
		\begin{exo}\textbf{(*)}\quad\\[0.2cm]
			D{\'e}terminer la borne sup{\'e}rieure et inf{\'e}rieure
			
			
			(si elles existent) de : $A=\{u_n \mid n\in\N\}$ en posant
			$u_n=2^n$ si $n$ est pair et  $u_n=2^{-n}$ sinon.
			
			\centering
			\rule{1\linewidth}{0.6pt}
		\end{exo}
		
				\begin{exo}\textbf{(**)}\quad\\[0.2cm]
			\begin{enumerate}
				\item D\'emontrer que si $r \in \Q$ et $ x \notin \Q $ alors $ r+x
				\notin \Q $ et si $r\not= 0$ alors $ r\times x \notin \Q $.
				\item Montrer que $\sqrt 2 \not\in\Q$,
				\item En d\'eduire : entre deux nombres rationnels distincts il y a toujours un nombre irrationnel.
				\item En déduire que l'ensemble des nombres irrationnels est dense dans l'ensemble des nombres réels. 
			\end{enumerate}
			\centering
			\rule{1\linewidth}{0.6pt}
		\end{exo}
		
		
	\end{minipage}	
	\hfill\vrule\hfill
	\begin{minipage}[c]{0.48\linewidth}
		\raggedright
		
		
		
		\begin{exo}\textbf{(**)}\quad\\[0.2cm]
			Soit $A$ et $B$ deux parties born\'ees de $\R$ non vide.
			
			\textbf{Vrai} ou \textbf{faux} ?
			\begin{enumerate}
				\item  $A \subset B \Rightarrow \sup A \leqslant \sup B$,
				\item $A \subset B \Rightarrow \inf A \leqslant \inf B$,
				\item $\sup (A\cup B) = \max(\sup A,\sup B)$,
				\item $\sup(A+B) < \sup A + \sup B$,
				\item $\sup(-A) = -\inf A$,
				\item $\sup A +\inf B \leqslant \sup(A+B)$.
			\end{enumerate}
			
			\centering
			\rule{1\linewidth}{0.6pt}
		\end{exo}
		
		
		\begin{exo}\textbf{(**)}\quad\\[0.2cm]
			Soit $x$ un r\'eel.
			\begin{enumerate}
				%\item Donner l'encadrement qui définit la partie entière $E(x)$.
				\item Soit $(u_n)_{n\in \N^*}$ la suite définie par $$u_n = \dfrac{E (x) + E (2x) + \ldots + E (nx)}{n^2}.$$ \\
				Donner un encadrement simple de $n^2 \times u_n$.
				\item En déduire que $(u_n)$ converge et calculer sa limite.
				\item En d\'eduire que $\Q$ est dense dans $\R$.
			\end{enumerate}
			
			\centering
			\rule{1\linewidth}{0.6pt}
		\end{exo}
		
		%\begin{exo}\quad\\[0.2cm]\centering \rule{1\linewidth}{0.6pt}\end{exo}	
		
		
	\end{minipage}
\end{minipage}


	

\end{document}