\documentclass[a4paper,10pt]{article}



\usepackage{fancyhdr} % pour personnaliser les en-têtes
\usepackage[utf8]{inputenc}
\usepackage[T1]{fontenc}
\usepackage{lastpage}
\usepackage[frenchb]{babel}
\usepackage{amsfonts,amssymb}
\usepackage{amsmath,amsthm,mathtools}
\usepackage{paralist}
\usepackage{xspace}
\usepackage{xcolor,multicol}
\usepackage{variations}
\usepackage{xypic}
\usepackage{eurosym}
\usepackage{graphicx}
\usepackage[np]{numprint}
\usepackage{hyperref} 
\usepackage{listings} % pour écrire des codes avec coloration syntaxique  

\usepackage{tikz}
\usetikzlibrary{calc, arrows, plotmarks,decorations.pathreplacing}
\usepackage{colortbl}
\usepackage{multirow}
\usepackage[top=2cm,bottom=1.5cm,right=2cm,left=1.5cm]{geometry}

\newtheorem{thm}{Théorème}
\newtheorem*{pro}{Propriété}
\newtheorem*{exemple}{Exemple}

\theoremstyle{definition}
\newtheorem*{remarque}{Remarque}
\theoremstyle{definition}
\newtheorem{exo}{Exercice}
\newtheorem{definition}{Définition}


\newcommand{\vtab}{\rule[-0.4em]{0pt}{1.2em}}
\newcommand{\V}{\overrightarrow}
\renewcommand{\thesection}{\Roman{section} }
\renewcommand{\thesubsection}{\arabic{subsection} }
\renewcommand{\thesubsubsection}{\alph{subsubsection} }
\newcommand*{\transp}[2][-3mu]{\ensuremath{\mskip1mu\prescript{\smash{\mathrm t\mkern#1}}{}{\mathstrut#2}}}%

\newcommand{\C}{\mathbb{C}}
\newcommand{\R}{\mathbb{R}}
\newcommand{\Q}{\mathbb{Q}}
\newcommand{\Z}{\mathbb{Z}}
\newcommand{\N}{\mathbb{N}}



\definecolor{vert}{RGB}{11,160,78}
\definecolor{rouge}{RGB}{255,120,120}
% Set the beginning of a LaTeX document
\pagestyle{fancy}
\lhead{Optimal Sup Spé, groupe IPESUP}\chead{Année~2021-2022}\rhead{Niveau: Première année de PCSI }\lfoot{M. Botcazou}\cfoot{\thepage}\rfoot{mail: ibotca52@gmail.com }\renewcommand{\headrulewidth}{0.4pt}\renewcommand{\footrulewidth}{0.4pt}

\begin{document}
	
	
	\begin{center}
		\Large \sc colle 13 = Espaces vectoriels, Applications linéaires et représentations matricielle
	\end{center}




\section*{Espaces vectoriels et applications linéaires:}\hfill\\\hfill\\
\begin{minipage}{1\linewidth}
	\begin{minipage}[t]{0.48\linewidth}
		\raggedright
		
		
		
		\begin{exo}\quad\\
			Soit $E$ un espace vectoriel et soient $E_1$ et $E_2$ deux sous-espaces vectoriels de dimension finie
			de $E$, on d\'efinit l'application $f\colon E_1\times E_2 \to E$ par $f(x_1,x_2)=x_1+x_2$.
			\begin{enumerate}
				\item Montrer que $f$ est lin\'eaire.
				\item D\'eterminer le noyau et l'image de $f$.
				\item Que donne le th\'eor\`eme du rang ?
			\end{enumerate}
			
			\centering
			\rule{1\linewidth}{0.6pt}
		\end{exo}
	
		\begin{exo}\quad\\
		Soit $E$ un espace vectoriel de dimension $n$ et
		$\phi $  une application lin\'eaire de  $E$  dans lui-m\^eme telle que  $\phi ^n=0$  et
		$\phi ^{n-1}\not = 0$.
		Soit  $x\in E$  tel que  $\phi ^{n-1}(x )\not = 0$. Montrer que la
		famille  $\{ x,\phi(x),\phi^2(x), \ldots ,\phi ^{n-1}(x)\} $  est une base de $E$.
			
		\centering
		\rule{1\linewidth}{0.6pt}
	\end{exo}


		\begin{exo}\quad\\
		Soient $f$ et $g$ deux endomorphismes de $E$ tels que $f \circ g = g \circ f$.
		Montrer que $Ker f$ et $Im f$ sont stables par $g$.
		
		\centering
		\rule{1\linewidth}{0.6pt}
	\end{exo}


		\begin{exo}\quad\\
		Soit $E$ l'espace vectoriel des fonctions de $\R$ dans $\R$. Soient $P$ le
		sous-espace des fonctions paires et $I$ le sous-espace des fonctions
		impaires. Montrer que $E=P\bigoplus I$. Donner l'expression du projecteur sur
		$P$ de direction $I$.
	
	\centering
	\rule{1\linewidth}{0.6pt}
\end{exo}
	
		
	
	\end{minipage}	
	\hfill\vrule\hfill
	\begin{minipage}[t]{0.48\linewidth}
		\raggedright
		
	\begin{exo}\quad\\
		Pour les applications lin\'eaires suivantes,  d\'eterminer $Ker f_i$ et 
		$Im f_i$. En d\'eduire si $f_i$ est injective, surjective, bijective.
		$$\begin{array}{ll}
		f_2 : \R^3 \to \R^3 & f_2(x,y,z)=(2x+y+z,y-z,x+y) \\
		f_3 : \R^2 \to \R^4 & f_3(x,y)=(y,0,x-7y,x+y) \\
		f_4 : \R_3[X] \to \R^3 & f_4(P) = \big( P(-1), P(0), P(1) \big) \\
		\end{array}
		$$
		
		\centering
		\rule{1\linewidth}{0.6pt}
	\end{exo}
		
	\begin{exo}\quad\\
		 Soit $E$ un espace vectoriel de dimension $3$, $\{e_1,e_2,e_3\}$ une base
		de $E$, et $t$ un param\`etre r\'eel. \\
		D\'emontrer que la donn\'ee de
		$\left\{
		\begin{array}{rcl}
		\phi(e_1) & = & e_1+e_2  \\
		\phi(e_2) & = & e_1-e_2  \\
		\phi(e_3) & = & e_1+t e_3
		\end{array}\right.$
		d\'efinit une application lin\'eaire
		$\phi$ de $E$ dans $E$. \'Ecrire le transform\'e du vecteur 
		$x=\alpha_1e_1+\alpha_2e_2+\alpha_3e_3$. Comment choisir $t$ pour que 
		$\phi$ soit injective ? surjective ?	
		
		\centering
		\rule{1\linewidth}{0.6pt}
	\end{exo}

\begin{exo}\quad\\
	Soit $E={\R}_{n}[X]$ l'espace vectoriel des polyn\^{o}mes de degr\'{e} $%
	\leq n$, et $f:E\rightarrow E$ d\'{e}finie par:
	$$f(P)=P+(1-X)P'. $$
	Montrer que $f$ est une application linéaire et donner une base de $Im f$ et de $Ker f.$
	
	\centering
	\rule{1\linewidth}{0.6pt}
\end{exo}



	
	\end{minipage}
\end{minipage}
\newpage
\section*{Espaces vectoriels et applications linéaires:}\hfill\\\hfill\\
\begin{minipage}{1\linewidth}
	\begin{minipage}[t]{0.48\linewidth}
		\raggedright
		
		
		
		\begin{exo}\quad\\
		Soit $f$ l'endomorphisme de $\R^2$ dont la matrice par
		rapport \`a la base canonique $(e_1, e_2)$ est
		$$A= \left( 
		\begin{array}{cc}
		11 & -6  \\
		12 & -6  \\
		\end{array}
		\right).$$
		Montrer que les vecteurs
		$$ e'_1 = 2e_1+3e_2,\quad e'_2 = 3e_1+4e_2,$$
		forment une base de $\R^2$ et calculer la matrice de $f$ par
		rapport \`a cette base.	
		
			\centering
			\rule{1\linewidth}{0.6pt}
		\end{exo}
		

		
		\begin{exo}\quad\\
		Soit $\R^2$ muni de la base canonique $\mathcal{B}=(\vec{i}, \vec{j})$.\\
		Soit $f : \R^2 \to \R^2$ la projection sur l'axe des abscisses $\R \vec{i}$ 
		parall\`element à $\R (\vec{i} + \vec{j})$.
		Déterminer $\textrm{Mat}_{\mathcal{B},\mathcal{B}}(f)$, la matrice de $f$ dans la base $(\vec{i}, \vec{j})$.
		
		Même question avec $\textrm{Mat}_{\mathcal{B}',\mathcal{B}}(f)$ où $\mathcal{B'}$ est la base 
		$(\vec{i} - \vec{j}, -2\vec{i}+3\vec{j})$ de $\R^2$.
		Même question avec $\textrm{Mat}_{\mathcal{B}',\mathcal{B}'}(f)$.	
			
			\centering
			\rule{1\linewidth}{0.6pt}
		\end{exo}
		
		
	\end{minipage}	
	\hfill\vrule\hfill
	\begin{minipage}[t]{0.48\linewidth}
		\raggedright
		
		\begin{exo}\quad\\
			Soit $f$ l'endomorphisme de $\R^2$ de matrice\\[0.25cm] $A=\begin{pmatrix} 2&\frac 23\\[0.25cm]
			-\frac 52&-\frac 23 \end{pmatrix}$ dans la base canonique. Soient\\[0.25cm] 
			$e_1 = \begin{pmatrix} -2 \\ 3\end{pmatrix}$
			et $e_2 = \begin{pmatrix} -2 \\ 5 \end{pmatrix}$.
			\begin{enumerate}
				\item Montrer que $\mathcal{B}'= (e_1, e_2)$ est une base de $\R^2$ et déterminer 
				$\text{Mat}_{\mathcal{B}'}(f)$.
				\item Calculer $A^n$ pour $n \in \N$.
				\item Déterminer l'ensemble des suites réelles qui \\[0.25cm]vérifient $\forall n \in \N$
				$\begin{cases} x_{n + 1} = 2x_n + \dfrac 23 y_n \\ y_{n + 1} = -\dfrac 52 x_n -
				\dfrac 23 y_n \end{cases}$.\\[0.25cm]
			\end{enumerate}
			
			
			\centering
			\rule{1\linewidth}{0.6pt}
		\end{exo}
		
		\begin{exo}\quad\\
			Trouver toutes les matrices de $\mathcal{M}_3(\R)$ qui vérifient
			\begin{enumerate}
				\item $M^2 = 0$ ;
				\item $M^2 = M$ ; 
				\item $M^2 = I$. 
			\end{enumerate}
			
			\centering
			\rule{1\linewidth}{0.6pt}
		\end{exo}

		
		
		
	\end{minipage}
\end{minipage}
\end{document}