\documentclass[a4paper,10pt]{article}



\usepackage{fancyhdr} % pour personnaliser les en-têtes
\usepackage[utf8]{inputenc}
\usepackage[T1]{fontenc}
\usepackage{lastpage}
\usepackage[frenchb]{babel}
\usepackage{amsfonts,amssymb,stmaryrd}
\usepackage{amsmath,amsthm,mathtools}
\usepackage{paralist}
\usepackage{xspace,xypic}
\usepackage{xcolor,multicol,tabularx}
\usepackage{variations}
\usepackage{xypic}
\usepackage{eurosym,multicol}
\usepackage{graphicx}
\usepackage{mathdots}%faire des points suspendus en diagonale
\usepackage[np]{numprint}
\usepackage{hyperref} 
\usepackage{relsize,exscale}
\usepackage{listings} % pour écrire des codes avec coloration syntaxique  

\usepackage{tikz}
\usetikzlibrary{calc, arrows, plotmarks,decorations.pathreplacing}
\usepackage{colortbl}
\usepackage{multirow}
\usepackage[top=2cm,bottom=1.5cm,right=2cm,left=1.5cm]{geometry}

\newtheorem{thm}{Théorème}
\newtheorem*{pro}{Propriété}
\newtheorem*{exemple}{Exemple}

\theoremstyle{definition}
\newtheorem*{remarque}{Remarque}
\theoremstyle{definition}
\newtheorem{exo}{Exercice}
\newtheorem{definition}{Définition}


\newcommand{\vtab}{\rule[-0.4em]{0pt}{1.2em}}
\newcommand{\V}{\overrightarrow}
\renewcommand{\thesection}{\Roman{section} }
\renewcommand{\thesubsection}{\arabic{subsection} }
\renewcommand{\thesubsubsection}{\alph{subsubsection} }
\newcommand*{\transp}[2][-3mu]{\ensuremath{\mskip1mu\prescript{\smash{\mathrm t\mkern#1}}{}{\mathstrut#2}}}%

\newcommand{\K}{\mathbb{K}}
\newcommand{\C}{\mathbb{C}}
\newcommand{\R}{\mathbb{R}}
\newcommand{\Q}{\mathbb{Q}}
\newcommand{\Z}{\mathbb{Z}}
\newcommand{\N}{\mathbb{N}}
\newcommand{\p}{\mathbb{P}}
\newcommand{\M}{\mathcal{M}}

\renewcommand{\Im}{\mathop{\mathrm{Im}}\nolimits}



\definecolor{vert}{RGB}{11,160,78}
\definecolor{rouge}{RGB}{255,120,120}
% Set the beginning of a LaTeX document
\pagestyle{fancy}
\lhead{Optimal Sup Spé, groupe IPESUP}\chead{Année~2021-2022}\rhead{Niveau: Première année de PCSI }\lfoot{M. Botcazou}\cfoot{\thepage}\rfoot{mail: ibotca52@gmail.com }\renewcommand{\headrulewidth}{0.4pt}\renewcommand{\footrulewidth}{0.4pt}

\begin{document}
 	

\begin{center}
\Large \sc colle 28 = Algèbre linéaire et réduction des endomorphismes
\end{center}

\section*{Exercices mixtes:}%[-0.25cm]

\raggedright

\begin{exo}\quad\\[0.25cm]
	Soit la matrice $A=\begin{pmatrix}
	1&0&1\\0&1&0\\1&0&1
	\end{pmatrix}$. 
	\begin{enumerate}
		\item  La matrice $A$ est-elle diagonalisable sur $\mathbb{R}$ ? Si oui, diagonaliser $A$.
		\item  La matrice $A$ est-elle inversible ?
	\end{enumerate}	
	
	\centering
	\rule{1\linewidth}{0.6pt}
\end{exo}

\begin{exo}\quad\\[0.25cm]
	Calculer les puissances n-ième de la matrice suivante : 
	$$A  =  \begin{pmatrix}
	1 & 1 \\
	0 &  2\\
	
\end{pmatrix} $$
	\centering
	\rule{1\linewidth}{0.6pt}
\end{exo}			




\begin{exo}\textbf{}\quad\\[0.25cm]
 On considère l'endomorphisme $f$ de $\R^3$ dont la matrice dans la
base canonique est
\[
A = \begin{pmatrix}
3&1&{ - 1}\\ 6&2&{ - 6}\\ 7&1&{ - 5}
\end{pmatrix}.
\]
\begin{enumerate}
	\item Montrer que $f$ admet deux valeurs propres réelles $\lambda_1
	< \lambda _2$
	\item Déterminer des vecteurs propres~$u_1$ et~$u_2$ associés aux
	valeurs propres~$\lambda_1$ et~$\lambda_2$.
	\item L'endomorphisme $f$ est-il diagonalisable ?
	\item On note
	\[ u_3 = \begin{pmatrix} 0 \\ 1 \\ 0 \end{pmatrix}. \]
	Montrer que $(u_1, u_2, u_3)$ est une base de $M_{3,1}(\R)$.
	\item Écrire la matrice $B$ de $f$ dans cette base .   
	\item Calculer $B^n$ pour tout $n \in \N$.
\end{enumerate}

\centering
\rule{1\linewidth}{0.6pt}
\end{exo}

\begin{exo}\textbf{}\quad\\[0.25cm]
  Pour tout $a \in \R$, on considère la matrice
\[
A(a) = \begin{pmatrix}
a&0&0&0\\
1&0&0&1\\
0&0&1&0\\
0&0&0&-1
\end{pmatrix} \in M_4(\R).
\]
\begin{enumerate}	
	\item Déterminer le polynôme caractéristique $\chi_{A(a)}(\lambda)$
	ainsi que les valeurs propres de $A(a)$.
	\item Justifier que si $a \not\in \{-1,0, 1\}$, alors $A(a)$ est
	diagonalisable.
	\item Étudier la diagonalisabilité de $A(0)$, $A(1)$ et $A(-1)$.\end{enumerate}
\centering
\rule{1\linewidth}{0.6pt}
\end{exo}

\begin{exo}\textbf{}\quad\\[0.25cm]
  \begin{enumerate}
	\item Soit $f$ l'endomorphisme de $\R^3$ dont la matrice dans la
	base canonique est donnée par
	\[
	\frac{1}{6}\begin{pmatrix}
	5 & -2 & 1\\
	-2 & 2 & 2\\
	1 & 2 & 5
	\end{pmatrix}.
	\]
	Démontrer que $f$ est une projection sur un plan vectoriel dont on
	donnera une équation cartésienne, parallèlement à une droite
	vectorielle dont on donnera une base.
	\item Écrire, dans la base canonique de $\R^3$, la matrice de la
	symétrie par rapport au plan d'équation $x+y+z=0$ parallèlement à
	la droite engendrée par le vecteur $(1,2,-2)$.
\end{enumerate}

\centering
\rule{1\linewidth}{0.6pt}
\end{exo}










\end{document}