\documentclass[a4paper,10pt]{article}



\usepackage{fancyhdr} % pour personnaliser les en-têtes
\usepackage[utf8]{inputenc}
\usepackage[T1]{fontenc}
\usepackage{lastpage}
\usepackage[frenchb]{babel}
\usepackage{amsfonts,amssymb}
\usepackage{amsmath,amsthm,mathtools}
\usepackage{paralist}
\usepackage{xspace}
\usepackage{xcolor,multicol}
\usepackage{variations}
\usepackage{xypic}
\usepackage{eurosym}
\usepackage{graphicx}
\usepackage{mathdots}%faire des points suspendus en diagonale
\usepackage[np]{numprint}
\usepackage{hyperref} 
\usepackage{listings} % pour écrire des codes avec coloration syntaxique  

\usepackage{tikz}
\usetikzlibrary{calc, arrows, plotmarks,decorations.pathreplacing}
\usepackage{colortbl}
\usepackage{multirow}
\usepackage[top=2cm,bottom=1.5cm,right=2cm,left=1.5cm]{geometry}

\newtheorem{thm}{Théorème}
\newtheorem*{pro}{Propriété}
\newtheorem*{exemple}{Exemple}

\theoremstyle{definition}
\newtheorem*{remarque}{Remarque}
\theoremstyle{definition}
\newtheorem{exo}{Exercice}
\newtheorem{definition}{Définition}


\newcommand{\vtab}{\rule[-0.4em]{0pt}{1.2em}}
\newcommand{\V}{\overrightarrow}
\renewcommand{\thesection}{\Roman{section} }
\renewcommand{\thesubsection}{\arabic{subsection} }
\renewcommand{\thesubsubsection}{\alph{subsubsection} }
\newcommand*{\transp}[2][-3mu]{\ensuremath{\mskip1mu\prescript{\smash{\mathrm t\mkern#1}}{}{\mathstrut#2}}}%

\newcommand{\K}{\mathbb{K}}
\newcommand{\C}{\mathbb{C}}
\newcommand{\R}{\mathbb{R}}
\newcommand{\Q}{\mathbb{Q}}
\newcommand{\Z}{\mathbb{Z}}
\newcommand{\N}{\mathbb{N}}

\renewcommand{\Im}{\mathop{\mathrm{Im}}\nolimits}



\definecolor{vert}{RGB}{11,160,78}
\definecolor{rouge}{RGB}{255,120,120}
% Set the beginning of a LaTeX document
\pagestyle{fancy}
\lhead{Optimal Sup Spé, groupe IPESUP}\chead{Année~2021-2022}\rhead{Niveau: Première année de PCSI }\lfoot{M. Botcazou}\cfoot{\thepage}\rfoot{mail: ibotca52@gmail.com }\renewcommand{\headrulewidth}{0.4pt}\renewcommand{\footrulewidth}{0.4pt}

\begin{document}
	
	
	\begin{center}
		\Large \sc colle 16 = Déterminants et dénombrement 
	\end{center}










\section*{Déterminants:}\hfill\\[-0.25cm]
\begin{minipage}{1\linewidth}
	\begin{minipage}[t]{0.48\linewidth}
		\raggedright
		
		
		
		\begin{exo}\quad\\
		Calculer en mettant en évidence la factorisation le déterminant suivant:
				$$ D = 
	\begin{vmatrix}
	1&\cos a& \cos 2a\\1&\cos b& \cos 2b\\1&\cos c& \cos 2c
	\end{vmatrix}
	$$
			
			\centering
			\rule{1\linewidth}{0.6pt}
		\end{exo}
		
		
		
		\begin{exo}\quad\\
			Calculer les déterminants des matrices suivantes :
			
			$$
			\begin{pmatrix}
			a&b&c\\c&a&b\\b&c&a
			\end{pmatrix}
			\quad
			\begin{pmatrix}
			10 & 0 & -5 & 15 \\ -2 & 7 & 3 & 0 \\ 8 & 14 & 0 & 2 \\ 0 & -21 & 1 & -1
			\end{pmatrix}
			$$
			\centering
			\rule{1\linewidth}{0.6pt}
		\end{exo}
	
	
			\begin{exo}\quad\\
		Soit $a$ un réel.
		On note $\Delta_n$ le déterminant suivant : 
		$$
		\Delta_n = 
		\left\vert
		\begin{matrix}
		a   &    0   & \cdots & 0      & n-1 \\
		0   &    a   & \ddots & \vdots & \vdots \\
		\vdots & \ddots & \ddots & 0      & 2 \\
		0   & \cdots &   0    & a      & 1 \\
		n-1  & \cdots &   2    & 1      & a
		\end{matrix}
		\right\vert
		$$
		\begin{enumerate}
			\item Calculer $\Delta_n$ en fonction de $\Delta_{n-1}$.
			\item Démontrer que : $\displaystyle \forall n\geq2\quad
			\Delta_n=a^n-a^{n-2}\sum_{i=1}^{n-1}{i^2}$.
		\end{enumerate}
		
		\centering
		\rule{1\linewidth}{0.6pt}
	\end{exo}

		
		
		
	\end{minipage}	
	\hfill\vrule\hfill
	\begin{minipage}[t]{0.48\linewidth}
		\raggedright
		
				\begin{exo}\quad\\[0.25cm]
			Soit $(a_{0},...,a_{n-1})\in\C^{n}$, $x\in\C$. Calculer
			$$
			\Delta_{n}=
			\left|
			\begin{matrix}
			x &  0    &        & a_{0}   \\
			-1 &\ddots &\ddots  &\vdots  \\
			&\ddots &x      & a_{n-2} \\
			0 &       & -1      & x+a_{n-1}
			\end{matrix}
			\right|
			$$
			\hfill\\[0.25cm]
			\centering
			\rule{1\linewidth}{0.6pt}
		\end{exo}
		

		
		\begin{exo}\quad\\
		Calculer les déterminants des matrices suivantes :
		
		$$
		\begin{pmatrix}
		a&a&b&0 \\  a&a&0&b \\  c&0&a&a \\ 0&c&a&a
		\end{pmatrix}
		\quad
		\begin{pmatrix}
		1&0&3&0&0 \\ 0&1&0&3&0 \\ a&0&a&0&3 \\ b&a&0&a&0 \\ 0&b&0&0&a  
		\end{pmatrix}
		$$
		\hfill\\[0.25cm]
			
			\centering
			\rule{1\linewidth}{0.6pt}
		\end{exo}
		
		\begin{exo} \textit{\textbf{Déterminant de Vandermonde}}\quad\\
		Montrer que
		$$\left|
		\begin{array}{ccccc}
		1 & t_1 & t_1^2 & \ldots & t_1^{n-1} \\
		1 & t_2 & t_2^2 & \ldots & t_2^{n-1} \\\
		\ldots&\ldots&\ldots& \ldots & \ldots \\
		1 & t_n & t_n^2 & \ldots & t_n^{n-1}
		\end{array}\right|
		= \prod_{1 \le i < j \le n} (t_j - t_i) $$
				\hfill\\[0.25cm]
			\centering
			\rule{1\linewidth}{0.6pt}
		\end{exo}	
		
	\end{minipage}
\end{minipage}

\section*{Dénombrement - Combinatoire:}\hfill\\[-0.25cm]
\begin{minipage}{1\linewidth}
	\begin{minipage}[t]{0.48\linewidth}
		\raggedright
		
		
		
		\begin{exo}\quad\\
			Dénombrer les anagrammes des mots suivants:
			$$MATHS, \quad RIRE,\quad ANANAS$$
			
			\centering
			\rule{1\linewidth}{0.6pt}
		\end{exo}
		
		
		
		\begin{exo}\quad\\
			Un damier est un plateau carré contenant 100 cases.
			\begin{enumerate}
				\item Combien y a-t-il de manières de placer 50 pièces blanches et 50 pièces noires sur ce damier ?
				\item Deux pièces sont dîtes côte à côte si l'une des arêtes de la case où elles se situent respectivement est en commun. Combien y a-t-il de manière de placer 50 pièces noires telles q'au moins deux pièces noires soient côte à côte. 
				\item Soient $n_1,n_2,n_3,n_4 \in\N$ tels que $n_1 + n_2 + n_3 +n_4 = 100$. On dispose de $n_1$ pièces noires, $n_2$ pièces blanches, $n_3$ pièces bleus et $n_4$ pièces rouges. Combien y a-t-il de manières différentes différentes de placer toutes ces pièces sur un damier. 
			\end{enumerate}
			\centering
			\rule{1\linewidth}{0.6pt}
		\end{exo}

		
		
		
		
	\end{minipage}	
	\hfill\vrule\hfill
	\begin{minipage}[t]{0.48\linewidth}
		\raggedright
		
		\begin{exo}\quad\\
			Lors d'une loterie de Noël, $300$ billets sont
			vendus aux enfants de l'école ; $4$ billets sont gagnants.
			J'achète $10$ billets, quelle est la probabilité pour que je gagne au moins un lot?
			
			\centering
			\rule{1\linewidth}{0.6pt}
		\end{exo}
		
		
		
		\begin{exo}\quad\\
			La probabilité pour une population d'être atteinte
			d'une maladie $A$ est $p$ donné; dans cette même population, un individu
			peut être atteint par une maladie $B$ avec une probabilité $q$ donnée aussi; 
			on suppose que les maladies sont indépendantes : quelle est la
			probabilité d'être atteint par l'une et l'autre de ces maladies?
			Quelle est la probabilité d'être atteint par l'une ou l'autre de ces
			maladies?
			
			\centering
			\rule{1\linewidth}{0.6pt}
		\end{exo}
		
		\begin{exo} \quad\\
			Dans un jeu de $52$ cartes, on prend une carte au hasard : les événements <<tirer un roi>> et 
			<<tirer un pique>> sont-ils indépendants? quelle est la probabilité de <<tirer un roi ou
			un pique>> ?
			
			\centering
			\rule{1\linewidth}{0.6pt}
		\end{exo}
		
		
		
		
		
		
	\end{minipage}
\end{minipage}


\end{document}