\documentclass[a4paper,10pt]{article}



\usepackage{fancyhdr} % pour personnaliser les en-têtes
\usepackage[utf8]{inputenc}
\usepackage[T1]{fontenc}
\usepackage{lastpage}
\usepackage[frenchb]{babel}
\usepackage{amsfonts,amssymb}
\usepackage{amsmath,amsthm,mathtools}
\usepackage{paralist}
\usepackage{xspace}
\usepackage{xcolor,multicol}
\usepackage{variations}
\usepackage{xypic}
\usepackage{eurosym}
\usepackage{graphicx}
\usepackage[np]{numprint}
\usepackage{hyperref} 
\usepackage{listings} % pour écrire des codes avec coloration syntaxique  

\usepackage{tikz}
\usetikzlibrary{calc, arrows, plotmarks,decorations.pathreplacing}
\usepackage{colortbl}
\usepackage{multirow}
\usepackage[top=2cm,bottom=1.5cm,right=2cm,left=1.5cm]{geometry}

\newtheorem{thm}{Théorème}
\newtheorem*{pro}{Propriété}
\newtheorem*{exemple}{Exemple}

\theoremstyle{definition}
\newtheorem*{remarque}{Remarque}
\theoremstyle{definition}
\newtheorem{exo}{Exercice}
\newtheorem{definition}{Définition}


\newcommand{\vtab}{\rule[-0.4em]{0pt}{1.2em}}
\newcommand{\V}{\overrightarrow}
\renewcommand{\thesection}{\Roman{section} }
\renewcommand{\thesubsection}{\arabic{subsection} }
\renewcommand{\thesubsubsection}{\alph{subsubsection} }
\newcommand*{\transp}[2][-3mu]{\ensuremath{\mskip1mu\prescript{\smash{\mathrm t\mkern#1}}{}{\mathstrut#2}}}%

\newcommand{\C}{\mathbb{C}}
\newcommand{\R}{\mathbb{R}}
\newcommand{\Q}{\mathbb{Q}}
\newcommand{\Z}{\mathbb{Z}}
\newcommand{\N}{\mathbb{N}}



\definecolor{vert}{RGB}{11,160,78}
\definecolor{rouge}{RGB}{255,120,120}
% Set the beginning of a LaTeX document
\pagestyle{fancy}
\lhead{Optimal Sup Spé, groupe IPESUP}\chead{Année~2021-2022}\rhead{Niveau: Première année de PCSI }\lfoot{M. Botcazou}\cfoot{\thepage/2}\rfoot{mail: ibotca52@gmail.com }\renewcommand{\headrulewidth}{0.4pt}\renewcommand{\footrulewidth}{0.4pt}

\begin{document}
	
	
	\begin{center}
		\Large \sc colle7 =  Matrices et fonctions dérivables
	\end{center}
	
\section *{Questions de cours:}

\noindent Soient $n,p\in\N$ et  $A,B\in\mathcal{M}_{n}\left(\R\right)$.
\begin{enumerate} 
\item Rappeler la définition d'une matrice élémentaire de  $\mathcal{M}_{n,p}\left(\R\right)$ et expliquer pourquoi toute matrice\\ $A\in\mathcal{M}_{n,p}\left(\R\right)$ peut se décomposer comme combinaison linéaire de matrices élémentaires de $\mathcal{M}_{n,p}\left(\R\right)$.
\item  Pour $(i,j)\in\{1,..,n\}^2$ donner le coefficient $AB_{ij}$ en fonction des coefficients des matrices A et B. Montrer que $Tr(AB) = Tr(BA)$.  
\item Montrer que $(AB)^T=B^TA^T$. Si $A$ est une matrice symétrique inversible, montrer que $A^{-1}$ est aussi une matrice symétriques. 
\item Démontrer la propriété suivante:
\begin{pro}\hfil\\
Soient $D\subset\R$, $f:D\rightarrow\C$ une fonction et $a\in D$. Si $f$ est dérivable en $a$, alors $f$ est continue en $a$.
\end{pro}
\item Soit $n\in\N$. Démontrer que la fonction $x \in\R \mapsto x^n$ est dérivable en tout point de $\R$ et donner sa fonction dérivée associée. 
\item Démontrer que la fonction $$\begin{array}{ccl}
\R & \rightarrow & \R\\
x & \longmapsto & \left\{\begin{array}{cl}
x\sin\left(\dfrac{1}{x}\right) & \text{si } x\neq 0\\
0& \text{sinon}
\end{array}\right.
\end{array}$$
\noindent est continue en $0$ mais non dérivable en $0$.
\end{enumerate}
\section*{Matrices:}
\begin{minipage}{1\linewidth}
\begin{minipage}[t]{0.48\linewidth}
\raggedright

\begin{exo}\quad\\
Soit $$A \ = \ \begin{pmatrix}
0 & 1 & -1\\
-1 &  2& -1\\
1 & -1 & 2
\end{pmatrix}$$
\begin{enumerate}
\item Montrer que le polynôme $P(X) = X^2-3X+2$ est anulateur de la matrice $A$.
\item Donner le reste de la division Euclidienne de $X^n$ par  $X^2-3X+2$ pour $n\geq 2$.
\item En déduire la valeur de $A^n$. 
\end{enumerate}

\centering
\rule{1\linewidth}{0.6pt}
\end{exo}

\begin{exo}\quad\\
Les affirmations suivantes sont-elles vraies ? 
\begin{enumerate}
\item $\forall A,B,C\in\mathcal{M}_{2}\left(\R\right) : Tr(ABC) = Tr(BAC)  $
\item $\exists A,B \in \mathcal{M}_{n}\left(\R\right) : AB - BA = I_n $
\item  Soient $A,B \in \mathcal{M}_{n}\left(\R\right) $  tels que $AB - BA = A $. Alors pour tout $n\in\N^*$ on a $Tr(A^n) = 0$
\end{enumerate}

\centering\rule{1\linewidth}{0.6pt}
\end{exo}

\begin{exo}\quad\\
Soient $A$ et $B$ deux matrices de tailles $n$ vérifiant $AB-BA=A$ .Montrer que pout tout entier naturel $k$
$$A^{k+1}B  - BA^{k+1}= (k+ 1)A^{k+1}$$


\centering\rule{1\linewidth}{0.6pt}
\end{exo}



\end{minipage}	
\hfill\vrule\hfill
\begin{minipage}[t]{0.48\linewidth}
\raggedright

\begin{exo}\quad\\
Soit  $A\in\mathcal{M}_{3}\left(\R\right)$ 
\begin{enumerate}
\item Pour tous $(i,j)\in\{1,2,3\}$ on note $E_{ij}$ une matrice élémentaire de $\mathcal{M}_{3}\left(\R\right)$. Expliquer ce que donne les produits matriciels $ I_3E_{ij}$ et $E_{ij}I_3$. 
\item Considérons le centre de $\mathcal{M}_{3}\left(\R\right)$ :
 $$ \mathcal{Z}\left(\mathcal{M}_{3}\left(\R\right)\right) : \left\{A\in\mathcal{M}_{3}\left(\R\right); \forall M\in\mathcal{M}_{3}\left(\R\right)  : MA  = AM \right \}$$
 Montrer que: 
 $$\mathcal{Z}\left(\mathcal{M}_{3}\left(\R\right)\right) = \left\{\lambda I_3 : \lambda\in\R \right \}$$
 
\end{enumerate}
\centering
\rule{1\linewidth}{0.6pt}
\end{exo}

\begin{exo}\quad\\
Calculer les puissances n-ième des matrices suivantes: $$A  =  \begin{pmatrix}
1 & 1 \\
0 &  2\\

\end{pmatrix} \ , \ A  =  \begin{pmatrix}
a & b \\
0 &  a\\

\end{pmatrix}  \ , \ A  =  \begin{pmatrix}
\cos(\theta) & -\sin(\theta)  \\
\sin(\theta) &  \cos(\theta) \\

\end{pmatrix}
$$


\centering\rule{1\linewidth}{0.6pt}
\end{exo}

\begin{exo}\quad\\
Soit $T$ une matrice triangulaire supérieure de taille $n\in\N$. Montrer que $T$ commute avec sa transposée, si et seulement si $T$ est diagonale.

\centering\rule{1\linewidth}{0.6pt}
\end{exo}






\end{minipage}
\end{minipage}
\newpage

\section*{Fonctions dérivables:}
\begin{minipage}{1\linewidth}
\begin{minipage}[t]{0.48\linewidth}
\raggedright

\begin{exo}\quad\\
Étudier la dérivabilité sur $\R$ des fonctions suivantes:
\begin{enumerate}
\item$x \longmapsto\left\{\begin{array}{cl}
(x-1)^2 & \text{Si } x\leq 1\\
(x-1)^3 & \text{Si } x> 1
\end{array}\right.$
\item $x \longmapsto\left\{\begin{array}{cl}
x^2+x & \text{Si } x\leq 1\\
ax^3+bx + 1 & \text{Si } x> 1
\end{array}\right.$  , $(a,b)\in\R^2$
\item $x \longmapsto \dfrac{|x|}{1 + |x^2-1|}$
\end{enumerate}
\centering\rule{1\linewidth}{0.6pt}
\end{exo}

\begin{exo}\quad\\
\noindent Soit $f\in\mathcal{C}^1\left(\R,\R\right)$. On fait l'hypothèse que:
 $$\forall x\in\R : f\circ f\left(x\right) = \dfrac{x}{4} +1$$
 \begin{enumerate}
\item Montrer que : $f'(x) = f'\left( \dfrac{x}{4} +1\right)  $ pour tout $x\in\R$.
\item En déduire de $f'$ est une fonction constante sur $\R$
\item Déterminer les fonctions $f\in\mathcal{C}^1\left(\R,\R\right)$ telles que $f\circ f\left(x\right) = \dfrac{x}{4} +1$ pour tout $x\in\R$.
 \end{enumerate}
\centering
\rule{1\linewidth}{0.6pt}
\end{exo}

\begin{exo}
Soit $a\in\R$ et $f:\R\rightarrow\R$ dérivable en a. \\
Que vaut $\lim\limits_{h\rightarrow 0}\dfrac{f(x+h)- f(x+h^2)}{h}$?

\centering\rule{1\linewidth}{0.6pt}
\end{exo}








\end{minipage}	
\hfill\vrule\hfill
\begin{minipage}[t]{0.48\linewidth}
\raggedright



\begin{exo}\quad\\
\begin{enumerate}
\item On note $f$ la fonction $x \ \mapsto \ e^{x^2}$. Montrer qu’il existe pour tout $n\in\N$ une fonction polynomiale $P_n$ pour laquelle $f^{(n)}(x) = P_ n(x)e^{x^2}$ pour tout $x\in\R$ ,puis déterminer le degré de $Pn$.
\item  On note $f$ la fonction $x \ \mapsto \ \dfrac{1}{1+x^2} $. Montrer qu’il existe pour tout $n\in\N$ une fonction polynomiale $P_n$ pour laquelle $f^{(n)}(x) = \dfrac{P_ n(x)}{(1+x^2)^{n+1}}$ pour tout $x\in\R$ ,puis déterminer le degré de $Pn$.
\end{enumerate}
\centering\rule{1\linewidth}{0.6pt}
\end{exo}

\begin{exo}\quad\\
Pour tous $n\in\N^*$, calculer la dérivée $n^{\text{ème}}$ \\de $x \mapsto  x^{n-1}\ln(1+x)$ sur $]-1,+\infty[$.

\centering\rule{1\linewidth}{0.6pt}
\end{exo}

\begin{exo}\quad\\
Montrer que pour tous $n\in\N*$ et $x\in\R*$:
$$\dfrac{d^n}{dx^n}\left(x^{(n-1)} \exp\left(\tiny\dfrac{1}{x}\right)\right) = \dfrac{(-1)^n}{x^{n+1}} \exp\left(\tiny\dfrac{1}{x}\right)$$

\centering\rule{1\linewidth}{0.6pt}
\end{exo}






\end{minipage}
\end{minipage}

\end{document}