\documentclass[11pt]{article}

 %Configuration de la feuille 
 
\usepackage{amsmath,amssymb,enumerate,graphicx,pgf,tikz,fancyhdr}
\usepackage[utf8]{inputenc}
\usetikzlibrary{arrows}
\usepackage{geometry}
\usepackage{tabvar}
\geometry{hmargin=2.2cm,vmargin=1.5cm}\pagestyle{fancy}
\lfoot{\bfseries http://www.bibmath.net}
\rfoot{\bfseries\thepage}
\cfoot{}
\renewcommand{\footrulewidth}{0.5pt} %Filet en bas de page

 %Macros utilisées dans la base de données d'exercices 

\newcommand{\mtn}{\mathbb{N}}
\newcommand{\mtns}{\mathbb{N}^*}
\newcommand{\mtz}{\mathbb{Z}}
\newcommand{\mtr}{\mathbb{R}}
\newcommand{\mtk}{\mathbb{K}}
\newcommand{\mtq}{\mathbb{Q}}
\newcommand{\mtc}{\mathbb{C}}
\newcommand{\mch}{\mathcal{H}}
\newcommand{\mcp}{\mathcal{P}}
\newcommand{\mcb}{\mathcal{B}}
\newcommand{\mcl}{\mathcal{L}}
\newcommand{\mcm}{\mathcal{M}}
\newcommand{\mcc}{\mathcal{C}}
\newcommand{\mcmn}{\mathcal{M}}
\newcommand{\mcmnr}{\mathcal{M}_n(\mtr)}
\newcommand{\mcmnk}{\mathcal{M}_n(\mtk)}
\newcommand{\mcsn}{\mathcal{S}_n}
\newcommand{\mcs}{\mathcal{S}}
\newcommand{\mcd}{\mathcal{D}}
\newcommand{\mcsns}{\mathcal{S}_n^{++}}
\newcommand{\glnk}{GL_n(\mtk)}
\newcommand{\mnr}{\mathcal{M}_n(\mtr)}
\DeclareMathOperator{\ch}{ch}
\DeclareMathOperator{\sh}{sh}
\DeclareMathOperator{\vect}{vect}
\DeclareMathOperator{\card}{card}
\DeclareMathOperator{\comat}{comat}
\DeclareMathOperator{\imv}{Im}
\DeclareMathOperator{\rang}{rg}
\DeclareMathOperator{\Fr}{Fr}
\DeclareMathOperator{\diam}{diam}
\DeclareMathOperator{\supp}{supp}
\newcommand{\veps}{\varepsilon}
\newcommand{\mcu}{\mathcal{U}}
\newcommand{\mcun}{\mcu_n}
\newcommand{\dis}{\displaystyle}
\newcommand{\croouv}{[\![}
\newcommand{\crofer}{]\!]}
\newcommand{\rab}{\mathcal{R}(a,b)}
\newcommand{\pss}[2]{\langle #1,#2\rangle}
 %Document 

\begin{document} 

%\begin{center}\textsc{{\huge https://www.bibmath.net/ressources/index.php?action=affiche&quoi=bde/analyse/une}}\end{center}

% Exercice 200


\vskip0.3cm\noindent\textsc{Exercice 1} - Majorée par une fonction affine - avec détails
\vskip0.2cm
Soit $F:[0,+\infty[\to\mathbb R$ une application uniformément continue. On se propose de démontrer qu'il existe
deux réels $a$ et $b$ tels que, pour tout $x\in[0,+\infty[$, on ait $F(x)\leq ax+b$. Pour cela, on commence par fixer $\eta_1>0$
tel que 
$$\forall (x,y)\in([0,+\infty[)^2,\ \big(|x-y|<\eta_1\implies |F(x)-F(y)|\leq 1\big).$$
On fixe également $x_0\in[0,+\infty[$.
\begin{enumerate}
\item Soit $n_0$ le plus petit entier tel que $\frac{x_0}{n_0}\leq \eta_1$; justifier l'existence de $n_0$ et démontrer que $n_0\leq \frac{x_0}{\eta_1}+1$.
\item Montrer que 
$$|F(x)-F(x_0)|\leq \sum_{k=0}^{n_0-1}\left|F\left(\frac{(k+1)x_0}{n_0}\right)-F\left(\frac{kx_0}{n_0}\right)\right|.$$
\item Conclure.
\item La fonction exponentielle est-elle uniformément continue sur $[0,+\infty[$?
\end{enumerate}


% Exercice 201


\vskip0.3cm\noindent\textsc{Exercice 2} - Majorée par une fonction affine
\vskip0.2cm
Soit $f:[0,+\infty[\to\mathbb R$ une fonction uniformément continue.
Montrer qu'il existe deux réels $a$ et $b$ tels que
$|f(x)|\leq ax+b$ pour tout $x\geq 0$.


% Exercice 3060


\vskip0.3cm\noindent\textsc{Exercice 3} - Une fonction définie "par morceaux"
\vskip0.2cm
Soit $f,g:\mathbb R\to\mathbb R$ deux fonctions continues. On définit une fonction $h:\mathbb R\to\mathbb R$ par 
$$h(x)=\left\{
\begin{array}{ll}
f(x)&\textrm{ si }x\in\mathbb Q\\
g(x)&\textrm{ si }x\in\mathbb R\backslash \mathbb Q.
\end{array}\right.$$
Démontrer que $h$ est continue en $x_0\in\mathbb R$ si et seulement si $f(x_0)=g(x_0)$.




\vskip0.5cm
\noindent{\small Cette feuille d'exercices a été conçue à l'aide du site \textsf{https://www.bibmath.net}}

%Vous avez accès aux corrigés de cette feuille par l'url : https://www.bibmath.net/ressources/justeunefeuille.php?id=26177
\end{document}