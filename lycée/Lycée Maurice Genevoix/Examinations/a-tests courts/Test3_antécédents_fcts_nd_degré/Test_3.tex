\documentclass[a4paper,10pt]{article}




\usepackage[utf8]{inputenc}
\usepackage[T1]{fontenc}
\usepackage{lastpage}
\usepackage{fancyhdr} % pour personnaliser les en-têtes
\usepackage[frenchb]{babel}
\usepackage{amsfonts,amssymb}
\usepackage{amsmath,amsthm}
\usepackage{paralist}
\usepackage{xspace}
\usepackage{xcolor}
\usepackage{variations}
\usepackage{xypic}
\usepackage{eurosym}
\usepackage{graphicx}
\usepackage[np]{numprint}
\usepackage{hyperref} 
\usepackage{listings} % pour écrire des codes avec coloration syntaxique  

\usepackage{tikz}
\usetikzlibrary{calc, arrows, plotmarks,decorations.pathreplacing}
\usepackage{colortbl}
\usepackage{multirow}
\usepackage[top=1.5cm,bottom=1.5cm,right=1.5cm,left=1.5cm]{geometry}



\theoremstyle{definition}
\newtheorem*{remarque}{Remarque}
\newtheorem{exo}{Exercice}
\newtheorem{definition}{Définition}


\newcommand{\vtab}{\rule[-0.4em]{0pt}{1.2em}}
\newcommand{\V}{\overrightarrow}
\renewcommand{\thesection}{\Roman{section} }
\renewcommand{\thesubsection}{\arabic{subsection} }
\renewcommand{\thesubsubsection}{\alph{subsubsection} }

\newcommand{\C}{\mathbb{C}}
\newcommand{\R}{\mathbb{R}}
\newcommand{\Q}{\mathbb{Q}}
\newcommand{\Z}{\mathbb{Z}}
\newcommand{\N}{\mathbb{N}}


\definecolor{vert}{RGB}{11,160,78}
\definecolor{rouge}{RGB}{255,120,120}

\pagestyle{fancy}
\lhead{}\chead{}\rhead{}\lfoot{Mr Botcazou}\cfoot{\thepage}\rfoot{\textbf{Tourner la page S.V.P.}}\renewcommand{\headrulewidth}{0pt}\renewcommand{\footrulewidth}{0.4pt}%Tourner la page S.V.P.



\begin{document}
	
	\leftline{\bfseries Lycée Maurice Genevoix \hfill Année~2021-2022}
	\leftline{\bfseries Nom: }
	\leftline{\bfseries Prénom:\hfill $Seconde_{.....}$}
	\rule[0.5ex]{\textwidth}{0.1mm}	
	
	\begin{center}
		\large \sc Test 2 : Fonctions polynomiales du second degré, Antécédents, tableau de signes et résolution d'équation\\[0.2cm]
		(30 minutes)
	\end{center}
\begin{exo} \textit{\textbf{Construire un tableau de signes double}}\\[0.2cm]
	Pour tout  $x\in\R$ on définit: $$h(x) = (3x+3)(3x-4)$$ \hfill\\\\[-0.5cm] Donner le tableau de signes de $h(x)$ en fonction de $x$ et en déduire\\ pour quelles valeurs de $x\in\R$ on a:\quad 
	$h(x) \geq 0$.\hfill\textbf{/3}\\[10cm]
\end{exo}




\begin{exo} \textit{\textbf{Trouver un point d'intersection}}\\[0.2cm]
	Pour tout  $x\in\R$ on définit:$$f(x) = (5x-4)(7x-2) \quad \text{et} \quad g(x) = (5x-4)^2$$
Résoudre dans $\R$ l'équation suivante:\hfill\textbf{/1.5} 
$$f(x) = g(x) $$	%\hfill\textbf{/1}\\\\

\end{exo}
\newpage
\lhead{}\chead{}\rhead{}\lfoot{Mr Botcazou}\cfoot{\thepage}\rfoot{\textbf{FIN.}}\renewcommand{\headrulewidth}{0pt}\renewcommand{\footrulewidth}{0.4pt}%Tourner la page S.V.P.

\begin{exo} \textit{\textbf{Utiliser la bonne forme pour résoudre une équation:}}\\[0.2cm]
	Pour tout  $x\in\R$ on définit: $$f(x)= 9x^2-12x-5 $$ 
	\hfill\\
	\begin{enumerate}
		\item Montrer que pour tout $x\in\R$ on a $f(x)= 9(x- \dfrac{5}{3})(x+ \dfrac{1}{3}) $ ~  	et en déduire \\
		les antécédents de $0$ par la fonction $f$. \hfill\textbf{/2.25}\\[7cm]
		
		\item Montrer que pour tout $x\in\R$ on a $f(x)= (3x- 2)^2-9 $ ~  	et en déduire \\
		les antécédents de $7$ par la fonction $f$. \hfill\textbf{/2.25}\\[7cm]
	\end{enumerate}
	
\end{exo}

\begin{exo} \hfill\textbf{/1}\\[0.2cm]
\noindent Pour tout  $x\in\R$ on définit:\quad $m(x) = -\dfrac{2}{3}x - 7 $  \\[0.2cm] 
\noindent Quelle est la nature de la fonction $m$ et donner pour quelles valeurs de $x$, l'image de $x$ par la fonction $m$ est inférieure strictement à $7$.

\end{exo}





\end{document}





