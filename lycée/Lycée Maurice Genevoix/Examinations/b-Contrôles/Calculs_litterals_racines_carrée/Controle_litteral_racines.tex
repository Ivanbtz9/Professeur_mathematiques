\documentclass[a4paper,10pt]{article}




\usepackage[utf8]{inputenc}
\usepackage[T1]{fontenc}
\usepackage{lastpage}
\usepackage{fancyhdr} % pour personnaliser les en-têtes
\usepackage[frenchb]{babel}
\usepackage{amsfonts,amssymb}
\usepackage{amsmath,amsthm}
\usepackage{paralist}
\usepackage{xspace}
\usepackage{xcolor}
\usepackage{variations}
\usepackage{xypic}
\usepackage{eurosym}
\usepackage{graphicx}
\usepackage[np]{numprint}
\usepackage{hyperref} 
\usepackage{listings} % pour écrire des codes avec coloration syntaxique  

\usepackage{tikz}
\usetikzlibrary{calc, arrows, plotmarks,decorations.pathreplacing}
\usepackage{colortbl}
\usepackage{multirow}
\usepackage[top=1.5cm,bottom=1.5cm,right=1.5cm,left=1.5cm]{geometry}



\theoremstyle{definition}
\newtheorem*{remarque}{Remarque}
\newtheorem{exo}{Exercice}
\newtheorem{definition}{Définition}


\newcommand{\vtab}{\rule[-0.4em]{0pt}{1.2em}}
\newcommand{\V}{\overrightarrow}
\renewcommand{\thesection}{\Roman{section} }
\renewcommand{\thesubsection}{\arabic{subsection} }
\renewcommand{\thesubsubsection}{\alph{subsubsection} }

\newcommand{\C}{\mathbb{C}}
\newcommand{\R}{\mathbb{R}}
\newcommand{\Q}{\mathbb{Q}}
\newcommand{\Z}{\mathbb{Z}}
\newcommand{\N}{\mathbb{N}}


\definecolor{vert}{RGB}{11,160,78}
\definecolor{rouge}{RGB}{255,120,120}

\pagestyle{fancy}
\lhead{}\chead{}\rhead{}\lfoot{}\cfoot{\thepage}\rfoot{\textbf{}}\renewcommand{\headrulewidth}{0pt}\renewcommand{\footrulewidth}{0.4pt}%Tourner la page S.V.P.



\begin{document}
	
	\leftline{\bfseries Lycée Maurice Genevoix \hfill Année~2021-2022}
	\leftline{\bfseries Nom: }
	\leftline{\bfseries Prénom:\hfill $Seconde_{.....}$}
	\rule[0.5ex]{\textwidth}{0.1mm}	
	
	\begin{center}
		\large \sc Contrôle sur le calcul littérale et les racines carrées\\
		(35 minutes)
	\end{center}



\begin{exo} \textit{\textbf{}}\\\\
\begin{enumerate}
	\item Simplifier les nombres suivants:\hfill\textbf{/2}\\\\
	\begin{enumerate}[$\square$]
		
		\item $\sqrt{28}\times \dfrac{2^2}{2^{-3}} \ = \  $.\dotfill \\\\
		
		\item $\sqrt{\dfrac{50}{2}}\times (5^{-2})^3 \ = \ $ .\dotfill \\\\\\
		
	\end{enumerate}
	\item Développer, réduire et ordonner l'expression littérale suivante:\hfill\textbf{/1.5}\\
	$$B \ = \ (5 - 2x)^2 - (7 + 3x)(8 - 2x)$$
	\bigskip \bigskip \bigskip \bigskip \bigskip \bigskip \bigskip \bigskip \bigskip \bigskip \bigskip \bigskip \bigskip \bigskip \bigskip \bigskip \bigskip
	\item 	Factoriser l'expression littérale suivante: \hfill\textbf{/1.5}\\
	$$C \ = \ (11 + 7y)(8 + 9v) - (4 + v)(7y + 11) $$
	\bigskip \bigskip \bigskip \bigskip \bigskip \bigskip \bigskip \bigskip \bigskip \bigskip \bigskip \bigskip \bigskip \bigskip \bigskip \bigskip 
	
	\newpage
	\item Donner le signe de $-\dfrac{5}{3}x+5$ dans un tableau de signes:\hfill\textbf{/2.5}\\\bigskip \bigskip \bigskip \bigskip \bigskip \bigskip \bigskip \bigskip \bigskip \bigskip 
\bigskip \bigskip \bigskip \bigskip \bigskip \bigskip \bigskip \bigskip \bigskip \bigskip \bigskip \bigskip \bigskip 
	\item Factoriser les expressions suivantes à l'aide des identités remarquables:\hfill\textbf{/1.5}\\ \\[0.5cm]
	\begin{enumerate}
		\item $36x^2 - 84x + 49 \ = \ $\bigskip \bigskip \bigskip \bigskip \bigskip \bigskip \bigskip \bigskip \bigskip \bigskip 
		\item $ 9u^2 - 49v^2\ = \ $ \bigskip \bigskip \bigskip \bigskip \bigskip \bigskip \bigskip \bigskip 
	\end{enumerate}
\item 	Résoudre dans $\R$ l'équation suivante: \hfill\textbf{/1}\\
$$(x-9)^2-64 = 0$$
\end{enumerate}


\end{exo}


\end{document}





